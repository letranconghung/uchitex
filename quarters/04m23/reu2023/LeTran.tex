\documentclass[openany, amssymb, psamsfonts]{amsart}
\usepackage{mathrsfs,comment}
\usepackage[usenames,dvipsnames]{color}
\usepackage[normalem]{ulem}
\usepackage{url}
\usepackage[all,arc,2cell]{xy}
\UseAllTwocells
\usepackage{enumerate}
%%% hyperref stuff is taken from AGT style file
\usepackage{hyperref} 
\usepackage{mathtools}
\usepackage{amsmath}
\usepackage[shortlabels]{enumitem}
\usepackage{esint}
\usepackage{pgfplots}
\pgfplotsset{compat = newest}
\hypersetup{%
  bookmarksnumbered=true,%
  bookmarks=true,%
  colorlinks=true,%
  linkcolor=blue,%
  citecolor=blue,%
  filecolor=blue,%
  menucolor=blue,%
  pagecolor=blue,%
  urlcolor=blue,%
  pdfnewwindow=true,%
  pdfstartview=FitBH}   
  
\let\fullref\autoref
%
%  \autoref is very crude.  It uses counters to distinguish environments
%  so that if say {lemma} uses the {theorem} counter, then autrorefs
%  which should come out Lemma X.Y in fact come out Theorem X.Y.  To
%  correct this give each its own counter eg:
%                 \newtheorem{theorem}{Theorem}[section]
%                 \newtheorem{lemma}{Lemma}[section]
%  and then equate the counters by commands like:
%                 \makeatletter
%                   \let\c@lemma\c@theorem
%                  \makeatother
%
%  To work correctly the environment name must have a corrresponding 
%  \XXXautorefname defined.  The following command does the job:
%
\def\makeautorefname#1#2{\expandafter\def\csname#1autorefname\endcsname{#2}}
%
%  Some standard autorefnames.  If the environment name for an autoref 
%  you need is not listed below, add a similar line to your TeX file:
%  
%\makeautorefname{equation}{Equation}%
\def\equationautorefname~#1\null{(#1)\null}
\makeautorefname{footnote}{footnote}%
\makeautorefname{item}{item}%
\makeautorefname{figure}{Figure}%
\makeautorefname{table}{Table}%
\makeautorefname{part}{Part}%
\makeautorefname{appendix}{Appendix}%
\makeautorefname{chapter}{Chapter}%
\makeautorefname{section}{Section}%
\makeautorefname{subsection}{Section}%
\makeautorefname{subsubsection}{Section}%
\makeautorefname{theorem}{Theorem}%
\makeautorefname{thm}{Theorem}%
\makeautorefname{cor}{Corollary}%
\makeautorefname{lem}{Lemma}%
\makeautorefname{prop}{Proposition}%
\makeautorefname{pro}{Property}
\makeautorefname{conj}{Conjecture}%
\makeautorefname{defn}{Definition}%
\makeautorefname{notn}{Notation}
\makeautorefname{notns}{Notations}
\makeautorefname{rem}{Remark}%
\makeautorefname{quest}{Question}%
\makeautorefname{exmp}{Example}%
\makeautorefname{ax}{Axiom}%
\makeautorefname{claim}{Claim}%
\makeautorefname{ass}{Assumption}%
\makeautorefname{asss}{Assumptions}%
\makeautorefname{con}{Construction}%
\makeautorefname{prob}{Problem}%
\makeautorefname{warn}{Warning}%
\makeautorefname{obs}{Observation}%
\makeautorefname{conv}{Convention}%


%
%                  *** End of hyperref stuff ***

%theoremstyle{plain} --- default
\newtheorem{thm}{Theorem}[section]
\newtheorem{cor}{Corollary}[section]
\newtheorem{prop}{Proposition}[section]
\newtheorem{lem}{Lemma}[section]
\newtheorem{prob}{Problem}[section]
\newtheorem{conj}{Conjecture}[section]
%\newtheorem{ass}{Assumption}[section]
%\newtheorem{asses}{Assumptions}[section]

\theoremstyle{definition}
\newtheorem{defn}{Definition}[section]
\newtheorem{ass}{Assumption}[section]
\newtheorem{asss}{Assumptions}[section]
\newtheorem{ax}{Axiom}[section]
\newtheorem{con}{Construction}[section]
\newtheorem{exmp}{Example}[section]
\newtheorem{notn}{Notation}[section]
\newtheorem{notns}{Notations}[section]
\newtheorem{pro}{Property}[section]
\newtheorem{quest}{Question}[section]
\newtheorem{rem}{Remark}[section]
\newtheorem{warn}{Warning}[section]
\newtheorem{sch}{Scholium}[section]
\newtheorem{obs}{Observation}[section]
\newtheorem{conv}{Convention}[section]

%%%% hack to get fullref working correctly
\makeatletter
\let\c@obs=\c@thm
\let\c@cor=\c@thm
\let\c@prop=\c@thm
\let\c@lem=\c@thm
\let\c@prob=\c@thm
\let\c@con=\c@thm
\let\c@conj=\c@thm
\let\c@defn=\c@thm
\let\c@notn=\c@thm
\let\c@notns=\c@thm
\let\c@exmp=\c@thm
\let\c@ax=\c@thm
\let\c@pro=\c@thm
\let\c@ass=\c@thm
\let\c@warn=\c@thm
\let\c@rem=\c@thm
\let\c@sch=\c@thm
\let\c@equation\c@thm
\numberwithin{equation}{section}
\makeatother

%%%% MATH SHORTHANDS %%%%
%% blackboard bold math capitals
\newcommand{\bbf}{\mathbb{F}}
\newcommand{\bbn}{\mathbb{N}}
\newcommand{\bbq}{\mathbb{Q}}
\newcommand{\bbr}{\mathbb{R}}
\newcommand{\bbz}{\mathbb{Z}}
\newcommand{\bbc}{\mathbb{C}}
\newcommand{\bbk}{\mathbb{K}}
\newcommand{\bbm}{\mathbb{M}}
\newcommand{\bbp}{\mathbb{P}}
\newcommand{\bbe}{\mathbb{E}}


\newcommand{\calb}{\mathcal{B}}
\newcommand{\calf}{\mathcal{F}}
\newcommand{\calt}{\mathcal{T}}
\newcommand{\call}{\mathcal{L}}

% \renewcommand{\phi}{\varphi}

% Universal Math Shortcuts
\newcommand{\st}{\hspace*{4pt}\text{s.t.}\hspace*{4pt}}
\newcommand{\pffwd}{\hspace*{2pt}\fbox{\(\Rightarrow\)}\hspace*{10pt}}
\newcommand{\pfbwd}{\hspace*{2pt}\fbox{\(\Leftarrow\)}\hspace*{10pt}}
\newcommand{\contra}{\ensuremath{\Rightarrow\Leftarrow}}

\let\oldforall\forall
\renewcommand{\forall}{\;\oldforall\; }
\let\oldexist\exists
\renewcommand{\exists}{\;\oldexist\; }
\newcommand\existu{\;\oldexist!\: }

\DeclareMathOperator{\Div}{div}

\let\implies\Rightarrow
\let\impliedby\Leftarrow
\let\iff\Leftrightarrow

\bibliographystyle{plain}

%--------Meta Data: Fill in your info------
\title{The Laplacian and The Heat Equation}

\author{Hung C. Le Tran}

\date{August 31 2023}

\begin{document}

\begin{abstract}
    In this paper, I explore the basics of Partial Differential Equations, especially the Laplace's, Poisson's and Heat Equations, and some probabilistic interpretations. 
\end{abstract}

\maketitle

\tableofcontents

\section{Preliminaries}
\subsection{Partial Differential Equations (PDEs)}
The study of PDEs in general is concerned with equations involving a function of two or more variables and its partial derivatives. Among the myriad of possible combinations of these components, we are most interested in those that are motivated and used to model and describe physical and probabilistic phenomena, e.g., transport equation, heat equation, wave equation, etc. Though some rigor will of course be expected, this paper will place more emphasis on interpretations, intuitions and why things should be true, instead of going through the tedium of rigor, as resources are vast for this use \cite{Evans}.
\subsection{Notation}
Through this paper, let $\Omega$ be an open subset of $\bbr^n$ and $u$ be a function $\Omega \to \bbr$. Then we denote the partial derivatives: \[
    \dfrac{\partial u}{\partial x_i}\eqqcolon u_{x_i},
    \dfrac{\partial^2 u}{\partial x_i \partial x_j}\eqqcolon u_{x_ix_j}, etc.
\]

A vector $\alpha = (\alpha_1, \alpha_2, \dots, \alpha_n)$ with each $ \alpha_i \geq 0 $ is a \textit{multiindex} of order $|\alpha| = \alpha_1 + \cdots + \alpha_n$. Then we can define: \[
    D^\alpha u(x) \coloneqq \dfrac{\partial^{|\alpha|} u}{\partial x_1^{\alpha_1} \partial x_2^{\alpha_2} \dots \partial x_n^{\alpha_n}}
\]
and $D^k u(x) \coloneqq \{D^\alpha u(x) \mid |\alpha| = k\}$ for $k \in \bbz_{\geq 0}$.

For $k = 1$, we view $D^1 u(x) = Du(x) = \nabla u(x) = (u_{x_1}, \cdots, u_{x_n})$ as a vector. For $k = 2$, we view $D^2 u(x)$ as the usual Hessian matrix.

With this, we are now interested in solving for $u: \Omega \to \bbr^n$ that satisfies some relationship \[
    F(D^k u(x), D^{k-1}u(x), \ldots, u(x), x) = 0 \;\;\;\:\text{in}\: \Omega
\]

By \textit{solving}, we mean finding all $u$ satisfying the above conditions, or possibly only a subset of all such solutions that satisfies auxiliary boundary conditions on $\partial \Omega$. Oftentimes we hope to find explicit and simple formulae for these solutions, but in the case that we can't, we shall deduce existence and properties of those solutions. Luckily for us, the PDEs under inspection for this paper do have explicit formulae that are relatively simple.

It should also be stated that in this paper, when derivatives are concerned, one can think of them as classical derivatives, meaning that the original function should be imposed some continuity conditions for such derivatives to exist in the first place. However, the \textit{theory of distributions} would hope to generalize this notion of derivatives further in the absence of such continuity conditions, a topic worth looking into.
\section{The Laplacian}
\begin{defn}
    Let $u: \Omega \to \bbr$. Then \textit{the Laplacian} of $u$ is \[
        \Delta u \coloneqq \sum_{i=1}^{n} u_{x_i x_i} = tr(D^2 u)
    \]
\end{defn}

This operator will be of utmost importance, so let us develop some intuition about what it means.

\subsection{Divergence of gradient}
The reader might have realized that $\Delta u = \Div (\nabla u)$, an intuition that will be clearer once put into a context:
\begin{equation} \label{laplace_intro}
    \Delta u = 0 \;\;\;\;\; \text{in}\: \Omega
\end{equation}

This is \textit{Laplace's Equation}. A standard interpretation of a function $u$ satisfying \eqref{laplace_intro} is that it represents a chemical concentration or heat \textit{in equilibrium}. This means that the net flux of the ``flow'' $\mathbf{F}$ over any sufficiently smooth boundary $\partial V$ is 0
\[
    \int_{\partial V} \mathbf{F} \cdot \nu dS = 0
\]
which, by Divergence Theorem, implies \[
    \int_{V} \Div \mathbf{F} = \int_{\partial V} \mathbf{F} \cdot \nu dS = 0
\]
The selection of $\partial V $ was arbitrary, so $\Div \mathbf{F} = 0$ in $\Omega$. In such physical systems, what would appropriately represent such ``flow'' $\mathbf{F}$? It is reasonable that the quantity flows from regions of higher to lower concentrations, therefore \[
    \mathbf{F} = -a \nabla u
\] for some proportionality constant $a>0$ that we can assume to be 1 WLOG. It follows that \[
    0 = \Div \mathbf{F} = \Div (\nabla u) = - \Delta u \implies \Delta u = 0
\]

\subsection{More intuition}
What if the RHS is not 0? The \textit{Poisson's Equation}
\begin{equation} \label{poisson}
    -\Delta u(x) = f(x) \;\;\;\;\; \text{in}\: \Omega
\end{equation}

describes to us a system (WLOG, heat) that is in equilibrium, however this time with an external impulse of heat $f$ throughout $\Omega$. $f(x) + \Delta u(x) = 0$ implies that whatever heat that was supposed to ``diverge'' out of point $x$, $\Delta u (x)$, is recompensated by the external impulse $f(x)$. More concretely, for sufficiently smooth $\partial V$, \[
    \int_{\partial V} -Du \cdot \nu dS = \int_V \Div (-Du) = \int_V f(x)
\]

It might be more natural for some readers to also visualize of a time axis at the mention of injecting heat into a system and letting it stabilize to equilibrium. Indeed, the non-homogeneous \textit{Heat Equation} describes such a phenomenon:
\begin{equation} \label{heat_intro}
    u_t(x, t) - \Delta_x u(x, t) = f(x, t) \;\;\;\;\; \text{in}\: \Omega \times [0, \infty)
\end{equation}
where $\Delta_x u(x, t)$ only sums over derivatives in $x$. Reshuffling \eqref{heat_intro}: \[u_t(x, t) = f(x, t) + \Delta_x u(x, t)\]
we can similarly interpret $u(x, t)$ to be the temperature at time $t$ at position $x$ on $\Omega$. At every moment in time, the change in temperature at a fixed $x$ is the flux flowing outwards with an instantaneous external impulse of heat $f(x, t)$ added on top of it. Understanding \eqref{heat_intro} as such, we can understand \eqref{poisson} as the steady-state heat equation, when there is no longer any fluctuation in temperature over time.

To summarize, we can understand $u$ that either satisfies \eqref{poisson} or \eqref{heat_intro} as the \textit{response} to impulses $f(x)$ or $f(x, t)$ respectively. The easier cases will be when there is no external impulse, i.e., $f \equiv 0$. Such a view of what these PDEs mean shall come in handy later.
\section{Laplace's Equation and Poisson's Equation}
\begin{defn}
    A $C^2$ function $u: \bar{\Omega} \to \bbr$ is \textit{harmonic} if it satisfies Laplace's Equation
    \begin{equation} \label{laplace}
        \Delta u = 0 \;\;\;\;\; \text{in}\: \Omega
    \end{equation}
\end{defn}

A quick proof can show that this equation is \textit{linear}, meaning that if $f, g: \bar{\Omega} \to \bbr$ are solutions to \eqref{laplace} then $f + g$ is also a solution. This serves as our motivation to find simple solutions, potentially with some symmetries, so as to build up to more complex solutions by linearly combining them.

\begin{prop}
    Let $u, v: \bbr^n \to \bbr$ such that \[
        \Delta u = 0, v(x) = u(Mx)
    \] for some orthogonal matrix $M$. Then $\Delta v(x) = 0$. In other words, Laplace's Equation is \textit{rotationally invariant}.
\end{prop}
\begin{proof}
    Using chain rule, we have
    \begin{align*}
        v_{x_i} (x)             & = \sum_{k=1}^{n}u_{x_k}(Mx)M_{ki}                          \\
        \implies v_{x_i x_j}(x) & =  \sum_{l=1}^{n}\sum_{k=1}^{n}u_{x_k x_l}(Mx)M_{ki}M_{lj}
    \end{align*}
    $M$ is orthogonal, so $MM^T = I \implies \sum_{i=1}^{n} M_{ki} M_{li} = \Delta_{kl}$ where $\Delta_{kl}$ is the Kronecker delta. Therefore,
    \begin{align*}
        \Delta v(x) & = \sum_{i=1}^{n}v_{x_i x_i}(x)                                             \\
                    & = \sum_{i=1}^{n} \sum_{l=1}^{n}\sum_{k=1}^{n}u_{x_k x_l}(Mx)M_{ki}M_{li}   \\
                    & = \sum_{l=1}^{n}\sum_{k=1}^{n}  \sum_{i=1}^{n} u_{x_k x_l}(Mx)M_{ki}M_{li} \\
                    & = \sum_{l=1}^{n}\sum_{k=1}^{n}  \sum_{i=1}^{n} u_{x_k x_l}(Mx)\Delta_{kl}  \\
                    & = \sum_{k=1}^{n}u_{x_k x_k}(Mx) = \Delta u(Mx) = 0
    \end{align*}
\end{proof}
\subsection{Fundamental solution of Laplace's Equation}
Since Laplace's Equation is rotationally invariant, we can first search for harmonic $u(x)$ that is radial: $u(x) = v(|x|) = v(r)$ in $\Omega = \bbr^n$. Minor computations yield \[
    0 = \Delta u = v''(r) + \dfrac{n-1}{r}v'(r)
\]
which implies, if $v' \neq 0$, \begin{align*}
    [\log(|v'|)]'       & = \dfrac{1-n}{r}                                            \\
    \implies \log(|v'|) & = (1-n)\log r + a                                           \\
    \implies v'         & = \dfrac{a}{r^{n-1}}                                        \\
    \implies v(r)       & = \begin{cases}
                                b \log r + c           & (n=2)      \\
                                \dfrac{b}{r^{n-2}} + c & (n \geq 3)
                            \end{cases}
\end{align*}
where $b, c \in \bbr$. The choices of $b$ and $c$ are not at all significant. Therefore, the sole reasoning for their choices below is to eliminate unnecessary complexities and the need for more scaling in computations later on.

\begin{defn}
    The function $\Phi: \bbr^n \backslash \{0\} \to \bbr$: \[
        \Phi (x) \coloneqq \begin{cases}
            -\dfrac{1}{2 \pi} \log |x|                      & (n=2)      \\
            \dfrac{1}{n(n-2)\alpha(n)} \dfrac{1}{|x|^{n-2}} & (n \geq 3)
        \end{cases}
    \]
    where $\alpha(n)$ is the volume of the unit ball in $\bbr^n$, is the \textit{fundamental solution of Laplace's Equation}. It is important to note that $\Phi$ is not well-defined at $0$ and is only harmonic on $\bbr^n \backslash \{0\}$.
\end{defn}

\subsection{Poisson's Equation}
It might be natural at this point to suggest a more complex solution for Laplace's Equation by convolving $\Phi$ and any $f$:
\[
    u(x) = \int_{\bbr^n} \Phi(x-y)f(y) dy
\]
After all, don't we have $\Delta \Phi(x) = 0$, so $\Delta \Phi(x - y) = 0$ for fixed $y$, and therefore integrating all these terms would also form a harmonic $u$? This is not possible due to the singularity at $y = x$, $\Phi(x-y)$ blows up and so swapping differentiation and integral signs ``$\Delta u(x) = \Delta \int_{\bbr^n} \Phi(x-y)f(y)dy = \int_{\bbr^n} \Delta (x-y)f(y) dy = 0$'' is unjustifiable and incorrect.

It is then somewhat surprising that convolving as above would yield us not a solution to Laplace's Equation, but to Poisson's! For simplicity, we assume that $f \in C^2_c(\bbr^n)$, a relatively strong condition.
\begin{thm} [Solution of Poisson's Equation]
    Given $f: \bbr^n \to \bbr$, then
    \begin{equation} \label{poisson_sln}
        u(x) = (\Phi * f)(x) = \int_{\bbr^n} \Phi(x -y) f(y) dy
    \end{equation}
    is $C^2$ and solves Poisson's Equation $-\Delta u = f$
\end{thm}
\begin{proof}
    1. We want to show that $u$ is $C^2$. By a quick change of variables (or, simply rewriting the convolution expression): \begin{equation*}
        u(x) = \int_{\bbr^n} \Phi(x -y) f(y) dy = \int_{\bbr^n} \Phi(y) f(x-y) dy
    \end{equation*}
    Rewriting like so enables us to evaluate derivatives of $x$ with derivatives of $f$, making good use of the $C^2_c$ conditions. For $h \neq 0$, \begin{align*}
        \dfrac{u(x + he_i) - u(x)}{h} &= \int_{\bbr^n} \Phi (y) \dfrac{f(x + he_i  -y) - f(x -y)}{h} dy \\
        \lim_{h \to 0} \dfrac{u(x + he_i) - u(x)}{h} &= \lim_{h \to 0} \int_{\bbr^n} \Phi (y) \dfrac{f(x + he_i  -y) - f(x -y)}{h} dy
    \end{align*}
    It remains for us to show that the swapping of limit and integral signs is justified. Since $f \in C^2_c(\bbr^n)$, its first and second derivatives are continuous and have compact support, which implies that they are bounded and uniformly continuous. Therefore, \[
    \dfrac{f( x + he_i - y) - f(x -y)} {h} \xrightarrow{unif.} \dfrac{\partial f}{\partial x_i} (x -y)
    \]
    which enables the swapping just like we wanted. Therefore $u$ is $C^1$. Applying the same argument again, we get that $u$ is $C^2$ \qed

    2. Following from the proof above, \[
    \Delta u(x) = \int_{\bbr^n} \Phi(y) \Delta_x f(x-y) dy
    \]
    is justified. Using this, we want to show that $- \Delta u = f$. We introduce some big-picture ideas:
    \begin{enumerate} [a)]
        \item $\Phi$ blows up at 0, so we shall isolate it inside a small ball
        \item The gradual goal will be to ``shift'' the $\Delta$ operator from $f$ to $\Phi$, using the Green's identity below, that is essentially the multivariable version of the integration by parts $\int_{a}^{b} u v'' dx = uv'\mid_a^b - \int_{a}^{b} u'v' dx$:
        \begin{equation} \label{greendudv}
            \int_{\Omega} u \Delta v dx = \int_{\partial \Omega} u\dfrac{\partial v}{\partial \nu}dS - \int_{\Omega} (Du \cdot Dv) dx 
        \end{equation}
        \item With these frequent swappings, it's important for us to estimate: \[
        |D\Phi (x)| \leq \dfrac{C}{|x|^{n-1}}, |D^2 \Phi(x)| \leq \dfrac{C}{|x|^n}
        \]
        which can be easily and explicitly derived from the $\Phi(x)$ expression.
    \end{enumerate}
    
    Let us start. Fix $\epsilon > 0$, we isolate the singularity:
    \[
    u(x) = \int_{B(0, \epsilon)} \Phi(y) \Delta_x f(x - y)dy + \int_{\bbr^n \backslash B(0, \epsilon)} \Phi(y) \Delta_x f(x - y)dy \eqqcolon I_\epsilon + J_\epsilon
    \]

    \textit{Estimating $I_\epsilon$}: From above, we know that second derivatives in $f$ are bounded, thus so is $\Delta_x f(x - y)$. Some complexity comes in to bound $\int_{B(0, \epsilon)} |\Phi(y)| dy$, but we readily change to polar coordinates:
    \begin{enumerate}
        \item For $n=2$,
        \begin{align*}
            \int_{B(0, \epsilon)} |\Phi(y)| dy &= \int_{B(0, \epsilon)} C |\log |y||dy \\
            &= C \int_{0}^{\epsilon} r |\log r| dr \\
            &\leq C \epsilon^2 \log \epsilon
        \end{align*}
        \item For $n \geq 3$, 
\begin{align*}
    \int_{B(0, \epsilon)} |\Phi(y)| dy &= \int_{B(0, \epsilon)} C_n \dfrac{1}{|y|^{n-2}} dy \\
            &= C_n A_n \int_{0}^{\epsilon} r^{n-1} r^{2-n} dr \\
            &= C_n \epsilon^2
\end{align*}
where $A_n$ is the area of the unit sphere $S^{n-1}$. This gives us \begin{equation} \label{I_e}
    |I_\epsilon| \leq \begin{cases}
        C\epsilon^2 |\log \epsilon| & (n=2) \\
        C \epsilon^2 & (n \geq 3)
    \end{cases}
\end{equation}
\end{enumerate}

\textit{Estimating $J_\epsilon$}: Noticing that $\Delta_x f(x-y) = \Delta_y f(x-y)$, $\Delta_y \Phi(y) = 0$ away from the origin and using \eqref{greendudv} twice,

\begin{align*}
    J_\epsilon &= \int_{\bbr^n \backslash B(0, \epsilon)} \Phi(y) \Delta_y f(x - y)dy \\
    &= \int_{\partial B(0, \epsilon)} \Phi(y) \dfrac{\partial f(x - y)}{\partial \nu} dS(y)  - \int_{\bbr^n \backslash B(0, \epsilon)} D\Phi(y) \cdot D_y f(x - y)dy \\
    &= \int_{\partial B(0, \epsilon)} \Phi(y) \dfrac{\partial f(x - y)}{\partial \nu} dS(y) + \Bigl(\int_{\bbr^n \backslash B(0, \epsilon)} \Delta_y \Phi(y) f(x - y)dy \\
    & - \int_{\partial B(0, \epsilon)} \dfrac{\partial \Phi(y)}{\partial \nu} f(x - y) dS(y) \Bigr) \\
    & = \int_{\partial B(0, \epsilon)} \Phi(y) \dfrac{\partial f(x - y)}{\partial \nu} dS(y) - \int_{\partial B(0, \epsilon)} \dfrac{\partial \Phi(y)}{\partial \nu} f(x - y) dS(y) \\
    & \eqqcolon K_\epsilon + L_\epsilon
\end{align*}
where $\nu$ denotes the \textit{inward pointing normal unit} along $\partial B(0, \epsilon)$, since we have flipped the boundary $\partial \bbr^n \backslash B(0, \epsilon)$ to $\partial B(0, \epsilon)$.

\textit{Estimating $K_\epsilon$}: Similar to above, first derivatives in $f$ are bounded and explicitly calculating $\int_{\partial B(0, \epsilon)} |\Phi (y)| dS(y)$ can show \begin{equation} \label{K_e}
    |K_\epsilon| \leq \begin{cases}
        C\epsilon |\log \epsilon| & (n=2) \\
        C \epsilon & (n \geq 3)
    \end{cases}
\end{equation}

\textit{Estimating $L_\epsilon$}: We can explicitly calculate this. On $\partial B(0, \epsilon)$, \begin{align*}
    \dfrac{\partial \Phi (y)}{\partial \nu} &= D\Phi(y) \cdot \nu \\
    &= \left( \dfrac{-1}{n \alpha(n)} \dfrac{y}{|y|^n} \right) \cdot \dfrac{-y}{|y|} \\
    &= \dfrac{1}{n \alpha(n) \epsilon^{n-1}}
\end{align*}
where $n \alpha(n) \epsilon^{n-1}$ is, critically, the surface area of the sphere $\partial B(0, \epsilon)$! Therefore, 
\begin{align} 
    L_\epsilon &= - \int_{\partial B(0, \epsilon)} \dfrac{1}{n \alpha(n) \epsilon^{n-1}} f(x - y) dS(y) \nonumber \\
    &= - \fint_{\partial B(0, \epsilon)} f(x-y) dS(y) \xrightarrow{\epsilon \to 0} -f(x) \label{L_e}
\end{align}
From \eqref{I_e}, \eqref{K_e}, \eqref{L_e} and letting $\epsilon \to 0$, we have $-\Delta u = f$
\end{proof}

\begin{rem}
    The above proof, though long, demonstrates the general direction that is similarly employed to prove some other theorems in this paper.
\end{rem}

\begin{obs} \label{obs_response}
    We can interpret \begin{equation} \label{impulse_response}
        - \Delta \Phi = \delta_0
    \end{equation}
    where $\Delta_0$ is the Dirac measure on $\bbr^n$, giving mass to the point 0. We can then compute \[
    - \Delta u(x) = \int_{\bbr^n} \Delta_x \Phi(x-y) f(y) dy = \int_{\bbr^n} \delta_x f(y) dy = f(x)
    \]

    In other words, $\Phi$ is the \textit{response} to the \textit{unit impulse} $\Delta_0$. What we have then done is to break up the impulse $f$ over $\bbr^n$ into a linear combination of these unit impulses over $\bbr^n$. Since $\Delta$ is linear, the response to the impulse $f$ would be the linear combination of the same coefficients of those unit impulse responses. This gives a natural reason why $u = \Phi * f$ should solve Poisson's Equation. This theme of breaking an impulse into more ``elementary'' impulses (in this case, unit impulses) and utilizing the linearity of the system to arrive at a solution will continue to be seen.
\end{obs}

\subsection{Mean value property}
Returning to Laplace's Equation and our understanding of it as heat in equilibrium, the reader might have a visualization of how temperature at a certain spot should be some average of the nearby points. Indeed, this is true.
\begin{thm} [Mean Value Property] \label{thm_mvp}
If $u \in C^2 (\Omega)$ is harmonic, then \[
u(x) = \fint_{\partial B(x, r)} u(y) dS(y) = \fint_{B(x, r)} u(y) dy
\]
for any ball $B(x, r) \subseteq \Omega$.
\end{thm}
\begin{proof}
    1. We want to show that the average of $u$ over sphere of any size $r$ stays constant ($u(x)$), so it's natural to attempt to show for $v(r) \coloneqq \fint_{\partial B(x, r)} u(y) dS(y), v'(r) = 0$. Re-parameterize $y$ in terms of $x$ and $r$, \begin{align*}
        v(r) &= \fint_{\partial B(0, 1)} u(x + rz) dS(z) \\
        \implies v'(r) &= \fint_{\partial B(0, 1)} Du(x + rz) \cdot z dS(z) \\
        &= \fint_{\partial B(x, r)} Du(y) \cdot \dfrac{y-x}{r} dS(y) \\
        &= \fint_{\partial B(x, r)} \dfrac{\partial u}{\partial \nu} dS(y) =^{\eqref{greendudv}} \dfrac{r}{n} \fint_{B(x, r)} \Delta u(y) dy = 0
    \end{align*}
    with the final factor $\frac{r}{n}$ to account for the scaling change between the averages. Therefore, $v$ is constant, with the value \[
    v \equiv \lim_{r \to 0} v(r) = \lim_{r \to 0} \fint_{\partial B(x, r)} u(y) dS(y) = u(x)
    \]
    2. Since $u(x)$ is the \textit{mean-value} for every sphere, using polar coordinates it is natural that it is also the \textit{mean-value} for every ball within $\Omega$.
\end{proof}
\begin{rem}
    The converse of \autoref{thm_mvp} holds. That is, if $u \in C^2(\Omega)$ satisfies \[
    u(x) = \fint_{\partial B(x, r)} u dS
    \]
    for each ball $B(x, r) \subseteq \Omega$ then $u$ is harmonic.
\end{rem}
\begin{rem}
Harmonic functions are therefore quite nice and smoothed out, as the value at each point is the average of that of its surrounding points. This suggests that there can't be local maxima (minima) within $\Omega$, as the average around the maxima (minima) will be lower (higher) than the maxima (minima), contradicting the mean-value property. We thus have the following.
\end{rem}
\begin{thm} [The Strong Maximum Principle]
    Suppose $u \in C^2(\Omega) \cap C(\bar{\Omega})$ is harmonic in $\Omega$, then $\max_{\bar{\Omega}} u = \max_{\partial \Omega} u$ (the maximum lies on the boundary). Moreover, if $\Omega$ is connected and there exists $x_0 \in \Omega$ such that $u(x_0) = \max_{\bar{\Omega}} u$ then $u$ is constant.
\end{thm}

\begin{proof}
(\cite{Evans}, pg. 27)
\end{proof}
\begin{rem}
A similar Strong Minimum Principle can be derived using the same approach.
\end{rem}

\begin{cor} [Uniqueness]
Let $g \in C(\partial \Omega), f \in C(\Omega)$, then there exists at most one solution $u \in C^2(\Omega) \cap C(\bar{\Omega})$ of Poisson's Equation with Dirichlet boundary conditions 
\begin{equation} \label{poisson_dirichlet}
    \begin{cases}
        \begin{aligned}
            -\Delta u &= f &&\:\text{in}\:  \Omega \\
            u &= g &&\:\text{on}\: \partial \Omega
        \end{aligned}
    \end{cases}
\end{equation}
\end{cor}
\begin{proof}
    Suppose $u, \tilde{u}$ satisfy the above then $w \coloneqq (u - \tilde{u})\in C^2(\Omega) \cap C(\bar{\Omega})$, is harmonic and is 0 on the boundary $\partial \Omega$. The Strong Maximum Principle implies $w \equiv 0$.
\end{proof}

\subsection{Regularity}
We will now prove that if $u \in C^2$ is harmonic, then $u \in C^\infty$ --- which is a very interesting statement given that the structure of the original PDE only has second-order derivatives, yet we are claiming that it is infinitely differentiable. We will state and prove a theorem that is slightly stronger than this.
\begin{thm} [Smoothness]
    If $u \in C(\Omega)$ is satisfies mean-value property for each ball $B(x, r) \subseteq \Omega$ then $u \in C^{\infty} (\Omega)$ 
\end{thm}

\begin{proof}
Let $\eta$ be the standard mollifier, which is a radial, $C^\infty$ function with support $B(0, 1)$ and $\int_{\bbr^n} \eta = 1$. Then $\eta_\epsilon$ is the rescaled $\eta$ such that its support is $B(0, \epsilon)$ while having $\int_{\bbr^n} \eta_\epsilon = 1$ still. Then $u^\epsilon \coloneqq \eta_\epsilon * u$ defined for $x \in \Omega_\epsilon \coloneqq \{x \in \Omega \mid d(x, \partial \Omega) > \epsilon\}$ is also $C^\infty$. $x \in \Omega_\epsilon$ simply means that it is possible to draw the ball $B(x, \epsilon) \subset \Omega$. Then we show that $u \in C^\infty$ by showing $u \equiv u^\epsilon$ in $\Omega_\epsilon$. Indeed, 
\begin{align*}
u^\epsilon(x) &= \int_{\Omega} \eta_\epsilon(x-y) u(y) dy \\
&= \int_{B(x, \epsilon)}  \eta_\epsilon(x-y) u(y) dy \\
&= \int_{0}^{\epsilon} \int_{\partial B(x, r)} \eta_\epsilon(r) u(y) dS(y) dr \\
&= \int_{0}^{\epsilon} \eta_\epsilon(r) \int_{\partial B(x, r)}  u(y) dS(y) dr \\
&= \int_{0}^{\epsilon} \eta_\epsilon(r) \left(\int_{\partial B(x, r)}  u(y) dS(y)\right) dr \\
&=^{(MVP)} \int_{0}^{\epsilon} \eta_\epsilon(r) (u(x) r^{n-1} A_n) dr \\
&= u(x) \int_{B(0, \epsilon)} \eta_\epsilon dy = u(x)
\end{align*}
\end{proof}
\subsection{Poisson's Equation with Dirichlet boundary conditions}
\eqref{poisson_dirichlet} imposes \textit{Dirichlet} boundary conditions on our Poisson's Equation, meaning it prescribes a value for our solution on the boundary $\partial \Omega$. We thus have to customize our solution \eqref{poisson_sln} to these specific conditions.
\begin{thm} [Representation Formula using Green's function]
If $u \in C^2(\bar{\Omega})$ solves \eqref{poisson_dirichlet} for $\Omega \subseteq \bbr^n$ open, bounded with $C^1 \partial \Omega$ then \begin{equation} \label{poisson_dirichlet_sln}
    u(x) = \int_{\Omega} f(y) G(x, y) dy - \int_{\partial \Omega} g(y) \dfrac{\partial G}{\partial \nu}(x, y) dS(y)  \;\;\;\;\; \text{in}\: \Omega
\end{equation}
with $G(x, y)$ constructed to satisfy:
\begin{equation} \label{green_construction}
    G(x, y) \coloneqq \Phi(y-x) - \phi^x(y) \:\text{for}\: x, y \in \Omega; x \neq y
\end{equation}
and the ``corrector function'' $\phi^x(y)$ found so that:
\begin{equation} \label{green_corrector}
    \begin{cases}
\begin{aligned}
    \Delta \phi^x (y) &= 0 &&\text{in}\: \Omega \\
    \phi^x (y) &= \Phi(y-x)  &&\text{on}\: \partial \Omega
\end{aligned}
\end{cases} 
\end{equation}
\end{thm}
\begin{proof}
(\cite{Evans}, pg. 34)
\end{proof}
\begin{rem}
We omit the proof and try to provide some intuition on what \eqref{poisson_dirichlet_sln} is doing. From \eqref{green_construction}, \eqref{green_corrector} and  \autoref{obs_response} on how $\Phi$ is the response to the unit impulse $\delta_0$, we can symbolically write for $G^x(y) \coloneqq G(x, y)$\begin{equation}
    \begin{cases}
        \begin{aligned}
        -\Delta G^x (y)= -\Delta (\Phi(y-x) - \phi^x(y)) &= \delta_x  &&\text{in}\: \Omega \\
        G^x (y)= \Phi(y-x) - \phi^x(y) &= 0  &&\text{on}\: \partial \Omega
        \end{aligned}
    \end{cases}
\end{equation}

$G^x(y)$ is therefore the response throughout $\Omega$ to the unit impulse at $x$ that vanishes at the boundary $\partial \Omega$. Thus, the first part of \eqref{poisson_dirichlet_sln} contributes to solve $-\Delta u = f$ per our previous approach. The second part handles the continuity of $u$ from in $\Omega$ to $u \equiv g$ on $\partial \Omega$.
\end{rem}
\begin{rem}
Oftentimes finding such $G$ will be difficult, and only for certain $\Omega$ having some sort of symmetry (e.g., the half space $\bbr^n_+$ or ball $B(x_0, r)$) explicit formulae for $G$ can be found. The symmetry enables us to make an informed guess of the corrector function $\phi^x(y)$ so that $G$ cancels out at the boundary.
\end{rem}

\section{Heat Equation}
\subsection{Fundamental solution of the Heat Equation}
In the same approach as for Laplace's Equation, we try to find a solution that is ``fundamental'' for the homogenous heat equation
\begin{equation} \label{heat}
    u_t - \Delta u = 0 \;\;\;\;\; \text{in}\: \Omega \times (0, \infty)
\end{equation}
with $\Omega \subseteq \bbr^n$ open, $u(x, t): \bar{\Omega} \times [0, \infty) \to \bbr$ unknown, and $\Delta u = \Delta_x u$ only in the spatial variables.

The first thing to notice is that there is a certain scaling ratio between $x$ and $t$, i.e., if $u(x, t)$ solves \eqref{heat} then $\tilde{u}(x, t) \coloneqq \lambda^\alpha u(\lambda x, \lambda^2 t)$ (invariant under dilation scaling) for any $\lambda, \alpha$ would also solve \eqref{heat}. Therefore, a family of solutions is characterized by a scaling factor $\alpha$ and a ``unit-less'' function $v$ defined when $\lambda = t^{-1/2}$:
\begin{equation}
    \tilde{u}(x, t) = t^{-\alpha/2}u\left(\frac{x}{\sqrt{t}}, 1\right) \eqqcolon t^{-\alpha/2} v\left(\frac{x}{\sqrt{t}}\right)
\end{equation}

We then opt for $v$ to be radial (similar to Laplace's), with a corresponding $\alpha = \frac{n}{2}$ so we can arrive at an explicit solution. Detailed derivations can be found in \cite{Evans}.
\begin{defn} The function $\Phi: \bbr^n \times (\bbr \backslash \{0\}) \to \bbr$ \begin{equation} \label{heat_fund}
    \Phi(x, t) \coloneqq \begin{cases}
        \begin{aligned}
            \frac{1}{(4\pi t)^{n/2}}e ^{-\frac{|x|^2}{4t}} &&\text{on} \: \bbr^n \times (0, \infty) \\
            0  &&\text{on} \: \bbr^n \times (-\infty, 0)
        \end{aligned}
    \end{cases}
\end{equation}
is the \textit{fundamental solution of the heat equation}.
\end{defn}

\begin{lem} \label{conv_1}
\begin{equation*}
    \forall t > 0, \int_{\bbr^n} \Phi(x, t) dx = 1
\end{equation*}
\end{lem}
\begin{rem}
    We wanted $\Phi$ to have the property above and this was the point of the additional scaling constant $\frac{1}{(4\pi)^{n/2}}$. Moreover, one can also show that $\Phi(x, t)$ is generally well-behaved around $(x \neq 0, t = 0)$, but have a singularity at the origin $(0, 0)$.
\end{rem}
\subsection{Initial-value problem}
We first visit the initial-value problem, the ``boundary'' for the time-axis \begin{equation} \label{heat_initial}
    \begin{cases}
        \begin{aligned}
            u_t - \Delta u &= 0 && \:\text{on}\: \bbr^n \times (0, \infty)\\
            u &= g && \:\text{on}\: \bbr^n \times \{t = 0\}  
        \end{aligned}
    \end{cases}
\end{equation}

In other words, \eqref{heat_initial} asks: With initial temperature $g$, how does heat propagate with no external impulse? We know that $\Phi(x, t)$ solves the heat equation away from the singularity $(0, 0)$, therefore so does $\Phi(x - y, t)$ for fixed $y \in \bbr^n$. Thus, convolving $\Phi$ over the spatial variables with any $g: \bbr^n \to \bbr$ should also solve the heat equation \textit{away from the singularity}. There are no blow-up issues for $t>0$, so this behavior is expected. Perhaps what's surprising is that this convolution, solves the initial-value problem, i.e., it approaches $g$ as $t \to 0$.

\begin{thm} [Solution of homogenous problem]
    Assume $g \in C(\bbr^n) \cap L^\infty (\bbr^n)$, and define for $t>0, x \in \bbr^n$, \begin{equation} \label{heat_initial_sln}
        u(x, t) = \int_{\bbr^n}\Phi(x-y, t) g(y) dy = \frac{1}{(4 \pi t)^{n/2}} \int_{\bbr^n} \Phi(x - y, t) g(y) dy
    \end{equation} then \begin{enumerate}
        \item $u \in C^{\infty} (\bbr^n \times (0, \infty))$ and solves the homogenous heat equation
        \item $\lim_{(x, t) \to (x^0, 0), x \in \bbr^n, t>0} u(x, t) = g(x_0)$ for each point $x_0 \in \bbr^n$.
    \end{enumerate}
\end{thm}

\begin{proof}
    We omit the proof for (1). See \cite{Evans}.

    To prove (2), fix $x_0 \in \bbr^n$ and $\epsilon > 0$. Using \autoref{conv_1}, we then estimate:
    \begin{align*}
        |u(x, t) - g(x_0)| &= |\int_{\bbr^n} \Phi(x-y, t) g(y) dy - g(x_0) \int_{\bbr^n} \Phi(x-y, t) dy| \\
        &\leq \int_{B(x_0, \delta)} \Phi(x-y, t) |g(y) - g(x_0)| dy \\
        &+ \int_{\bbr^n \backslash B(x_0, \delta)} \Phi(x-y, t) |g(y) - g(x_0)| \eqqcolon I + J
    \end{align*}
    for some $\delta$. We will then control $I$ through the continuity of $g$ and control $J$ through the decay of $\Phi$ when $y$ is farther away.

    Since $g \in C(\bbr^n)$, choose $\delta = \delta(\epsilon)$ such that $|y- x_0| < \delta \implies |g(y) - g(x_0)| < \epsilon$. Then \[
    I \leq \epsilon \int_{B(x_0, \delta)} \Phi(x-y, t) \leq \epsilon
    \]

    To control $J$, we see that $y$ is at least $\delta$ away from $x_0$, while $x \to x_0$ is arbitrarily close to $x_0$. This is convenient as $\Phi(x-y, t) = C e^{-\frac{|x-y|^2}{4t}}$ and thus the ``speed of decay'' has a lower bound. Concretely, if $|x - x_0| < \frac{\delta}{2}$ then $|y-x| \geq \frac{1}{2} |y - x_0|$. Therefore,
    \begin{align*}
    J &\leq 2 ||g||_{L^\infty} \int_{\bbr^n \backslash B(x_0, \delta)} \Phi(x-y, t) dy \\
    &\leq \frac{C}{t^{n/2}}  \int_{\bbr^n \backslash B(x_0, \delta)} e^{-\frac{|x-y|^2}{4t}} dy \\
    &\leq \frac{C}{t^{n/2}}  \int_{\bbr^n \backslash B(x_0, \delta)} e^{-\frac{|y-x_0|^2}{16t}} dy \\
    &= \frac{C}{t^{n/2}} \int_{\delta}^{\infty} e^{-\frac{r^2}{16t}} r^{n-1} dr \xrightarrow{t \to 0^+} 0 
    \end{align*}
    with the last convergence owing to $\exp$ decaying much faster than polynomials.

    Therefore if $|x-x_0| < \frac{\delta}{2}$ and $t>0$ small enough, $|u(x, t) - g(x_0)| < 2\epsilon$.
\end{proof}

\begin{obs}
We can write 
\begin{equation}
\begin{cases}
    \begin{aligned}
        \Phi_t - \Delta \Phi &= 0 && \:\text{in}\: \bbr^n \times (0, \infty)\\
        \Phi &= \delta_0 &&\:\text{on}\: \bbr^n \times \{t = 0\} 
    \end{aligned}
\end{cases}
\end{equation}
which informally makes, on $\bbr^n \times \{t = 0\}$, $u(x) = \int_{\bbr^n} \delta_x g(y) dy = g(x)$. In other words, this means that $\Phi$ is the representation of how heat propagates with an initial unit impulse  $\delta_0$ and no external impulse. It is then understandable that the (linear) propagation of initial temperature $g$ is just the integration of all these responses with corresponding coefficients!
\end{obs}

\subsection{Nonhomogenous problem}
The nonhomogenous, initial-value problem is 
\begin{equation} \label{heat_nonhomogenous}
    \begin{cases}
        \begin{aligned}
            u_t - \Delta u &= f && \:\text{in}\: \bbr^n \times (0, \infty) \\
            u &= 0 && \:\text{on}\: \bbr^n \times \{t = 0\}
        \end{aligned}
    \end{cases}
\end{equation}
This time, the problem asks for how heat propagates given no initial temperature, but at every time $\tau$, an \textit{external} impulse $f^\tau(x) \coloneqq f(x, \tau)$ is fed into $\bbr^n$. To some, this might look very different from the homogenous problem from above; after all, $0$ and the functions $f$ (or $g$) are switched! However, due to the linearity of the system, there is no difference between the solution when injecting the external impulse $f(x, t)$ and linearly combining solutions of ``new'' propagation processes that start at different times with (shifted) initial conditions as exactly these impulses. Concretely, for fixed $\tau$, consider \eqref{heat_initial_sln} shifted in time with time-adjusted initial conditions:
\begin{equation}
u^\tau(x, t) = \int_{\bbr^n} \Phi(x-y, t-\tau) f^\tau(y) dy  = \int_{\bbr^n} \Phi(x-y, t-\tau) f(y, \tau) dy
\end{equation}
then $u^s$ would solve the homogenous heat equation on $\bbr^n \times (\tau, \infty)$ with initial conditions $f^\tau$. To solve \eqref{heat_nonhomogenous}, one simply linearly combines these processes throughout time.

\begin{thm} [Solution of nonhomogenous problem] Define
\begin{equation} \label{heat_nonhomogenous_sln}
    u(x, t) \coloneqq \int_{0}^{t} \int_{\bbr^n} \Phi(x-y, t - \tau) f(y, \tau) dy d\tau
\end{equation}
Then,
\begin{enumerate}
    \item $u_t(x, t) - \Delta u(x, t) = f(x, t)$ on $\bbr^n \times (0, \infty)$
    \item $\lim_{(x, t) \to (x_0, 0), x \in \bbr^n, t >0} u(x, t) = 0$ for each point $x_0 \in \bbr^n$
\end{enumerate}
\end{thm}
\begin{proof}
    Intuitions are provided above. See (\cite{Evans}, pg. 50).
\end{proof}

\begin{rem}
Of course, we can combine \eqref{heat_initial_sln} and \eqref{heat_nonhomogenous_sln}:
\begin{equation}
u(x, t) = \int_{\bbr^n} \Phi(x- y, t) g(y) dy + \int_{0}^{t} \int_{\bbr^n} \Phi(x - y, t - \tau) f(y, \tau) dy d\tau
\end{equation}
to solve \begin{equation}
    \begin{cases}
    \begin{aligned}
        u_t - \Delta u &= f && \:\text{in}\: \bbr^n \times (0, \infty) \\
        u &= g && \:\text{on}\: \bbr^n \times \{t = 0\}
    \end{aligned}
    \end{cases}
\end{equation}
\end{rem}
\begin{rem}
There are analogues of the theorems of Mean Value Property, Strong Maximum Principle, Uniqueness and Regularity for the heat equation, which are not that surprising given that Laplace's Equation is the steady-state heat equation. See \cite{Evans} for more.
\end{rem}

\section{Fourier Transform}
Throughout the previous sections, a clear motif that has been used is that we break complex impulses into smaller ``elementary'' components, solving for these components and then building back up the more complex solution, of course provided that the differential operator is linear. So far, these ``elementary'' components have been Dirac delta unit impulses, but some readers might have noticed another common technique of doing this exact procedure, yet with a different family of ``elementary'' components. The technique in this section was first used by Joseph Fourier (1768 – 1830) to study the heat equation. We will arrive at the fundamental solution of the heat equation via this different approach.

\subsection{Definition and properties}
\begin{defn} [Fourier transform]
    For $u \in L^1(\bbr^n)$, its \textit{Fourier transform} is \begin{equation}
        \hat{u}(y) \coloneqq \frac{1}{(2\pi)^{n/2}} \int_{\bbr^n} e^{-ix \cdot y} u(x) dx
    \end{equation}
    and \textit{inverse Fourier transform} is
    \begin{equation}
    \check{u} (y) \coloneqq \frac{1}{(2\pi)^{n/2}} \int_{\bbr^n} e^{ix \cdot y} u(x) dx
    \end{equation}
\end{defn}

\begin{rem}
Since $|e^{\pm ix \cdot y}| \leq 1$ and $u \in L^1(\bbr^n)$, the two integrals are well-defined. Moreover, we can similarly define the Fourier transform and inverse Fourier transform for $L^2$ functions as the convergence of those of $L^1 \cap L^2$ functions \cite{Evans}.
\end{rem}

\begin{lem}
For $u, v \in L^2(\bbr^n)$, the following properties hold:
\begin{enumerate}
    \item $\widehat{D^\alpha u} = (iy)^\alpha \hat{u}$
    \item $\widehat{u * v} = (2 \pi)^{n/2} \hat{u} \hat{v}$
\end{enumerate}
\end{lem}


































\newpage
% \section{Discrete Interpretations}
% \begin{defn} [Harmonic Function]
%     Recall that a function \(f\) on \(\bbz^d\) is harmonic at \(x\) if \(f(x)\) equals the average of \(f\) on its nearest neighbors. If \(U\) is an open subset of \(\bbr^d\), we will say that \(f\) is \textbf{harmonic} in \(U\) if and only if it is continuous and satisfies the following \textbf{mean value property}: for every \(x \in U\), and every \(0 < \epsilon < dist(x, \partial U)\),
%     \begin{equation}
%         f(x) = MV(f; x, \epsilon) = \int_{|y-x| = \epsilon}f(y) ds(y)
%     \end{equation}
% \end{defn}

% \begin{rem}
%     Value at \(x\) equals to average of ball radius \(\epsilon\) around \(x\) for all \(\epsilon\)
% \end{rem}

% \begin{defn} [Laplacian]
%     \[
%         \Delta f(x) = \lim_{e \to 0} \dfrac{1}{\epsilon^2}\sum_{y \in \bbz^d, |y| = 1} [f(x + \epsilon y) - f(x)]
%     \]
% \end{defn}
% \begin{rem}
%     Just taking in each direction, not the whole ball!
% \end{rem}
% \begin{prop} [Representing Laplacian in partial derivatives]
%     Suppose \(f\) is \(C^2\) in a neighborhood of \(x\) in \(\bbr^d\). Then \(\Delta f(x) \)exists at \(x\) and \[
%         \Delta f(x) = \sum_{j=1}^{d} \partial_{jj} f(x)
%     \]
% \end{prop}
% \begin{proof}
%     This comes naturally from the above definition of \(\Delta f(x)\), as well as the approximation one can make from the \(C^2\) smoothness of \(f(x)\).
% \end{proof}

% \begin{prop}
%     If \(f\) is \(C^2\) in a neighborhood of \(x\), then \begin{equation}
%         \dfrac{1}{2d}  \Delta f(x) = \lim_{\epsilon \to 0} \dfrac{MV(f; x, \epsilon) - f(x)}{\epsilon^2}
%     \end{equation}
% \end{prop}

% \begin{thm}

%     \textbf{BIG THEOREM! Stating the equivalence of a harmonic function with its Laplacian operator}

%     A function in a domain \(U\) is harmonic if and only if \(f\) is \(C^2\) with \(\Delta f(x) = 0 \forall x \in U\)
% \end{thm}

% We first state here without full justification that $\Delta u(x)$ represents the difference between the average of $u$ ``around'' the point $x$ and the value $u(x)$. Perhaps this would be clearer if we take a brief look at the \textit{discrete Laplacian}, which we
% define on the lattice grid for a function $u: U \subseteq \bbz^d \to \bbr$

% \[\call u (x) \coloneqq \dfrac{1}{d} \sum_{i=1}^{n} \bigl[[u(x+e_i) - u(x)] - [u(x) - u(x-e_i)]\bigr] = \dfrac{1}{2d} \sum_{i=1}^{n} [u(x \pm e_i) - u(x)] \]
% where the Laplacian still represents the (scaled) sum of derivatives (now differences) of $u$. Further elaboration of this interpretation of the Laplacian would be presented in later sections, yet its intuition would serve us well to understand the following equations, \textit{Poisson's Equation} and \textit{Heat Equation} respectively, better:

\newpage
\section*{Acknowledgments} Words cannot express my gratitude to my mentor, Professor Beniada Shabani --- without whom none of this would have been possible. Her unwavering patience in our meetings, deep understanding of the material and above all, her immense enthusiasm for mathematics have been an enormous source of inspiration for me. I would also like to thank Professors Daniil Rudenko and L\'{a}szl\'{o} Babai for the tremendous Apprentice Program, Professor Greg Lawler for the Probability and Analysis series, and Professor Peter May for organizing this invaluable experience. Lastly, I would like to extend my thanks to the lecturers, TAs, friends and upperclassmen for an amazing REU.

\section{Bibliography}
\begin{thebibliography}{9}
    \bibitem{Evans} Lawrence C. Evans. Partial Differential Equations. Graduate Studies in Mathematics, American Mathematical Society. 2010.

\end{thebibliography}
\end{document}

