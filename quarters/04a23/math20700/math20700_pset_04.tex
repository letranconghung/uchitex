\documentclass[a4paper, 12pt]{article}
%%%%%%%%%%%%%%%%%%%%%%%%%%%%%%%%%%%%%%%%%%%%%%%%%%%%%%%%%%%%%%%%%%%%%%%%%%%%%%%
%                                Basic Packages                               %
%%%%%%%%%%%%%%%%%%%%%%%%%%%%%%%%%%%%%%%%%%%%%%%%%%%%%%%%%%%%%%%%%%%%%%%%%%%%%%%

% Gives us multiple colors.
\usepackage[usenames,dvipsnames,pdftex]{xcolor}
% Lets us style link colors.
\usepackage{hyperref}
% Lets us import images and graphics.
\usepackage{graphicx}
% Lets us use figures in floating environments.
\usepackage{float}
% Lets us create multiple columns.
\usepackage{multicol}
% Gives us better math syntax.
\usepackage{amsmath,amsfonts,mathtools,amsthm,amssymb}
% Lets us strikethrough text.
\usepackage{cancel}
% Lets us edit the caption of a figure.
\usepackage{caption}
% Lets us import pdf directly in our tex code.
\usepackage{pdfpages}
% Lets us do algorithm stuff.
\usepackage[ruled,vlined,linesnumbered]{algorithm2e}
% Use a smiley face for our qed symbol.
\usepackage{tikzsymbols}
% \usepackage{fullpage} %%smaller margins
\usepackage[shortlabels]{enumitem}

\setlist[enumerate]{font={\bfseries}} % global settings, for all lists

\usepackage{setspace}
\usepackage[margin=1in, headsep=12pt]{geometry}
\usepackage{wrapfig}
\usepackage{listings}
\usepackage{parskip}

\definecolor{codegreen}{rgb}{0,0.6,0}
\definecolor{codegray}{rgb}{0.5,0.5,0.5}
\definecolor{codepurple}{rgb}{0.58,0,0.82}
\definecolor{backcolour}{rgb}{0.95,0.95,0.95}

\lstdefinestyle{mystyle}{
    backgroundcolor=\color{backcolour},   
    commentstyle=\color{codegreen},
    keywordstyle=\color{magenta},
    numberstyle=\tiny\color{codegray},
    stringstyle=\color{codepurple},
    basicstyle=\ttfamily\footnotesize,
    breakatwhitespace=false,         
    breaklines=true,                 
    captionpos=b,                    
    keepspaces=true,                 
    numbers=left,                    
    numbersep=5pt,                  
    showspaces=false,                
    showstringspaces=false,
    showtabs=false,                  
    tabsize=2,
    numbers=none
}

\lstset{style=mystyle}
\def\class{article}


%%%%%%%%%%%%%%%%%%%%%%%%%%%%%%%%%%%%%%%%%%%%%%%%%%%%%%%%%%%%%%%%%%%%%%%%%%%%%%%
%                                Basic Settings                               %
%%%%%%%%%%%%%%%%%%%%%%%%%%%%%%%%%%%%%%%%%%%%%%%%%%%%%%%%%%%%%%%%%%%%%%%%%%%%%%%

%%%%%%%%%%%%%
%  Symbols  %
%%%%%%%%%%%%%

\let\implies\Rightarrow
\let\impliedby\Leftarrow
\let\iff\Leftrightarrow
\let\epsilon\varepsilon
%%%%%%%%%%%%
%  Tables  %
%%%%%%%%%%%%

\setlength{\tabcolsep}{5pt}
\renewcommand\arraystretch{1.5}

%%%%%%%%%%%%%%
%  SI Unitx  %
%%%%%%%%%%%%%%

\usepackage{siunitx}
\sisetup{locale = FR}

%%%%%%%%%%
%  TikZ  %
%%%%%%%%%%

\usepackage[framemethod=TikZ]{mdframed}
\usepackage{tikz}
\usepackage{tikz-cd}
\usepackage{tikzsymbols}

\usetikzlibrary{intersections, angles, quotes, calc, positioning}
\usetikzlibrary{arrows.meta}

\tikzset{
    force/.style={thick, {Circle[length=2pt]}-stealth, shorten <=-1pt}
}

%%%%%%%%%%%%%%%
%  PGF Plots  %
%%%%%%%%%%%%%%%

\usepackage{pgfplots}
\pgfplotsset{width=10cm, compat=newest}

%%%%%%%%%%%%%%%%%%%%%%%
%  Center Title Page  %
%%%%%%%%%%%%%%%%%%%%%%%

\usepackage{titling}
\renewcommand\maketitlehooka{\null\mbox{}\vfill}
\renewcommand\maketitlehookd{\vfill\null}

%%%%%%%%%%%%%%%%%%%%%%%%%%%%%%%%%%%%%%%%%%%%%%%%%%%%%%%
%  Create a grey background in the middle of the PDF  %
%%%%%%%%%%%%%%%%%%%%%%%%%%%%%%%%%%%%%%%%%%%%%%%%%%%%%%%

\usepackage{eso-pic}
\newcommand\definegraybackground{
    \definecolor{reallylightgray}{HTML}{FAFAFA}
    \AddToShipoutPicture{
        \ifthenelse{\isodd{\thepage}}{
            \AtPageLowerLeft{
                \put(\LenToUnit{\dimexpr\paperwidth-222pt},0){
                    \color{reallylightgray}\rule{222pt}{297mm}
                }
            }
        }
        {
            \AtPageLowerLeft{
                \color{reallylightgray}\rule{222pt}{297mm}
            }
        }
    }
}

%%%%%%%%%%%%%%%%%%%%%%%%
%  Modify Links Color  %
%%%%%%%%%%%%%%%%%%%%%%%%

\hypersetup{
    % Enable highlighting links.
    colorlinks,
    % Change the color of links to blue.
    urlcolor=blue,
    % Change the color of citations to black.
    citecolor={black},
    % Change the color of url's to blue with some black.
    linkcolor={blue!80!black}
}

%%%%%%%%%%%%%%%%%%
% Fix WrapFigure %
%%%%%%%%%%%%%%%%%%

\newcommand{\wrapfill}{\par\ifnum\value{WF@wrappedlines}>0
        \parskip=0pt
        \addtocounter{WF@wrappedlines}{-1}%
        \null\vspace{\arabic{WF@wrappedlines}\baselineskip}%
        \WFclear
    \fi}

%%%%%%%%%%%%%%%%%
% Multi Columns %
%%%%%%%%%%%%%%%%%

\let\multicolmulticols\multicols
\let\endmulticolmulticols\endmulticols

\RenewDocumentEnvironment{multicols}{mO{}}
{%
    \ifnum#1=1
        #2%
    \else % More than 1 column
        \multicolmulticols{#1}[#2]
    \fi
}
{%
    \ifnum#1=1
    \else % More than 1 column
        \endmulticolmulticols
    \fi
}

\newlength{\thickarrayrulewidth}
\setlength{\thickarrayrulewidth}{5\arrayrulewidth}


%%%%%%%%%%%%%%%%%%%%%%%%%%%%%%%%%%%%%%%%%%%%%%%%%%%%%%%%%%%%%%%%%%%%%%%%%%%%%%%
%                           School Specific Commands                          %
%%%%%%%%%%%%%%%%%%%%%%%%%%%%%%%%%%%%%%%%%%%%%%%%%%%%%%%%%%%%%%%%%%%%%%%%%%%%%%%

%%%%%%%%%%%%%%%%%%%%%%%%%%%
%  Initiate New Counters  %
%%%%%%%%%%%%%%%%%%%%%%%%%%%

\newcounter{lecturecounter}

%%%%%%%%%%%%%%%%%%%%%%%%%%
%  Helpful New Commands  %
%%%%%%%%%%%%%%%%%%%%%%%%%%

\makeatletter

\newcommand\resetcounters{
    % Reset the counters for subsection, subsubsection and the definition
    % all the custom environments.
    \setcounter{subsection}{0}
    \setcounter{subsubsection}{0}
    \setcounter{definition0}{0}
    \setcounter{paragraph}{0}
    \setcounter{theorem}{0}
    \setcounter{claim}{0}
    \setcounter{corollary}{0}
    \setcounter{proposition}{0}
    \setcounter{lemma}{0}
    \setcounter{exercise}{0}
    \setcounter{problem}{0}
    
    \setcounter{subparagraph}{0}
    % \@ifclasswith\class{nocolor}{
    %     \setcounter{definition}{0}
    % }{}
}

%%%%%%%%%%%%%%%%%%%%%
%  Lecture Command  %
%%%%%%%%%%%%%%%%%%%%%

\usepackage{xifthen}

% EXAMPLE:
% 1. \lecture{Oct 17 2022 Mon (08:46:48)}{Lecture Title}
% 2. \lecture[4]{Oct 17 2022 Mon (08:46:48)}{Lecture Title}
% 3. \lecture{Oct 17 2022 Mon (08:46:48)}{}
% 4. \lecture[4]{Oct 17 2022 Mon (08:46:48)}{}
% Parameters:
% 1. (Optional) lecture number.
% 2. Time and date of lecture.
% 3. Lecture Title.
\def\@lecture{}
\def\@lectitle{}
\def\@leccount{}
\newcommand\lecture[3]{
    \newpage

    % Check if user passed the lecture title or not.
    \def\@leccount{Lecture #1}
    \ifthenelse{\isempty{#3}}{
        \def\@lecture{Lecture #1}
        \def\@lectitle{Lecture #1}
    }{
        \def\@lecture{Lecture #1: #3}
        \def\@lectitle{#3}
    }

    \setcounter{section}{#1}
    \renewcommand\thesubsection{#1.\arabic{subsection}}
    
    \phantomsection
    \addcontentsline{toc}{section}{\@lecture}
    \resetcounters

    \begin{mdframed}
        \begin{center}
            \Large \textbf{\@leccount}
            
            \vspace*{0.2cm}
            
            \large \@lectitle
            
            
            \vspace*{0.2cm}

            \normalsize #2
        \end{center}
    \end{mdframed}

}

%%%%%%%%%%%%%%%%%%%%
%  Import Figures  %
%%%%%%%%%%%%%%%%%%%%

\usepackage{import}
\pdfminorversion=7

% EXAMPLE:
% 1. \incfig{limit-graph}
% 2. \incfig[0.4]{limit-graph}
% Parameters:
% 1. The figure name. It should be located in figures/NAME.tex_pdf.
% 2. (Optional) The width of the figure. Example: 0.5, 0.35.
\newcommand\incfig[2][1]{%
    \def\svgwidth{#1\columnwidth}
    \import{./figures/}{#2.pdf_tex}
}

\begingroup\expandafter\expandafter\expandafter\endgroup
\expandafter\ifx\csname pdfsuppresswarningpagegroup\endcsname\relax
\else
    \pdfsuppresswarningpagegroup=1\relax
\fi

%%%%%%%%%%%%%%%%%
% Fancy Headers %
%%%%%%%%%%%%%%%%%

\usepackage{fancyhdr}

% Force a new page.
\newcommand\forcenewpage{\clearpage\mbox{~}\clearpage\newpage}

% This command makes it easier to manage my headers and footers.
\newcommand\createintro{
    % Use roman page numbers (e.g. i, v, vi, x, ...)
    \pagenumbering{roman}

    % Display the page style.
    \maketitle
    % Make the title pagestyle empty, meaning no fancy headers and footers.
    \thispagestyle{empty}
    % Create a newpage.
    \newpage

    % Input the intro.tex page if it exists.
    \IfFileExists{intro.tex}{ % If the intro.tex file exists.
        % Input the intro.tex file.
        \textbf{Course}: MATH 16300: Honors Calculus III

\textbf{Section}: 43

\textbf{Professor}: Minjae Park

\textbf{At}: The University of Chicago

\textbf{Quarter}: Spring 2023

\textbf{Course materials}: Calculus by Spivak (4th Edition), Calculus On Manifolds by Spivak

\vspace{1cm}
\textbf{Disclaimer}: This document will inevitably contain some mistakes, both simple typos and serious logical and mathematical errors. Take what you read with a grain of salt as it is made by an undergraduate student going through the learning process himself. If you do find any error, I would really appreciate it if you can let me know by email at \href{mailto:conghungletran@gmail.com}{conghungletran@gmail.com}.

        % Make the pagestyle fancy for the intro.tex page.
        \pagestyle{fancy}

        % Remove the line for the header.
        \renewcommand\headrulewidth{0pt}

        % Remove all header stuff.
        \fancyhead{}

        % Add stuff for the footer in the center.
        % \fancyfoot[C]{
        %   \textit{For more notes like this, visit
        %   \href{\linktootherpages}{\shortlinkname}}. \\
        %   \vspace{0.1cm}
        %   \hrule
        %   \vspace{0.1cm}
        %   \@author, \\
        %   \term: \academicyear, \\
        %   Last Update: \@date, \\
        %   \faculty
        % }

        \newpage
    }{ % If the intro.tex file doesn't exist.
        % Force a \newpageage.
        % \forcenewpage
        \newpage
    }

    % Remove the center stuff we did above, and replace it with just the page
    % number, which is still in roman numerals.
    \fancyfoot[C]{\thepage}
    % Add the table of contents.
    \tableofcontents
    % Force a new page.
    \newpage

    % Move the page numberings back to arabic, from roman numerals.
    \pagenumbering{arabic}
    % Set the page number to 1.
    \setcounter{page}{1}

    % Add the header line back.
    \renewcommand\headrulewidth{0.4pt}
    % In the top right, add the lecture title.
    \fancyhead[R]{\footnotesize \@lecture}
    % In the top left, add the author name.
    \fancyhead[L]{\footnotesize \@author}
    % In the bottom center, add the page.
    \fancyfoot[C]{\thepage}
    % Add a nice gray background in the middle of all the upcoming pages.
    % \definegraybackground
}

\makeatother


%%%%%%%%%%%%%%%%%%%%%%%%%%%%%%%%%%%%%%%%%%%%%%%%%%%%%%%%%%%%%%%%%%%%%%%%%%%%%%%
%                               Custom Commands                               %
%%%%%%%%%%%%%%%%%%%%%%%%%%%%%%%%%%%%%%%%%%%%%%%%%%%%%%%%%%%%%%%%%%%%%%%%%%%%%%%

%%%%%%%%%%%%
%  Circle  %
%%%%%%%%%%%%

\newcommand*\circled[1]{\tikz[baseline= (char.base)]{
        \node[shape=circle,draw,inner sep=1pt] (char) {#1};}
}

%%%%%%%%%%%%%%%%%%%
%  Todo Commands  %
%%%%%%%%%%%%%%%%%%%

% \usepackage{xargs}
% \usepackage[colorinlistoftodos]{todonotes}

% \makeatletter

% \@ifclasswith\class{working}{
%     \newcommandx\unsure[2][1=]{\todo[linecolor=red,backgroundcolor=red!25,bordercolor=red,#1]{#2}}
%     \newcommandx\change[2][1=]{\todo[linecolor=blue,backgroundcolor=blue!25,bordercolor=blue,#1]{#2}}
%     \newcommandx\info[2][1=]{\todo[linecolor=OliveGreen,backgroundcolor=OliveGreen!25,bordercolor=OliveGreen,#1]{#2}}
%     \newcommandx\improvement[2][1=]{\todo[linecolor=Plum,backgroundcolor=Plum!25,bordercolor=Plum,#1]{#2}}

%     \newcommand\listnotes{
%         \newpage
%         \listoftodos[Notes]
%     }
% }{
%     \newcommandx\unsure[2][1=]{}
%     \newcommandx\change[2][1=]{}
%     \newcommandx\info[2][1=]{}
%     \newcommandx\improvement[2][1=]{}

%     \newcommand\listnotes{}
% }

% \makeatother

%%%%%%%%%%%%%
%  Correct  %
%%%%%%%%%%%%%

% EXAMPLE:
% 1. \correct{INCORRECT}{CORRECT}
% Parameters:
% 1. The incorrect statement.
% 2. The correct statement.
\definecolor{correct}{HTML}{009900}
\newcommand\correct[2]{{\color{red}{#1 }}\ensuremath{\to}{\color{correct}{ #2}}}


%%%%%%%%%%%%%%%%%%%%%%%%%%%%%%%%%%%%%%%%%%%%%%%%%%%%%%%%%%%%%%%%%%%%%%%%%%%%%%%
%                                 Environments                                %
%%%%%%%%%%%%%%%%%%%%%%%%%%%%%%%%%%%%%%%%%%%%%%%%%%%%%%%%%%%%%%%%%%%%%%%%%%%%%%%

\usepackage{varwidth}
\usepackage{thmtools}
\usepackage[most,many,breakable]{tcolorbox}

\tcbuselibrary{theorems,skins,hooks}
\usetikzlibrary{arrows,calc,shadows.blur}

%%%%%%%%%%%%%%%%%%%
%  Define Colors  %
%%%%%%%%%%%%%%%%%%%

% color prototype
% \definecolor{color}{RGB}{45, 111, 177}

% ESSENTIALS: 
\definecolor{myred}{HTML}{c74540}
\definecolor{myblue}{HTML}{072b85}
\definecolor{mygreen}{HTML}{388c46}
\definecolor{myblack}{HTML}{000000}

\colorlet{definition_color}{myred}

\colorlet{theorem_color}{myblue}
\colorlet{lemma_color}{myblue}
\colorlet{prop_color}{myblue}
\colorlet{corollary_color}{myblue}
\colorlet{claim_color}{myblue}

\colorlet{proof_color}{myblack}
\colorlet{example_color}{myblack}
\colorlet{exercise_color}{myblack}

% MISCS: 
%%%%%%%%%%%%%%%%%%%%%%%%%%%%%%%%%%%%%%%%%%%%%%%%%%%%%%%%%
%  Create Environments Styles Based on Given Parameter  %
%%%%%%%%%%%%%%%%%%%%%%%%%%%%%%%%%%%%%%%%%%%%%%%%%%%%%%%%%

% \mdfsetup{skipabove=1em,skipbelow=0em}

%%%%%%%%%%%%%%%%%%%%%%
%  Helpful Commands  %
%%%%%%%%%%%%%%%%%%%%%%

% EXAMPLE:
% 1. \createnewtheoremstyle{thmdefinitionbox}{}{}
% 2. \createnewtheoremstyle{thmtheorembox}{}{}
% 3. \createnewtheoremstyle{thmproofbox}{qed=\qedsymbol}{
%       rightline=false, topline=false, bottomline=false
%    }
% Parameters:
% 1. Theorem name.
% 2. Any extra parameters to pass directly to declaretheoremstyle.
% 3. Any extra parameters to pass directly to mdframed.
\newcommand\createnewtheoremstyle[3]{
    \declaretheoremstyle[
        headfont=\bfseries\sffamily, bodyfont=\normalfont, #2,
        mdframed={
                #3,
            },
    ]{#1}
}

% EXAMPLE:
% 1. \createnewcoloredtheoremstyle{thmdefinitionbox}{definition}{}{}
% 2. \createnewcoloredtheoremstyle{thmexamplebox}{example}{}{
%       rightline=true, leftline=true, topline=true, bottomline=true
%     }
% 3. \createnewcoloredtheoremstyle{thmproofbox}{proof}{qed=\qedsymbol}{backgroundcolor=white}
% Parameters:
% 1. Theorem name.
% 2. Color of theorem.
% 3. Any extra parameters to pass directly to declaretheoremstyle.
% 4. Any extra parameters to pass directly to mdframed.

% change backgroundcolor to #2!5 if user wants a colored backdrop to theorem environments. It's a cool color theme, but there's too much going on in the page.
\newcommand\createnewcoloredtheoremstyle[4]{
    \declaretheoremstyle[
        headfont=\bfseries\sffamily\color{#2},
        bodyfont=\normalfont,
        headpunct=,
        headformat = \NAME~\NUMBER\NOTE \hfill\smallskip\linebreak,
        #3,
        mdframed={
                outerlinewidth=0.75pt,
                rightline=false,
                leftline=false,
                topline=false,
                bottomline=false,
                backgroundcolor=white,
                skipabove = 5pt,
                skipbelow = 0pt,
                linecolor=#2,
                innertopmargin = 0pt,
                innerbottommargin = 0pt,
                innerrightmargin = 4pt,
                innerleftmargin= 6pt,
                leftmargin = -6pt,
                #4,
            },
    ]{#1}
}



%%%%%%%%%%%%%%%%%%%%%%%%%%%%%%%%%%%
%  Create the Environment Styles  %
%%%%%%%%%%%%%%%%%%%%%%%%%%%%%%%%%%%

\makeatletter
\@ifclasswith\class{nocolor}{
    % Environments without color.

    % ESSENTIALS:
    \createnewtheoremstyle{thmdefinitionbox}{}{}
    \createnewtheoremstyle{thmtheorembox}{}{}
    \createnewtheoremstyle{thmproofbox}{qed=\qedsymbol}{}
    \createnewtheoremstyle{thmcorollarybox}{}{}
    \createnewtheoremstyle{thmlemmabox}{}{}
    \createnewtheoremstyle{thmclaimbox}{}{}
    \createnewtheoremstyle{thmexamplebox}{}{}

    % MISCS: 
    \createnewtheoremstyle{thmpropbox}{}{}
    \createnewtheoremstyle{thmexercisebox}{}{}
    \createnewtheoremstyle{thmexplanationbox}{}{}
    \createnewtheoremstyle{thmremarkbox}{}{}
    
    % STYLIZED MORE BELOW
    \createnewtheoremstyle{thmquestionbox}{}{}
    \createnewtheoremstyle{thmsolutionbox}{qed=\qedsymbol}{}
}{
    % Environments with color.

    % ESSENTIALS: definition, theorem, proof, corollary, lemma, claim, example
    \createnewcoloredtheoremstyle{thmdefinitionbox}{definition_color}{}{leftline=false}
    \createnewcoloredtheoremstyle{thmtheorembox}{theorem_color}{}{leftline=false}
    \createnewcoloredtheoremstyle{thmproofbox}{proof_color}{qed=\qedsymbol}{}
    \createnewcoloredtheoremstyle{thmcorollarybox}{corollary_color}{}{leftline=false}
    \createnewcoloredtheoremstyle{thmlemmabox}{lemma_color}{}{leftline=false}
    \createnewcoloredtheoremstyle{thmpropbox}{prop_color}{}{leftline=false}
    \createnewcoloredtheoremstyle{thmclaimbox}{claim_color}{}{leftline=false}
    \createnewcoloredtheoremstyle{thmexamplebox}{example_color}{}{}
    \createnewcoloredtheoremstyle{thmexplanationbox}{example_color}{qed=\qedsymbol}{}
    \createnewcoloredtheoremstyle{thmremarkbox}{theorem_color}{}{}

    \createnewcoloredtheoremstyle{thmmiscbox}{black}{}{}

    \createnewcoloredtheoremstyle{thmexercisebox}{exercise_color}{}{}
    \createnewcoloredtheoremstyle{thmproblembox}{theorem_color}{}{leftline=false}
    \createnewcoloredtheoremstyle{thmsolutionbox}{mygreen}{qed=\qedsymbol}{}
}
\makeatother

%%%%%%%%%%%%%%%%%%%%%%%%%%%%%
%  Create the Environments  %
%%%%%%%%%%%%%%%%%%%%%%%%%%%%%
\declaretheorem[numberwithin=section, style=thmdefinitionbox,     name=Definition]{definition}
\declaretheorem[numberwithin=section, style=thmtheorembox,     name=Theorem]{theorem}
\declaretheorem[numbered=no,          style=thmexamplebox,     name=Example]{example}
\declaretheorem[numberwithin=section, style=thmtheorembox,       name=Claim]{claim}
\declaretheorem[numberwithin=section, style=thmcorollarybox,   name=Corollary]{corollary}
\declaretheorem[numberwithin=section, style=thmpropbox,        name=Proposition]{proposition}
\declaretheorem[numberwithin=section, style=thmlemmabox,       name=Lemma]{lemma}
\declaretheorem[numberwithin=section, style=thmexercisebox,    name=Exercise]{exercise}
\declaretheorem[numbered=no,          style=thmproofbox,       name=Proof]{proof0}
\declaretheorem[numbered=no,          style=thmexplanationbox, name=Explanation]{explanation}
\declaretheorem[numbered=no,          style=thmsolutionbox,    name=Solution]{solution}
\declaretheorem[numberwithin=section,          style=thmproblembox,     name=Problem]{problem}
\declaretheorem[numbered=no,          style=thmmiscbox,    name=Intuition]{intuition}
\declaretheorem[numbered=no,          style=thmmiscbox,    name=Goal]{goal}
\declaretheorem[numbered=no,          style=thmmiscbox,    name=Recall]{recall}
\declaretheorem[numbered=no,          style=thmmiscbox,    name=Motivation]{motivation}
\declaretheorem[numbered=no,          style=thmmiscbox,    name=Remark]{remark}
\declaretheorem[numbered=no,          style=thmmiscbox,    name=Observe]{observe}
\declaretheorem[numbered=no,          style=thmmiscbox,    name=Question]{question}


%%%%%%%%%%%%%%%%%%%%%%%%%%%%
%  Edit Proof Environment  %
%%%%%%%%%%%%%%%%%%%%%%%%%%%%

\renewenvironment{proof}[2][\proofname]{
    % \vspace{-12pt}
    \begin{proof0} [#2]
        }{\end{proof0}}

\theoremstyle{definition}

\newtheorem*{notation}{Notation}
\newtheorem*{previouslyseen}{As previously seen}
\newtheorem*{property}{Property}
% \newtheorem*{intuition}{Intuition}
% \newtheorem*{goal}{Goal}
% \newtheorem*{recall}{Recall}
% \newtheorem*{motivation}{Motivation}
% \newtheorem*{remark}{Remark}
% \newtheorem*{observe}{Observe}

\author{Hung C. Le Tran}


%%%% MATH SHORTHANDS %%%%
%% blackboard bold math capitals
\DeclareMathOperator*{\esssup}{ess\,sup}
\DeclareMathOperator*{\Hom}{Hom}
\newcommand{\bbf}{\mathbb{F}}
\newcommand{\bbn}{\mathbb{N}}
\newcommand{\bbq}{\mathbb{Q}}
\newcommand{\bbr}{\mathbb{R}}
\newcommand{\bbz}{\mathbb{Z}}
\newcommand{\bbc}{\mathbb{C}}
\newcommand{\bbk}{\mathbb{K}}
\newcommand{\bbm}{\mathbb{M}}
\newcommand{\bbp}{\mathbb{P}}
\newcommand{\bbe}{\mathbb{E}}

\newcommand{\bfw}{\mathbf{w}}
\newcommand{\bfx}{\mathbf{x}}
\newcommand{\bfX}{\mathbf{X}}
\newcommand{\bfy}{\mathbf{y}}
\newcommand{\bfyhat}{\mathbf{\hat{y}}}

\newcommand{\calb}{\mathcal{B}}
\newcommand{\calf}{\mathcal{F}}
\newcommand{\calt}{\mathcal{T}}
\newcommand{\call}{\mathcal{L}}
\renewcommand{\phi}{\varphi}

% Universal Math Shortcuts
\newcommand{\st}{\hspace*{2pt}\text{s.t.}\hspace*{2pt}}
\newcommand{\pffwd}{\hspace*{2pt}\fbox{\(\Rightarrow\)}\hspace*{10pt}}
\newcommand{\pfbwd}{\hspace*{2pt}\fbox{\(\Leftarrow\)}\hspace*{10pt}}
\newcommand{\contra}{\ensuremath{\Rightarrow\Leftarrow}}
\newcommand{\cvgn}{\xrightarrow{n \to \infty}}
\newcommand{\cvgj}{\xrightarrow{j \to \infty}}

\newcommand{\im}{\mathrm{im}}
\newcommand{\innerproduct}[2]{\langle #1, #2 \rangle}
\newcommand*{\conj}[1]{\overline{#1}}

% https://tex.stackexchange.com/questions/438612/space-between-exists-and-forall
% https://tex.stackexchange.com/questions/22798/nice-looking-empty-set
\let\oldforall\forall
\renewcommand{\forall}{\;\oldforall\; }
\let\oldexist\exists
\renewcommand{\exists}{\;\oldexist\; }
\newcommand\existu{\;\oldexist!\: }
\let\oldemptyset\emptyset
\let\emptyset\varnothing


\renewcommand{\_}[1]{\underline{#1}}
\DeclarePairedDelimiter{\abs}{\lvert}{\rvert}
\DeclarePairedDelimiter{\norm}{\lVert}{\rVert}
\DeclarePairedDelimiter\ceil{\lceil}{\rceil}
\DeclarePairedDelimiter\floor{\lfloor}{\rfloor}
\setlength\parindent{0pt}
\setlength{\headheight}{12.0pt}
\addtolength{\topmargin}{-12.0pt}


% Default skipping, change if you want more spacing
% \thinmuskip=3mu
% \medmuskip=4mu plus 2mu minus 4mu
% \thickmuskip=5mu plus 5mu

% \DeclareMathOperator{\ext}{ext}
% \DeclareMathOperator{\bridge}{bridge}
\title{MATH 20700: Honors Analysis in Rn I \\ \large Problem Set 4}
\date{22 Oct 2023}
\author{Hung Le Tran}
\begin{document}
\maketitle
\setcounter{section}{4}
\textbf{Textbook: Pugh's Real Mathematical Analysis}

\textit{Collaborators: Lucio, Hung Pham, Duc}
\begin{problem} [3.39 \redtext{done}]
Consider the characteristic functions $f(x)$ and $g(x)$ of the intervals $[1, 4]$ and $[2, 5]$. The derivatives $f'$ and $g'$ exist almost everywhere. The integration-by-parts formula says that \[
    \int_0^3 f(x) g'(x) dx = f(3)g(3) - f(0)g(0) - \int_0^3f'(x) g(x) dx
\]
But both integrals are zero, while $f(3) g(3) - f(0)g(0) = 1$. Where is the error?
\end{problem}
\begin{solution}
    $f = \chi_{[1, 4]}, g = \chi_{[2, 5]}$ are not differentiable on $[0, 3]$ to be able to apply IBP. Specifically, $f$ is not differentiable at 1:
    \[
        \lim_{t \to 1+} \frac{f(t) - f(1)}{t - 1} = 0
    \]
    \[
        \lim_{t \to 1-} \frac{f(t) - f(1)}{t - 1} = \lim_{t \to 1-}\frac{-1}{t - 1} = \infty
    \]
    Similarly, $g$ is not differentiable at 2.

    Therefore IBP is not applicable.
\end{solution}
\begin{problem} [3.42 \redtext{done}]
Suppose that $\psi: [c, d] \to [a, b]$ is continuous and for every zero set $Z \subset [a, b]$, $\psi^{pre}(Z)$ is a zero set in $[c, d]$.
\begin{enumerate} [(a)]
    \item If $f$ is RI, prove that $f \circ \psi$ is RI.
    \item Derive Corollary 32 from (a)
\end{enumerate}
\end{problem}
\begin{solution}
    $f: [a, b] \to \bbr$

    \textbf{(a)} Let $D$ be the set of discontinuities of $f$. Since $f$ is RI, $D$ is a zero set.

    Since $\psi$ is continuous, $\psi^{Pre}(D)$ is the set of discontinuities of $f \circ \psi$ (Pugh). $D$ is a zero set, so it follows that $\psi^{Pre}(D)$ is a zero set too. Thus the set of discontinuities of $f \circ \psi$ is a zero set. $f \circ \psi$ is RI. \qed

    \textbf{(b)} Want to show Corollary 32: If $f: [a, b] \to \bbr$ is RI and $\psi: [c, d] \to [a, b]$ is a homeomorphism whose inverse satisfies a Lipschitz condition then $f \circ \psi$ is RI.

    Let $\psi^{-1}$ be $L-$Lipschitz.

    We want to show that $\psi$ satisfies the conditions above in (a), i.e., that for every set $Z \subset [a, b]$, $\psi^{Pre}(Z)$ is a zero set.

    Fix $\epsilon > 0$. Given zero set $Z \subset [a, b]$. Then $Z$ can be covered by finitely many open intervals $\{(a_i, b_i) \mid i \leq k\}$ whose lengths add up to less than $\epsilon / L$. Since $\psi$ is a homeomorphism, $\psi^{-1}(Z)$ is covered by the finitely many open intervals $(a'_i, b'_i) = \psi^{-1}[(a_i, b_i)]$. And we have \[
        \sum |b'_i - a'_i| \leq \sum L(b_i - a_i) < L \frac{\epsilon}{L} = \epsilon
    \]
    so $\psi^{Pre}(Z) = \psi^{-1}(Z)$ is indeed a zero set. Now we can apply (a) to get that $f \circ \psi$ is RI.
\end{solution}
\begin{problem} [3.43 \redtext{done}]
Let $\psi(x) = x \sin (1/x)$ for $0 < x \leq 1$ and $\psi(0) = 0$.
\begin{enumerate} [(a)]
    \item If $f: [-1, 1] \to \bbr$ is RI, prove that $f \circ \psi$ is RI.
    \item What happens for $\psi(x) = \sqrt{x} \sin 1/x$?
\end{enumerate}
\end{problem}
\begin{solution}
    \textbf{(a)} 
    $\psi$ is differentiable on (0, 1):
    \[
        \psi'(x) = \sin(1/x) + x\cos(1/x)(-1)x^{-2} = \sin(1/x) - \cos(1/x)/x.
    \]
    Therefore, $\psi$ has critical points satisfying: \[
    \sin(1/x_c) = \cos(1/x_c)/x_c
    \]
    Crucially, at these critical points, $\sin(1/x_c) \neq 0 $ because if so then $\cos(1/x_c) = 0 $ too, a contradiction.
    \begin{align*}
        \psi''(x) &= \cos(1/x)(-x^{-2}) - \frac{x(-\sin(1/x))(-x^{-2}) - \cos(1/x)}{x^2} \\
        &= -\cos(1/x)x^{-2} - \sin(1/x)x^{-3} + \cos(1/x)x^{-2}\\
        &= -\sin(1/x)x^{-3} \\
        \implies \psi''(x_c) &= -\sin(1/x_c)x_c^{-3} \neq 0
    \end{align*}
    WLOG, assume $\psi''(x_c) > 0$. Since $\psi''$ is continuous, there exists $\delta_1 > 0$ such that $t \in (x_c, x_c + \delta_1) \implies \psi''(t) > 0$, which implies \[
    \psi'(t) = \psi'(x_c) + \int_{x_c}^{x_c+t} \psi''(s) ds > \psi'(x_c) > 0
    \]
    Similarly, there also exist $\delta_2 > 0$ such that $t \in (x_c - \delta_2, x_c) \implies \psi''(t) > 0$, which implies $\psi'(t) < 0$ on $(x_c - \delta_2, x_c)$. Therefore, it can be concluded that for each $x_c$, there exists an open interval $(x_c - \min\{\delta_1, \delta_2\}, x_c + \min\{\delta_1, \delta_2\})$ that has no other critical points. Since $\bbq$ is dense in $\bbr$, there exists $q_c \in \bbq \cap (x_c - \min\{\delta_1, \delta_2\}, x_c + \min\{\delta_1, \delta_2\})$.

    Let $D$ be the set of critical points of $\psi$ on $(0, 1]$. The mapping above from $x_c \mapsto q_c$ describes an injective mapping from $D \to \bbq$. Thus $D$ is countable.

    We may now enumerate the countable critical points with the endpoints 0 and 1:
    \[
    x_1 = 0 < x_2 < x_3 < \cdots < 1
    \]

    Consider $\psi$ restricted to $I_k = (x_k, x_{k+1})$. Since $\psi'$ is continuous and can't achieve 0 in the interval (otherwise there would be another critical point in between), it has to be positive throughout or negative throughout. WLOG, suppose $\psi' > 0$ on $I_k$. Therefore $\psi$ is monotone thus surjective onto $(\psi(x_{k+1}), \psi(x_k))$. Using Inverse Function Theorem, $\psi$ is a homeomorphism. In fact, a $C^1$ homeomorphism, since $\psi'$ is continuous. Using Corollary 15 (Section 3.1), $\psi$ is a $C^1$ diffeomorphism.

    Now, we want to show that $\psi^{-1}$ maps zero sets to zero sets on $I_k$. Restrict $\psi$ to $I_k$, call it $\psi_k = \psi|_k$.

    $I_k = \bigcup_{n \in \bbn} [x_{k+1} + 1/n, x_{k} - 1/n] \eqqcolon \bigcup_{n \in \bbn}J_n$. On each compact $J_n$, we know that $\psi_k'$ is continuous so $\frac{1}{\psi_k'}$ is continuous. It is continuous on compact $J_n$ so it is bounded by $L$. Then Inverse Function Theorem implies \[
    |(\psi_k^{-1})'| = \abs*{\frac{1}{\psi_k' \circ \psi_k}} \leq L
    \]
    Therefore $\psi_k^{-1}$ is $L$-Lipschitz. As shown in 3.42(b), this implies that $\psi_k^{-1}$ maps zero sets to zero sets.

    However, this is only on $J_n$. But if we let $Z$ be a zero set in $I_k$, then $Z = Z \cap I_k = \bigcup (Z \cap J_n) \eqqcolon \bigcup Z_n$. Each $Z_n$ is a zero set so \[
    \psi_k^{-1}(Z) = \bigcup \psi_k^{-1}(Z_n) 
    \]
    is a countable union of zero sets, and is therefore a zero set.

    Therefore, $\psi_k^{-1}$ maps zero sets to zero sets.

    Now, take zero set $Y$ in [-1, 1], then 
    \begin{align*}
        \psi^{Pre}(Y) &= \bigcup_{k\in \bbn}\psi^{Pre}(Y \cap \psi(I_k)) \cup \bigcup_{k\in \bbn}\psi^{Pre}(Y \cap \{\psi(x_k)\}) \\
        &= \bigcup_{k \in \bbn} \psi_k^{-1}(Y \cap \psi(I_k)) \cup \bigcup_{k\in \bbn}\psi^{Pre}(Y \cap \{\psi(x_k)\})
    \end{align*}
    is a union of:
\begin{itemize}
    \item a countable union of zero sets, and 
    \item a countable union of finite sets
\end{itemize}
    is therefore a zero set.
\end{solution}

Then, let $E$ be the set of discontinuities of $f$. Since $f$ is RI, $E$ is zero set. Since $\psi$ is continuous, it follows that the set of discontinuities of $f \circ \psi$ is $\psi^{Pre}(E)$, which is a zero set. $f \circ \psi$ is therefore RI.


\textbf{(b)} The result still holds. It only remains for us to prove that $\psi''$ is continuous on (0, 1) and $\psi'' \neq 0$ at critical points.
\[
\psi'(x) = x^{-3/2}(\sin(1/x)x - 2\cos(1/x))
\]
\[
\psi''(x) = -4x^{-7/2}(\sin(1/x)x^2 -4\cos(1/x)x + 4\sin(1/x))
\]
So $\psi''(x)$ is continuous. Critical points satisfy:
\[
\sin(1/x_c)x_c = 2\cos(1/x_c) \implies \cos(1/x_c) \neq 0, \tan(1/x_c) = 2/x_c
\]
so 
\[
    \begin{aligned}
        &0 &&= \psi''(x_c) \\
        \Leftrightarrow &0 &&= -4x_c^{-7/2}(\sin(1/x_c)x_c^2 -4\cos(1/x_c)x_c + 4\sin(1/x_c))\\
        \Leftrightarrow &0 &&= \sin(1/x_c)(x_c^2 + 4) - 4\cos(1/x_c)x_c \\
        \Leftrightarrow &\tan(1/x_c)(x_c^2 + 4) &&= 4x_c \\
        \Leftrightarrow & \frac{2(x_c^2 + 4)}{x_c} &&=4x_c \\
        \Leftrightarrow & x_c^2 + 4 &&= 2x_c^2 \\
        \Leftrightarrow & x_c && \in \{\pm 2\} \\
    \end{aligned}
\]
so $\psi'' \neq 0$ on $(0, 1)$. Therefore, the same proof in (a) applies. $f \circ \psi$ is therefore RI.
\begin{problem} [4.4a \redtext{done}]
If $f_n: \bbr \to \bbr$ is uniformly continuous for each $n \in \bbn$ and if $f_n \unicvg f$ as $n \to \infty$, prove or disprove that $f$ is uniformly continuous.
\end{problem}
\begin{solution}
    WTS $f$ is uniformly continuous.

    Fix $\epsilon > 0$. Since $f_n \unicvg f$, there exists $N =  N_\epsilon \in \bbn$ such that \[
        \forall n \geq N_\epsilon, d_{sup}(f_n, f) < \epsilon /3
    \]
    Since $f_N$ is uniformly continuous, there exists $\delta = \delta_{N_\epsilon} = \delta_\epsilon$ such that \[
        |x - y| < \delta \implies |f_N(x) - f_N(y)| <
        \epsilon / 3
    \]
    Then we can estimate, for $|x - y| < \delta$:
    \[
        |f(x) - f(y)| \leq |f(x) - f_N(x)| + |f_N(x) - f_N(y) | + |f_N(y) - f(x) | \leq 3 \epsilon /3 = \epsilon
    \]
    $f$ is therefore uniformly continuous.
\end{solution}
\begin{problem} [4.8 \redtext{done}]
Is the sequence of functions $f_n: \bbr \to \bbr$ defined by \[
    f_n(x) = \cos(n+ x) + \log \left(1 + \frac{1}{\sqrt{n+2}} \sin^2(n^nx)\right)
\]
equicontinuous? Prove or disprove.
\end{problem}
\begin{solution}
    (Joshua's hint on Discord)

    We want to show separately that $a_n(x) = \cos(n+x)$, $b_n(x) = \log \left(1 + \frac{1}{\sqrt{n+2}} \sin^2(n^nx)\right)$ are equicontinuous.

    \textbf{1.} \[
        |a_n'(x)| = |-\sin(n+x)| \leq 1
    \]
    Since $a_n(x)$ is continuous and differentiable everywhere on $\bbr$, MVT gives us the equicontinuity of $\{f_n\}$: given $\epsilon > 0$, then choose $\delta = \epsilon$, then \[
        |s - t| < \delta \implies |\cos(n+ s) - \cos(n+ t)| = |-\sin(n+ \theta)| |s - t| < 1\delta < \epsilon
    \]
    so $\cos(n+x)$ is equicontinuous.

    \textbf{2.} WTS $(b_n) $ is uniformly equicontinuous.

    \textbf{2.1.} WTS each $b_n$ is uniformly continuous.

    We do this by showing that $b_n$ is Lipschitz. \[
    |b_n'(x)| = \abs*{\frac{1}{1+\frac{1}{\sqrt{n+2}}\sin^2(n^nx)}\left(\frac{2n^n}{\sqrt{n+2}}\sin(n^nx)\cos(n^nx)\right)} \leq \abs*{\frac{n^n}{\sqrt{n+2}}\sin(2n^nx)} \leq C_n
    \]
    Similar to \textbf{1.}, $b_n$ is $C_n$-Lipschitz. Given $\epsilon > 0$, then we can choose $\delta_n = \epsilon /C_n$, then \[
    |s - t| < \delta_n \implies |b_n(s) - b_n(t)| \leq C_n |s-t| < \epsilon
    \]
    Therefore $b_n$ is uniformly continuous.

    \textbf{2.2.} WTS $(b_n)$ is uniformly Cauchy.

    Note that for fixed $n$, \[
    0 = \log 1 \leq |b_n(x)| \leq \log\left(1 + \frac{1}{\sqrt{n+2}}\right)
    \]

    Therefore, for all $\epsilon > 0$, choose $N \in \bbn$ such that $ \log\left(1 + \frac{1}{\sqrt{n+2}} \right)< \epsilon /2$. Then for all $n, m \geq N$, for all $x \in \bbr$,
    \begin{align*}
        |b_n(x) - b_m(x)| &\leq |b_n(x)| + |b_m(x)| \\
        &\leq 2\log\left(1 + \frac{1}{\sqrt{N+2}}\right) < \epsilon
    \end{align*}
    Therefore for all $n, m \geq N$, \[
    d_{sup}(b_n, b_m) < \epsilon
    \]
    $(b_n)$ is therefore uniformly Cauchy.

    \textbf{2.3.} Given $\epsilon > 0$, then since $(b_n)$ is Cauchy, there exists $N \in \bbn$ such that \[
    n, m \geq N \implies d_{sup}(b_n, b_m) < \epsilon /3
    \]

    Since each $\{b_n\}_{1 \leq n \leq N}$ is uniformly continuous, there exists $\{\delta_n\}_{1\leq n \leq N}$ such that \[
    |s - t| < \delta_n \implies |b_n(s) - b_n(t)| < \epsilon /3
    \]
    Take $\delta = \min\{\delta_n : 1 \leq n \leq N\}$. Then, for $|s - t| < \delta$, 
    \begin{itemize}
        \item If $1 \leq n \leq N$ then $|b_n(s) - b_n(t)| < \epsilon / 3 < \epsilon$, since $\delta \leq \delta_n$. 
        \item If $n > N$ then \[
        |b_n(s) - b_n(t)| \leq |b_n(s) - b_N(s)| + |b_N(s) - b_N(t)| + |b_N(t) - b_n(t) < 3\epsilon / 3 = \epsilon
        \]
    \end{itemize}

    It follows that \[
    |s - t| < \delta \implies |b_n(s) - b_n(t)| < \epsilon \forall n \in \bbn
    \]
    $(b_n)$ is therefore uniformly equicontinuous.

    \textbf{3.} Since $(a_n)$ and $(b_n)$ are both uniformly equicontinuous, given $\epsilon > 0$, there exists $\delta_a, \delta_b > 0$ such that \[
    |s-t| < \delta_a \implies |a_n(s) - a_n(t)| < \epsilon /2, |s-t| < \delta_b \implies |b_n(s) - b_n(t)| < \epsilon/2
    \]
    Take $\delta = \min\{\delta_a, \delta_b\}$ then \[
    |s-t| < \delta \implies |f_n(s) - f_n(t)| \leq |a_n(s) - a_n(t)| + |b_n(s) - b_n(t)| < 2\epsilon / 2 = \epsilon 
    \]
    $(f_n)$ is therefore uniformly equicontinuous.
\end{solution}
\begin{problem} [4.9 \redtext{done}]
If $f: \bbr \to \bbr$ is continuous and the sequence $f_n(x) = f(nx)$ is equicontinuous, what can be said about $f$?
\end{problem}
\begin{solution}
    WTS $f$ is a constant function, i.e. $f(x) = C \forall x$ for some $C \in \bbr$.

    Suppose not. Then there exists $a \neq b \in \bbr$ such that $f(a) \neq f(b)$.

    Let $\epsilon = \frac{|f(b) - f(a)|}{2}$. Then there exists $\delta$ such that \[
        |x-y| < \delta \implies |f(nx) - f(ny)| < \epsilon \forall n \in \bbn
    \]
    Choose $N$ sufficiently large, such that $|\frac{b-a}{N}| < \delta$, i.e. $N > \frac{|b-a|}{\delta}$. Then that implies \[
        |b/N - a/N| < \delta \implies |f(b) - f(a)| = |f(N(b/N)) - f(N(a/N))| < \epsilon = \frac{|f(b) - f(a)|}{2} \contra
    \]
    It follows that $f$ is a constant function.
\end{solution}

\begin{problem} [4.14 \redtext{done}]
Recall from Exercise 2.78 that a metric space $M$ is chain connected if for each $\epsilon > 0$ and each $p, q \in M$ there is a chain $p = p_0, \dots, p_n = q$ in $M$ such that \[
    d(p_{k-1}, p_k) < \epsilon \:\text{for}\:  1\leq k \leq n.
\]
A family $\calf$ of functions $f: M \to \bbr$ is bounded at $p \in M$ if the set $\{f(p) : f \in \calf\}$ is bounded in $\bbr$.

Show that $M$ is chain connected if and only if pointwise boundedness of an equicontinuous family at one point of $M$ implies pointwise boundedness at every point of $M$.
\end{problem}
\begin{solution}
    \pffwd Let $M$ be chain connected. And let $\calf$ be an equicontinuous family of functions that is bounded at some $p \in M$, i.e. $\{f(p) : f \in \calf\}$ is bounded in $\bbr$. WTS $\calf$ is pointwise bounded at every point $q \in M$.

    $\{f(p) : f \in \calf\}$ is bounded in $\bbr$. Thus there exists $L \in \bbr$ such that $|f(p)| < L$ for all $f \in \calf$

    Take arbitrary $q \in M$. Fix $\epsilon = 1$. Since $\calf$ is equicontinuous, there exists $\delta > 0$ such that \[
        |x- y| < \delta \implies |f(x) - f(y)| < 1 \forall f \in \calf
    \]
    Since $M$ is chain connected, there exists a chain $p = p_0, \dots, p_N = q$ in $M$ such that $d(p_{k-1}, p_k) < \delta \forall 1 \leq k \leq N$.
    Now we can make the estimate for all $f \in \calf$,  \[
        |f(q) - f(p)| \leq \sum_{k=1}^{N}|f(p_{k}) - f(p_{k-1})| < N
    \]
    therefore for all $f \in \calf$ \[
        |f(q)| \leq N + |f(p)| < N + L
    \]
    so $\calf$ is pointwise bounded at $q \in M$. \qed

    \pfbwd We have that for all equicontinuous family $\calf$ that is pointwise bounded at one point $p \in M$, it is pointwise boundedness at every point in $M$.

    % If $M$ has one element, then we have $M$ is chain-connected trivially. Consider when it has more than one.

    % Construct $\calf = \{f_n(x) = n \inf_{y \in M \backslash \{x\}} d(x, y) \}_{n \in \bbn}$. Note that $\inf_{y \in M \backslash \{x\}, y \neq x} d(x, y)$ is well-defined because the set $\{d(x, y) : y \in M \backslash \{x\}\}$ is non-empty and has 0 as a lower bound.

    % We first show that $\calf$ is indeed equicontinuous. Indeed, given $\epsilon > 0$, we can choose $\delta = \epsilon/2$. Then, 
    % \[
    % d(x, y) < \delta \implies |f_n(x) - f_n(y)| \leq |f_n(x)| + |f_n(y)| 
    % \]
    Let's prove by contradiction. Suppose $M$ is not chain-connected. Then there exists $\epsilon_0 > 0; p, q \in M$ such that there is no $\epsilon_0-$chain from $p$ to $q$. Let $A$ be the set of points that can be $\epsilon_0$-chain-connected from $p$, and let $B$ be the set of points that cannot. Then $p \in A, q \in B$. Also, \[
    M = A \sqcup B
    \]
    so if $c \in M$ then $c \in A$ or $c \in B$. Define family $\calf = \{f_n: M \to \bbr\}$ as follows \[
    f_n(x) = \begin{cases}
        0 &\:\text{for}\: x \in A \\
        n &\:\text{for}\: x \in B
    \end{cases}
    \]
    then $f_n(p) = 0 \forall n \in \bbn$, since $p \in A$. Thus $\calf$ is pointwise bounded at $p$.

    Also note that for all $a \in A, b \in B$, we have $d(a, b) \geq \epsilon_0$. Because otherwise, since there exists an $\epsilon_0$-chain from $p$ to $a$, we can concatenate this chain with $b$ as the last node to get an $\epsilon_0$-chain from $p$ to $b$, making $b \in A$.
    
    Now, given any $\epsilon > 0$, we can choose $\delta = \epsilon_0$. Then for $d(s, t) < \delta$, since if one of $s$ and $t$ is in $A$ and the other in $B$ then $d(s, t) \geq \epsilon_0 > \delta$, so they must be both in $A$ or both in $B$. Therefore $f_n(s) = f_n(t) \implies d(f_n(s), f_n(t)) = 0 < \epsilon$.

    Therefore, $\calf$ is uniformly equicontinuous.

    The problem's assumption then yields that $\calf$ is pointwise bounded at every point in $M$. But $\{f_n(q)\} = \bbn$ is clearly NOT bounded! \contra

    By contradiction, $M$ must therefore be chain-connected.
\end{solution}

\begin{problem} [4.23 a-c \redtext{done}]
Let $M$ be a compact metric space, and let $(i_n)$ be a sequence of isometries $i_n : M \to M$. \begin{enumerate} [(a)]
    \item Prove that there exists a subsequence $i_{n_k}$ that converges to an isometry $i$ as $k\to \infty$.
    \item Infer that the space of self-isometries of $M$ is compact.
    \item Does the inverse isometry $i_{n_k}^{-1}$ converge to $i^{-1}$? (Proof or counterexample.)
\end{enumerate}
\end{problem}
\begin{solution}
    \textbf{(a)} We want to show that the familiy of functions $i_n: X \to Y$ is uniformly equicontinuous. In fact, given $\epsilon > 0$, we can choose $\delta = \epsilon$. Then, \[
        d(x, y) < \delta \implies d(i_n(x), i_n(y)) = d(x, y) < \delta = \epsilon \forall n \in \bbn
    \]
    which gives us uniformly equicontinuous $(i_n)$.

    Now we want to show that the space of self-isometries $S = \{i: M \to M\}$ is complete. As a result from above, we've seen that each self-isometry is continuous. Thus, since $C^0(M, \bbr)$ is complete, if the sequence of isometries $i_n$ is Cauchy then $i_n \cvgn i \in C^0$. Now we want to show $i \in S$. Indeed, since $(i_n)$ is uniformly equicontinuous, given $\epsilon > 0$, there exists $N = N_\epsilon \in \bbn$ such that for all $n \geq N$, $d_{sup}(i_n, i) < \epsilon / 2$. It follows that, for all $x, y \in M$,
    \[
        d(i(x), i(y)) \leq d(i(x), i_N(x)) + d(i_N(x), i_N(y)) + d(i_N(y), i(y)) < \epsilon + d(x, y)
    \]
    and \[
        d(x, y) = d(i_N(x), i_N(y)) \leq d(i_N(x), i(x)) + d(i(x), i(y)) + d(i(y), i_N(y)) < \epsilon + d(i(x),i(y))
    \]
    We therefore have the inequalities:
    \[
        d(x, y) - \epsilon < d(i(x), i(y)) < d(x, y) + \epsilon
    \]
    that holds for all $\epsilon > 0; x, y \in M$. It follows that $d(i(x), i(y)) = d(x, y) \forall x, y \in M$ and $i$ is indeed an isometry.

    Therefore the space of self-isometries $S$ is complete.


    For each $x \in M$, we have that $(i_n(x))_{n \in \bbn}$ lies in a compact subset of $M$, namely $M$ itself. Therefore, using Theorem 37 (Section 4.8), $(i_n)$ has a uniformly Cauchy subsequence. Since $S$ is complete, the uniformly Cauchy subsequence is convergent. \qed

    \textbf{(b)} Restating what we proved from (a): for all $\{i_n\} \subseteq S$, there exists a convergent subsequence $\{i_{n_k}\}$. $S$, the space of self-isometries from $M$ to $M$, is therefore compact. \qed

    \textbf{(c)} (With Prof Wilkinson's provided hint)
    We first show that the inverse $i_n^{-1}$ is well-defined, i.e., that all self-isometries $f: M \to M$ for $M$ compact is a bijection.

    Take arbitrary $x_0 \in M$ and arbitrary isometry $f: M \to M$. Then we can construct the following sequence:
    \[
        x_1 = f(x_0), x_2 = (f \circ f)(x_0), \dots, x_n = f^{\circ n}(x_0), \dots
    \]
    Clearly, $\{x_n\} \subseteq f(M)$. Since $M$ is compact, there exists a subsequence $(x_{n_j})_{j \in \bbn}$ that converges to some $x \in M$.

    Fix $\epsilon > 0$. Then there exists $J \in \bbn$ such that for all $j \geq J$, \[
        d(x_{n_j}, x) < \epsilon/2
    \]
    Then choose $J_1 > J_2 \geq J$. Let $N_1 = n_{J_1} > N_2 = n_{J_2}$ then we have \[
        d(x_{N_1}, x), d(x_{N_2}, x) < \epsilon / 2 \implies d(x_{N_1}, x_{N_2}) \leq d(x_{N_1}, x) + d(x_{N_2}, x) < \epsilon
    \]
    But by construction of $(x_n)$, \[
        \epsilon > d(x_{N_1}, x_{N_2}) = d(f^{\circ N_1}(x_0), f^{\circ N_2}(x_0)) = d(f^{\circ (N_1 - N_2)}(x_0), x_0) = d(x_{N_2 - N_1}, x_0)
    \]
    It follows that for all $\epsilon > 0$, there always exists $K = K_\epsilon, x_K \in M$ such that $d(x_K, x_0) < \epsilon$. Therefore $x_0$ is a cluster point of $f(M)$.

    From above, every isometry is continuous. $f$ is therefore continuous and $M$ is compact so $f(M)$ is compact, and therefore closed. $x_0$ is a cluster point of a closed set and is therefore in the set.

    It follows that every self-isometry of $M$ compact is surjective. It is clear that every isometry is also injective, because \[
    x \neq y \implies d(x, y) > 0 \implies d(f(x), f(y)) = d(x, y) > 0 \implies f(x) \neq f(y)
    \]
    Thus, every self-isometry of $M$ compact is bijective and $f^{-1}$ is well-defined. It is also clear that $f^{-1}$ is an isometry: \[
    d(f^{-1}(x), f^{-1}(y)) = d(f(f^{-1}(x)), f(f^{-1})(y)) = d(x, y)
    \]

    Now, we want to show that yes, if self-isometries $(i_{n_k})$ converge to $i$ then $(i_{n_k}^{-1})$ also converge to $i^{-1}$.

    For simplicity, relabel $i_{n_k}$ as $g_k$. Then since $g_k$ converges to $i$, given $\epsilon > 0$, there exists $K \in \bbn$ such that for all $k \geq K$, \[
    \sup_{x \in M}\{d_M(g_k(x), i(x))\} < \epsilon
    \]
    Then for all $k \geq K$, \begin{align*}
    \sup_{x \in M}\{d(g_k^{-1}(x), i^{-1}(x))\} &= \sup_{x \in M}\{d(i(g_k^{-1}(x)), x) \} \\
    &= \sup_{x \in M}\{d(i(g_k^{-1}(x)), g_k(g_k^{-1}(x)))\} \\
    &\leq \sup_{x \in M}\{d(i(x), g_k(x))\} < \epsilon
    \end{align*}
    Therefore $g_k^{-1}$ does converge to $i^{-1}$.
\end{solution}

\begin{problem} [Prelim 4.8 \redtext{done}]
Let $h: [0, 1) \to \bbr$ be a uniformly continuous function where [0, 1) is the half-open interval. Prove that there is a unique continuous map $g: [0, 1] \to \bbr$ such that $g(x) = h(x)$ for all $x \in [0, 1)$.
\end{problem}
\begin{solution}
    Let $a_n \coloneqq h\left(1-\frac{1}{n}\right)$. We want to show that $(a_n)$ is Cauchy. Given $\epsilon > 0$, then since $h$ is uniformly continuous, there exists $\delta_1 > 0$ such that \[
        |x-y| < \delta_1 \implies |h(x) - h(y)| < \epsilon
    \]
    Choose $N_1 \in \bbn$ such that $2/N_1 < \delta_1$. It then follows that for $m, n \geq N_1$, \[
        \abs*{\left(1-\frac{1}{m}\right) - \left(1 - \frac{1}{n}\right)} \leq \frac{2}{\min\{m, n\}} \leq \frac{2}{N_1} < \delta_1
    \]
    which implies \[
        |a_m - a_n| = |h(1 - 1/m) - h(1-1/n)| < \epsilon
    \]
    $(a_n)$ is therefore Cauchy and converges to unique $L \in \bbr$.

    We now want to show that $\lim_{t \to 1-} g(t) = \lim_{t \to 1-} h(t) = L$.

    Indeed, given $\epsilon > 0$, there exists $\delta_2 > 0$ such that \[
        |x - y| < \delta_2 \implies |h(x) - h(y)| < \epsilon / 2
    \]
    Furthermore, since $(a_n)$ converges to $L$, there exists $N_2 \in \bbn$ such that for all $n \in \bbn$, \[
        |a_n - L| < \epsilon/2
    \]
    Therefore, for $0 < 1 - t < \frac{\delta_2}{2}$, we can choose $N_3 \geq N_2$ such that $\frac{1}{N_3} < \frac{\delta_2}{2}$. Then
    \[
        |t - (1-1/N_3)| \leq |1-t| + |1/N_3| < 2\delta_2/2 = \delta_2
    \]
    which implies
    \[
        |h(t) - L| \leq |h(t) - h(1-1/N_3)| + |h(1-1/N_3) - L| < \epsilon/2 + \epsilon/2 = \epsilon
    \]
    In short, there exists $\delta = \delta_2/2$ such that \[
        1 - t < \delta \implies |h(t) - L| < \epsilon
    \]
    Therefore $\lim_{h \to 1-} h(t) = L$.

    Since $g$ is continuous, it follows that $g(1) = \lim_{t \to 1-}g(t) = L$ is unique. $g$ is therefore unique, since it is already defined on [0, 1).
\end{solution}

\begin{problem} [Prelim 4.12 \redtext{done}]
Let $f: [0, 1] \to \bbr$ be continuously differentiable, with $f(0) = 0$. Prove that \[
    \norm{f}^2 \leq \int_0^1 (f'(x))^2 dx
\]
where $\norm{f} = \sup\{|f(t)|: 0\leq t \leq 1\}$.
\end{problem}
\begin{solution}
    Since $f$ is continuously differentiable, $|f|$ is continuous on $[0, 1]$. Therefore it achieves its maximum at $x_0 \in [0, 1]$, i.e. \[
        \norm{f} = \sup\{|f(t)|: 0\leq t \leq 1\} = |f(x_0)|
    \]
    Therefore, using 2nd FTC and Cauchy-Schwarz,
    \begin{align*}
        \norm{f}^2 & = |f(x_0)|^2                                                      \\
                   & = \abs*{f(0) + \int_{0}^{x_0}f'(t) dt}^2                          \\
                   & = \abs*{\int_{0}^{x_0}f'(t) dt}^2                                 \\
                   & \leq \left(\int_{0}^{x_0}(f'(t))^2 dt\right) \int_{0}^{x_0}1^2 dt \\
                   & \leq \left(\int_{0}^{1}(f'(t))^2 dt \right) x_0                   \\
                   & \leq \int_{0}^{1}(f'(t))^2 dt
    \end{align*}
    since $x_0 \in [0, 1]$.
\end{solution}

\begin{problem} [Prelim 4.20 \redtext{done}]
Let $(g_n)$ be a sequence of RI functions from [0, 1] into $\bbr$ such that $|g_n(x)| \leq 1$ for all $n, x$. Define \[
    G_n(x) = \int_0^xg_n(t) dt.
\]
Prove that a subsequence of $(G_n)$ converges uniformly.
\end{problem}
\begin{solution}
    Note that $G_n$ is only defined on $[0, 1]$.

    We first want to show that $G_n$ is uniformly equicontinuous. Indeed,
    \begin{align*}
        \abs{G_n(x) - G_n(y)} & = \abs*{\int_{0}^{x}g_n(t)dt - \int_{0}^{y}g_n(t)dt} \\
                              & = \abs*{\int_{x}^{y}g_n(t) dt}                       \\
                              & \leq \abs*{\int_{x}^{y}\abs{g_n(t)}dt}               \\
                              & \leq \abs{x-y}
    \end{align*}
    Thus, given $\epsilon > 0$, we can choose $\delta = \epsilon$, then for all $G_n$, \[
        \abs{G_n(x) - G_n(y)} \leq \abs{x-y} < \delta = \epsilon
    \]

    Secondly, we want to show that $G_n$ is bounded. Indeed,
    \[
        |G_n(x)| = \left|\int_{0}^{x}g_n(t) dt \right| \leq \left|\int_{0}^{x}|g_n(t)| dt\right| \leq \abs*{x} \leq 1
    \]

    Therefore, combining the 2 facts above, using Arzela-Ascoli, it follows that $(G_n)$ has a convergent subsequence.
\end{solution}
\end{document}