\documentclass[a4paper, 12pt]{article}
%%%%%%%%%%%%%%%%%%%%%%%%%%%%%%%%%%%%%%%%%%%%%%%%%%%%%%%%%%%%%%%%%%%%%%%%%%%%%%%
%                                Basic Packages                               %
%%%%%%%%%%%%%%%%%%%%%%%%%%%%%%%%%%%%%%%%%%%%%%%%%%%%%%%%%%%%%%%%%%%%%%%%%%%%%%%

% Gives us multiple colors.
\usepackage[usenames,dvipsnames,pdftex]{xcolor}
% Lets us style link colors.
\usepackage{hyperref}
% Lets us import images and graphics.
\usepackage{graphicx}
% Lets us use figures in floating environments.
\usepackage{float}
% Lets us create multiple columns.
\usepackage{multicol}
% Gives us better math syntax.
\usepackage{amsmath,amsfonts,mathtools,amsthm,amssymb}
% Lets us strikethrough text.
\usepackage{cancel}
% Lets us edit the caption of a figure.
\usepackage{caption}
% Lets us import pdf directly in our tex code.
\usepackage{pdfpages}
% Lets us do algorithm stuff.
\usepackage[ruled,vlined,linesnumbered]{algorithm2e}
% Use a smiley face for our qed symbol.
\usepackage{tikzsymbols}
% \usepackage{fullpage} %%smaller margins
\usepackage[shortlabels]{enumitem}

\usepackage{setspace}
\usepackage[margin=1in, headsep=12pt]{geometry}
\usepackage{wrapfig}
\def\class{article}


%%%%%%%%%%%%%%%%%%%%%%%%%%%%%%%%%%%%%%%%%%%%%%%%%%%%%%%%%%%%%%%%%%%%%%%%%%%%%%%
%                                Basic Settings                               %
%%%%%%%%%%%%%%%%%%%%%%%%%%%%%%%%%%%%%%%%%%%%%%%%%%%%%%%%%%%%%%%%%%%%%%%%%%%%%%%

%%%%%%%%%%%%%
%  Symbols  %
%%%%%%%%%%%%%

\let\implies\Rightarrow
\let\impliedby\Leftarrow
\let\iff\Leftrightarrow
\let\epsilon\varepsilon
%%%%%%%%%%%%
%  Tables  %
%%%%%%%%%%%%

\setlength{\tabcolsep}{5pt}
\renewcommand\arraystretch{1.5}

%%%%%%%%%%%%%%
%  SI Unitx  %
%%%%%%%%%%%%%%

\usepackage{siunitx}
\sisetup{locale = FR}

%%%%%%%%%%
%  TikZ  %
%%%%%%%%%%

\usepackage[framemethod=TikZ]{mdframed}
\usepackage{tikz}
\usepackage{tikz-cd}
\usepackage{tikzsymbols}

\usetikzlibrary{intersections, angles, quotes, calc, positioning}
\usetikzlibrary{arrows.meta}

\tikzset{
    force/.style={thick, {Circle[length=2pt]}-stealth, shorten <=-1pt}
}

%%%%%%%%%%%%%%%
%  PGF Plots  %
%%%%%%%%%%%%%%%

\usepackage{pgfplots}
\pgfplotsset{width=10cm, compat=newest}

%%%%%%%%%%%%%%%%%%%%%%%
%  Center Title Page  %
%%%%%%%%%%%%%%%%%%%%%%%

\usepackage{titling}
\renewcommand\maketitlehooka{\null\mbox{}\vfill}
\renewcommand\maketitlehookd{\vfill\null}

%%%%%%%%%%%%%%%%%%%%%%%%%%%%%%%%%%%%%%%%%%%%%%%%%%%%%%%
%  Create a grey background in the middle of the PDF  %
%%%%%%%%%%%%%%%%%%%%%%%%%%%%%%%%%%%%%%%%%%%%%%%%%%%%%%%

\usepackage{eso-pic}
\newcommand\definegraybackground{
    \definecolor{reallylightgray}{HTML}{FAFAFA}
    \AddToShipoutPicture{
        \ifthenelse{\isodd{\thepage}}{
            \AtPageLowerLeft{
                \put(\LenToUnit{\dimexpr\paperwidth-222pt},0){
                    \color{reallylightgray}\rule{222pt}{297mm}
                }
            }
        }
        {
            \AtPageLowerLeft{
                \color{reallylightgray}\rule{222pt}{297mm}
            }
        }
    }
}

%%%%%%%%%%%%%%%%%%%%%%%%
%  Modify Links Color  %
%%%%%%%%%%%%%%%%%%%%%%%%

\hypersetup{
    % Enable highlighting links.
    colorlinks,
    % Change the color of links to blue.
    urlcolor=blue,
    % Change the color of citations to black.
    citecolor={black},
    % Change the color of url's to blue with some black.
    linkcolor={blue!80!black}
}

%%%%%%%%%%%%%%%%%%
% Fix WrapFigure %
%%%%%%%%%%%%%%%%%%

\newcommand{\wrapfill}{\par\ifnum\value{WF@wrappedlines}>0
        \parskip=0pt
        \addtocounter{WF@wrappedlines}{-1}%
        \null\vspace{\arabic{WF@wrappedlines}\baselineskip}%
        \WFclear
    \fi}

%%%%%%%%%%%%%%%%%
% Multi Columns %
%%%%%%%%%%%%%%%%%

\let\multicolmulticols\multicols
\let\endmulticolmulticols\endmulticols

\RenewDocumentEnvironment{multicols}{mO{}}
{%
    \ifnum#1=1
        #2%
    \else % More than 1 column
        \multicolmulticols{#1}[#2]
    \fi
}
{%
    \ifnum#1=1
    \else % More than 1 column
        \endmulticolmulticols
    \fi
}

\newlength{\thickarrayrulewidth}
\setlength{\thickarrayrulewidth}{5\arrayrulewidth}


%%%%%%%%%%%%%%%%%%%%%%%%%%%%%%%%%%%%%%%%%%%%%%%%%%%%%%%%%%%%%%%%%%%%%%%%%%%%%%%
%                           School Specific Commands                          %
%%%%%%%%%%%%%%%%%%%%%%%%%%%%%%%%%%%%%%%%%%%%%%%%%%%%%%%%%%%%%%%%%%%%%%%%%%%%%%%

%%%%%%%%%%%%%%%%%%%%%%%%%%%
%  Initiate New Counters  %
%%%%%%%%%%%%%%%%%%%%%%%%%%%

\newcounter{lecturecounter}

%%%%%%%%%%%%%%%%%%%%%%%%%%
%  Helpful New Commands  %
%%%%%%%%%%%%%%%%%%%%%%%%%%

\makeatletter

\newcommand\resetcounters{
    % Reset the counters for subsection, subsubsection and the definition
    % all the custom environments.
    \setcounter{subsection}{0}
    \setcounter{subsubsection}{0}
    \setcounter{paragraph}{0}
    \setcounter{subparagraph}{0}
    \setcounter{theorem}{0}
    \setcounter{claim}{0}
    \setcounter{corollary}{0}
    \setcounter{lemma}{0}
    \setcounter{exercise}{0}

    \@ifclasswith\class{nocolor}{
        \setcounter{definition}{0}
    }{}
}

%%%%%%%%%%%%%%%%%%%%%
%  Lecture Command  %
%%%%%%%%%%%%%%%%%%%%%

\usepackage{xifthen}

% EXAMPLE:
% 1. \lecture{Oct 17 2022 Mon (08:46:48)}{Lecture Title}
% 2. \lecture[4]{Oct 17 2022 Mon (08:46:48)}{Lecture Title}
% 3. \lecture{Oct 17 2022 Mon (08:46:48)}{}
% 4. \lecture[4]{Oct 17 2022 Mon (08:46:48)}{}
% Parameters:
% 1. (Optional) lecture number.
% 2. Time and date of lecture.
% 3. Lecture Title.
\def\@lecture{}
\def\@lectitle{}
\def\@leccount{}
\newcommand\lecture[3]{
    %   % Add 1 to the lecture counter.
    %   \addtocounter{lecturecounter}{1}

    % Set the section number to the lecture counter.
    \setcounter{section}{#1}
    \renewcommand\thesubsection{#1.\arabic{subsection}}

    % Reset the counters.
    \resetcounters

    % Check if user passed the lecture title or not.
    \def\@leccount{Lecture #1}
    \ifthenelse{\isempty{#3}}{
        \def\@lecture{Lecture #1}
    }{
        \def\@lecture{Lecture #1: #3}
        \def\@lectitle{#3}
    }

    % Display the information like the following:
    %                                                  Oct 17 2022 Mon (08:49:10)
    % ---------------------------------------------------------------------------
    % Lecture 1: Lecture Title
    \newpage
    \begin{mdframed}
        % \section*{\@lecture}
        \begin{center}
            \Large \textbf{\@leccount}

            \vspace*{0.2cm}

            \large\@lectitle

            \vspace*{0.2cm}
            \normalsize #2
        \end{center}
    \end{mdframed}
    \addcontentsline{toc}{section}{\@lecture}
}

%%%%%%%%%%%%%%%%%%%%
%  Import Figures  %
%%%%%%%%%%%%%%%%%%%%

\usepackage{import}
\pdfminorversion=7

% EXAMPLE:
% 1. \incfig{limit-graph}
% 2. \incfig[0.4]{limit-graph}
% Parameters:
% 1. The figure name. It should be located in figures/NAME.tex_pdf.
% 2. (Optional) The width of the figure. Example: 0.5, 0.35.
\newcommand\incfig[2][1]{%
    \def\svgwidth{#1\columnwidth}
    \import{./figures/}{#2.pdf_tex}
}

\begingroup\expandafter\expandafter\expandafter\endgroup
\expandafter\ifx\csname pdfsuppresswarningpagegroup\endcsname\relax
\else
    \pdfsuppresswarningpagegroup=1\relax
\fi

%%%%%%%%%%%%%%%%%
% Fancy Headers %
%%%%%%%%%%%%%%%%%

\usepackage{fancyhdr}

% Force a new page.
\newcommand\forcenewpage{\clearpage\mbox{~}\clearpage\newpage}

% This command makes it easier to manage my headers and footers.
\newcommand\createintro{
    % Use roman page numbers (e.g. i, v, vi, x, ...)
    \pagenumbering{roman}

    % Display the page style.
    \maketitle
    % Make the title pagestyle empty, meaning no fancy headers and footers.
    \thispagestyle{empty}
    % Create a newpage.
    \newpage

    % Input the intro.tex page if it exists.
    \IfFileExists{intro.tex}{ % If the intro.tex file exists.
        % Input the intro.tex file.
        \textbf{Course}: COURSE

\textbf{Section}: SECTION

\textbf{Professor}: PROFESSOR

\textbf{At}: The University of Chicago

\textbf{Quarter}: QUARTER

\textbf{Course materials}: COURSE_MATERIALS

\vspace{1cm}
\textbf{Disclaimer}: This document will inevitably contain some mistakes, both simple typos and serious logical and mathematical errors. Take what you read with a grain of salt as it is made by an undergraduate student going through the learning process himself. If you do find any error, I would really appreciate it if you can let me know by email at \href{mailto:conghungletran@gmail.com}{conghungletran@gmail.com}.

        % Make the pagestyle fancy for the intro.tex page.
        \pagestyle{fancy}

        % Remove the line for the header.
        \renewcommand\headrulewidth{0pt}

        % Remove all header stuff.
        \fancyhead{}

        % Add stuff for the footer in the center.
        % \fancyfoot[C]{
        %   \textit{For more notes like this, visit
        %   \href{\linktootherpages}{\shortlinkname}}. \\
        %   \vspace{0.1cm}
        %   \hrule
        %   \vspace{0.1cm}
        %   \@author, \\
        %   \term: \academicyear, \\
        %   Last Update: \@date, \\
        %   \faculty
        % }
    }{ % If the intro.tex file doesn't exist.
        % Force a \newpageage.
        \forcenewpage
    }

    % Create a new page.
    \newpage

    % Remove the center stuff we did above, and replace it with just the page
    % number, which is still in roman numerals.
    \fancyfoot[C]{\thepage}
    % Add the table of contents.
    \tableofcontents
    % Force a new page.
    \forcenewpage

    % Move the page numberings back to arabic, from roman numerals.
    \pagenumbering{arabic}
    % Set the page number to 1.
    \setcounter{page}{1}

    % Add the header line back.
    \renewcommand\headrulewidth{0.4pt}
    % In the top right, add the lecture title.
    \fancyhead[R]{\footnotesize \@lecture}
    % In the top left, add the author name.
    \fancyhead[L]{\footnotesize \@author}
    % In the bottom center, add the page.
    \fancyfoot[C]{\thepage}
    % Add a nice gray background in the middle of all the upcoming pages.
    % \definegraybackground
}

\makeatother


%%%%%%%%%%%%%%%%%%%%%%%%%%%%%%%%%%%%%%%%%%%%%%%%%%%%%%%%%%%%%%%%%%%%%%%%%%%%%%%
%                               Custom Commands                               %
%%%%%%%%%%%%%%%%%%%%%%%%%%%%%%%%%%%%%%%%%%%%%%%%%%%%%%%%%%%%%%%%%%%%%%%%%%%%%%%

%%%%%%%%%%%%
%  Circle  %
%%%%%%%%%%%%

\newcommand*\circled[1]{\tikz[baseline=(char.base)]{
        \node[shape=circle,draw,inner sep=1pt] (char) {#1};}
}

%%%%%%%%%%%%%%%%%%%
%  Todo Commands  %
%%%%%%%%%%%%%%%%%%%

% \usepackage{xargs}
% \usepackage[colorinlistoftodos]{todonotes}

% \makeatletter

% \@ifclasswith\class{working}{
%     \newcommandx\unsure[2][1=]{\todo[linecolor=red,backgroundcolor=red!25,bordercolor=red,#1]{#2}}
%     \newcommandx\change[2][1=]{\todo[linecolor=blue,backgroundcolor=blue!25,bordercolor=blue,#1]{#2}}
%     \newcommandx\info[2][1=]{\todo[linecolor=OliveGreen,backgroundcolor=OliveGreen!25,bordercolor=OliveGreen,#1]{#2}}
%     \newcommandx\improvement[2][1=]{\todo[linecolor=Plum,backgroundcolor=Plum!25,bordercolor=Plum,#1]{#2}}

%     \newcommand\listnotes{
%         \newpage
%         \listoftodos[Notes]
%     }
% }{
%     \newcommandx\unsure[2][1=]{}
%     \newcommandx\change[2][1=]{}
%     \newcommandx\info[2][1=]{}
%     \newcommandx\improvement[2][1=]{}

%     \newcommand\listnotes{}
% }

% \makeatother

%%%%%%%%%%%%%
%  Correct  %
%%%%%%%%%%%%%

% EXAMPLE:
% 1. \correct{INCORRECT}{CORRECT}
% Parameters:
% 1. The incorrect statement.
% 2. The correct statement.
\definecolor{correct}{HTML}{009900}
\newcommand\correct[2]{{\color{red}{#1 }}\ensuremath{\to}{\color{correct}{ #2}}}


%%%%%%%%%%%%%%%%%%%%%%%%%%%%%%%%%%%%%%%%%%%%%%%%%%%%%%%%%%%%%%%%%%%%%%%%%%%%%%%
%                                 Environments                                %
%%%%%%%%%%%%%%%%%%%%%%%%%%%%%%%%%%%%%%%%%%%%%%%%%%%%%%%%%%%%%%%%%%%%%%%%%%%%%%%

\usepackage{varwidth}
\usepackage{thmtools}
\usepackage[most,many,breakable]{tcolorbox}

\tcbuselibrary{theorems,skins,hooks}
\usetikzlibrary{arrows,calc,shadows.blur}

%%%%%%%%%%%%%%%%%%%
%  Define Colors  %
%%%%%%%%%%%%%%%%%%%

\definecolor{myblue}{RGB}{45, 111, 177}
\definecolor{mygreen}{RGB}{56, 140, 70}
\definecolor{myred}{RGB}{199, 68, 64}
\definecolor{mypurple}{RGB}{197, 92, 212}

% ESSENTIALS: 
\definecolor{definition_color}{HTML}{c74540}

\definecolor{theorem_color}{HTML}{00007B}
\colorlet{proof_color}{theorem_color}
\colorlet{prop_color}{theorem_color}

\colorlet{corollary_color}{mypurple!85!black}

\definecolor{lemma_color}{HTML}{983b0f}

\definecolor{example_color}{HTML}{2A7F7F}
\colorlet{exercise_color}{example_color}

\colorlet{claim_color}{mygreen!85!black}

% MISCS: 
%%%%%%%%%%%%%%%%%%%%%%%%%%%%%%%%%%%%%%%%%%%%%%%%%%%%%%%%%
%  Create Environments Styles Based on Given Parameter  %
%%%%%%%%%%%%%%%%%%%%%%%%%%%%%%%%%%%%%%%%%%%%%%%%%%%%%%%%%

\mdfsetup{skipabove=1em,skipbelow=0em}

%%%%%%%%%%%%%%%%%%%%%%
%  Helpful Commands  %
%%%%%%%%%%%%%%%%%%%%%%

% EXAMPLE:
% 1. \createnewtheoremstyle{thmdefinitionbox}{}{}
% 2. \createnewtheoremstyle{thmtheorembox}{}{}
% 3. \createnewtheoremstyle{thmproofbox}{qed=\qedsymbol}{
%       rightline=false, topline=false, bottomline=false
%    }
% Parameters:
% 1. Theorem name.
% 2. Any extra parameters to pass directly to declaretheoremstyle.
% 3. Any extra parameters to pass directly to mdframed.
\newcommand\createnewtheoremstyle[3]{
    \declaretheoremstyle[
        headfont=\bfseries\sffamily, bodyfont=\normalfont, #2,
        mdframed={
                #3,
            },
    ]{#1}
}

% EXAMPLE:
% 1. \createnewcoloredtheoremstyle{thmdefinitionbox}{definition}{}{}
% 2. \createnewcoloredtheoremstyle{thmexamplebox}{example}{}{
%       rightline=true, leftline=true, topline=true, bottomline=true
%     }
% 3. \createnewcoloredtheoremstyle{thmproofbox}{proof}{qed=\qedsymbol}{backgroundcolor=white}
% Parameters:
% 1. Theorem name.
% 2. Color of theorem.
% 3. Any extra parameters to pass directly to declaretheoremstyle.
% 4. Any extra parameters to pass directly to mdframed.

% change backgroundcolor to #2!5 if user wants a colored backdrop to theorem environments. It's a cool color theme, but there's too much going on in the page.
\newcommand\createnewcoloredtheoremstyle[4]{
    \declaretheoremstyle[
        headfont=\bfseries\sffamily\color{#2}, bodyfont=\normalfont, #3,
        mdframed={
                linewidth=2pt,
                rightline=false, leftline=true, topline=false, bottomline=false,
                linecolor=#2, backgroundcolor=white, #4,
            },
    ]{#1}
}



%%%%%%%%%%%%%%%%%%%%%%%%%%%%%%%%%%%
%  Create the Environment Styles  %
%%%%%%%%%%%%%%%%%%%%%%%%%%%%%%%%%%%

\makeatletter
\@ifclasswith\class{nocolor}{
    % Environments without color.

    % ESSENTIALS:
    \createnewtheoremstyle{thmdefinitionbox}{}{}
    \createnewtheoremstyle{thmtheorembox}{}{}
    \createnewtheoremstyle{thmproofbox}{qed=\qedsymbol}{}
    \createnewtheoremstyle{thmcorollarybox}{}{}
    \createnewtheoremstyle{thmlemmabox}{}{}
    \createnewtheoremstyle{thmclaimbox}{}{}
    \createnewtheoremstyle{thmexamplebox}{}{}

    % MISCS: 
    \createnewtheoremstyle{thmpropbox}{}{}
    \createnewtheoremstyle{thmexercisebox}{}{}
    \createnewtheoremstyle{thmexplanationbox}{}{}
    \createnewtheoremstyle{thmremarkbox}{}{}

    % STYLIZED MORE BELOW
    \createnewtheoremstyle{thmquestionbox}{}{}
    \createnewtheoremstyle{thmsolutionbox}{qed=\qedsymbol}{}
}{
    % Environments with color.

    % ESSENTIALS: definition, theorem, proof, corollary, lemma, claim, example
    \createnewcoloredtheoremstyle{thmdefinitionbox}{definition_color}{}{
        leftline=true
    }
    \createnewcoloredtheoremstyle{thmtheorembox}{theorem_color}{}{
        leftline=true
    }
    \createnewcoloredtheoremstyle{thmproofbox}{proof_color}{qed=\qedsymbol}{backgroundcolor=white}
    \createnewcoloredtheoremstyle{thmcorollarybox}{corollary_color}{}{backgroundcolor=white}
    \createnewcoloredtheoremstyle{thmlemmabox}{lemma_color}{}{backgroundcolor=white}
    \createnewcoloredtheoremstyle{thmclaimbox}{claim_color}{}{}
    \createnewcoloredtheoremstyle{thmexamplebox}{example_color}{}{backgroundcolor=white}
    \createnewcoloredtheoremstyle{thmexplanationbox}{example_color}{qed=\qedsymbol}{backgroundcolor=white}
    \createnewcoloredtheoremstyle{thmremarkbox}{theorem_color}{}{backgroundcolor=white}

    \createnewcoloredtheoremstyle{thmmiscbox}{black}{}{leftline=false,backgroundcolor=white}


    \createnewcoloredtheoremstyle{thmpropbox}{prop_color}{}{backgroundcolor=white}
    \createnewcoloredtheoremstyle{thmexercisebox}{exercise_color}{}{backgroundcolor=white}

    \createnewcoloredtheoremstyle{thmproblembox}{myred}{}{backgroundcolor=white}
    \createnewcoloredtheoremstyle{thmsolutionbox}{mygreen}{qed=\qedsymbol}{backgroundcolor=white}
}
\makeatother

%%%%%%%%%%%%%%%%%%%%%%%%%%%%%
%  Create the Environments  %
%%%%%%%%%%%%%%%%%%%%%%%%%%%%%
\declaretheorem[numberwithin=section, style=thmdefinitionbox,     name=Definition]{definition0}
\declaretheorem[numberwithin=section, style=thmtheorembox,     name=Theorem]{theorem}
\declaretheorem[numbered=no,          style=thmexamplebox,     name=Example]{example}
\declaretheorem[numberwithin=section, style=thmclaimbox,       name=Claim]{claim}
\declaretheorem[numberwithin=section, style=thmcorollarybox,   name=Corollary]{corollary}
\declaretheorem[numberwithin=section, style=thmpropbox,        name=Proposition]{proposition}
\declaretheorem[numberwithin=section, style=thmlemmabox,       name=Lemma]{lemma}
\declaretheorem[numberwithin=section, style=thmexercisebox,    name=Exercise]{exercise}
\declaretheorem[numbered=no,          style=thmproofbox,       name=Proof]{replacementproof}
\declaretheorem[numbered=no,          style=thmexplanationbox, name=Explanation]{explanation}
\declaretheorem[numbered=no,          style=thmsolutionbox,    name=Solution]{solution}
\declaretheorem[numberwithin=section,          style=thmproblembox,     name=Problem]{problem}
\declaretheorem[numbered=no,          style=thmmiscbox,    name=Intuition]{intuition}
\declaretheorem[numbered=no,          style=thmmiscbox,    name=Goal]{goal}
\declaretheorem[numbered=no,          style=thmmiscbox,    name=Recall]{recall}
\declaretheorem[numbered=no,          style=thmmiscbox,    name=Motivation]{motivation}
\declaretheorem[numbered=no,          style=thmmiscbox,    name=Remark]{remark}
\declaretheorem[numbered=no,          style=thmmiscbox,    name=Observe]{observe}




%%%% FANCY GRAPHICS:

% \makeatletter
% \@ifclasswith\class{nocolor}{
%   % Environments without color.

%   \newtheorem*{note}{Note}

%   \declaretheorem[numberwithin=section, style=thmdefinitionbox, name=Definition]{definition}
%   \declaretheorem[numberwithin=section, style=thmquestionbox,   name=Question]{question}
%   \declaretheorem[numberwithin=section, style=thmsolutionbox,   name=Solution]{solution}
% }{
%   % Environments with color.

%   \newtcbtheorem[number within=section]{Definition}{Definition}{
%     enhanced,
%     before skip=2mm,
%     after skip=2mm,
%     colback=red!5,
%     colframe=red!80!black,
%     colbacktitle=red!75!black,
%     boxrule=0.5mm,
%     attach boxed title to top left={
%       xshift=1cm,
%       yshift*=1mm-\tcboxedtitleheight
%     },
%     varwidth boxed title*=-3cm,
%     boxed title style={
%       interior engine=empty,
%       frame code={
%         \path[fill=tcbcolback]
%         ([yshift=-1mm,xshift=-1mm]frame.north west)
%         arc[start angle=0,end angle=180,radius=1mm]
%         ([yshift=-1mm,xshift=1mm]frame.north east)
%         arc[start angle=180,end angle=0,radius=1mm];
%         \path[left color=tcbcolback!60!black,right color=tcbcolback!60!black,
%         middle color=tcbcolback!80!black]
%         ([xshift=-2mm]frame.north west) -- ([xshift=2mm]frame.north east)
%         [rounded corners=1mm]-- ([xshift=1mm,yshift=-1mm]frame.north east)
%         -- (frame.south east) -- (frame.south west)
%         -- ([xshift=-1mm,yshift=-1mm]frame.north west)
%         [sharp corners]-- cycle;
%       },
%     },
%     fonttitle=\bfseries,
%     title={#2},
%     #1
%   }{def}

%   \NewDocumentEnvironment{definition}{O{}O{}}
%     {\begin{Definition}{#1}{#2}}{\end{Definition}}

%   \newtcolorbox{note}[1][]{%
%     enhanced jigsaw,
%     colback=gray!20!white,%
%     colframe=gray!80!black,
%     size=small,
%     boxrule=1pt,
%     title=\textbf{Note:-},
%     halign title=flush center,
%     coltitle=black,
%     breakable,
%     drop shadow=black!50!white,
%     attach boxed title to top left={xshift=1cm,yshift=-\tcboxedtitleheight/2,yshifttext=-\tcboxedtitleheight/2},
%     minipage boxed title=1.5cm,
%     boxed title style={%
%       colback=white,
%       size=fbox,
%       boxrule=1pt,
%       boxsep=2pt,
%       underlay={%
%         \coordinate (dotA) at ($(interior.west) + (-0.5pt,0)$);
%         \coordinate (dotB) at ($(interior.east) + (0.5pt,0)$);
%         \begin{scope}
%           \clip (interior.north west) rectangle ([xshift=3ex]interior.east);
%           \filldraw [white, blur shadow={shadow opacity=60, shadow yshift=-.75ex}, rounded corners=2pt] (interior.north west) rectangle (interior.south east);
%         \end{scope}
%         \begin{scope}[gray!80!black]
%           \fill (dotA) circle (2pt);
%           \fill (dotB) circle (2pt);
%         \end{scope}
%       },
%     },
%     #1,
%   }

%   \newtcbtheorem{Question}{Question}{enhanced,
%     breakable,
%     colback=white,
%     colframe=myblue!80!black,
%     attach boxed title to top left={yshift*=-\tcboxedtitleheight},
%     fonttitle=\bfseries,
%     title=\textbf{Question:-},
%     boxed title size=title,
%     boxed title style={%
%       sharp corners,
%       rounded corners=northwest,
%       colback=tcbcolframe,
%       boxrule=0pt,
%     },
%     underlay boxed title={%
%       \path[fill=tcbcolframe] (title.south west)--(title.south east)
%       to[out=0, in=180] ([xshift=5mm]title.east)--
%       (title.center-|frame.east)
%       [rounded corners=\kvtcb@arc] |-
%       (frame.north) -| cycle;
%     },
%     #1
%   }{def}

%   \NewDocumentEnvironment{question}{O{}O{}}
%   {\begin{Question}{#1}{#2}}{\end{Question}}

%   \newtcolorbox{Solution}{enhanced,
%     breakable,
%     colback=white,
%     colframe=mygreen!80!black,
%     attach boxed title to top left={yshift*=-\tcboxedtitleheight},
%     title=\textbf{Solution:-},
%     boxed title size=title,
%     boxed title style={%
%       sharp corners,
%       rounded corners=northwest,
%       colback=tcbcolframe,
%       boxrule=0pt,
%     },
%     underlay boxed title={%
%       \path[fill=tcbcolframe] (title.south west)--(title.south east)
%       to[out=0, in=180] ([xshift=5mm]title.east)--
%       (title.center-|frame.east)
%       [rounded corners=\kvtcb@arc] |-
%       (frame.north) -| cycle;
%     },
%   }

%   \NewDocumentEnvironment{solution}{O{}O{}}
%   {\vspace{-10pt}\begin{Solution}{#1}{#2}}{\end{Solution}}
% }
% \makeatother


%%%%% END OF FANCY GRAPHICS %%%%%%%%%%



%%%%%%%%%%%%%%%%%%%%%%%%%%%%
%  Edit Proof Environment  %
%%%%%%%%%%%%%%%%%%%%%%%%%%%%

\renewenvironment{proof}[2][\proofname]{
    % \vspace{-12pt}
    \begin{replacementproof} [#2]
}{\end{replacementproof}}

\newenvironment{definition}[1]{
    \begin{definition0}[#1]

    \hfill
        
    \vspace{0.2cm}

}{

    \vspace{0.2cm}
    \end{definition0}
}


\theoremstyle{definition}

\newtheorem*{notation}{Notation}
\newtheorem*{previouslyseen}{As previously seen}
\newtheorem*{property}{Property}
% \newtheorem*{intuition}{Intuition}
% \newtheorem*{goal}{Goal}
% \newtheorem*{recall}{Recall}
% \newtheorem*{motivation}{Motivation}
% \newtheorem*{remark}{Remark}
% \newtheorem*{observe}{Observe}

\author{Cong Hung Le Tran}


%%%% MATH SHORTHANDS %%%%
%% blackboard bold math capitals
\newcommand{\bbf}{\mathbb{F}}
\newcommand{\bbn}{\mathbb{N}}
\newcommand{\bbq}{\mathbb{Q}}
\newcommand{\bbr}{\mathbb{R}}
\newcommand{\bbz}{\mathbb{Z}}
\newcommand{\bbc}{\mathbb{C}}
\newcommand{\bbk}{\mathbb{K}}
\newcommand{\bbm}{\mathbb{M}}
\renewcommand{\phi}{\varphi}
\newcommand{\st}{\;\text{such that}\;}


% MATH 20250 %
\newcommand{\Hom}{\mathrm{Hom}}
\newcommand{\im}{\mathrm{im}}

% https://tex.stackexchange.com/questions/438612/space-between-exists-and-forall
\let\oldforall\forall
\renewcommand{\forall}{\;\oldforall\; }
\let\oldexist\exists
\renewcommand{\exists}{\;\oldexist\; }
\newcommand\existu{\;\oldexist!\: }


\renewcommand{\_}[1]{\underline{ #1 }}
\DeclarePairedDelimiter{\abs}{\lvert}{\rvert}
\DeclarePairedDelimiter{\norm}{\lVert}{\rVert}
\setlength\parindent{0pt}
\setlength{\headheight}{12.0pt}
\addtolength{\topmargin}{-12.0pt}


% Default skipping, change if you want more spacing
% \thinmuskip=3mu
% \medmuskip=4mu plus 2mu minus 4mu
% \thickmuskip=5mu plus 5mu



% \DeclareMathOperator{\ext}{ext}
% \DeclareMathOperator{\bridge}{bridge}
\title{MATH 20700: Honors Analysis in Rn I \\ \large Problem Set 5}
\date{31 Oct 2023}
\author{Hung Le Tran}
\begin{document}
\maketitle
\setcounter{section}{5}
\textbf{Textbook: Pugh's Real Mathematical Analysis}
\begin{problem} [2.13 \redtext{done}]
Assume $f: M \to N$  is a function from one metric space to another, satisfying the following condition: If a sequence $(p_n) \subseteq M$ converges then the sequence $(f(p_n)) \subseteq N$ converges. Prove that $f$ is continuous.
\end{problem}
\begin{solution}
    Let $(p_n) \subseteq M$ and $p_n \cvgn p \in M$. Then WTS $(f(p_n)) \cvgn f(p)$. Consider $(q_n) \subseteq M$ defined as follows: \[
        q_{2i -1} = p_i, q_{2i} = p \forall i \in \bbn
    \]
    then $(q_n)$ is a well-defined sequence in $M$. Does it converge? Yes. Given $\epsilon > 0$, since $p_n \cvgn p$, there exists $N_1 \in \bbn $ such that \[
        \forall n \geq N_1, d(p_n, p) < \epsilon
    \]
    Then choose $N_2 = 2N_1$, then for all $n \geq N_2$, we have if $n$ is even then \[
        d(q_n, p) = d(p, p) = 0 < \epsilon
    \]
    and if $n$ is odd \[
        \frac{n+ 1}{2} > N_1 \implies d(q_n, p) = d(p_{\frac{n+1}{2}}, p) < \epsilon
    \]
    Therefore $q_n \cvgn p$. It follows that $f(q_n) \cvgn y \in N$. WTS $y = f(p)$.

    Suppose not, then since $f(q_n) \cvgn y$, for $\epsilon' = d(y, f(p))/2$, there exists $N_3$ such that for all $n \geq N_3$, \[
        d(f(q_n),  y) < \epsilon' / 2
    \]
    Take $N_4 = 2N_3 - 1 \geq N_3$ then \[
        d(y, f(p))/2 > d(f(q_{N_4}), y) = d(f(q_{(2N_3 - 1)}), y) = d(f(p), y) \contra
    \]
    It follows that $y = f(p)$. Therefore $f(q_n) \cvgn f(p)$.

    But then $(f(p_n))$ is a convergent subsequence of $(f(q_n))$, so it has to converge to the same limit. Thus $f(p_n) \cvgn f(p)$. $f$ is therefore continuous.
\end{solution}
\begin{problem} [2.27 \redtext{done}]
If $S, T \subseteq M$, a metric space, and $S \subseteq T$, prove that \begin{enumerate} [(a)]
    \item $\overline{S} \subseteq \overline{T}$
    \item $int(S) \subseteq int(T)$
\end{enumerate}
\end{problem}
\begin{solution}
    \textbf{(a)} Let $s \in \overline{S}$. Then there exists a sequence $(p_n) \subseteq S$ such that $p_n \cvgn s$. But $S \subseteq T \implies (p_n) \subseteq T$ too. And $p_n \cvgn s \in M$ so $s$ is a limit point of $T$. In other words, $s \in \overline{T}$. Therefore $\overline{S} \subseteq \overline{T}$. \qed

    \textbf{(b)} Recall that $int(S)$ is the largest open set in $M$ that is a subset of $S$. Let $s \in int(S)$. Then there exists $r > 0$ such that \[
        B_M(s, r) \subseteq S
    \]
    But $S \subseteq T \subseteq M$ so $B_M(s, r) \subseteq T$.

    Suppose that $s \not \in int(T)$ then we can construct \[
        I = int(T) \cup B_M(s, r)
    \]
    is a union of open sets in $M$ and is therefore open. Also, $int(T), B_M(s, r) \subseteq T \implies I \subseteq T$.

    However, $int(T) \subseteq I, s \in I, s \not \in int(T)$ so $int(T)$ is not the largest open set in $M$ that is a subset of $T$. \contra

    It follows that $s \in int(T)$. Therefore $int(S) \subseteq int(T)$.
\end{solution}
\begin{problem} [2.41 \redtext{done}]
Let $\norm{\cdot}$ be any norm on $\bbr^m$ and let $B = \{x \in \bbr^m : \norm{x} \leq 1\}$. Prove that $B$ is compact. [Hint: It suffices to show that $B$ is closed and bounded with respect to the Euclidean matrix.]
\end{problem}
\begin{solution}
    Notation: Let $|\cdot|$ be the standard Euclidean norm.

    $\norm{\cdot}$ induces a metric $d$ on $\bbr^m$. Consider the identity map between 2 metric spaces \[
        id: (\bbr^m, d_E) \to (\bbr^m, d), x \mapsto x.
    \]
    It is clearly a bijection.

    % We want to show that $id$ is a homeomorphism.
    Denote $e_i \in \bbr^m, e_i = (0, \dots, 1, \dots, 0)$ with its $i^{th}$ entry as 1, and the rest as 0. Define \[
        L \coloneqq \max_{1 \leq i \leq m}{\norm{e_i}} < \infty
    \]

    Note that $B = id(B) = id^{-1}(B)$.

    \textbf{1.} Claim that $B$ is closed in $(\bbr^m, d_E)$.

    For any $a, b \in (\bbr^m, d)$, we have:
    \begin{align*}
        d(a, b) & = \norm{a - b}                                 \\
                & = \norm{(a_1, \dots, a_m) - (b_1, \dots, b_m)} \\
                & = \norm*{\sum_{i=1}^m (a_i-b_i) e_i}           \\
                & \leq \sum_{i=1}^m |a_i-b_i|\norm{e_i}          \\
                & \leq L\sum_{i=1}^m |a_i-b_i|                   \\
                & \leq Lm \sqrt{\sum_{i=1}^m (a_i-b_i)^2}        \\
                & \leq Lm d_E(a, b)
    \end{align*}
    so $id$ is $Lm-$Lipschitz. $id$ is therefore continuous. $B$ is the closed unit ball in $(\bbr^m, d)$, so its preimage, $B$ itself, is closed in $(\bbr^m, d_E)$.

    \textbf{2.} Claim that $B$ is bounded in $(\bbr^m, d_E)$.

    \textbf{2.1.}
    Take $S = S^{m-1}(\bbr^m, d_E) \subset \bbr^m$. It is compact. Since $id$ is continuous, $S = id(S)$ is compact in $(\bbr^m, d)$. Clearly, $0 \not \in S$.

    Claim that there exists $c > 0$ such that $d(u, 0) \geq c \forall u \in S$.

    Suppose not. Then for all $n \in \bbn$, there exists $u_n \in S$ such that \[
        d(u_n, 0) < \frac{1}{n}
    \]
    $(u_n)$ is a sequence in $(\bbr^m, d)$. Trivially, $u_n \cvgn 0$ in $(\bbr^m, d)$. However, since $S$ is compact in $(\bbr^m, d)$, there exists a subsequence $(u_{n_j})$ that converges in $S$. But since $u_n \cvgn 0$,  $u_{n_j} \cvgj 0$ too. But $0 \not \in S$. \contra

    It follows that there does exist such a $c > 0$. Which means \[
        \forall u \in S, \norm{u} = d(u, 0) \geq c
    \]

    \textbf{2.2.} For any $v \in B \subseteq \bbr^m$, let $w = \frac{1}{|v|}v$. Then $|w| = \frac{|v|}{|v|} = 1 \implies w \in S$. Then
    \begin{align*}
        |v| & = \frac{\norm{v}}{\norm{\frac{1}{|v|}v}}                            \\
            & = \frac{\norm{v}}{\norm{w}} \leq \frac{1}{c} \norm{v} = \frac{1}{c}
    \end{align*}

    Therefore $B$ is bounded in in $(\bbr^m, d_E)$.

    \textbf{3.} Since $B$ is closed and bounded in  $(\bbr^m, d_E)$, it follows that $B$ is compact in $(\bbr^m, d_E)$ (H-B). $id$ is continuous, so $B = id(B)$ is compact in $(\bbr^m, d)$.
\end{solution}

\begin{problem} [2.96 \redtext{done}]
If $A \subseteq B \subseteq C$, $A$ is dense in $B$, $B$ is dense in $C$, prove that $A$ is dense in $C$.
\end{problem}
\begin{solution}
    Let $c \in C$ and $\epsilon > 0$.

    Since $B$ is dense in $C$, there exists $b \in B$ such that $d(b, c) < \epsilon/2$.

    Since $A$ is dense in $B$, there exists $a \in A$ such that $d(a, b) < \epsilon / 2$. Thus, we can always pick $a \in A$ satisfying:
    \[
        d(a, c) \leq d(a, b) + d(b, c) < \epsilon
    \]
    Thus $A$ is dense in $C$.
\end{solution}
\begin{problem} [3.37 \redtext{done}]
Suppose that $f: \bbr \to [-M, M]$ has no jump discontinuities. Does $f$ have the intermediate value property? (Proof or counterexample)
\end{problem}
\begin{solution}
    No. Define \[
        f(x) = \begin{cases}
            \sin(1/x) & \:\text{for}\: x > 0 \\
            0         & \:\text{for}\: x = 0 \\
            2         & \:\text{for}\: x < 0
        \end{cases}
    \]
    Then \begin{enumerate}
        \item $|\sin(1/x)|, |0|, |2| \leq 2 \implies |f| \leq 2$.
        \item $f = \sin(1/x)$ is continuous on $ x > 0$ and $f = 2$ is continuous on $x < 0$.
        \item Claim that $f$ is discontinuous at 0, but it is not a jump discontinuity. We want to show that $\lim_{x \to 0+} f(x)$ doesn't exist.

              Suppose that it does, that $\lim_{x \to 0+} f(x) = L$. Choose $\epsilon = 1$, then there exists $\delta > 0$ such that $x \in (0, \delta) \implies |f(x) - L| < \epsilon$.

              Choose $N \in \bbn$ sufficiently large such that $\frac{1}{N} < \delta$, then we can construct $x_1, x_2 \in (0, \delta)$ \[
                  x_1 = \frac{1}{2N\pi + \frac{\pi}{2}}, x_2 = \frac{1}{2N\pi - \frac{\pi}{2}}
              \]
              which yields \[
                  f(x_1) = \sin(1/x_1) = 1, f(x_2) =  \sin(1/x_2) = -1
              \]
              But \[
                  2 = |f(x_1) - f(x_2)| \leq |f(x_1) - L| + |f(x_2) - L| < 2\epsilon = 2 \contra
              \]
              Therefore $\lim_{x \to 0+}f(x)$ doesn't exist.

              The non-existence of the right limit of $f$ at 0 implies that it is a nonjump discontinuous. (A jump discontinuity requires both right and left limits to exist).
        \item $f$ does not have the intermediate value property. There doesn't exist $x_0 \in \bbr$ such that $f(x_0) = 1.5$.
    \end{enumerate}
\end{solution}
\begin{problem} [4.34a \redtext{done}]
Consider the ODE $y' = 2\sqrt{|y|}$ where $y \in \bbr$. Show that there are many solutions to this ODE, all with the same initial condition $y(0) = 0$. Not only does $y(t) = 0$ solve the ODE, but also $y(t) = t^2$ does for $t \geq 0$.
\end{problem}
\begin{solution}
    WTS that every member of the family of functions \[
        \calf \coloneqq \left\{y_{a, b}: (-1, 1) \to \bbr,  y_{a, b}(x) = \begin{cases}
            -(t-a)^2 & \:\text{for}\: t \in (-1, a) \\
            0        & \:\text{for}\: t \in [a, b]  \\
            (t-b)^2  & \:\text{for}\: t \in (b, 1)
        \end{cases} \Bigg| a \in (-1, 0), b \in (0, 1)\right\}
    \]
    is a solution to the ODE $y' = 2\sqrt{|y|}$. Take any $y = y_{a, b}$. Then
    \begin{enumerate}
        \item $y(0) = 0$ by definition.
        \item On $(-1, a), y' = 2(a-t) = 2\sqrt{(t-a)^2} = 2\sqrt{|y|}$
        \item On $(b, 1), y' = 2(t-b) = 2\sqrt{(t-b)^2} = 2\sqrt{|y|}$
        \item On $(a, b), y' = 0 = 2\sqrt{|y|}$
        \item We have\[
                  \lim_{t \to a-}\frac{y(t) - y(a)}{t-a} = \lim_{t \to a-} \frac{-(t-a)^2}{t-a} = \lim_{t \to a-} (a-t) = 0
              \]
              \[
                  \lim_{t \to a+}\frac{y(t) - y(a)}{t-a} = 0
              \]
              It follows that $y'(a) = 0 = \sqrt{|y(a)|}$.
        \item  We have\[
                  \lim_{t \to b+}\frac{y(t) - y(b)}{t-b} = \lim_{t \to b+} \frac{(t-b)^2}{t-b} = \lim_{t \to b+} (t-b) = 0
              \]
              \[
                  \lim_{t \to b-}\frac{y(t) - y(b)}{t-b} = 0
              \]
              It follows that $y'(b) = 0 = \sqrt{|y(b)|}$.
    \end{enumerate}
    Therefore $y = y_{a, b}$ solves the ODE with initial condition $y(0) = 0$.
\end{solution}
\begin{problem} [5.1 \redtext{done}]
Let $T: V \to W$ be a linear transformation, and let $p \in V$ be given. Prove that the following are equivalent: \begin{enumerate} [(a)]
    \item $T$ is continuous at the origin.
    \item $T$ is continuous at $p$.
    \item $T$ is continuous at at least one point of $V$.
\end{enumerate}
\end{problem}

\begin{solution}
    WTS (c) implies (a). Then suppose $T$ is continuous at some $q \in V$. Then for all $\epsilon > 0$, there exists $\delta > 0$ such that \[
        |u - q| < \delta \implies |Tu - Tq| < \epsilon
    \]
    Use the same $\delta$. Then \[
        |v - 0| < \delta \implies |(q+v) - q| < \delta \implies |T(q+ v) - Tq| < \epsilon \implies |T(v)| < \epsilon
    \]
    And $T(0) = 0$ so that implies $|T(v) - T(0)| < \epsilon$.

    $T$ is therefore continuous at the origin. Therefore (c) implies (a).

    (b) implies (c) and (a) implies (c), since $0, p \in V$. By Theorem 2, (a) implies that $f$ is continuous everywhere, which implies (b) and (c).
\end{solution}
\begin{problem} [5.4 \redtext{done}]
The \textbf{conorm} of a linear transformation $T: \bbr^n \to \bbr^m$ is \[
    \mathfrak{m}(T) = \inf \left\{\frac{|Tv|}{|v|} : v \neq 0\right\}
\]
It is the \textbf{minimum stretch} that $T$ imparts to vectors in $\bbr^n$. Let $U$ be the unit ball in $\bbr^n$.\begin{enumerate} [(a)]
    \item Show that the norm and conorm of $T$ are the radii of the smallest ball that contains TU and, when $n = m$, the largest ball contained in $TU$.
    \item If $T$ is an isomorphism, prove that $\mathfrak{m}(T) = \norm{T^{-1}}^{-1}$.
    \item If $m = n, T = I + S$, and $\norm{S} < 1$, prove that $\mathfrak{m}(T) > 0$. [Hint: The inequality $\abs{u + v} \geq \abs{u} - \abs{v}$ is useful because it implies $|Tu| \geq |u| - |Su|$.] How can you infer that $T$ is an isomorphism?
    \item If the norm and conorm of $T$ are equal, what can you say about $T$?
\end{enumerate}
\end{problem}
\begin{solution}
    Reiterate that $U$ is the closed unit ball (Chapter 1):
    \[
        U = \{v : |v| \leq 1\}
    \]
    Observe that for a fixed $k > 0$, for all $v \in \bbr^n, v \neq 0$, there exists $u \in \bbr^n$ such that $|u| = k$ and \[
        \frac{|Tv|}{|v|} = \frac{|Tu|}{|u|}
    \]
    with \[
        u \coloneqq k\frac{v}{|v|} (\implies |u| = k|v|/|v| = k)
    \]
    It follows that \[
        \left\{\frac{|Tv|}{|v|} : v \neq 0\right\} = \left\{\frac{|Tv|}{|v|} : v \in U\right\}= \left\{\frac{|Tv|}{|v|} : |v| = 1\right\} = \left\{|Tv| : |v| = 1\right\}
    \]
    Intuitively, one can always project any $v \neq 0$ or $v \in U$ onto the unit sphere (and vice versa). $T$ being a linear transformation keeps the quotient $\frac{|Tv|}{|v|}$ invariant.

    Then, \[
        \norm{T} = \sup \left\{\frac{|Tv|}{|v|} : v \neq 0 \right\} = \sup_{|u| = 1}\{|Tu|\}
    \]
    Similarly,
    \[
        \mathfrak{m}(T) = \inf_{|u| = 1}\{|Tu|\}
    \]


    \textbf{(a)}
    Take $u \in \bbr^n$ with $|u| = 1$. Then $u, -u \in U$. Therefore any ball that contains $TU$ has to have diameter:
    \[
        diam \geq |Tu - (T(-u))| = 2|Tu|
    \]
    Therefore the radius $R$ of the smallest ball that contains $TU$ has to satisfy:
    \[
        R \geq \frac{1}{2}\sup_{|u| = 1}\{2 |Tu|\} = \sup_{|u| = 1}\{|Tu|\} = \norm{T}
    \]
    Similarly, any ball that is contained in $TU$ has to have diameter: \[
        diam \leq 2|Tu|
    \]
    so the radius $r$ of the largest ball that is contained in $TU$ has to satisfy: \[
        r \leq \frac{1}{2}\inf_{|u| = 1}\{2 |Tu|\} = \inf_{|u| = 1}\{|Tu|\} = \mathfrak{m}(T)
    \]
    We show that equality can be achieved with closed ball $B_1$ of radius $\norm{T}$ and closed ball $B_2$ of radius $\mathfrak{m}(T)$, centered at the origin, i.e., $B_1$ is sufficient to contain $TU$, and $B_2$ is sufficient to be contained in $TU$.

    \textbf{1.} If $\norm{T} = \infty \implies TU \subseteq \bbr^m = B_1$ and we're done.

    If $\norm{T} < \infty$, WTS \[
        \sup_{|u| = 1}\{|Tu|\} \geq \sup_{v \in U} \{|Tv|\}
    \]
    For all $v \in U$, there exists $u = v/|v|$, which satisfies $|u| = 1$. Then \[
        |Tu| = |Tv||u|/|v| = |Tv|/|v| \geq |Tv|
    \]
    It follows that  \[
        \norm{T} = \sup_{|u| = 1}\{|Tu|\} \geq \sup_{v \in U} \{|Tv|\}
    \]
    so $B_1$ is sufficient to contain $TU$. \qed

    \textbf{2.} If $m = n$, then $T: \bbr^n \to \bbr^n$.
    
    \textbf{Case 1:} $\ker(T) \neq \{0\}$, i.e., $T$ has a non-trivial kernel. That means there exists $w \in \bbr^n, w \neq 0$ such that $Tw = 0 \implies |Tw| = 0$. Then there exists $u = w/|w|$ with $|u| = 1 \implies |Tu| = 0$ too. Thus \[
    \mathfrak{m}(T) = \inf_{|u| = 1} \{|Tu|\} = 0
    \]
    Ball $B_2$ with radius 0 is trivially contained in $TU$.

    \textbf{Case 2: }$\ker(T) = \{0\}$, i.e., $T$ has a trivial kernel. Then since $T: \bbr^n \to \bbr^n$, it is an isomorphism. Therefore, if $w$ satisfies $|Tw| \leq \mathfrak{m}(T)$ (i.e., $w \in B_2$), then \[
    |Tw| \leq \inf_{v \neq 0} \left\{\frac{|Tv|}{|v|}\right\} \leq \frac{|Tw|}{|w|} \implies |w| \leq 1
    \]
    therefore $w \in U \implies Tw \in TU \implies B_2 \subseteq TU$. \qed

    \textbf{(b)} $T$ is an isomorphism. We first claim for a set $A = \{x : x > 0\}$ that \[
    \inf A = 1/\sup \{1/x: x \in A\} \eqqcolon 1/S
    \]
    First, $S \geq 1/x \forall x \in A \implies 1/S \leq x \forall x \in A$ so $1/S$ is a lower bound of $A$. Suppose $\inf A < 1/S$ then there exists $y \in A: 0 \leq \inf A < y < 1/S$. Then \[
    y < 1/S \implies 1/y > S \geq 1/y \contra
    \]
    and we are done with our claim. Since $T$ is an isomorphism, $\ker(T) = \{0\} \implies |Tv|/|v| > 0 \forall v \neq 0$.

    Then, \begin{align*}
    \mathfrak{m}(T) &= \inf_{v \neq 0}\{|Tv|/|v|\} \\
    &= \frac{1}{\sup_{v \neq 0}\{|v|/|Tv|\}} \\
    &= \frac{1}{\sup_{w \neq 0}\{|T^{-1}w|/|w|\}}\\
    &= \norm{T^{-1}}^{-1} \qed
    \end{align*}

    \textbf{(c)} Since $T = I + S$, we have \[
    Tu = Iu + Su = u + Su \implies |u| = |Tu - Su| \leq |Tu| + |Su| \implies |Tu| \geq |u| - |Su|
    \]
    Therefore, \[
    \mathfrak{m}(T) = \inf_{|u| = 1}\{|Tu|\} \geq \inf_{|u| = 1}\{|u| - |Su|\} = 1 - \mathfrak{m}(S) \geq 1 - \norm{S} > 0
    \]
    as required.

    Then if $\ker(T) \neq \{0\} \implies \exists v \neq 0: Tv = 0 \implies \mathfrak{m}(T) = 0$, a contradiction. So $\ker(T) = \{0\}$.
    
    $m = n \implies T$ is an isomorphism.

    \textbf{(d)} $\norm{T} = \mathfrak{m}(T) \implies \frac{|Tv|}{|v|} = c \in \bbr \forall v \neq 0$, i.e. \[
    |Tv| = c|v|
    \]
    $T$ scales norm of all vectors with constant $c$.
\end{solution}
\end{document}