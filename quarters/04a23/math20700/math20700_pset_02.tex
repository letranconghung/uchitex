\documentclass[a4paper, 12pt]{article}
%%%%%%%%%%%%%%%%%%%%%%%%%%%%%%%%%%%%%%%%%%%%%%%%%%%%%%%%%%%%%%%%%%%%%%%%%%%%%%%
%                                Basic Packages                               %
%%%%%%%%%%%%%%%%%%%%%%%%%%%%%%%%%%%%%%%%%%%%%%%%%%%%%%%%%%%%%%%%%%%%%%%%%%%%%%%

% Gives us multiple colors.
\usepackage[usenames,dvipsnames,pdftex]{xcolor}
% Lets us style link colors.
\usepackage{hyperref}
% Lets us import images and graphics.
\usepackage{graphicx}
% Lets us use figures in floating environments.
\usepackage{float}
% Lets us create multiple columns.
\usepackage{multicol}
% Gives us better math syntax.
\usepackage{amsmath,amsfonts,mathtools,amsthm,amssymb}
% Lets us strikethrough text.
\usepackage{cancel}
% Lets us edit the caption of a figure.
\usepackage{caption}
% Lets us import pdf directly in our tex code.
\usepackage{pdfpages}
% Lets us do algorithm stuff.
\usepackage[ruled,vlined,linesnumbered]{algorithm2e}
% Use a smiley face for our qed symbol.
\usepackage{tikzsymbols}
% \usepackage{fullpage} %%smaller margins
\usepackage[shortlabels]{enumitem}

\usepackage{setspace}
\usepackage[margin=1in, headsep=12pt]{geometry}
\usepackage{wrapfig}
\def\class{article}


%%%%%%%%%%%%%%%%%%%%%%%%%%%%%%%%%%%%%%%%%%%%%%%%%%%%%%%%%%%%%%%%%%%%%%%%%%%%%%%
%                                Basic Settings                               %
%%%%%%%%%%%%%%%%%%%%%%%%%%%%%%%%%%%%%%%%%%%%%%%%%%%%%%%%%%%%%%%%%%%%%%%%%%%%%%%

%%%%%%%%%%%%%
%  Symbols  %
%%%%%%%%%%%%%

\let\implies\Rightarrow
\let\impliedby\Leftarrow
\let\iff\Leftrightarrow
\let\epsilon\varepsilon
%%%%%%%%%%%%
%  Tables  %
%%%%%%%%%%%%

\setlength{\tabcolsep}{5pt}
\renewcommand\arraystretch{1.5}

%%%%%%%%%%%%%%
%  SI Unitx  %
%%%%%%%%%%%%%%

\usepackage{siunitx}
\sisetup{locale = FR}

%%%%%%%%%%
%  TikZ  %
%%%%%%%%%%

\usepackage[framemethod=TikZ]{mdframed}
\usepackage{tikz}
\usepackage{tikz-cd}
\usepackage{tikzsymbols}

\usetikzlibrary{intersections, angles, quotes, calc, positioning}
\usetikzlibrary{arrows.meta}

\tikzset{
    force/.style={thick, {Circle[length=2pt]}-stealth, shorten <=-1pt}
}

%%%%%%%%%%%%%%%
%  PGF Plots  %
%%%%%%%%%%%%%%%

\usepackage{pgfplots}
\pgfplotsset{width=10cm, compat=newest}

%%%%%%%%%%%%%%%%%%%%%%%
%  Center Title Page  %
%%%%%%%%%%%%%%%%%%%%%%%

\usepackage{titling}
\renewcommand\maketitlehooka{\null\mbox{}\vfill}
\renewcommand\maketitlehookd{\vfill\null}

%%%%%%%%%%%%%%%%%%%%%%%%%%%%%%%%%%%%%%%%%%%%%%%%%%%%%%%
%  Create a grey background in the middle of the PDF  %
%%%%%%%%%%%%%%%%%%%%%%%%%%%%%%%%%%%%%%%%%%%%%%%%%%%%%%%

\usepackage{eso-pic}
\newcommand\definegraybackground{
    \definecolor{reallylightgray}{HTML}{FAFAFA}
    \AddToShipoutPicture{
        \ifthenelse{\isodd{\thepage}}{
            \AtPageLowerLeft{
                \put(\LenToUnit{\dimexpr\paperwidth-222pt},0){
                    \color{reallylightgray}\rule{222pt}{297mm}
                }
            }
        }
        {
            \AtPageLowerLeft{
                \color{reallylightgray}\rule{222pt}{297mm}
            }
        }
    }
}

%%%%%%%%%%%%%%%%%%%%%%%%
%  Modify Links Color  %
%%%%%%%%%%%%%%%%%%%%%%%%

\hypersetup{
    % Enable highlighting links.
    colorlinks,
    % Change the color of links to blue.
    urlcolor=blue,
    % Change the color of citations to black.
    citecolor={black},
    % Change the color of url's to blue with some black.
    linkcolor={blue!80!black}
}

%%%%%%%%%%%%%%%%%%
% Fix WrapFigure %
%%%%%%%%%%%%%%%%%%

\newcommand{\wrapfill}{\par\ifnum\value{WF@wrappedlines}>0
        \parskip=0pt
        \addtocounter{WF@wrappedlines}{-1}%
        \null\vspace{\arabic{WF@wrappedlines}\baselineskip}%
        \WFclear
    \fi}

%%%%%%%%%%%%%%%%%
% Multi Columns %
%%%%%%%%%%%%%%%%%

\let\multicolmulticols\multicols
\let\endmulticolmulticols\endmulticols

\RenewDocumentEnvironment{multicols}{mO{}}
{%
    \ifnum#1=1
        #2%
    \else % More than 1 column
        \multicolmulticols{#1}[#2]
    \fi
}
{%
    \ifnum#1=1
    \else % More than 1 column
        \endmulticolmulticols
    \fi
}

\newlength{\thickarrayrulewidth}
\setlength{\thickarrayrulewidth}{5\arrayrulewidth}


%%%%%%%%%%%%%%%%%%%%%%%%%%%%%%%%%%%%%%%%%%%%%%%%%%%%%%%%%%%%%%%%%%%%%%%%%%%%%%%
%                           School Specific Commands                          %
%%%%%%%%%%%%%%%%%%%%%%%%%%%%%%%%%%%%%%%%%%%%%%%%%%%%%%%%%%%%%%%%%%%%%%%%%%%%%%%

%%%%%%%%%%%%%%%%%%%%%%%%%%%
%  Initiate New Counters  %
%%%%%%%%%%%%%%%%%%%%%%%%%%%

\newcounter{lecturecounter}

%%%%%%%%%%%%%%%%%%%%%%%%%%
%  Helpful New Commands  %
%%%%%%%%%%%%%%%%%%%%%%%%%%

\makeatletter

\newcommand\resetcounters{
    % Reset the counters for subsection, subsubsection and the definition
    % all the custom environments.
    \setcounter{subsection}{0}
    \setcounter{subsubsection}{0}
    \setcounter{paragraph}{0}
    \setcounter{subparagraph}{0}
    \setcounter{theorem}{0}
    \setcounter{claim}{0}
    \setcounter{corollary}{0}
    \setcounter{lemma}{0}
    \setcounter{exercise}{0}

    \@ifclasswith\class{nocolor}{
        \setcounter{definition}{0}
    }{}
}

%%%%%%%%%%%%%%%%%%%%%
%  Lecture Command  %
%%%%%%%%%%%%%%%%%%%%%

\usepackage{xifthen}

% EXAMPLE:
% 1. \lecture{Oct 17 2022 Mon (08:46:48)}{Lecture Title}
% 2. \lecture[4]{Oct 17 2022 Mon (08:46:48)}{Lecture Title}
% 3. \lecture{Oct 17 2022 Mon (08:46:48)}{}
% 4. \lecture[4]{Oct 17 2022 Mon (08:46:48)}{}
% Parameters:
% 1. (Optional) lecture number.
% 2. Time and date of lecture.
% 3. Lecture Title.
\def\@lecture{}
\def\@lectitle{}
\def\@leccount{}
\newcommand\lecture[3]{
    %   % Add 1 to the lecture counter.
    %   \addtocounter{lecturecounter}{1}

    % Set the section number to the lecture counter.
    \setcounter{section}{#1}
    \renewcommand\thesubsection{#1.\arabic{subsection}}

    % Reset the counters.
    \resetcounters

    % Check if user passed the lecture title or not.
    \def\@leccount{Lecture #1}
    \ifthenelse{\isempty{#3}}{
        \def\@lecture{Lecture #1}
    }{
        \def\@lecture{Lecture #1: #3}
        \def\@lectitle{#3}
    }

    % Display the information like the following:
    %                                                  Oct 17 2022 Mon (08:49:10)
    % ---------------------------------------------------------------------------
    % Lecture 1: Lecture Title
    \newpage
    \begin{mdframed}
        % \section*{\@lecture}
        \begin{center}
            \Large \textbf{\@leccount}

            \vspace*{0.2cm}

            \large\@lectitle

            \vspace*{0.2cm}
            \normalsize #2
        \end{center}
    \end{mdframed}
    \addcontentsline{toc}{section}{\@lecture}
}

%%%%%%%%%%%%%%%%%%%%
%  Import Figures  %
%%%%%%%%%%%%%%%%%%%%

\usepackage{import}
\pdfminorversion=7

% EXAMPLE:
% 1. \incfig{limit-graph}
% 2. \incfig[0.4]{limit-graph}
% Parameters:
% 1. The figure name. It should be located in figures/NAME.tex_pdf.
% 2. (Optional) The width of the figure. Example: 0.5, 0.35.
\newcommand\incfig[2][1]{%
    \def\svgwidth{#1\columnwidth}
    \import{./figures/}{#2.pdf_tex}
}

\begingroup\expandafter\expandafter\expandafter\endgroup
\expandafter\ifx\csname pdfsuppresswarningpagegroup\endcsname\relax
\else
    \pdfsuppresswarningpagegroup=1\relax
\fi

%%%%%%%%%%%%%%%%%
% Fancy Headers %
%%%%%%%%%%%%%%%%%

\usepackage{fancyhdr}

% Force a new page.
\newcommand\forcenewpage{\clearpage\mbox{~}\clearpage\newpage}

% This command makes it easier to manage my headers and footers.
\newcommand\createintro{
    % Use roman page numbers (e.g. i, v, vi, x, ...)
    \pagenumbering{roman}

    % Display the page style.
    \maketitle
    % Make the title pagestyle empty, meaning no fancy headers and footers.
    \thispagestyle{empty}
    % Create a newpage.
    \newpage

    % Input the intro.tex page if it exists.
    \IfFileExists{intro.tex}{ % If the intro.tex file exists.
        % Input the intro.tex file.
        \textbf{Course}: COURSE

\textbf{Section}: SECTION

\textbf{Professor}: PROFESSOR

\textbf{At}: The University of Chicago

\textbf{Quarter}: QUARTER

\textbf{Course materials}: COURSE_MATERIALS

\vspace{1cm}
\textbf{Disclaimer}: This document will inevitably contain some mistakes, both simple typos and serious logical and mathematical errors. Take what you read with a grain of salt as it is made by an undergraduate student going through the learning process himself. If you do find any error, I would really appreciate it if you can let me know by email at \href{mailto:conghungletran@gmail.com}{conghungletran@gmail.com}.

        % Make the pagestyle fancy for the intro.tex page.
        \pagestyle{fancy}

        % Remove the line for the header.
        \renewcommand\headrulewidth{0pt}

        % Remove all header stuff.
        \fancyhead{}

        % Add stuff for the footer in the center.
        % \fancyfoot[C]{
        %   \textit{For more notes like this, visit
        %   \href{\linktootherpages}{\shortlinkname}}. \\
        %   \vspace{0.1cm}
        %   \hrule
        %   \vspace{0.1cm}
        %   \@author, \\
        %   \term: \academicyear, \\
        %   Last Update: \@date, \\
        %   \faculty
        % }
    }{ % If the intro.tex file doesn't exist.
        % Force a \newpageage.
        \forcenewpage
    }

    % Create a new page.
    \newpage

    % Remove the center stuff we did above, and replace it with just the page
    % number, which is still in roman numerals.
    \fancyfoot[C]{\thepage}
    % Add the table of contents.
    \tableofcontents
    % Force a new page.
    \forcenewpage

    % Move the page numberings back to arabic, from roman numerals.
    \pagenumbering{arabic}
    % Set the page number to 1.
    \setcounter{page}{1}

    % Add the header line back.
    \renewcommand\headrulewidth{0.4pt}
    % In the top right, add the lecture title.
    \fancyhead[R]{\footnotesize \@lecture}
    % In the top left, add the author name.
    \fancyhead[L]{\footnotesize \@author}
    % In the bottom center, add the page.
    \fancyfoot[C]{\thepage}
    % Add a nice gray background in the middle of all the upcoming pages.
    % \definegraybackground
}

\makeatother


%%%%%%%%%%%%%%%%%%%%%%%%%%%%%%%%%%%%%%%%%%%%%%%%%%%%%%%%%%%%%%%%%%%%%%%%%%%%%%%
%                               Custom Commands                               %
%%%%%%%%%%%%%%%%%%%%%%%%%%%%%%%%%%%%%%%%%%%%%%%%%%%%%%%%%%%%%%%%%%%%%%%%%%%%%%%

%%%%%%%%%%%%
%  Circle  %
%%%%%%%%%%%%

\newcommand*\circled[1]{\tikz[baseline=(char.base)]{
        \node[shape=circle,draw,inner sep=1pt] (char) {#1};}
}

%%%%%%%%%%%%%%%%%%%
%  Todo Commands  %
%%%%%%%%%%%%%%%%%%%

% \usepackage{xargs}
% \usepackage[colorinlistoftodos]{todonotes}

% \makeatletter

% \@ifclasswith\class{working}{
%     \newcommandx\unsure[2][1=]{\todo[linecolor=red,backgroundcolor=red!25,bordercolor=red,#1]{#2}}
%     \newcommandx\change[2][1=]{\todo[linecolor=blue,backgroundcolor=blue!25,bordercolor=blue,#1]{#2}}
%     \newcommandx\info[2][1=]{\todo[linecolor=OliveGreen,backgroundcolor=OliveGreen!25,bordercolor=OliveGreen,#1]{#2}}
%     \newcommandx\improvement[2][1=]{\todo[linecolor=Plum,backgroundcolor=Plum!25,bordercolor=Plum,#1]{#2}}

%     \newcommand\listnotes{
%         \newpage
%         \listoftodos[Notes]
%     }
% }{
%     \newcommandx\unsure[2][1=]{}
%     \newcommandx\change[2][1=]{}
%     \newcommandx\info[2][1=]{}
%     \newcommandx\improvement[2][1=]{}

%     \newcommand\listnotes{}
% }

% \makeatother

%%%%%%%%%%%%%
%  Correct  %
%%%%%%%%%%%%%

% EXAMPLE:
% 1. \correct{INCORRECT}{CORRECT}
% Parameters:
% 1. The incorrect statement.
% 2. The correct statement.
\definecolor{correct}{HTML}{009900}
\newcommand\correct[2]{{\color{red}{#1 }}\ensuremath{\to}{\color{correct}{ #2}}}


%%%%%%%%%%%%%%%%%%%%%%%%%%%%%%%%%%%%%%%%%%%%%%%%%%%%%%%%%%%%%%%%%%%%%%%%%%%%%%%
%                                 Environments                                %
%%%%%%%%%%%%%%%%%%%%%%%%%%%%%%%%%%%%%%%%%%%%%%%%%%%%%%%%%%%%%%%%%%%%%%%%%%%%%%%

\usepackage{varwidth}
\usepackage{thmtools}
\usepackage[most,many,breakable]{tcolorbox}

\tcbuselibrary{theorems,skins,hooks}
\usetikzlibrary{arrows,calc,shadows.blur}

%%%%%%%%%%%%%%%%%%%
%  Define Colors  %
%%%%%%%%%%%%%%%%%%%

\definecolor{myblue}{RGB}{45, 111, 177}
\definecolor{mygreen}{RGB}{56, 140, 70}
\definecolor{myred}{RGB}{199, 68, 64}
\definecolor{mypurple}{RGB}{197, 92, 212}

% ESSENTIALS: 
\definecolor{definition_color}{HTML}{c74540}

\definecolor{theorem_color}{HTML}{00007B}
\colorlet{proof_color}{theorem_color}
\colorlet{prop_color}{theorem_color}

\colorlet{corollary_color}{mypurple!85!black}

\definecolor{lemma_color}{HTML}{983b0f}

\definecolor{example_color}{HTML}{2A7F7F}
\colorlet{exercise_color}{example_color}

\colorlet{claim_color}{mygreen!85!black}

% MISCS: 
%%%%%%%%%%%%%%%%%%%%%%%%%%%%%%%%%%%%%%%%%%%%%%%%%%%%%%%%%
%  Create Environments Styles Based on Given Parameter  %
%%%%%%%%%%%%%%%%%%%%%%%%%%%%%%%%%%%%%%%%%%%%%%%%%%%%%%%%%

\mdfsetup{skipabove=1em,skipbelow=0em}

%%%%%%%%%%%%%%%%%%%%%%
%  Helpful Commands  %
%%%%%%%%%%%%%%%%%%%%%%

% EXAMPLE:
% 1. \createnewtheoremstyle{thmdefinitionbox}{}{}
% 2. \createnewtheoremstyle{thmtheorembox}{}{}
% 3. \createnewtheoremstyle{thmproofbox}{qed=\qedsymbol}{
%       rightline=false, topline=false, bottomline=false
%    }
% Parameters:
% 1. Theorem name.
% 2. Any extra parameters to pass directly to declaretheoremstyle.
% 3. Any extra parameters to pass directly to mdframed.
\newcommand\createnewtheoremstyle[3]{
    \declaretheoremstyle[
        headfont=\bfseries\sffamily, bodyfont=\normalfont, #2,
        mdframed={
                #3,
            },
    ]{#1}
}

% EXAMPLE:
% 1. \createnewcoloredtheoremstyle{thmdefinitionbox}{definition}{}{}
% 2. \createnewcoloredtheoremstyle{thmexamplebox}{example}{}{
%       rightline=true, leftline=true, topline=true, bottomline=true
%     }
% 3. \createnewcoloredtheoremstyle{thmproofbox}{proof}{qed=\qedsymbol}{backgroundcolor=white}
% Parameters:
% 1. Theorem name.
% 2. Color of theorem.
% 3. Any extra parameters to pass directly to declaretheoremstyle.
% 4. Any extra parameters to pass directly to mdframed.

% change backgroundcolor to #2!5 if user wants a colored backdrop to theorem environments. It's a cool color theme, but there's too much going on in the page.
\newcommand\createnewcoloredtheoremstyle[4]{
    \declaretheoremstyle[
        headfont=\bfseries\sffamily\color{#2}, bodyfont=\normalfont, #3,
        mdframed={
                linewidth=2pt,
                rightline=false, leftline=true, topline=false, bottomline=false,
                linecolor=#2, backgroundcolor=white, #4,
            },
    ]{#1}
}



%%%%%%%%%%%%%%%%%%%%%%%%%%%%%%%%%%%
%  Create the Environment Styles  %
%%%%%%%%%%%%%%%%%%%%%%%%%%%%%%%%%%%

\makeatletter
\@ifclasswith\class{nocolor}{
    % Environments without color.

    % ESSENTIALS:
    \createnewtheoremstyle{thmdefinitionbox}{}{}
    \createnewtheoremstyle{thmtheorembox}{}{}
    \createnewtheoremstyle{thmproofbox}{qed=\qedsymbol}{}
    \createnewtheoremstyle{thmcorollarybox}{}{}
    \createnewtheoremstyle{thmlemmabox}{}{}
    \createnewtheoremstyle{thmclaimbox}{}{}
    \createnewtheoremstyle{thmexamplebox}{}{}

    % MISCS: 
    \createnewtheoremstyle{thmpropbox}{}{}
    \createnewtheoremstyle{thmexercisebox}{}{}
    \createnewtheoremstyle{thmexplanationbox}{}{}
    \createnewtheoremstyle{thmremarkbox}{}{}

    % STYLIZED MORE BELOW
    \createnewtheoremstyle{thmquestionbox}{}{}
    \createnewtheoremstyle{thmsolutionbox}{qed=\qedsymbol}{}
}{
    % Environments with color.

    % ESSENTIALS: definition, theorem, proof, corollary, lemma, claim, example
    \createnewcoloredtheoremstyle{thmdefinitionbox}{definition_color}{}{
        leftline=true
    }
    \createnewcoloredtheoremstyle{thmtheorembox}{theorem_color}{}{
        leftline=true
    }
    \createnewcoloredtheoremstyle{thmproofbox}{proof_color}{qed=\qedsymbol}{backgroundcolor=white}
    \createnewcoloredtheoremstyle{thmcorollarybox}{corollary_color}{}{backgroundcolor=white}
    \createnewcoloredtheoremstyle{thmlemmabox}{lemma_color}{}{backgroundcolor=white}
    \createnewcoloredtheoremstyle{thmclaimbox}{claim_color}{}{}
    \createnewcoloredtheoremstyle{thmexamplebox}{example_color}{}{backgroundcolor=white}
    \createnewcoloredtheoremstyle{thmexplanationbox}{example_color}{qed=\qedsymbol}{backgroundcolor=white}
    \createnewcoloredtheoremstyle{thmremarkbox}{theorem_color}{}{backgroundcolor=white}

    \createnewcoloredtheoremstyle{thmmiscbox}{black}{}{leftline=false,backgroundcolor=white}


    \createnewcoloredtheoremstyle{thmpropbox}{prop_color}{}{backgroundcolor=white}
    \createnewcoloredtheoremstyle{thmexercisebox}{exercise_color}{}{backgroundcolor=white}

    \createnewcoloredtheoremstyle{thmproblembox}{myred}{}{backgroundcolor=white}
    \createnewcoloredtheoremstyle{thmsolutionbox}{mygreen}{qed=\qedsymbol}{backgroundcolor=white}
}
\makeatother

%%%%%%%%%%%%%%%%%%%%%%%%%%%%%
%  Create the Environments  %
%%%%%%%%%%%%%%%%%%%%%%%%%%%%%
\declaretheorem[numberwithin=section, style=thmdefinitionbox,     name=Definition]{definition0}
\declaretheorem[numberwithin=section, style=thmtheorembox,     name=Theorem]{theorem}
\declaretheorem[numbered=no,          style=thmexamplebox,     name=Example]{example}
\declaretheorem[numberwithin=section, style=thmclaimbox,       name=Claim]{claim}
\declaretheorem[numberwithin=section, style=thmcorollarybox,   name=Corollary]{corollary}
\declaretheorem[numberwithin=section, style=thmpropbox,        name=Proposition]{proposition}
\declaretheorem[numberwithin=section, style=thmlemmabox,       name=Lemma]{lemma}
\declaretheorem[numberwithin=section, style=thmexercisebox,    name=Exercise]{exercise}
\declaretheorem[numbered=no,          style=thmproofbox,       name=Proof]{replacementproof}
\declaretheorem[numbered=no,          style=thmexplanationbox, name=Explanation]{explanation}
\declaretheorem[numbered=no,          style=thmsolutionbox,    name=Solution]{solution}
\declaretheorem[numberwithin=section,          style=thmproblembox,     name=Problem]{problem}
\declaretheorem[numbered=no,          style=thmmiscbox,    name=Intuition]{intuition}
\declaretheorem[numbered=no,          style=thmmiscbox,    name=Goal]{goal}
\declaretheorem[numbered=no,          style=thmmiscbox,    name=Recall]{recall}
\declaretheorem[numbered=no,          style=thmmiscbox,    name=Motivation]{motivation}
\declaretheorem[numbered=no,          style=thmmiscbox,    name=Remark]{remark}
\declaretheorem[numbered=no,          style=thmmiscbox,    name=Observe]{observe}




%%%% FANCY GRAPHICS:

% \makeatletter
% \@ifclasswith\class{nocolor}{
%   % Environments without color.

%   \newtheorem*{note}{Note}

%   \declaretheorem[numberwithin=section, style=thmdefinitionbox, name=Definition]{definition}
%   \declaretheorem[numberwithin=section, style=thmquestionbox,   name=Question]{question}
%   \declaretheorem[numberwithin=section, style=thmsolutionbox,   name=Solution]{solution}
% }{
%   % Environments with color.

%   \newtcbtheorem[number within=section]{Definition}{Definition}{
%     enhanced,
%     before skip=2mm,
%     after skip=2mm,
%     colback=red!5,
%     colframe=red!80!black,
%     colbacktitle=red!75!black,
%     boxrule=0.5mm,
%     attach boxed title to top left={
%       xshift=1cm,
%       yshift*=1mm-\tcboxedtitleheight
%     },
%     varwidth boxed title*=-3cm,
%     boxed title style={
%       interior engine=empty,
%       frame code={
%         \path[fill=tcbcolback]
%         ([yshift=-1mm,xshift=-1mm]frame.north west)
%         arc[start angle=0,end angle=180,radius=1mm]
%         ([yshift=-1mm,xshift=1mm]frame.north east)
%         arc[start angle=180,end angle=0,radius=1mm];
%         \path[left color=tcbcolback!60!black,right color=tcbcolback!60!black,
%         middle color=tcbcolback!80!black]
%         ([xshift=-2mm]frame.north west) -- ([xshift=2mm]frame.north east)
%         [rounded corners=1mm]-- ([xshift=1mm,yshift=-1mm]frame.north east)
%         -- (frame.south east) -- (frame.south west)
%         -- ([xshift=-1mm,yshift=-1mm]frame.north west)
%         [sharp corners]-- cycle;
%       },
%     },
%     fonttitle=\bfseries,
%     title={#2},
%     #1
%   }{def}

%   \NewDocumentEnvironment{definition}{O{}O{}}
%     {\begin{Definition}{#1}{#2}}{\end{Definition}}

%   \newtcolorbox{note}[1][]{%
%     enhanced jigsaw,
%     colback=gray!20!white,%
%     colframe=gray!80!black,
%     size=small,
%     boxrule=1pt,
%     title=\textbf{Note:-},
%     halign title=flush center,
%     coltitle=black,
%     breakable,
%     drop shadow=black!50!white,
%     attach boxed title to top left={xshift=1cm,yshift=-\tcboxedtitleheight/2,yshifttext=-\tcboxedtitleheight/2},
%     minipage boxed title=1.5cm,
%     boxed title style={%
%       colback=white,
%       size=fbox,
%       boxrule=1pt,
%       boxsep=2pt,
%       underlay={%
%         \coordinate (dotA) at ($(interior.west) + (-0.5pt,0)$);
%         \coordinate (dotB) at ($(interior.east) + (0.5pt,0)$);
%         \begin{scope}
%           \clip (interior.north west) rectangle ([xshift=3ex]interior.east);
%           \filldraw [white, blur shadow={shadow opacity=60, shadow yshift=-.75ex}, rounded corners=2pt] (interior.north west) rectangle (interior.south east);
%         \end{scope}
%         \begin{scope}[gray!80!black]
%           \fill (dotA) circle (2pt);
%           \fill (dotB) circle (2pt);
%         \end{scope}
%       },
%     },
%     #1,
%   }

%   \newtcbtheorem{Question}{Question}{enhanced,
%     breakable,
%     colback=white,
%     colframe=myblue!80!black,
%     attach boxed title to top left={yshift*=-\tcboxedtitleheight},
%     fonttitle=\bfseries,
%     title=\textbf{Question:-},
%     boxed title size=title,
%     boxed title style={%
%       sharp corners,
%       rounded corners=northwest,
%       colback=tcbcolframe,
%       boxrule=0pt,
%     },
%     underlay boxed title={%
%       \path[fill=tcbcolframe] (title.south west)--(title.south east)
%       to[out=0, in=180] ([xshift=5mm]title.east)--
%       (title.center-|frame.east)
%       [rounded corners=\kvtcb@arc] |-
%       (frame.north) -| cycle;
%     },
%     #1
%   }{def}

%   \NewDocumentEnvironment{question}{O{}O{}}
%   {\begin{Question}{#1}{#2}}{\end{Question}}

%   \newtcolorbox{Solution}{enhanced,
%     breakable,
%     colback=white,
%     colframe=mygreen!80!black,
%     attach boxed title to top left={yshift*=-\tcboxedtitleheight},
%     title=\textbf{Solution:-},
%     boxed title size=title,
%     boxed title style={%
%       sharp corners,
%       rounded corners=northwest,
%       colback=tcbcolframe,
%       boxrule=0pt,
%     },
%     underlay boxed title={%
%       \path[fill=tcbcolframe] (title.south west)--(title.south east)
%       to[out=0, in=180] ([xshift=5mm]title.east)--
%       (title.center-|frame.east)
%       [rounded corners=\kvtcb@arc] |-
%       (frame.north) -| cycle;
%     },
%   }

%   \NewDocumentEnvironment{solution}{O{}O{}}
%   {\vspace{-10pt}\begin{Solution}{#1}{#2}}{\end{Solution}}
% }
% \makeatother


%%%%% END OF FANCY GRAPHICS %%%%%%%%%%



%%%%%%%%%%%%%%%%%%%%%%%%%%%%
%  Edit Proof Environment  %
%%%%%%%%%%%%%%%%%%%%%%%%%%%%

\renewenvironment{proof}[2][\proofname]{
    % \vspace{-12pt}
    \begin{replacementproof} [#2]
}{\end{replacementproof}}

\newenvironment{definition}[1]{
    \begin{definition0}[#1]

    \hfill
        
    \vspace{0.2cm}

}{

    \vspace{0.2cm}
    \end{definition0}
}


\theoremstyle{definition}

\newtheorem*{notation}{Notation}
\newtheorem*{previouslyseen}{As previously seen}
\newtheorem*{property}{Property}
% \newtheorem*{intuition}{Intuition}
% \newtheorem*{goal}{Goal}
% \newtheorem*{recall}{Recall}
% \newtheorem*{motivation}{Motivation}
% \newtheorem*{remark}{Remark}
% \newtheorem*{observe}{Observe}

\author{Cong Hung Le Tran}


%%%% MATH SHORTHANDS %%%%
%% blackboard bold math capitals
\newcommand{\bbf}{\mathbb{F}}
\newcommand{\bbn}{\mathbb{N}}
\newcommand{\bbq}{\mathbb{Q}}
\newcommand{\bbr}{\mathbb{R}}
\newcommand{\bbz}{\mathbb{Z}}
\newcommand{\bbc}{\mathbb{C}}
\newcommand{\bbk}{\mathbb{K}}
\newcommand{\bbm}{\mathbb{M}}
\renewcommand{\phi}{\varphi}
\newcommand{\st}{\;\text{such that}\;}


% MATH 20250 %
\newcommand{\Hom}{\mathrm{Hom}}
\newcommand{\im}{\mathrm{im}}

% https://tex.stackexchange.com/questions/438612/space-between-exists-and-forall
\let\oldforall\forall
\renewcommand{\forall}{\;\oldforall\; }
\let\oldexist\exists
\renewcommand{\exists}{\;\oldexist\; }
\newcommand\existu{\;\oldexist!\: }


\renewcommand{\_}[1]{\underline{ #1 }}
\DeclarePairedDelimiter{\abs}{\lvert}{\rvert}
\DeclarePairedDelimiter{\norm}{\lVert}{\rVert}
\setlength\parindent{0pt}
\setlength{\headheight}{12.0pt}
\addtolength{\topmargin}{-12.0pt}


% Default skipping, change if you want more spacing
% \thinmuskip=3mu
% \medmuskip=4mu plus 2mu minus 4mu
% \thickmuskip=5mu plus 5mu



% \DeclareMathOperator{\ext}{ext}
% \DeclareMathOperator{\bridge}{bridge}
\title{MATH 20700: Honors Analysis in Rn I \\ \large Problem Set 2}
\date{06 Oct 2023}
\author{Hung Le Tran}
\begin{document}
\maketitle
\setcounter{section}{2}
\textbf{Textbook: Pugh's Real Mathematical Analysis}

\textit{Collaborators: Lucio, Duc, Hung Pham}
\begin{problem} [2.30]
Consider a two-point set $M = \{a, b\}$ whose topology consists of the two sets, $M$ and the empty set. Why does this topology not arise from a metric on $M$?
\end{problem}
\begin{solution}
    Suppose there exists a metric $d$ that induces the topology $\calt = \{\{a, b\}, \emptyset\}$ on $M$. Let $\epsilon = d(a, b)/2$, then $B_M(x, \epsilon) = \{a\}$, so $\{a\}$ is an open set. But $\{a\} \not \in \calt$, \contra

    Therefore $\calt$ can't be a topology induced by a metric.
\end{solution}

\begin{problem} [2.36]
Construct a set with exactly three cluster points.
\end{problem}
\begin{solution}
    \[
        S = \{1 - \frac{1}{n} \mid n \in \bbn\} \cup \{ 3 - \frac{1}{n} \mid n \in \bbn\} \cup \{5 - \frac{1}{n} \mid n \in \bbn\}
    \]
    $S$ clearly has cluster points 1, 3, 5. The set of cluster points $S'$ is a subset of $\lim S = \{1, 3, 5\}$. So it has none other than the 3 above.
\end{solution}

\begin{problem} [2.44]
Consider a function $f: M \to \bbr$. Its graph is the set \[
    \{(p, y) \in M \times \bbr : y = fp\}
\]

\begin{enumerate} [(a)]
    \item Prove that if $f$ is continuous then its graph is closed (as a subset of $M \times \bbr$).
    \item Prove that if $f$ is continuous and $M$ is compact then its graph is compact
    \item Prove that if the graph of $f$ is compact then $f$ is continuous
    \item What if the graph is merely closed? Give an example of a discontinuous function $f: \bbr \to \bbr$ whose graph is closed.
\end{enumerate}
\end{problem}

\begin{solution}

    We freely use Pugh's Theorem 17 (Ch2.3) in this problem.

    Let $G = \{(p, y) \in M \times \bbr : y = fp\} $ be the graph. \par

    \textbf{(a)} Let $\{(p_n, y_n)\} \subset G$ be a sequence that converges to $(p, y) \in M \times \bbr$. WLOG, we use the metric $d_{max}$ on $M \times \bbr$. WTS $(p, y) \in G$.

    $(p_n, y_n) \cvgn (p, y) \implies p_n \cvgn p, y_n \cvgn y$. But $f$ is continuous, so $y_n = f(p_n) \cvgn f(p) \implies y = f(p) \implies (p, y) \in G$. \qed \par

    \textbf{(b)} Let $\{(p_n, y_n)\} \subset G$. WTS there exists a subsequence $\{(p_{n_j}, y_{n_j})\}$ that converges in $G$.

    Since $M$ is compact and $\{p_n\} \subset M$, there exists a subsequence $\{p_{n_j}\}$ such that $ p_{n_j} \cvgj p \in M$.

    $f$ is continuous, so $f(p_{n_j}) \xrightarrow{j \to \infty} f(p) \Leftrightarrow y_{n_j} \xrightarrow{j \to \infty} f(p) \eqqcolon y$.
    This implies \[
        (p_{n_j}, y_{n_j}) \cvgj (p, y) \in G \qed
    \]

    \textbf{(c)} Let $\{p_n\} \subset M$ such that $p_n \xrightarrow{n \to \infty} p \in M$. WTS $f(p_n) \xrightarrow{n \to \infty} f(p)$.

    Suppose not:
    \[
        \neg (\forall \epsilon > 0, \exists N \in \bbn, \forall n \geq N, d(f(p_n), f(p)) < \epsilon)
    \]
    \[
        \Leftrightarrow \exists \epsilon > 0, \forall N \in \bbn, \exists n \geq N, d(f(p_n), f(p)) \geq \epsilon
    \]
    Then we construct another sequence $q_n$, defined as follows:
    Choose $N_0 = 1$ and choose $N_1 \geq N_0$ satisfying the conditions above. Assign $q_1 = p_{N_1}$.
    Inductively, choose $N_{k+1} \geq N_{k}$ satisfying the conditions above, and assign $q_{k+1} = p_{N_{k+1}}$.

    By constructing $\{q_n\} \subset M$ (essentially a particular subsequence of $\{p_n\}$) this way, \[
        \forall n \in \bbn, d(f(q_n), f(p)) \geq \epsilon
    \]

    Since $G$ is compact, sequence $\{(q_n, f(q_n))\} \subset G$ has subsequence $\{(q_{n_j}, f(q_{n_j}))\}$ such that $(q_{n_j}, f(q_{n_j})) \cvgj (q, f(q)) \in G$, which implies $q_{n_j} \cvgj q, f(q_{n_j}) \cvgj f(q)$. But $\{q_{n_j}\}$ is simply a (sub) subsequence of $\{p_n\}$, so $q = p \implies f(q_{n_j}) \cvgj f(p)$.

    But $\forall n \in \bbn, d(f(q_n), f(p)) \geq \epsilon$. \contra \qed

    \textbf{(d)} Define \[
        f(x) = \begin{cases}
            0             & \:\text{for}\: x = 0, \\
            \frac{1}{|x|} & \:\text{otherwise}.\:
        \end{cases}
    \]
    then $f$ is discontinuous at $0$.

    We show that $G$ is indeed closed. Let $\{(p_n, f(p_n))\} \subset M \times \bbr$ such that $(p_n, f(p_n)) \cvgn (p, q) \in M \times \bbr$. WTS $(p, q) \in G \Leftrightarrow q = f(p)$.

    $(p_n, f(p_n)) \cvgn (p, q) \Leftrightarrow p_n \cvgn p, f(p_n) \cvgn q$.

    If $p \neq 0$, let $\epsilon = |p|/2$, then $\exists N \in \bbn$ such that $\forall n \geq N, |p|/2 < |p_n/2| < 3|p|/2$. In particular, they are non-zero. Since $f$ is trivially continuous on $\bbr \backslash \{0\}$, it is then clear that $f(p_n) \cvgn f(p)$, so $q = f(p)$.

    If $p = 0$, with $p_n \cvgn 0, f(p_n)$ diverges. In particular, given any $M > 0$, let $\epsilon = \frac{1}{2M}$ then there exists $N \in \bbn$ such that $\forall n \geq N, |p_n| < \epsilon = \frac{1}{2M} \implies f(p_N) > 2M > M$. So there can't be $(p_n, f(p_n)) \cvgn (0, q)$ in the first place.
\end{solution}

\begin{problem} [2.49*]
Construct a subset $A \subset \bbr$ and a continuous bijection $f: A \to A$ that is not a homeomorphism. [Hint: By Theorem 36, $A$ must be noncompact]
\end{problem}
\begin{solution}
Let $A = [0, 1)  \cup [2, 3) \cup [4, 5) \cup \dots = \bigcup_{n=0}^{\infty} [2n, 2n+1)$

And construct map $f$ that (continuously):
\begin{enumerate}
    \item Scales and shifts [0, 1) to [0, 1/2)
    \item Scales and shifts [2, 3) to [1/2, 1)
    \item Shifts [4, 5) to [2, 3), [6, 7) to [4, 5), etc., in short shifts $[2n, 2n+1)$ to $[2n-2, 2n-1)$ for all $n \geq 2$
\end{enumerate}

Then $f$ trivially maintains sequential convergence (therefore is continuous) and is a bijection. So it is a continuous bijection.

However, $f$ sends [2, 2.5) to [1/2, 3/4), an open set in $A$ to a nonopen set in $A$, so $f^{-1}$ is not continuous. It follows that $f$ is not homeo.
\end{solution}

\begin{problem} [2.56]
Prove that the 2-sphere is not homeomorphic to the plane.
\end{problem}
\begin{solution}
    We want to first show that the 2-sphere is compact, by showing that it is closed and bounded in $\bbr^3$.

    It is clear that the (unit) 2-sphere is bounded by the $2 \times 2 \times 2$ box centered at the origin, so it is bounded.

    Let $\{p_n\} \subset S^2$ such that $p_n \cvgn p \in \bbr^3$. We now want to show that $p \in S^2$. Recall that $S^2 = \{z \in \bbr^3 \mid d(0, z) = 1\}$

    For all $\epsilon > 0$, there exists $N \in \bbn$ such that $\forall n \geq N, d(p_n, p) < \epsilon$. But $d(0, p_n) = 1 \forall n \in \bbn$. It follows that \[
        1 - \epsilon < d(0, p_N) - d(p_N, p) \leq d(0, p) \leq d(0, p_N) + d(p_N, p) < 1 + \epsilon
    \]
    which holds for all $\epsilon > 0$. It follows that $d(0, p) = 1$, so $p \in S^2$. Therefore the 2-sphere is closed.

    Since the 2-sphere is closed and bounded in $\bbr^3$, it is compact (H-B).

    On the other hand, $\bbr^2$ is unbounded in $\bbr^2$, so it is not compact.

    Therefore the 2-sphere (a compact set) can't be homeomorphic to the plane (a noncompact set).
\end{solution}

\begin{problem} [2.66]
$\ $
\begin{enumerate} [(a)]
    \item Prove that every connected open subset of $\bbr^m$ is path-connected.
    \item Is the same true for open connected subsets of the circle?
    \item What about connected nonopen subsets of the circle?
\end{enumerate}
\end{problem}
\begin{solution}

    \textbf{(a)}
    Let $U \subset \bbr^m$ be connected and open. WTS $U$ is path-connected.

    We first want to show that any ball in $\bbr^m$ is path-connected. Let $p, q \in B(x, r) \subset U$ then we can construct \[
        f: [0, 1] \to B(x, r), t \mapsto p + (q-p)t
    \]
    that is a continuous path from $p$ to $q$ as required.

    WTS for every $p, q \in U$, there exists a continuous path from $p$ to $q$. Fix $p, q \in U$.
    
    Then let $A$ be the set of points $z \in U$ such that there exists a continuous path from $p$ to $z$. We want to show that $A$ is clopen. Note that $p \in A$, so $A$ is non-empty.

    1. Show that $A$ is open.
    
    Let $z \in A$ be arbitrary. Since $U$ is open, we can draw $B_U(z, r) \subset U$.
    
    We want to show that $B_U(z, r) \subset A$. Choose any $s \in B(z, r)$, then we can concatenate the continuous path from $p_0$ to $z$ (since $z \in A$) and the continuous path from $z$ to $s$ (since the ball is path-connected) to get a continuous path from $p_0$ to $s$.
    
    More concretely, if there exists continuous path $f_{p, z}: [0, 1] \to U$ from $p$ to $z$, and $ f_{z, s}: [0, 1] \to U$ from $z$ to $s$ then we can construct \[
        f_{p, s}(t) = \begin{cases}
            f_{p, z} (2t) & \:\text{for}\: t < 0.5    \\
            f_{z, s} (2t-1) & \:\text{for}\: t \geq 0.5
        \end{cases}
    \]
    is a continuous path from $p$ to $s$. So $s \in A$. It follows that $B(z, r) \subset A$. Therefore $A$ is open.

    2. Show that $A$ is closed.

    WTS $A^c$, the set of points $y \in U$ such that there doesn't exist a continuous path from $p$ to $y$, is open. 
    
    Let $y \in A^c$ be arbitrary. Since $U$ is open, we can draw $B_U(y, r) \subset U$.
    
    WTS $B_U(y, r) \subset A^c$. Choose any $s \in B(y, r)$. Suppose not, that there exists a continuous path from $p$ to $s$, then since the ball is path-connected, it follows that there also exists a continuous path from $s$ to $y$. By concatenating these 2 paths, we can get a continuous path from $p$ to $y$. \contra

    So there doesn't exist a continuous path from $p$ to $s$ for all $s \in B(y, r)$. It follows that $B_U(y, r) \subset A^c$, so $A^c$ is open. Thus $A$ is closed.

    3. Therefore $A$ is a nonempty clopen subset of connected $U$, so $A = U$. It follows that $U$ is path-connected. \qed

    \textbf{(b)} Yes. WLOG, use $S^1$.

    We now want to show that any ball $B(x, r) \subset S^1$ is path-connected. If $r \leq 2$ then $B(x, r)$ is an arc of $S^1$, while if $r > 2$ then $B(x, r) = S^1$. Both of which can be parameterized as $\{(\cos \theta, \sin \theta)\}$, and are therefore trivially path-connected.

    Apply the same proof in (a). \qed

    \textbf{(c)} Yes, it is path-connected.

    Let $V$ be a connected, nonopen subset of the circle, and $p \neq q$ be arbitrary points in $V$. $p$ and $q$ then partition $S^1$ into a major closed arc $A_1$ and a minor closed arc $A_2$ that satisfies: \[
        A_1 \cup A_2 = S^1, A_1 \cap A_2 = \{p, q\}
    \]

    \textbf{Case 1:} If $A_1 \subset V$ or $A_2 \subset V$, then there clearly exists a continuous path in $V$ from $p$ to $q$, namely through these continuous closed arcs (that can be easily parameterized into $\{(\cos \theta, \sin \theta)\}$ on a closed interval of $\theta$)

    \textbf{Case 2:} There exists $k \in A_1, l \in A_2$ such that $k, l \not \in V$. Then $k, l$ similarly partitions $S^1$ into 2 closed arcs $A_3, A_4$ such that \[
        A_3 \cup A_4 = S^1, A_3 \cap A_4 = \{k, l\}
    \]

    Now consider $A'_3 \coloneqq A_3 \backslash \{k, l\}, A'_4 \coloneqq A_4 \backslash \{k, l\}$, which are open sets in $S^1$ that are image sets of $(\cos \theta, \sin \theta)$ on an open interval of $\theta$. Then \[
        (V \cap A'_3) \cup (V \cap A'_4) = V \cap (A'_3 \cup A'_4) = V \cap (S^1 \backslash \{k, l\}) = V
    \]
    However, $A'_3 \cap A'_4 = \emptyset$ so $(V \cap A'_3) \cap (V \cap A'_4) = \emptyset$. Furthermore, $V \cap A'_3$ and $V \cap A'_4$ in the subspace topology of $V$, as they are intersections of $V$ with an open set in the bigger topology of $S^1$. It follows that $(V \cap A'_3) \sqcup (V \cap A'_4)$ partitions $V$ into 2 non-empty open sets, which are therefore proper clopen, making $V$ disconnected, \contra.

    It follows that only case 1 is valid, and there exists a continuous path from any $p$ to $q$.
\end{solution}

\begin{problem} [2.78]
$(p_1, \dots, p_n)$ is an $\epsilon$-chain in a metric space $M$ if for each $i$ we have $p_i \in M$ and $d(p_i, p_{i+1}) < \epsilon$. The metric space is chain-connected if for each $\epsilon > 0$ and each pair of points $p, q \in M$ there is an $\epsilon-$chain from $p$ to $q$.

\begin{enumerate} [(a)]
    \item Show that every connected metric space is chain-connected.
    \item Show that if $M$ is compact and chain-connected then it is connected.
    \item Is $\bbr \backslash \bbz$ chain-connected?
    \item If $M$ is complete and chain-connected, is it connected?
\end{enumerate}
\end{problem}
\begin{solution}

    \textbf{(a)} Suppose not. Then there exists $\epsilon > 0$ and $p_0, q_0 \in M$ such that there doesn't exist an $\epsilon-$chain from $p_0$ to $q_0$. We emphasize that this $\epsilon$ is now fixed.

    Let $D$ be the set of such points in $M$ that have no $\epsilon-$chain from $p_0$. Define $C = M \backslash D$. Then $p_0 \in C, q_0 \in D$. WTS $C, D$ are open.

    We first consider $C$. Take arbitrary $c \in C$, then let its corresponding $\epsilon-$chain from $p_0$ be $(p_1, \dots, p_n)$.

    Draw $B_M(c, \epsilon)$. We show that $C$ is open by showing that indeed we can draw for arbitrary $c$, $B_M(c, \epsilon) \subset C$.

    Indeed, $\forall c' \in B_M(c, \epsilon)$, there exists an $\epsilon$-chain from $p_0$ to $c'$, namely $(p_1, \dots, p_n, c')$, which is well-defined since $d(c', p_n) = d(c', c) < \epsilon$.

    Therefore $c' \in C$. So $B_M(c, \epsilon) \subset C$.

    We now consider $D$. The proof is very similar. Take arbitrary $d \in D$.

    Draw $B_M(d, \epsilon)$. We show that $D$ is open by showing that we can draw for arbitrary $d$, $B_M(d, \epsilon) \subset D$.

    Indeed, $\forall d' \in B_M(d, \epsilon)$, if there exists an $\epsilon-$ chain from $p_0$ to $d'$, for example, $(q_1, \dots, q_n)$, then one can form an $\epsilon-$chain from $p_0$ to $d$, which would be $(q_1, \dots, q_n, d)$, a well-defined chain since $d(d, q_n) = d(d, d') < \epsilon$, which would be a contradiction since $d \in D$.

    Therefore there doesn't exist an $\epsilon-$chain from $p_0$ to $d'$. So $(d' \in B_M(d, \epsilon) \implies d' \in D)$. $D$ is therefore open.

    In conclusion, we have \[
        M = C \sqcup D
    \]
    where $C$ and $D$ are both open, which make them both clopen. But $p_0 \not \in D, q_0 \not \in C$, so they are proper clopen subsets of $M$, making $M$ disconnected. \contra

    It follows that $M$ must be chain-connected. \qed

    \textbf{(b)} Let $M$ be compact and chain-connected. WTS $M$ is connected.

    Suppose not. Then there exists $A, B \subset M$ such that \[
        M = A \sqcup B
    \]
    and $A, B \neq \emptyset$ and are clopen. Since $A, B \neq \emptyset, \exists a \in A, b \in B$.

    We shall construct a sequence $\{t_n\} \subset A$ as follows:

    Set $\epsilon = \frac{1}{n}$, then there exists (finite) $\epsilon-$chain $(p_{n, 1}, p_{n, 2}, \dots, p_{n, N(n)})$ from $a$ to $b$, i.e. $p_{n, 1} = a, p_{n, N(n) = b}$. ($N(n) \geq 2$, since $a \neq b$).

    Let $T(n) = \{1 \leq k \leq N(n) \mid B_M(p_{n, k}, \frac{1}{n}) \subset A\}$, i.e., the set of indices of points in the chain whose $\epsilon-$ neighborhoods are still within $A$. Clearly $1 \in T(n)$ and it has an upper bound $N(n)$, so there exists a maximum value $k(n) \in \bbn$.

    Assign $t_n \coloneqq p_{n, k(n)} \in A$.

    Since $A$ is a closed subset of compact $M$, it follows that $A$ is also compact. Therefore, there exists a subsequence $\{t_{n_j}\} $ such that \[
        t_{n_j} \cvgj t \in A
    \]

    Since $A$ is open, we can draw a ball $B_M(t, r) \subset A$.

    Since $t_{n_j} \cvgj t$, there exists $J_1\in \bbn$ such that $\forall j > J_1, d(t_{n_j}, t) < \frac{r}{2}$.

    Choose $J = \ceil{\max\{J_1, \frac{4}{r}\}} + 1$. Let $L = n_J$. Then \[
        \frac{1}{L} \leq \frac{1}{J} < \frac{r}{4}
    \]

    Consider the original path from $a$ to $b$ with $\epsilon = \frac{1}{L}$:
    \[
        (a = p_{L, 1}; p_{L, 2}; \dots ; p_{L, k(L)} = t_{L} ; p_{L, k(L) + 1} ; \dots ; p_{L, N(L)} = b)
    \]
    where we pay special attention to the node after $t_L$.

    We see that \begin{align*}
        d(t, p_{L, k(L) + 1}) & \leq d(t, t_L) + d(t_L, p_{L, k(L) + 1})   \\
                              & < \frac{r}{2} + \frac{1}{L}                \\
                              & < \frac{r}{2} + \frac{r}{4} = \frac{3r}{4}
    \end{align*}

    It trivially follows that $B_M(p_{L, k(L) + 1}, \frac{1}{L}) \subset B_M(t, r)$.

    But $B_M(t, r) \subset A$, so $B_M(p_{L, k(L) + 1}) \subset A$, which contradicts with $k(L)$ being the maximum index at which the node of the path is still in $A$ \contra.

    It follows that $M$ must be connected. \qed

    \textbf{(c)} Yes.

    Given $\epsilon > 0; a, b \in \bbr \backslash \bbz$. WLOG, assume $\epsilon < 1$, since a larger $\epsilon'$-chain can be composed through nodes consisting of $\ceil{\frac{\epsilon'}{\epsilon}}$ subnodes of the $\epsilon-$ chain.

    We will recursively define an $\epsilon-$chain from $a$ to $b$ as follows:
    \[
        p_1 \coloneqq a \\
    \]
    If $b - p_n < \epsilon, p_{n+1} \coloneqq b$, else
    \[
        p_{n+1} = \begin{cases}
            p_n + \epsilon   & \:\text{if}\: p_n  < \ceil{p_n} - \epsilon                              \\
            p_n + \epsilon/2 & \:\text{if}\: \ceil{p_n} - \epsilon \leq p_n < \ceil{p_n} - \epsilon /2 \\
            p_n + \epsilon   & \:\text{if}\: \ceil{p_n} - \epsilon /2 \leq p_n < \ceil{p_n}
        \end{cases}
    \]

    It is firstly trivial that $p_{n+1}$, as constructed, is not in $\bbr \backslash \bbz$. It is also guaranteed that after at most $\ceil{\frac{b-a}{\epsilon / 2}}$ terms, $p_n$ will come within $\epsilon$ of $b$, and therefore setting $p_{n+1} = b$, completing the $\epsilon-$chain from $a$ to $b$. \qed

    \textbf{(d)} No.

    Counter-example: 

    Define \[
    A = \{(x, 1/x) \mid x > 0\}, B = \{(x, -1/x) \mid x < 0\}
    \]
    and $M = A \sqcup B$. We show that $M$ is complete and chain-connected, but not connected.

    First, $A$ and $B$ are closed in the ambient metric space $\bbr^2$ (similar to problem 2.44d). So $M = A \sqcup B$ is also closed in $\bbr^2$. $\bbr^2$ is complete so $M$ is complete.

    Second, $M$ is chain-connected. Suppose we are given $\epsilon > 0; p, q \in M$. If $p$ and $q$ are both in $A$ or are both in $B$, then the chain is trivial. If $p \in A$ and $q \in B$ then we can define $x_1 = -\epsilon/3, x_2 = \epsilon/3$ then \[
    d_E((x_1, 1/|x_1|), (x_2, 1/|x_2|)) = 2\epsilon / 3 < \epsilon
    \]
    so we can make and concatenate $\epsilon-$chains from $p$ to $(x_1, 1/x_1)$, from $(x_1, 1/x_1)$ to $(x_2, -1/x_2)$, and from $(x_2, -1/x_2)$ to $q$.

    Lastly, it is not connected. $A = A \cap M$, $A$ is closed in $\bbr^2$ so $A$ is closed in $M$. It is also open in $M$ because for each $p = (a, 1/a) \in A$, we can draw $B_M((a, 1/a), 1/4a) \subset A$ trivially. So $A$ is clopen and proper.
\end{solution}

\begin{problem} [2.117 f]
Fold a piece of paper in half.
\begin{enumerate} [(a)]
    \item Is this a continuous transformation of one rectangle into another?
    \item Is it injective?
    \item Draw an open set in the target rectangle, and find its preimage in the original rectangle. Is it open?
    \item What if the open set meets the crease?

          The \textbf{baker's transformation} is a similar mapping. A rectangle of dough is stretched to twice its length and then folded back on itself. Is the transformation continuous? A formula for the baker's transformation in one variable is \[
              f: [0, 1] \to [0, 1], f(x) = 1 - |1 - 2x|
          \]
          The $\mathbf{n^{th}}$ iterate of $f$ is $f^n = f \circ f \circ \dots \circ f, n$ times. The \textbf{orbit} of a point $x$ is \[
              \{x, f(x), f^2(x), \dots, f^n(x), \dots\}
          \]
          $f^{\circ n}$ is $f^n$
    \item If $x$ is rational prove that the orbit of $x$ is a finite set.
    \item If $x$ is irrational what is the orbit?
\end{enumerate}
\end{problem}
\begin{solution}
    \textbf{(a)} WLOG, Let the rectangle be $[0, 2] \times [0, 1]$, and let the crease be at $x = 1$. Then the transformation is \[
        f(x, y) = \begin{cases}
            (x, y)   & \:\text{for}\: x < 1    \\
            (2-x, y) & \:\text{for}\: x \geq 1
        \end{cases}
    \]
    which folds the rectangle $[0, 2] \times [0,1]$ into the rectangle $[0, 1] \times [0, 1]$. The transformation is continuous, since the stepwise function on the $x-$coordinate is continuous and $y$ is trivially continuous. \qed

    \textbf{(b)} No, it is not injective.
    \[
        f(0.5, 0) = f(1.5, 0) = (0.5, 0) \qed
    \]

    \textbf{(c)} Yes.

    Let the open set be $U = \{(x, y)\} \subset [0, 1] \times [0, 1]$. Define $g(x, y) = (2-x, y)$ then \[
        f^{Pre}(U) = U \cup g(U)
    \]
    Note that $g$ is a homeomorphism, so $g(U)$ is also open. $f^{Pre}(U)$ is a union of open sets and is therefore open. \qed

    \textbf{(d)} It doesn't matter if the open set meets the crease. The above proof still holds. \qed

    \textbf{(e)} We freely use the fact that if $x \in [0, 1]$, $x$ can be decomposed into the unique following binary representation:
    \[
    x = \frac{a_1}{2} + \frac{a_2}{2^2} + \cdots + \frac{a_n}{2^n} + \cdots
    \]
    i.e., each $x \in [0, 1]$ corresponds to a unique sequence $\{a_{x, n}\} $ with $a_{x, n} \in \{0, 1\}$.

    Note that 0 corresponds to the zero-sequence, while 1 corresponds to the one-sequence.

    Let $x \in [0, 1]$. To be concise, we label $\{a_{x, n}\}$ as $\{a_n\}$. We pay attention to how the sequence changes as $f$ is applied continually to $x$.
    
    Rewriting $f$, we have: \[
    f(x) = \begin{cases}
        2x & \:\text{for}\: x < 0.5 \\
        2 - 2x & \:\text{for}\: x \geq 0.5
    \end{cases}
    \]
    
    When $x < 0.5$, it is clear that $a_1 = 0$. Then we have \[
     2x = 2 \left(\frac{a_2}{2^2} + \cdots \right) = \frac{a_2}{2} + \frac{a_3}{2^2} + \cdots
    \]
    which corresponds to sequence $(a_2, a_3, \cdots)$ (sequence shifts left)

    $(a_1, a_2, \dots ) \mapsto (a_2, a_3, \dots)$.
    
    When $x \geq 0.5$, it is clear that $a_1 = 1$. We have \[
    2 - 2x = \left(1 + \frac{1}{2} + \frac{1}{2^2} + \cdots\right) - \left(1 + \frac{a_2}{2} + \frac{a_3}{2^2} + \cdots \right) = \frac{1-a_2}{2} + \frac{1-a_3}{2} + \cdots
    \]
    which corresponds to sequence $(1-a_2, 1-a_3, \cdots)$ (sequence shifts left then subtracted from one-sequence).

    Now, we also freely use the fact that if $x$ is rational then the sequence eventually repeats itself, i.e. there exists $N, k \in \bbn$ such that $\forall n \geq N, a_{x, n} = a_{x, n+k}$. Otherwise, the sequence doesn't.

    Let $x$ be such a rational, then there exists fixed $N, k \in \bbn$ satisfying the conditions above.

    Then, after $N-1$ applications of $f$, from the only 2 possible transitions above, we either get the sequence \[
    A_1 = (a_N, a_{N+1}, \cdots)
    \]
    or \[
    A_2 = (1 - a_N, 1 - a_{N+1}, \cdots)
    \]
    
    WLOG, suppose we get $A_1$. $k$ steps later, we either get $A_1$ or $A_2$. 
    
    If we get $A_1$, that means we have entered a loop, and so values $f^{N-1 + kj}(x)$ for all $j \in \bbn$ have already appeared as $f^{N-1 + k}(x)$ in the orbit of $x$. Thus the orbit is finite ($\leq N-1 + k$).

    If we get $A_2$, $k$ steps later, we either get $A_1$ or $A_2$. It's trivial now that there is now a loop of either length $k$ or length $2k$. So the orbit is finite. \qed

    \textbf{(f)} It is clear that the orbit of $x$ is countable. If $x$ is irrational, then we claim that its orbit of $x$ is a denumerable (infinite countable) subset of $[0, 1]$.

    Suppose not. Then the orbit is finite. Let $N$ be the maximum index such that $f^N(x)$ is in the orbit of $x$. This implies that \[
    f^{N+1}(x) = f^k(x)
    \]
    for some $k \leq N$. Then $f^{N+2}(x) = f^{k+1}(x), \dots, f^{2N+1-k}(x) = f^N(x), \dots$

    In short, 
    \begin{equation} \label{loop}
        f^{k + (N+1 - k)t}(x) = f^k(x)
    \end{equation}
    for all $t \in \bbn$.

    Let $f^k(x) = (b_{k+1}, b_{k+2}, \dots)$ ($b_i$ can either all be $a_i$ or all be $1-a_i$, this doesn't matter). Then by referring to the 2 possible transitions we have as represented above in part (e), after $2(N+1-k)t, t \in \bbn$ transitions (an even number), we must reach \[
    f^{k + 2(N+1-k)t}(x) = (b_{k + 2(N+1-k)t + 1}, b_{k + 2(N+1-k)t + 2}, \dots)
    \]
    or more specifically, not $(1 - b_{k + 2(N+1-k)t + 1}, 1 - b_{k + 2(N+1-k)t + 2}, \dots)$, since the number of transitions is even.

    Then \eqref{loop} implies that \[
    b_{k+1} = b_{k+2(N+1-k)t + 1}, b_{k+1} = b_{k+2(N+1-k)t + 1}, \dots
    \]
    which is to suggest that $\{b_n\}$ eventually repeats, with maximum period of $2(N+1-k)$.

    This implies that $x$ is rational. \contra

    So the orbit of $x$ must be an infinite countable set (subset of [0, 1])
\end{solution}
\begin{problem} [2.81* (EC)]
The topologist's sine curve is the set \[
    \{(x, y) \mid x = 0 \:\text{and}\:  |y| \leq 1 \:\text{or}\: 0 < x < 1 \:\text{and}\:  y = \sin (\frac{1}{x})\}
\]
(It is the union of a circular arc and the topologist's sine curve.) Prove that it is path-connected but not locally path-connected. ($M$ is \textbf{locally path-connected} if for each $p \in M$ and each neighborhood $U$ of $p$ there is a path-connected subneighborhood $V$ of $p$.)
\end{problem}
\begin{solution}
\end{solution}
\end{document}