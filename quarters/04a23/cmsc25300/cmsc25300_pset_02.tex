\documentclass[a4paper, 12pt]{article}
%%%%%%%%%%%%%%%%%%%%%%%%%%%%%%%%%%%%%%%%%%%%%%%%%%%%%%%%%%%%%%%%%%%%%%%%%%%%%%%
%                                Basic Packages                               %
%%%%%%%%%%%%%%%%%%%%%%%%%%%%%%%%%%%%%%%%%%%%%%%%%%%%%%%%%%%%%%%%%%%%%%%%%%%%%%%

% Gives us multiple colors.
\usepackage[usenames,dvipsnames,pdftex]{xcolor}
% Lets us style link colors.
\usepackage{hyperref}
% Lets us import images and graphics.
\usepackage{graphicx}
% Lets us use figures in floating environments.
\usepackage{float}
% Lets us create multiple columns.
\usepackage{multicol}
% Gives us better math syntax.
\usepackage{amsmath,amsfonts,mathtools,amsthm,amssymb}
% Lets us strikethrough text.
\usepackage{cancel}
% Lets us edit the caption of a figure.
\usepackage{caption}
% Lets us import pdf directly in our tex code.
\usepackage{pdfpages}
% Lets us do algorithm stuff.
\usepackage[ruled,vlined,linesnumbered]{algorithm2e}
% Use a smiley face for our qed symbol.
\usepackage{tikzsymbols}
% \usepackage{fullpage} %%smaller margins
\usepackage[shortlabels]{enumitem}

\usepackage{setspace}
\usepackage[margin=1in, headsep=12pt]{geometry}
\usepackage{wrapfig}
\def\class{article}


%%%%%%%%%%%%%%%%%%%%%%%%%%%%%%%%%%%%%%%%%%%%%%%%%%%%%%%%%%%%%%%%%%%%%%%%%%%%%%%
%                                Basic Settings                               %
%%%%%%%%%%%%%%%%%%%%%%%%%%%%%%%%%%%%%%%%%%%%%%%%%%%%%%%%%%%%%%%%%%%%%%%%%%%%%%%

%%%%%%%%%%%%%
%  Symbols  %
%%%%%%%%%%%%%

\let\implies\Rightarrow
\let\impliedby\Leftarrow
\let\iff\Leftrightarrow
\let\epsilon\varepsilon
%%%%%%%%%%%%
%  Tables  %
%%%%%%%%%%%%

\setlength{\tabcolsep}{5pt}
\renewcommand\arraystretch{1.5}

%%%%%%%%%%%%%%
%  SI Unitx  %
%%%%%%%%%%%%%%

\usepackage{siunitx}
\sisetup{locale = FR}

%%%%%%%%%%
%  TikZ  %
%%%%%%%%%%

\usepackage[framemethod=TikZ]{mdframed}
\usepackage{tikz}
\usepackage{tikz-cd}
\usepackage{tikzsymbols}

\usetikzlibrary{intersections, angles, quotes, calc, positioning}
\usetikzlibrary{arrows.meta}

\tikzset{
    force/.style={thick, {Circle[length=2pt]}-stealth, shorten <=-1pt}
}

%%%%%%%%%%%%%%%
%  PGF Plots  %
%%%%%%%%%%%%%%%

\usepackage{pgfplots}
\pgfplotsset{width=10cm, compat=newest}

%%%%%%%%%%%%%%%%%%%%%%%
%  Center Title Page  %
%%%%%%%%%%%%%%%%%%%%%%%

\usepackage{titling}
\renewcommand\maketitlehooka{\null\mbox{}\vfill}
\renewcommand\maketitlehookd{\vfill\null}

%%%%%%%%%%%%%%%%%%%%%%%%%%%%%%%%%%%%%%%%%%%%%%%%%%%%%%%
%  Create a grey background in the middle of the PDF  %
%%%%%%%%%%%%%%%%%%%%%%%%%%%%%%%%%%%%%%%%%%%%%%%%%%%%%%%

\usepackage{eso-pic}
\newcommand\definegraybackground{
    \definecolor{reallylightgray}{HTML}{FAFAFA}
    \AddToShipoutPicture{
        \ifthenelse{\isodd{\thepage}}{
            \AtPageLowerLeft{
                \put(\LenToUnit{\dimexpr\paperwidth-222pt},0){
                    \color{reallylightgray}\rule{222pt}{297mm}
                }
            }
        }
        {
            \AtPageLowerLeft{
                \color{reallylightgray}\rule{222pt}{297mm}
            }
        }
    }
}

%%%%%%%%%%%%%%%%%%%%%%%%
%  Modify Links Color  %
%%%%%%%%%%%%%%%%%%%%%%%%

\hypersetup{
    % Enable highlighting links.
    colorlinks,
    % Change the color of links to blue.
    urlcolor=blue,
    % Change the color of citations to black.
    citecolor={black},
    % Change the color of url's to blue with some black.
    linkcolor={blue!80!black}
}

%%%%%%%%%%%%%%%%%%
% Fix WrapFigure %
%%%%%%%%%%%%%%%%%%

\newcommand{\wrapfill}{\par\ifnum\value{WF@wrappedlines}>0
        \parskip=0pt
        \addtocounter{WF@wrappedlines}{-1}%
        \null\vspace{\arabic{WF@wrappedlines}\baselineskip}%
        \WFclear
    \fi}

%%%%%%%%%%%%%%%%%
% Multi Columns %
%%%%%%%%%%%%%%%%%

\let\multicolmulticols\multicols
\let\endmulticolmulticols\endmulticols

\RenewDocumentEnvironment{multicols}{mO{}}
{%
    \ifnum#1=1
        #2%
    \else % More than 1 column
        \multicolmulticols{#1}[#2]
    \fi
}
{%
    \ifnum#1=1
    \else % More than 1 column
        \endmulticolmulticols
    \fi
}

\newlength{\thickarrayrulewidth}
\setlength{\thickarrayrulewidth}{5\arrayrulewidth}


%%%%%%%%%%%%%%%%%%%%%%%%%%%%%%%%%%%%%%%%%%%%%%%%%%%%%%%%%%%%%%%%%%%%%%%%%%%%%%%
%                           School Specific Commands                          %
%%%%%%%%%%%%%%%%%%%%%%%%%%%%%%%%%%%%%%%%%%%%%%%%%%%%%%%%%%%%%%%%%%%%%%%%%%%%%%%

%%%%%%%%%%%%%%%%%%%%%%%%%%%
%  Initiate New Counters  %
%%%%%%%%%%%%%%%%%%%%%%%%%%%

\newcounter{lecturecounter}

%%%%%%%%%%%%%%%%%%%%%%%%%%
%  Helpful New Commands  %
%%%%%%%%%%%%%%%%%%%%%%%%%%

\makeatletter

\newcommand\resetcounters{
    % Reset the counters for subsection, subsubsection and the definition
    % all the custom environments.
    \setcounter{subsection}{0}
    \setcounter{subsubsection}{0}
    \setcounter{paragraph}{0}
    \setcounter{subparagraph}{0}
    \setcounter{theorem}{0}
    \setcounter{claim}{0}
    \setcounter{corollary}{0}
    \setcounter{lemma}{0}
    \setcounter{exercise}{0}

    \@ifclasswith\class{nocolor}{
        \setcounter{definition}{0}
    }{}
}

%%%%%%%%%%%%%%%%%%%%%
%  Lecture Command  %
%%%%%%%%%%%%%%%%%%%%%

\usepackage{xifthen}

% EXAMPLE:
% 1. \lecture{Oct 17 2022 Mon (08:46:48)}{Lecture Title}
% 2. \lecture[4]{Oct 17 2022 Mon (08:46:48)}{Lecture Title}
% 3. \lecture{Oct 17 2022 Mon (08:46:48)}{}
% 4. \lecture[4]{Oct 17 2022 Mon (08:46:48)}{}
% Parameters:
% 1. (Optional) lecture number.
% 2. Time and date of lecture.
% 3. Lecture Title.
\def\@lecture{}
\def\@lectitle{}
\def\@leccount{}
\newcommand\lecture[3]{
    %   % Add 1 to the lecture counter.
    %   \addtocounter{lecturecounter}{1}

    % Set the section number to the lecture counter.
    \setcounter{section}{#1}
    \renewcommand\thesubsection{#1.\arabic{subsection}}

    % Reset the counters.
    \resetcounters

    % Check if user passed the lecture title or not.
    \def\@leccount{Lecture #1}
    \ifthenelse{\isempty{#3}}{
        \def\@lecture{Lecture #1}
    }{
        \def\@lecture{Lecture #1: #3}
        \def\@lectitle{#3}
    }

    % Display the information like the following:
    %                                                  Oct 17 2022 Mon (08:49:10)
    % ---------------------------------------------------------------------------
    % Lecture 1: Lecture Title
    \newpage
    \begin{mdframed}
        % \section*{\@lecture}
        \begin{center}
            \Large \textbf{\@leccount}

            \vspace*{0.2cm}

            \large\@lectitle

            \vspace*{0.2cm}
            \normalsize #2
        \end{center}
    \end{mdframed}
    \addcontentsline{toc}{section}{\@lecture}
}

%%%%%%%%%%%%%%%%%%%%
%  Import Figures  %
%%%%%%%%%%%%%%%%%%%%

\usepackage{import}
\pdfminorversion=7

% EXAMPLE:
% 1. \incfig{limit-graph}
% 2. \incfig[0.4]{limit-graph}
% Parameters:
% 1. The figure name. It should be located in figures/NAME.tex_pdf.
% 2. (Optional) The width of the figure. Example: 0.5, 0.35.
\newcommand\incfig[2][1]{%
    \def\svgwidth{#1\columnwidth}
    \import{./figures/}{#2.pdf_tex}
}

\begingroup\expandafter\expandafter\expandafter\endgroup
\expandafter\ifx\csname pdfsuppresswarningpagegroup\endcsname\relax
\else
    \pdfsuppresswarningpagegroup=1\relax
\fi

%%%%%%%%%%%%%%%%%
% Fancy Headers %
%%%%%%%%%%%%%%%%%

\usepackage{fancyhdr}

% Force a new page.
\newcommand\forcenewpage{\clearpage\mbox{~}\clearpage\newpage}

% This command makes it easier to manage my headers and footers.
\newcommand\createintro{
    % Use roman page numbers (e.g. i, v, vi, x, ...)
    \pagenumbering{roman}

    % Display the page style.
    \maketitle
    % Make the title pagestyle empty, meaning no fancy headers and footers.
    \thispagestyle{empty}
    % Create a newpage.
    \newpage

    % Input the intro.tex page if it exists.
    \IfFileExists{intro.tex}{ % If the intro.tex file exists.
        % Input the intro.tex file.
        \textbf{Course}: COURSE

\textbf{Section}: SECTION

\textbf{Professor}: PROFESSOR

\textbf{At}: The University of Chicago

\textbf{Quarter}: QUARTER

\textbf{Course materials}: COURSE_MATERIALS

\vspace{1cm}
\textbf{Disclaimer}: This document will inevitably contain some mistakes, both simple typos and serious logical and mathematical errors. Take what you read with a grain of salt as it is made by an undergraduate student going through the learning process himself. If you do find any error, I would really appreciate it if you can let me know by email at \href{mailto:conghungletran@gmail.com}{conghungletran@gmail.com}.

        % Make the pagestyle fancy for the intro.tex page.
        \pagestyle{fancy}

        % Remove the line for the header.
        \renewcommand\headrulewidth{0pt}

        % Remove all header stuff.
        \fancyhead{}

        % Add stuff for the footer in the center.
        % \fancyfoot[C]{
        %   \textit{For more notes like this, visit
        %   \href{\linktootherpages}{\shortlinkname}}. \\
        %   \vspace{0.1cm}
        %   \hrule
        %   \vspace{0.1cm}
        %   \@author, \\
        %   \term: \academicyear, \\
        %   Last Update: \@date, \\
        %   \faculty
        % }
    }{ % If the intro.tex file doesn't exist.
        % Force a \newpageage.
        \forcenewpage
    }

    % Create a new page.
    \newpage

    % Remove the center stuff we did above, and replace it with just the page
    % number, which is still in roman numerals.
    \fancyfoot[C]{\thepage}
    % Add the table of contents.
    \tableofcontents
    % Force a new page.
    \forcenewpage

    % Move the page numberings back to arabic, from roman numerals.
    \pagenumbering{arabic}
    % Set the page number to 1.
    \setcounter{page}{1}

    % Add the header line back.
    \renewcommand\headrulewidth{0.4pt}
    % In the top right, add the lecture title.
    \fancyhead[R]{\footnotesize \@lecture}
    % In the top left, add the author name.
    \fancyhead[L]{\footnotesize \@author}
    % In the bottom center, add the page.
    \fancyfoot[C]{\thepage}
    % Add a nice gray background in the middle of all the upcoming pages.
    % \definegraybackground
}

\makeatother


%%%%%%%%%%%%%%%%%%%%%%%%%%%%%%%%%%%%%%%%%%%%%%%%%%%%%%%%%%%%%%%%%%%%%%%%%%%%%%%
%                               Custom Commands                               %
%%%%%%%%%%%%%%%%%%%%%%%%%%%%%%%%%%%%%%%%%%%%%%%%%%%%%%%%%%%%%%%%%%%%%%%%%%%%%%%

%%%%%%%%%%%%
%  Circle  %
%%%%%%%%%%%%

\newcommand*\circled[1]{\tikz[baseline=(char.base)]{
        \node[shape=circle,draw,inner sep=1pt] (char) {#1};}
}

%%%%%%%%%%%%%%%%%%%
%  Todo Commands  %
%%%%%%%%%%%%%%%%%%%

% \usepackage{xargs}
% \usepackage[colorinlistoftodos]{todonotes}

% \makeatletter

% \@ifclasswith\class{working}{
%     \newcommandx\unsure[2][1=]{\todo[linecolor=red,backgroundcolor=red!25,bordercolor=red,#1]{#2}}
%     \newcommandx\change[2][1=]{\todo[linecolor=blue,backgroundcolor=blue!25,bordercolor=blue,#1]{#2}}
%     \newcommandx\info[2][1=]{\todo[linecolor=OliveGreen,backgroundcolor=OliveGreen!25,bordercolor=OliveGreen,#1]{#2}}
%     \newcommandx\improvement[2][1=]{\todo[linecolor=Plum,backgroundcolor=Plum!25,bordercolor=Plum,#1]{#2}}

%     \newcommand\listnotes{
%         \newpage
%         \listoftodos[Notes]
%     }
% }{
%     \newcommandx\unsure[2][1=]{}
%     \newcommandx\change[2][1=]{}
%     \newcommandx\info[2][1=]{}
%     \newcommandx\improvement[2][1=]{}

%     \newcommand\listnotes{}
% }

% \makeatother

%%%%%%%%%%%%%
%  Correct  %
%%%%%%%%%%%%%

% EXAMPLE:
% 1. \correct{INCORRECT}{CORRECT}
% Parameters:
% 1. The incorrect statement.
% 2. The correct statement.
\definecolor{correct}{HTML}{009900}
\newcommand\correct[2]{{\color{red}{#1 }}\ensuremath{\to}{\color{correct}{ #2}}}


%%%%%%%%%%%%%%%%%%%%%%%%%%%%%%%%%%%%%%%%%%%%%%%%%%%%%%%%%%%%%%%%%%%%%%%%%%%%%%%
%                                 Environments                                %
%%%%%%%%%%%%%%%%%%%%%%%%%%%%%%%%%%%%%%%%%%%%%%%%%%%%%%%%%%%%%%%%%%%%%%%%%%%%%%%

\usepackage{varwidth}
\usepackage{thmtools}
\usepackage[most,many,breakable]{tcolorbox}

\tcbuselibrary{theorems,skins,hooks}
\usetikzlibrary{arrows,calc,shadows.blur}

%%%%%%%%%%%%%%%%%%%
%  Define Colors  %
%%%%%%%%%%%%%%%%%%%

\definecolor{myblue}{RGB}{45, 111, 177}
\definecolor{mygreen}{RGB}{56, 140, 70}
\definecolor{myred}{RGB}{199, 68, 64}
\definecolor{mypurple}{RGB}{197, 92, 212}

% ESSENTIALS: 
\definecolor{definition_color}{HTML}{c74540}

\definecolor{theorem_color}{HTML}{00007B}
\colorlet{proof_color}{theorem_color}
\colorlet{prop_color}{theorem_color}

\colorlet{corollary_color}{mypurple!85!black}

\definecolor{lemma_color}{HTML}{983b0f}

\definecolor{example_color}{HTML}{2A7F7F}
\colorlet{exercise_color}{example_color}

\colorlet{claim_color}{mygreen!85!black}

% MISCS: 
%%%%%%%%%%%%%%%%%%%%%%%%%%%%%%%%%%%%%%%%%%%%%%%%%%%%%%%%%
%  Create Environments Styles Based on Given Parameter  %
%%%%%%%%%%%%%%%%%%%%%%%%%%%%%%%%%%%%%%%%%%%%%%%%%%%%%%%%%

\mdfsetup{skipabove=1em,skipbelow=0em}

%%%%%%%%%%%%%%%%%%%%%%
%  Helpful Commands  %
%%%%%%%%%%%%%%%%%%%%%%

% EXAMPLE:
% 1. \createnewtheoremstyle{thmdefinitionbox}{}{}
% 2. \createnewtheoremstyle{thmtheorembox}{}{}
% 3. \createnewtheoremstyle{thmproofbox}{qed=\qedsymbol}{
%       rightline=false, topline=false, bottomline=false
%    }
% Parameters:
% 1. Theorem name.
% 2. Any extra parameters to pass directly to declaretheoremstyle.
% 3. Any extra parameters to pass directly to mdframed.
\newcommand\createnewtheoremstyle[3]{
    \declaretheoremstyle[
        headfont=\bfseries\sffamily, bodyfont=\normalfont, #2,
        mdframed={
                #3,
            },
    ]{#1}
}

% EXAMPLE:
% 1. \createnewcoloredtheoremstyle{thmdefinitionbox}{definition}{}{}
% 2. \createnewcoloredtheoremstyle{thmexamplebox}{example}{}{
%       rightline=true, leftline=true, topline=true, bottomline=true
%     }
% 3. \createnewcoloredtheoremstyle{thmproofbox}{proof}{qed=\qedsymbol}{backgroundcolor=white}
% Parameters:
% 1. Theorem name.
% 2. Color of theorem.
% 3. Any extra parameters to pass directly to declaretheoremstyle.
% 4. Any extra parameters to pass directly to mdframed.

% change backgroundcolor to #2!5 if user wants a colored backdrop to theorem environments. It's a cool color theme, but there's too much going on in the page.
\newcommand\createnewcoloredtheoremstyle[4]{
    \declaretheoremstyle[
        headfont=\bfseries\sffamily\color{#2}, bodyfont=\normalfont, #3,
        mdframed={
                linewidth=2pt,
                rightline=false, leftline=true, topline=false, bottomline=false,
                linecolor=#2, backgroundcolor=white, #4,
            },
    ]{#1}
}



%%%%%%%%%%%%%%%%%%%%%%%%%%%%%%%%%%%
%  Create the Environment Styles  %
%%%%%%%%%%%%%%%%%%%%%%%%%%%%%%%%%%%

\makeatletter
\@ifclasswith\class{nocolor}{
    % Environments without color.

    % ESSENTIALS:
    \createnewtheoremstyle{thmdefinitionbox}{}{}
    \createnewtheoremstyle{thmtheorembox}{}{}
    \createnewtheoremstyle{thmproofbox}{qed=\qedsymbol}{}
    \createnewtheoremstyle{thmcorollarybox}{}{}
    \createnewtheoremstyle{thmlemmabox}{}{}
    \createnewtheoremstyle{thmclaimbox}{}{}
    \createnewtheoremstyle{thmexamplebox}{}{}

    % MISCS: 
    \createnewtheoremstyle{thmpropbox}{}{}
    \createnewtheoremstyle{thmexercisebox}{}{}
    \createnewtheoremstyle{thmexplanationbox}{}{}
    \createnewtheoremstyle{thmremarkbox}{}{}

    % STYLIZED MORE BELOW
    \createnewtheoremstyle{thmquestionbox}{}{}
    \createnewtheoremstyle{thmsolutionbox}{qed=\qedsymbol}{}
}{
    % Environments with color.

    % ESSENTIALS: definition, theorem, proof, corollary, lemma, claim, example
    \createnewcoloredtheoremstyle{thmdefinitionbox}{definition_color}{}{
        leftline=true
    }
    \createnewcoloredtheoremstyle{thmtheorembox}{theorem_color}{}{
        leftline=true
    }
    \createnewcoloredtheoremstyle{thmproofbox}{proof_color}{qed=\qedsymbol}{backgroundcolor=white}
    \createnewcoloredtheoremstyle{thmcorollarybox}{corollary_color}{}{backgroundcolor=white}
    \createnewcoloredtheoremstyle{thmlemmabox}{lemma_color}{}{backgroundcolor=white}
    \createnewcoloredtheoremstyle{thmclaimbox}{claim_color}{}{}
    \createnewcoloredtheoremstyle{thmexamplebox}{example_color}{}{backgroundcolor=white}
    \createnewcoloredtheoremstyle{thmexplanationbox}{example_color}{qed=\qedsymbol}{backgroundcolor=white}
    \createnewcoloredtheoremstyle{thmremarkbox}{theorem_color}{}{backgroundcolor=white}

    \createnewcoloredtheoremstyle{thmmiscbox}{black}{}{leftline=false,backgroundcolor=white}


    \createnewcoloredtheoremstyle{thmpropbox}{prop_color}{}{backgroundcolor=white}
    \createnewcoloredtheoremstyle{thmexercisebox}{exercise_color}{}{backgroundcolor=white}

    \createnewcoloredtheoremstyle{thmproblembox}{myred}{}{backgroundcolor=white}
    \createnewcoloredtheoremstyle{thmsolutionbox}{mygreen}{qed=\qedsymbol}{backgroundcolor=white}
}
\makeatother

%%%%%%%%%%%%%%%%%%%%%%%%%%%%%
%  Create the Environments  %
%%%%%%%%%%%%%%%%%%%%%%%%%%%%%
\declaretheorem[numberwithin=section, style=thmdefinitionbox,     name=Definition]{definition0}
\declaretheorem[numberwithin=section, style=thmtheorembox,     name=Theorem]{theorem}
\declaretheorem[numbered=no,          style=thmexamplebox,     name=Example]{example}
\declaretheorem[numberwithin=section, style=thmclaimbox,       name=Claim]{claim}
\declaretheorem[numberwithin=section, style=thmcorollarybox,   name=Corollary]{corollary}
\declaretheorem[numberwithin=section, style=thmpropbox,        name=Proposition]{proposition}
\declaretheorem[numberwithin=section, style=thmlemmabox,       name=Lemma]{lemma}
\declaretheorem[numberwithin=section, style=thmexercisebox,    name=Exercise]{exercise}
\declaretheorem[numbered=no,          style=thmproofbox,       name=Proof]{replacementproof}
\declaretheorem[numbered=no,          style=thmexplanationbox, name=Explanation]{explanation}
\declaretheorem[numbered=no,          style=thmsolutionbox,    name=Solution]{solution}
\declaretheorem[numberwithin=section,          style=thmproblembox,     name=Problem]{problem}
\declaretheorem[numbered=no,          style=thmmiscbox,    name=Intuition]{intuition}
\declaretheorem[numbered=no,          style=thmmiscbox,    name=Goal]{goal}
\declaretheorem[numbered=no,          style=thmmiscbox,    name=Recall]{recall}
\declaretheorem[numbered=no,          style=thmmiscbox,    name=Motivation]{motivation}
\declaretheorem[numbered=no,          style=thmmiscbox,    name=Remark]{remark}
\declaretheorem[numbered=no,          style=thmmiscbox,    name=Observe]{observe}




%%%% FANCY GRAPHICS:

% \makeatletter
% \@ifclasswith\class{nocolor}{
%   % Environments without color.

%   \newtheorem*{note}{Note}

%   \declaretheorem[numberwithin=section, style=thmdefinitionbox, name=Definition]{definition}
%   \declaretheorem[numberwithin=section, style=thmquestionbox,   name=Question]{question}
%   \declaretheorem[numberwithin=section, style=thmsolutionbox,   name=Solution]{solution}
% }{
%   % Environments with color.

%   \newtcbtheorem[number within=section]{Definition}{Definition}{
%     enhanced,
%     before skip=2mm,
%     after skip=2mm,
%     colback=red!5,
%     colframe=red!80!black,
%     colbacktitle=red!75!black,
%     boxrule=0.5mm,
%     attach boxed title to top left={
%       xshift=1cm,
%       yshift*=1mm-\tcboxedtitleheight
%     },
%     varwidth boxed title*=-3cm,
%     boxed title style={
%       interior engine=empty,
%       frame code={
%         \path[fill=tcbcolback]
%         ([yshift=-1mm,xshift=-1mm]frame.north west)
%         arc[start angle=0,end angle=180,radius=1mm]
%         ([yshift=-1mm,xshift=1mm]frame.north east)
%         arc[start angle=180,end angle=0,radius=1mm];
%         \path[left color=tcbcolback!60!black,right color=tcbcolback!60!black,
%         middle color=tcbcolback!80!black]
%         ([xshift=-2mm]frame.north west) -- ([xshift=2mm]frame.north east)
%         [rounded corners=1mm]-- ([xshift=1mm,yshift=-1mm]frame.north east)
%         -- (frame.south east) -- (frame.south west)
%         -- ([xshift=-1mm,yshift=-1mm]frame.north west)
%         [sharp corners]-- cycle;
%       },
%     },
%     fonttitle=\bfseries,
%     title={#2},
%     #1
%   }{def}

%   \NewDocumentEnvironment{definition}{O{}O{}}
%     {\begin{Definition}{#1}{#2}}{\end{Definition}}

%   \newtcolorbox{note}[1][]{%
%     enhanced jigsaw,
%     colback=gray!20!white,%
%     colframe=gray!80!black,
%     size=small,
%     boxrule=1pt,
%     title=\textbf{Note:-},
%     halign title=flush center,
%     coltitle=black,
%     breakable,
%     drop shadow=black!50!white,
%     attach boxed title to top left={xshift=1cm,yshift=-\tcboxedtitleheight/2,yshifttext=-\tcboxedtitleheight/2},
%     minipage boxed title=1.5cm,
%     boxed title style={%
%       colback=white,
%       size=fbox,
%       boxrule=1pt,
%       boxsep=2pt,
%       underlay={%
%         \coordinate (dotA) at ($(interior.west) + (-0.5pt,0)$);
%         \coordinate (dotB) at ($(interior.east) + (0.5pt,0)$);
%         \begin{scope}
%           \clip (interior.north west) rectangle ([xshift=3ex]interior.east);
%           \filldraw [white, blur shadow={shadow opacity=60, shadow yshift=-.75ex}, rounded corners=2pt] (interior.north west) rectangle (interior.south east);
%         \end{scope}
%         \begin{scope}[gray!80!black]
%           \fill (dotA) circle (2pt);
%           \fill (dotB) circle (2pt);
%         \end{scope}
%       },
%     },
%     #1,
%   }

%   \newtcbtheorem{Question}{Question}{enhanced,
%     breakable,
%     colback=white,
%     colframe=myblue!80!black,
%     attach boxed title to top left={yshift*=-\tcboxedtitleheight},
%     fonttitle=\bfseries,
%     title=\textbf{Question:-},
%     boxed title size=title,
%     boxed title style={%
%       sharp corners,
%       rounded corners=northwest,
%       colback=tcbcolframe,
%       boxrule=0pt,
%     },
%     underlay boxed title={%
%       \path[fill=tcbcolframe] (title.south west)--(title.south east)
%       to[out=0, in=180] ([xshift=5mm]title.east)--
%       (title.center-|frame.east)
%       [rounded corners=\kvtcb@arc] |-
%       (frame.north) -| cycle;
%     },
%     #1
%   }{def}

%   \NewDocumentEnvironment{question}{O{}O{}}
%   {\begin{Question}{#1}{#2}}{\end{Question}}

%   \newtcolorbox{Solution}{enhanced,
%     breakable,
%     colback=white,
%     colframe=mygreen!80!black,
%     attach boxed title to top left={yshift*=-\tcboxedtitleheight},
%     title=\textbf{Solution:-},
%     boxed title size=title,
%     boxed title style={%
%       sharp corners,
%       rounded corners=northwest,
%       colback=tcbcolframe,
%       boxrule=0pt,
%     },
%     underlay boxed title={%
%       \path[fill=tcbcolframe] (title.south west)--(title.south east)
%       to[out=0, in=180] ([xshift=5mm]title.east)--
%       (title.center-|frame.east)
%       [rounded corners=\kvtcb@arc] |-
%       (frame.north) -| cycle;
%     },
%   }

%   \NewDocumentEnvironment{solution}{O{}O{}}
%   {\vspace{-10pt}\begin{Solution}{#1}{#2}}{\end{Solution}}
% }
% \makeatother


%%%%% END OF FANCY GRAPHICS %%%%%%%%%%



%%%%%%%%%%%%%%%%%%%%%%%%%%%%
%  Edit Proof Environment  %
%%%%%%%%%%%%%%%%%%%%%%%%%%%%

\renewenvironment{proof}[2][\proofname]{
    % \vspace{-12pt}
    \begin{replacementproof} [#2]
}{\end{replacementproof}}

\newenvironment{definition}[1]{
    \begin{definition0}[#1]

    \hfill
        
    \vspace{0.2cm}

}{

    \vspace{0.2cm}
    \end{definition0}
}


\theoremstyle{definition}

\newtheorem*{notation}{Notation}
\newtheorem*{previouslyseen}{As previously seen}
\newtheorem*{property}{Property}
% \newtheorem*{intuition}{Intuition}
% \newtheorem*{goal}{Goal}
% \newtheorem*{recall}{Recall}
% \newtheorem*{motivation}{Motivation}
% \newtheorem*{remark}{Remark}
% \newtheorem*{observe}{Observe}

\author{Cong Hung Le Tran}


%%%% MATH SHORTHANDS %%%%
%% blackboard bold math capitals
\newcommand{\bbf}{\mathbb{F}}
\newcommand{\bbn}{\mathbb{N}}
\newcommand{\bbq}{\mathbb{Q}}
\newcommand{\bbr}{\mathbb{R}}
\newcommand{\bbz}{\mathbb{Z}}
\newcommand{\bbc}{\mathbb{C}}
\newcommand{\bbk}{\mathbb{K}}
\newcommand{\bbm}{\mathbb{M}}
\renewcommand{\phi}{\varphi}
\newcommand{\st}{\;\text{such that}\;}


% MATH 20250 %
\newcommand{\Hom}{\mathrm{Hom}}
\newcommand{\im}{\mathrm{im}}

% https://tex.stackexchange.com/questions/438612/space-between-exists-and-forall
\let\oldforall\forall
\renewcommand{\forall}{\;\oldforall\; }
\let\oldexist\exists
\renewcommand{\exists}{\;\oldexist\; }
\newcommand\existu{\;\oldexist!\: }


\renewcommand{\_}[1]{\underline{ #1 }}
\DeclarePairedDelimiter{\abs}{\lvert}{\rvert}
\DeclarePairedDelimiter{\norm}{\lVert}{\rVert}
\setlength\parindent{0pt}
\setlength{\headheight}{12.0pt}
\addtolength{\topmargin}{-12.0pt}


% Default skipping, change if you want more spacing
% \thinmuskip=3mu
% \medmuskip=4mu plus 2mu minus 4mu
% \thickmuskip=5mu plus 5mu



% \DeclareMathOperator{\ext}{ext}
% \DeclareMathOperator{\bridge}{bridge}
\title{CMSC 25300: Mathematical Foundations of ML \\ \large Problem Set 2}
\date{15 Oct 2023}
\author{Hung Le Tran}
\begin{document}
\maketitle
\setcounter{section}{2}
\begin{problem} [Problem 1]
\end{problem}
\begin{solution}
    (a)

    No, they are not linearly independent:
    \[
    (-2) \begin{bmatrix}
        1 \\ 3 \\ -1 \\ 2
    \end{bmatrix} + 1 \begin{bmatrix}
        0 \\ 2 \\ 1 \\ 4
    \end{bmatrix} + 1 \begin{bmatrix}
        1 \\ 4  \\ 0 \\ 2
    \end{bmatrix} - 1 \begin{bmatrix}
    -1 \\ 0 \\ 3 \\ 2
    \end{bmatrix} = 0
    \]

    (b)
    Since $\{\mathbf{x_1, x_2, x_3, x_4}\}$ are not linearly independent, we can have at most 3 linearly independent vectors in the column space.

    We want to show that $\{\mathbf{x_1, x_2, x_3}\}$ are indeed linearly independent.

    Suppose there there exists $\alpha_1, \alpha_2, \alpha_3$ such that \[
        \alpha_1 \begin{bmatrix}
            1 \\ 3 \\ -1 \\ 2
        \end{bmatrix} + \alpha_2 \begin{bmatrix}
            0 \\ 2 \\ 1 \\ 4
        \end{bmatrix} + \alpha_3 \begin{bmatrix}
            1 \\ 4  \\ 0 \\ 2
        \end{bmatrix} = 0
    \]
    then that implies \[
    \begin{cases}
        \begin{aligned}
            \alpha_1 + \alpha_3 &= 0 \\
            3 \alpha_1 + 2 \alpha_2 + 4 \alpha_3 &= 0 \\
            - \alpha_1 + \alpha_2 &= 0 \\
            2 \alpha_1 + 4 \alpha_2 + 2\alpha_3 &= 0\\
        \end{aligned}
    \end{cases}
    \]
    The first and third equation implies $\alpha_1 = \alpha_2 = -\alpha_3 \equiv a$. The third equation then implies \[
    0 = 3a + 2a - 4a = a \implies a = 0 \implies \alpha_1 = \alpha_2 = \alpha_3 = 0
    \]
    $\{\mathbf{x_1, x_2, x_3}\}$ are indeed linearly independent.

    (c) No. $\{\mathbf{x_1, x_2, x_3, x_4}\}$ are \textbf{NOT} linearly independent, one can replace any of $\mathbf{x_1, x_2}$ or $\mathbf{x_3}$ with $\mathbf{x_4}$ and linearly independent set in (c) would still be linearly independent.

    In short, any set of 3 column vectors of $\mathbf{X}$ is a linearly independent set.

    (d) Since the largest possible set of linearly independent column vectors of $\mathbf{X}$ has 3 elements, $Rank(\mathbf{X}) = 3$
\end{solution}

\begin{problem} [Problem 2]
\end{problem}
\begin{solution}
    (a) Yes, they are. If there exists $a_1, a_2$ such that \[
    a_1 \begin{bmatrix}
        1.4 \\
        -1.4\\
        1.4\\
        -1.4
    \end{bmatrix} + a_2 \begin{bmatrix}
    -1.4\\
    1.4\\
    1.4\\
    1.4
    \end{bmatrix} = 0
    \]
    then \[
    1.4 a_1 - 1.4a_2 = 0, 1.4a_1 + 1.4a_2 = 0
    \]
    which easily implies $a_1 = a_2 = 0$.

    (b) No, they are not.

    \[
    \begin{bmatrix}
        4 \\ 5 \\ 13
    \end{bmatrix} - 2 \begin{bmatrix}
        2 \\ 2 \\ 8
    \end{bmatrix} = \begin{bmatrix}
    0 \\ 1 \\ -3
    \end{bmatrix}
    \]

    (c) Yes, they are. If there exists $a_1, a_2, a_3$ such that \[
    a_1 \begin{bmatrix}
        -1 \\ 0 \\ 1
    \end{bmatrix} + a_2 \begin{bmatrix}
    1 \\ -1 \\ 0
    \end{bmatrix} + a_3 \begin{bmatrix}
    -1 \\ -1 \\ 0
    \end{bmatrix} = 0
    \]
    then \[
        \begin{cases}
        \begin{aligned}
        -a_1 + a_2 - a_3 &= 0 \\
        -a_2 -a_3 &= 0 \\
        a_1 &= 0
        \end{aligned}
        \end{cases}
    \]
    
    Substituting $a_1 = 0$ from the last equation into the first 2, it's easily to see that $a_2 = a_3 = 0$ too.

    (d) $Rank(\mathbf{X}) = 2$. The 2 column vectors of $\mathbf{X}$ are not linearly independent, because if they were, the second vector has to be a scaled version of the first column vector, which it clearly isn't ($\frac{6}{4} \neq \frac{14}{-6}$). This is also the maximum number of linearly independent column vectors for $\mathbf{X}$, so its rank must also be 2.
\end{solution}

\begin{problem} [Problem 3]
\end{problem}
\begin{solution}
(a)\[
f(\bfw) = (3\bfx)^T\bfw - 2\bfw^T \bfx = \bfx^T \bfw
\]
so \[
\nabla_\bfw f = \bfx
\]

(b) \[
f(\bfw) = (\bfw - 2\bfx)^T (\bfw - \bfx) = \bfw^T \bfw - 3\bfx^T \bfw + 2\bfx^T \bfx = (\bfw - 3 \bfx)^T \bfw + 2\bfx^T \bfx
\]
so \[
\nabla_\bfw f = \bfw - 3 \bfx
\]

(c) \[
f(\bfw) = \bfx^T \begin{bmatrix}
    2 & 3 \\
    5 & 7
\end{bmatrix} \bfw = \bfx^T \begin{bmatrix}
    2 & 5 \\
    3 & 7
\end{bmatrix}^T \bfw = \left(\begin{bmatrix}
    2 & 5 \\
    3 & 7
\end{bmatrix} \bfx \right)^T \bfw
\]
so \[
\nabla_\bfw f = \begin{bmatrix}
    2 & 5 \\
    3 & 7
\end{bmatrix} \bfx
\]

(d)
\[
f(\bfw) = \bfw^T \begin{bmatrix}
    1 & 4 \\
    4 & 16
\end{bmatrix} \bfw
\]

For $f$ to make sense in the first place, $\bfw$ must have shape $2 \times 1$.

Therefore \[
f(\bfw) = \begin{bmatrix}
w_1 & w_2  
\end{bmatrix}
\begin{bmatrix}
    1 & 4 \\
    4 & 16
\end{bmatrix} \begin{bmatrix}
w_1 \\
w_2
\end{bmatrix} = \begin{bmatrix}
    w_1 + 4w_2 \\
    4w_1 + 16w_2
\end{bmatrix} \begin{bmatrix}
w_1 \\
w_2
\end{bmatrix} = \begin{bmatrix}
    w_1^2 + 5w_1w_2 + 4w_2^2 \\
    4w_1^2 + 20w_1w_2 + 16w_2^2
\end{bmatrix}
\]
yielding
\[
\nabla_\bfw f(\bfw) = \begin{bmatrix}
    2w_1 + 5w_2 \\
    20w_1 + 32w_2
\end{bmatrix} = \begin{bmatrix}
    2 & 5 \\
    20 & 32
\end{bmatrix} \bfw
\]

(e)
\[
f(\bfw) = \bfw^T \begin{bmatrix}
    2 & 3\\
    4 & -1
\end{bmatrix} \bfw
\]

For $f$ to make sense in the first place, $\bfw$ must have shape $2 \times 1$.

Therefore \[
f(\bfw) = \begin{bmatrix}
w_1 & w_2  
\end{bmatrix}
\begin{bmatrix}
    2 & 3 \\
    4 & -1
\end{bmatrix} \begin{bmatrix}
w_1 \\
w_2
\end{bmatrix} = \begin{bmatrix}
    2w_1 + 3w_2 \\
    4w_1 - w_2
\end{bmatrix} \begin{bmatrix}
w_1 \\
w_2
\end{bmatrix} = \begin{bmatrix}
    2w_1^2 + 5w_1w_2 + 3w_2^2 \\
    4w_1^2 + 3w_1w_2 - w_2^2
\end{bmatrix}
\]
yielding
\[
\nabla_\bfw f(\bfw) = \begin{bmatrix}
    4w_1 + 5w_2 \\
    3w_1 - 2w_2
\end{bmatrix} = \begin{bmatrix}
    4 & 5\\
    3 & 2
\end{bmatrix} \bfw
\]

\end{solution}

\begin{problem} [Problem 4]
\end{problem}
\begin{solution}
(a)
\begin{lstlisting} [language=python]
import numpy as np
# #### Part a #####
# Load in training data and labels
# File available on Canvas
face_data_dict = np.load ("face_emotion_data.npz")
X = face_data_dict ["X"]
y = face_data_dict ["y"]
n , p = X.shape


# Solve the least - squares solution . weights is the array of
# weight coefficients
# TODO : find weights
weights = np.linalg.inv((X.T @ X)) @ (X.T) @ y

print(f"Part 4a. Found weights:\n {weights}")
\end{lstlisting}
(b)
We've found weight $\bfw$ from part (a). To classify a new face $\bfx_0$:
\[
\bfyhat(\bfx_0) = \begin{cases}
    +1 & \:\text{if}\: \bfx_0^T \bfw \geq 0 \\
    -1 & \:\text{if}\: \bfx_0^T \bfw < 0
\end{cases}
\]

(c)
\begin{lstlisting} [language=python]
    def lstsq_cv_err (features : np.ndarray , labels : np.ndarray , subset_count : int =8) -> float :
    """ Estimate the error of a least - squares classifier using cross - validation . Use subset_count different
    train / test splits with each subset acting as the holdout set once.
    
    Parameters :
        features ( np . ndarray ) : dataset features as a 2 D array with shape ( sample_count , feature_count )
        labels ( np . ndarray ) : dataset class labels (+1/ -1) as a 1 D array with length ( sample_count )
        subset_count ( int ) : number of subsets to divide the dataset into
    
    Note : assumes that subset_count divides the dataset evenly
   
    Returns :
    cls_err ( float ) : estimated classification errorrate of least - squares method
    """
    sample_count , feature_count = features.shape
    subset_size = sample_count // subset_count
    # Reshape arrays for easier subset - level manipulation
    features = features . reshape ( subset_count , subset_size ,
    feature_count )
    labels = labels . reshape ( subset_count , subset_size )
    subset_idcs = np . arange ( subset_count )
    train_set_size = ( subset_count - 1) * subset_size
    subset_err_counts = np . zeros ( subset_count )
    for i in range ( subset_count ) :
        # TODO : select relevant dataset
        # fit and evaluate a linear model ,
        # then store errors in subset_err_counts [ i ]
        X = np.row_stack([features[j] for j in subset_idcs if j != i])
        y = np.concatenate([labels[j] for j in subset_idcs if j != i])
        w = np.linalg.inv(np.dot(X.T, X)) @ (X.T) @ y
        yhat = np.dot(features[i], w) > 0
        ytrue = labels[i] > 0
        subset_err_counts[i] = np.sum(yhat ^ ytrue)
    # Average over the entire dataset to find the classification error
    cls_err = np . sum ( subset_err_counts ) / ( subset_count *
    subset_size )
    return cls_err
    # Run on the dataset with all features included
full_feat_cv_err = lstsq_cv_err ( face_features, face_labels)
print (f"Error estimate :{full_feat_cv_err *100:.3f}%")
\end{lstlisting}

(d) I would first find the weight using all 9 features. Then, I would remove the feature that has the smallest corresponding weight. This implies that that feature has the least correlation to the final prediction. Of course, this is given that all features are properly scaled to the same range/order of magnitude. Find the weight using the remaining 8 features. Continue doing this process until the error rate exceeds some acceptable threshold value.

(e)
\begin{lstlisting} [language=python]
def lstsq_cv_err_least_index (features : np.ndarray , labels : np.ndarray , subset_count : int =8) -> float :
    sample_count , feature_count = features.shape
    subset_size = sample_count // subset_count
    # Reshape arrays for easier subset - level manipulation
    features = features . reshape ( subset_count , subset_size ,
    feature_count )
    labels = labels . reshape ( subset_count , subset_size )
    subset_idcs = np . arange ( subset_count )
    train_set_size = ( subset_count - 1) * subset_size
    subset_err_counts = np . zeros ( subset_count )

    sum_w = np.zeros(feature_count)
    for i in range ( subset_count ) :
        X = np.row_stack([features[j] for j in subset_idcs if j != i])
        y = np.concatenate([labels[j] for j in subset_idcs if j != i])
        w = np.linalg.inv(np.dot(X.T, X)) @ (X.T) @ y
        
        sum_w += np.abs(w)
        
        yhat = np.dot(features[i], w) > 0
        ytrue = labels[i] > 0
        subset_err_counts[i] = np.sum(yhat ^ ytrue)
    # Average over the entire dataset to find the classification error
    cls_err = np . sum ( subset_err_counts ) / ( subset_count *
    subset_size )
    # returns feature index with least weight
    li = np.argmin(sum_w)
    return (cls_err, li)


err = 0
featuresNotUsed = set()
sample_count , feature_count = face_features.shape
while(err < 0.06 and len(featuresNotUsed) <= feature_count - 1):
    features_used = face_features
    features_used = np.array([[row[j] for j in range(feature_count) if j not in featuresNotUsed] for row in face_features])
    cls_err, li = lstsq_cv_err_least_index ( features_used, face_labels)
    print (f"Error estimate :{cls_err *100:.3f}%, features not used: {featuresNotUsed}, feature to remove (in sub array): {li}")
    
    # converting feature index in subarray to feature index in original array
    actualIndex = -1
    count = -1
    while(count < li):
        actualIndex += 1
        while(actualIndex in featuresNotUsed):
            actualIndex += 1
        count += 1
        
    featuresNotUsed.add(actualIndex)
    err = cls_err
\end{lstlisting}
which prints out 
\begin{lstlisting}
Error estimate :4.688%, features not used: set(), feature to remove (in sub array): 5
Error estimate :4.688%, features not used: {5}, feature to remove (in sub array): 4
Error estimate :4.688%, features not used: {4, 5}, feature to remove (in sub array): 5
Error estimate :4.688%, features not used: {4, 5, 7}, feature to remove (in sub array): 5
Error estimate :6.250%, features not used: {8, 4, 5, 7}, feature to remove (in sub array): 1
\end{lstlisting}

So my choice of features (to keep CV error rate below 6 \%) would be all features except for 5th, 6th and 8th feature (4th, 5th and 7th in 0-index).
\end{solution}

\begin{problem} [Problem 5]
\end{problem}
\begin{solution}
\mbox{}

\begin{lstlisting} [language=python]
import numpy as np
import matplotlib . pyplot as plt
# File available on Canvas
data = np.load ('polydata_2D.npz')
x1 = np.ravel(data['x1'])
x2 = np.ravel(data['x2'])
y = data ['y']
N = x1.size
p = np.zeros((3, N))
for d in [1,2,3]:
    # Generate the X matrix for this d
    # Find the least - squares weight matrix w_d
    # Evaluate the best - fit polynomial at each point ( x1 , x2 )
    # and store the result in the corresponding column of p
    # Plot the degree 1 surface
    X = None
    if d == 1:
        X = np.column_stack([x1, x2, np.ones(N)])
    elif d==2:
        X = np.column_stack([x1, x1*x1, x2, x2*x2, np.ones(N)])
    else:
        X = np.column_stack([x1, x1*x1, x1*x1*x1, x2, x2*x2, x2*x2*x2, np.ones(N)])

    w_d = np.linalg.inv(np.matmul(X.T, X)) @ (X.T) @ y
    p[d-1] = X @ w_d
    
Z1 = p [0 ,:].reshape(data['x1'].shape )
ax = plt.axes (projection = '3d')
ax.scatter(data ['x1'] , data ['x2'], y)
ax.plot_surface(data ['x1'] , data ['x2'] , Z1 , color = 'orange')
plt.show ()
# Plot the degree 2 surface
Z2 = p [1 ,:]. reshape  (data ['x1' ].shape)
ax = plt.axes (projection = '3d')
ax.scatter(data['x1'] , data ['x2'], y )
ax.plot_surface (data ['x1'] , data ['x2'] , Z2, color = 'orange')
plt.show ()
# Plot the degree 3 surface
Z3 = p [2 ,:].reshape (data ['x1'].shape )
ax = plt.axes (projection = '3d')
ax.scatter (data ['x1'] , data['x2'] , y )
ax.plot_surface (data ['x1'] , data ['x2'] , Z3 , color = 'orange')
plt.show ()
\end{lstlisting}
\end{solution}
\end{document}