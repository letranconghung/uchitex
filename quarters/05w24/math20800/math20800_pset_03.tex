\documentclass[a4paper, 12pt]{article}
%%%%%%%%%%%%%%%%%%%%%%%%%%%%%%%%%%%%%%%%%%%%%%%%%%%%%%%%%%%%%%%%%%%%%%%%%%%%%%%
%                                Basic Packages                               %
%%%%%%%%%%%%%%%%%%%%%%%%%%%%%%%%%%%%%%%%%%%%%%%%%%%%%%%%%%%%%%%%%%%%%%%%%%%%%%%

% Gives us multiple colors.
\usepackage[usenames,dvipsnames,pdftex]{xcolor}
% Lets us style link colors.
\usepackage{hyperref}
% Lets us import images and graphics.
\usepackage{graphicx}
% Lets us use figures in floating environments.
\usepackage{float}
% Lets us create multiple columns.
\usepackage{multicol}
% Gives us better math syntax.
\usepackage{amsmath,amsfonts,mathtools,amsthm,amssymb}
% Lets us strikethrough text.
\usepackage{cancel}
% Lets us edit the caption of a figure.
\usepackage{caption}
% Lets us import pdf directly in our tex code.
\usepackage{pdfpages}
% Lets us do algorithm stuff.
\usepackage[ruled,vlined,linesnumbered]{algorithm2e}
% Use a smiley face for our qed symbol.
\usepackage{tikzsymbols}
% \usepackage{fullpage} %%smaller margins
\usepackage[shortlabels]{enumitem}

\usepackage{setspace}
\usepackage[margin=1in, headsep=12pt]{geometry}
\usepackage{wrapfig}
\def\class{article}


%%%%%%%%%%%%%%%%%%%%%%%%%%%%%%%%%%%%%%%%%%%%%%%%%%%%%%%%%%%%%%%%%%%%%%%%%%%%%%%
%                                Basic Settings                               %
%%%%%%%%%%%%%%%%%%%%%%%%%%%%%%%%%%%%%%%%%%%%%%%%%%%%%%%%%%%%%%%%%%%%%%%%%%%%%%%

%%%%%%%%%%%%%
%  Symbols  %
%%%%%%%%%%%%%

\let\implies\Rightarrow
\let\impliedby\Leftarrow
\let\iff\Leftrightarrow
\let\epsilon\varepsilon
%%%%%%%%%%%%
%  Tables  %
%%%%%%%%%%%%

\setlength{\tabcolsep}{5pt}
\renewcommand\arraystretch{1.5}

%%%%%%%%%%%%%%
%  SI Unitx  %
%%%%%%%%%%%%%%

\usepackage{siunitx}
\sisetup{locale = FR}

%%%%%%%%%%
%  TikZ  %
%%%%%%%%%%

\usepackage[framemethod=TikZ]{mdframed}
\usepackage{tikz}
\usepackage{tikz-cd}
\usepackage{tikzsymbols}

\usetikzlibrary{intersections, angles, quotes, calc, positioning}
\usetikzlibrary{arrows.meta}

\tikzset{
    force/.style={thick, {Circle[length=2pt]}-stealth, shorten <=-1pt}
}

%%%%%%%%%%%%%%%
%  PGF Plots  %
%%%%%%%%%%%%%%%

\usepackage{pgfplots}
\pgfplotsset{width=10cm, compat=newest}

%%%%%%%%%%%%%%%%%%%%%%%
%  Center Title Page  %
%%%%%%%%%%%%%%%%%%%%%%%

\usepackage{titling}
\renewcommand\maketitlehooka{\null\mbox{}\vfill}
\renewcommand\maketitlehookd{\vfill\null}

%%%%%%%%%%%%%%%%%%%%%%%%%%%%%%%%%%%%%%%%%%%%%%%%%%%%%%%
%  Create a grey background in the middle of the PDF  %
%%%%%%%%%%%%%%%%%%%%%%%%%%%%%%%%%%%%%%%%%%%%%%%%%%%%%%%

\usepackage{eso-pic}
\newcommand\definegraybackground{
    \definecolor{reallylightgray}{HTML}{FAFAFA}
    \AddToShipoutPicture{
        \ifthenelse{\isodd{\thepage}}{
            \AtPageLowerLeft{
                \put(\LenToUnit{\dimexpr\paperwidth-222pt},0){
                    \color{reallylightgray}\rule{222pt}{297mm}
                }
            }
        }
        {
            \AtPageLowerLeft{
                \color{reallylightgray}\rule{222pt}{297mm}
            }
        }
    }
}

%%%%%%%%%%%%%%%%%%%%%%%%
%  Modify Links Color  %
%%%%%%%%%%%%%%%%%%%%%%%%

\hypersetup{
    % Enable highlighting links.
    colorlinks,
    % Change the color of links to blue.
    urlcolor=blue,
    % Change the color of citations to black.
    citecolor={black},
    % Change the color of url's to blue with some black.
    linkcolor={blue!80!black}
}

%%%%%%%%%%%%%%%%%%
% Fix WrapFigure %
%%%%%%%%%%%%%%%%%%

\newcommand{\wrapfill}{\par\ifnum\value{WF@wrappedlines}>0
        \parskip=0pt
        \addtocounter{WF@wrappedlines}{-1}%
        \null\vspace{\arabic{WF@wrappedlines}\baselineskip}%
        \WFclear
    \fi}

%%%%%%%%%%%%%%%%%
% Multi Columns %
%%%%%%%%%%%%%%%%%

\let\multicolmulticols\multicols
\let\endmulticolmulticols\endmulticols

\RenewDocumentEnvironment{multicols}{mO{}}
{%
    \ifnum#1=1
        #2%
    \else % More than 1 column
        \multicolmulticols{#1}[#2]
    \fi
}
{%
    \ifnum#1=1
    \else % More than 1 column
        \endmulticolmulticols
    \fi
}

\newlength{\thickarrayrulewidth}
\setlength{\thickarrayrulewidth}{5\arrayrulewidth}


%%%%%%%%%%%%%%%%%%%%%%%%%%%%%%%%%%%%%%%%%%%%%%%%%%%%%%%%%%%%%%%%%%%%%%%%%%%%%%%
%                           School Specific Commands                          %
%%%%%%%%%%%%%%%%%%%%%%%%%%%%%%%%%%%%%%%%%%%%%%%%%%%%%%%%%%%%%%%%%%%%%%%%%%%%%%%

%%%%%%%%%%%%%%%%%%%%%%%%%%%
%  Initiate New Counters  %
%%%%%%%%%%%%%%%%%%%%%%%%%%%

\newcounter{lecturecounter}

%%%%%%%%%%%%%%%%%%%%%%%%%%
%  Helpful New Commands  %
%%%%%%%%%%%%%%%%%%%%%%%%%%

\makeatletter

\newcommand\resetcounters{
    % Reset the counters for subsection, subsubsection and the definition
    % all the custom environments.
    \setcounter{subsection}{0}
    \setcounter{subsubsection}{0}
    \setcounter{paragraph}{0}
    \setcounter{subparagraph}{0}
    \setcounter{theorem}{0}
    \setcounter{claim}{0}
    \setcounter{corollary}{0}
    \setcounter{lemma}{0}
    \setcounter{exercise}{0}

    \@ifclasswith\class{nocolor}{
        \setcounter{definition}{0}
    }{}
}

%%%%%%%%%%%%%%%%%%%%%
%  Lecture Command  %
%%%%%%%%%%%%%%%%%%%%%

\usepackage{xifthen}

% EXAMPLE:
% 1. \lecture{Oct 17 2022 Mon (08:46:48)}{Lecture Title}
% 2. \lecture[4]{Oct 17 2022 Mon (08:46:48)}{Lecture Title}
% 3. \lecture{Oct 17 2022 Mon (08:46:48)}{}
% 4. \lecture[4]{Oct 17 2022 Mon (08:46:48)}{}
% Parameters:
% 1. (Optional) lecture number.
% 2. Time and date of lecture.
% 3. Lecture Title.
\def\@lecture{}
\def\@lectitle{}
\def\@leccount{}
\newcommand\lecture[3]{
    %   % Add 1 to the lecture counter.
    %   \addtocounter{lecturecounter}{1}

    % Set the section number to the lecture counter.
    \setcounter{section}{#1}
    \renewcommand\thesubsection{#1.\arabic{subsection}}

    % Reset the counters.
    \resetcounters

    % Check if user passed the lecture title or not.
    \def\@leccount{Lecture #1}
    \ifthenelse{\isempty{#3}}{
        \def\@lecture{Lecture #1}
    }{
        \def\@lecture{Lecture #1: #3}
        \def\@lectitle{#3}
    }

    % Display the information like the following:
    %                                                  Oct 17 2022 Mon (08:49:10)
    % ---------------------------------------------------------------------------
    % Lecture 1: Lecture Title
    \newpage
    \begin{mdframed}
        % \section*{\@lecture}
        \begin{center}
            \Large \textbf{\@leccount}

            \vspace*{0.2cm}

            \large\@lectitle

            \vspace*{0.2cm}
            \normalsize #2
        \end{center}
    \end{mdframed}
    \addcontentsline{toc}{section}{\@lecture}
}

%%%%%%%%%%%%%%%%%%%%
%  Import Figures  %
%%%%%%%%%%%%%%%%%%%%

\usepackage{import}
\pdfminorversion=7

% EXAMPLE:
% 1. \incfig{limit-graph}
% 2. \incfig[0.4]{limit-graph}
% Parameters:
% 1. The figure name. It should be located in figures/NAME.tex_pdf.
% 2. (Optional) The width of the figure. Example: 0.5, 0.35.
\newcommand\incfig[2][1]{%
    \def\svgwidth{#1\columnwidth}
    \import{./figures/}{#2.pdf_tex}
}

\begingroup\expandafter\expandafter\expandafter\endgroup
\expandafter\ifx\csname pdfsuppresswarningpagegroup\endcsname\relax
\else
    \pdfsuppresswarningpagegroup=1\relax
\fi

%%%%%%%%%%%%%%%%%
% Fancy Headers %
%%%%%%%%%%%%%%%%%

\usepackage{fancyhdr}

% Force a new page.
\newcommand\forcenewpage{\clearpage\mbox{~}\clearpage\newpage}

% This command makes it easier to manage my headers and footers.
\newcommand\createintro{
    % Use roman page numbers (e.g. i, v, vi, x, ...)
    \pagenumbering{roman}

    % Display the page style.
    \maketitle
    % Make the title pagestyle empty, meaning no fancy headers and footers.
    \thispagestyle{empty}
    % Create a newpage.
    \newpage

    % Input the intro.tex page if it exists.
    \IfFileExists{intro.tex}{ % If the intro.tex file exists.
        % Input the intro.tex file.
        \textbf{Course}: COURSE

\textbf{Section}: SECTION

\textbf{Professor}: PROFESSOR

\textbf{At}: The University of Chicago

\textbf{Quarter}: QUARTER

\textbf{Course materials}: COURSE_MATERIALS

\vspace{1cm}
\textbf{Disclaimer}: This document will inevitably contain some mistakes, both simple typos and serious logical and mathematical errors. Take what you read with a grain of salt as it is made by an undergraduate student going through the learning process himself. If you do find any error, I would really appreciate it if you can let me know by email at \href{mailto:conghungletran@gmail.com}{conghungletran@gmail.com}.

        % Make the pagestyle fancy for the intro.tex page.
        \pagestyle{fancy}

        % Remove the line for the header.
        \renewcommand\headrulewidth{0pt}

        % Remove all header stuff.
        \fancyhead{}

        % Add stuff for the footer in the center.
        % \fancyfoot[C]{
        %   \textit{For more notes like this, visit
        %   \href{\linktootherpages}{\shortlinkname}}. \\
        %   \vspace{0.1cm}
        %   \hrule
        %   \vspace{0.1cm}
        %   \@author, \\
        %   \term: \academicyear, \\
        %   Last Update: \@date, \\
        %   \faculty
        % }
    }{ % If the intro.tex file doesn't exist.
        % Force a \newpageage.
        \forcenewpage
    }

    % Create a new page.
    \newpage

    % Remove the center stuff we did above, and replace it with just the page
    % number, which is still in roman numerals.
    \fancyfoot[C]{\thepage}
    % Add the table of contents.
    \tableofcontents
    % Force a new page.
    \forcenewpage

    % Move the page numberings back to arabic, from roman numerals.
    \pagenumbering{arabic}
    % Set the page number to 1.
    \setcounter{page}{1}

    % Add the header line back.
    \renewcommand\headrulewidth{0.4pt}
    % In the top right, add the lecture title.
    \fancyhead[R]{\footnotesize \@lecture}
    % In the top left, add the author name.
    \fancyhead[L]{\footnotesize \@author}
    % In the bottom center, add the page.
    \fancyfoot[C]{\thepage}
    % Add a nice gray background in the middle of all the upcoming pages.
    % \definegraybackground
}

\makeatother


%%%%%%%%%%%%%%%%%%%%%%%%%%%%%%%%%%%%%%%%%%%%%%%%%%%%%%%%%%%%%%%%%%%%%%%%%%%%%%%
%                               Custom Commands                               %
%%%%%%%%%%%%%%%%%%%%%%%%%%%%%%%%%%%%%%%%%%%%%%%%%%%%%%%%%%%%%%%%%%%%%%%%%%%%%%%

%%%%%%%%%%%%
%  Circle  %
%%%%%%%%%%%%

\newcommand*\circled[1]{\tikz[baseline=(char.base)]{
        \node[shape=circle,draw,inner sep=1pt] (char) {#1};}
}

%%%%%%%%%%%%%%%%%%%
%  Todo Commands  %
%%%%%%%%%%%%%%%%%%%

% \usepackage{xargs}
% \usepackage[colorinlistoftodos]{todonotes}

% \makeatletter

% \@ifclasswith\class{working}{
%     \newcommandx\unsure[2][1=]{\todo[linecolor=red,backgroundcolor=red!25,bordercolor=red,#1]{#2}}
%     \newcommandx\change[2][1=]{\todo[linecolor=blue,backgroundcolor=blue!25,bordercolor=blue,#1]{#2}}
%     \newcommandx\info[2][1=]{\todo[linecolor=OliveGreen,backgroundcolor=OliveGreen!25,bordercolor=OliveGreen,#1]{#2}}
%     \newcommandx\improvement[2][1=]{\todo[linecolor=Plum,backgroundcolor=Plum!25,bordercolor=Plum,#1]{#2}}

%     \newcommand\listnotes{
%         \newpage
%         \listoftodos[Notes]
%     }
% }{
%     \newcommandx\unsure[2][1=]{}
%     \newcommandx\change[2][1=]{}
%     \newcommandx\info[2][1=]{}
%     \newcommandx\improvement[2][1=]{}

%     \newcommand\listnotes{}
% }

% \makeatother

%%%%%%%%%%%%%
%  Correct  %
%%%%%%%%%%%%%

% EXAMPLE:
% 1. \correct{INCORRECT}{CORRECT}
% Parameters:
% 1. The incorrect statement.
% 2. The correct statement.
\definecolor{correct}{HTML}{009900}
\newcommand\correct[2]{{\color{red}{#1 }}\ensuremath{\to}{\color{correct}{ #2}}}


%%%%%%%%%%%%%%%%%%%%%%%%%%%%%%%%%%%%%%%%%%%%%%%%%%%%%%%%%%%%%%%%%%%%%%%%%%%%%%%
%                                 Environments                                %
%%%%%%%%%%%%%%%%%%%%%%%%%%%%%%%%%%%%%%%%%%%%%%%%%%%%%%%%%%%%%%%%%%%%%%%%%%%%%%%

\usepackage{varwidth}
\usepackage{thmtools}
\usepackage[most,many,breakable]{tcolorbox}

\tcbuselibrary{theorems,skins,hooks}
\usetikzlibrary{arrows,calc,shadows.blur}

%%%%%%%%%%%%%%%%%%%
%  Define Colors  %
%%%%%%%%%%%%%%%%%%%

\definecolor{myblue}{RGB}{45, 111, 177}
\definecolor{mygreen}{RGB}{56, 140, 70}
\definecolor{myred}{RGB}{199, 68, 64}
\definecolor{mypurple}{RGB}{197, 92, 212}

% ESSENTIALS: 
\definecolor{definition_color}{HTML}{c74540}

\definecolor{theorem_color}{HTML}{00007B}
\colorlet{proof_color}{theorem_color}
\colorlet{prop_color}{theorem_color}

\colorlet{corollary_color}{mypurple!85!black}

\definecolor{lemma_color}{HTML}{983b0f}

\definecolor{example_color}{HTML}{2A7F7F}
\colorlet{exercise_color}{example_color}

\colorlet{claim_color}{mygreen!85!black}

% MISCS: 
%%%%%%%%%%%%%%%%%%%%%%%%%%%%%%%%%%%%%%%%%%%%%%%%%%%%%%%%%
%  Create Environments Styles Based on Given Parameter  %
%%%%%%%%%%%%%%%%%%%%%%%%%%%%%%%%%%%%%%%%%%%%%%%%%%%%%%%%%

\mdfsetup{skipabove=1em,skipbelow=0em}

%%%%%%%%%%%%%%%%%%%%%%
%  Helpful Commands  %
%%%%%%%%%%%%%%%%%%%%%%

% EXAMPLE:
% 1. \createnewtheoremstyle{thmdefinitionbox}{}{}
% 2. \createnewtheoremstyle{thmtheorembox}{}{}
% 3. \createnewtheoremstyle{thmproofbox}{qed=\qedsymbol}{
%       rightline=false, topline=false, bottomline=false
%    }
% Parameters:
% 1. Theorem name.
% 2. Any extra parameters to pass directly to declaretheoremstyle.
% 3. Any extra parameters to pass directly to mdframed.
\newcommand\createnewtheoremstyle[3]{
    \declaretheoremstyle[
        headfont=\bfseries\sffamily, bodyfont=\normalfont, #2,
        mdframed={
                #3,
            },
    ]{#1}
}

% EXAMPLE:
% 1. \createnewcoloredtheoremstyle{thmdefinitionbox}{definition}{}{}
% 2. \createnewcoloredtheoremstyle{thmexamplebox}{example}{}{
%       rightline=true, leftline=true, topline=true, bottomline=true
%     }
% 3. \createnewcoloredtheoremstyle{thmproofbox}{proof}{qed=\qedsymbol}{backgroundcolor=white}
% Parameters:
% 1. Theorem name.
% 2. Color of theorem.
% 3. Any extra parameters to pass directly to declaretheoremstyle.
% 4. Any extra parameters to pass directly to mdframed.

% change backgroundcolor to #2!5 if user wants a colored backdrop to theorem environments. It's a cool color theme, but there's too much going on in the page.
\newcommand\createnewcoloredtheoremstyle[4]{
    \declaretheoremstyle[
        headfont=\bfseries\sffamily\color{#2}, bodyfont=\normalfont, #3,
        mdframed={
                linewidth=2pt,
                rightline=false, leftline=true, topline=false, bottomline=false,
                linecolor=#2, backgroundcolor=white, #4,
            },
    ]{#1}
}



%%%%%%%%%%%%%%%%%%%%%%%%%%%%%%%%%%%
%  Create the Environment Styles  %
%%%%%%%%%%%%%%%%%%%%%%%%%%%%%%%%%%%

\makeatletter
\@ifclasswith\class{nocolor}{
    % Environments without color.

    % ESSENTIALS:
    \createnewtheoremstyle{thmdefinitionbox}{}{}
    \createnewtheoremstyle{thmtheorembox}{}{}
    \createnewtheoremstyle{thmproofbox}{qed=\qedsymbol}{}
    \createnewtheoremstyle{thmcorollarybox}{}{}
    \createnewtheoremstyle{thmlemmabox}{}{}
    \createnewtheoremstyle{thmclaimbox}{}{}
    \createnewtheoremstyle{thmexamplebox}{}{}

    % MISCS: 
    \createnewtheoremstyle{thmpropbox}{}{}
    \createnewtheoremstyle{thmexercisebox}{}{}
    \createnewtheoremstyle{thmexplanationbox}{}{}
    \createnewtheoremstyle{thmremarkbox}{}{}

    % STYLIZED MORE BELOW
    \createnewtheoremstyle{thmquestionbox}{}{}
    \createnewtheoremstyle{thmsolutionbox}{qed=\qedsymbol}{}
}{
    % Environments with color.

    % ESSENTIALS: definition, theorem, proof, corollary, lemma, claim, example
    \createnewcoloredtheoremstyle{thmdefinitionbox}{definition_color}{}{
        leftline=true
    }
    \createnewcoloredtheoremstyle{thmtheorembox}{theorem_color}{}{
        leftline=true
    }
    \createnewcoloredtheoremstyle{thmproofbox}{proof_color}{qed=\qedsymbol}{backgroundcolor=white}
    \createnewcoloredtheoremstyle{thmcorollarybox}{corollary_color}{}{backgroundcolor=white}
    \createnewcoloredtheoremstyle{thmlemmabox}{lemma_color}{}{backgroundcolor=white}
    \createnewcoloredtheoremstyle{thmclaimbox}{claim_color}{}{}
    \createnewcoloredtheoremstyle{thmexamplebox}{example_color}{}{backgroundcolor=white}
    \createnewcoloredtheoremstyle{thmexplanationbox}{example_color}{qed=\qedsymbol}{backgroundcolor=white}
    \createnewcoloredtheoremstyle{thmremarkbox}{theorem_color}{}{backgroundcolor=white}

    \createnewcoloredtheoremstyle{thmmiscbox}{black}{}{leftline=false,backgroundcolor=white}


    \createnewcoloredtheoremstyle{thmpropbox}{prop_color}{}{backgroundcolor=white}
    \createnewcoloredtheoremstyle{thmexercisebox}{exercise_color}{}{backgroundcolor=white}

    \createnewcoloredtheoremstyle{thmproblembox}{myred}{}{backgroundcolor=white}
    \createnewcoloredtheoremstyle{thmsolutionbox}{mygreen}{qed=\qedsymbol}{backgroundcolor=white}
}
\makeatother

%%%%%%%%%%%%%%%%%%%%%%%%%%%%%
%  Create the Environments  %
%%%%%%%%%%%%%%%%%%%%%%%%%%%%%
\declaretheorem[numberwithin=section, style=thmdefinitionbox,     name=Definition]{definition0}
\declaretheorem[numberwithin=section, style=thmtheorembox,     name=Theorem]{theorem}
\declaretheorem[numbered=no,          style=thmexamplebox,     name=Example]{example}
\declaretheorem[numberwithin=section, style=thmclaimbox,       name=Claim]{claim}
\declaretheorem[numberwithin=section, style=thmcorollarybox,   name=Corollary]{corollary}
\declaretheorem[numberwithin=section, style=thmpropbox,        name=Proposition]{proposition}
\declaretheorem[numberwithin=section, style=thmlemmabox,       name=Lemma]{lemma}
\declaretheorem[numberwithin=section, style=thmexercisebox,    name=Exercise]{exercise}
\declaretheorem[numbered=no,          style=thmproofbox,       name=Proof]{replacementproof}
\declaretheorem[numbered=no,          style=thmexplanationbox, name=Explanation]{explanation}
\declaretheorem[numbered=no,          style=thmsolutionbox,    name=Solution]{solution}
\declaretheorem[numberwithin=section,          style=thmproblembox,     name=Problem]{problem}
\declaretheorem[numbered=no,          style=thmmiscbox,    name=Intuition]{intuition}
\declaretheorem[numbered=no,          style=thmmiscbox,    name=Goal]{goal}
\declaretheorem[numbered=no,          style=thmmiscbox,    name=Recall]{recall}
\declaretheorem[numbered=no,          style=thmmiscbox,    name=Motivation]{motivation}
\declaretheorem[numbered=no,          style=thmmiscbox,    name=Remark]{remark}
\declaretheorem[numbered=no,          style=thmmiscbox,    name=Observe]{observe}




%%%% FANCY GRAPHICS:

% \makeatletter
% \@ifclasswith\class{nocolor}{
%   % Environments without color.

%   \newtheorem*{note}{Note}

%   \declaretheorem[numberwithin=section, style=thmdefinitionbox, name=Definition]{definition}
%   \declaretheorem[numberwithin=section, style=thmquestionbox,   name=Question]{question}
%   \declaretheorem[numberwithin=section, style=thmsolutionbox,   name=Solution]{solution}
% }{
%   % Environments with color.

%   \newtcbtheorem[number within=section]{Definition}{Definition}{
%     enhanced,
%     before skip=2mm,
%     after skip=2mm,
%     colback=red!5,
%     colframe=red!80!black,
%     colbacktitle=red!75!black,
%     boxrule=0.5mm,
%     attach boxed title to top left={
%       xshift=1cm,
%       yshift*=1mm-\tcboxedtitleheight
%     },
%     varwidth boxed title*=-3cm,
%     boxed title style={
%       interior engine=empty,
%       frame code={
%         \path[fill=tcbcolback]
%         ([yshift=-1mm,xshift=-1mm]frame.north west)
%         arc[start angle=0,end angle=180,radius=1mm]
%         ([yshift=-1mm,xshift=1mm]frame.north east)
%         arc[start angle=180,end angle=0,radius=1mm];
%         \path[left color=tcbcolback!60!black,right color=tcbcolback!60!black,
%         middle color=tcbcolback!80!black]
%         ([xshift=-2mm]frame.north west) -- ([xshift=2mm]frame.north east)
%         [rounded corners=1mm]-- ([xshift=1mm,yshift=-1mm]frame.north east)
%         -- (frame.south east) -- (frame.south west)
%         -- ([xshift=-1mm,yshift=-1mm]frame.north west)
%         [sharp corners]-- cycle;
%       },
%     },
%     fonttitle=\bfseries,
%     title={#2},
%     #1
%   }{def}

%   \NewDocumentEnvironment{definition}{O{}O{}}
%     {\begin{Definition}{#1}{#2}}{\end{Definition}}

%   \newtcolorbox{note}[1][]{%
%     enhanced jigsaw,
%     colback=gray!20!white,%
%     colframe=gray!80!black,
%     size=small,
%     boxrule=1pt,
%     title=\textbf{Note:-},
%     halign title=flush center,
%     coltitle=black,
%     breakable,
%     drop shadow=black!50!white,
%     attach boxed title to top left={xshift=1cm,yshift=-\tcboxedtitleheight/2,yshifttext=-\tcboxedtitleheight/2},
%     minipage boxed title=1.5cm,
%     boxed title style={%
%       colback=white,
%       size=fbox,
%       boxrule=1pt,
%       boxsep=2pt,
%       underlay={%
%         \coordinate (dotA) at ($(interior.west) + (-0.5pt,0)$);
%         \coordinate (dotB) at ($(interior.east) + (0.5pt,0)$);
%         \begin{scope}
%           \clip (interior.north west) rectangle ([xshift=3ex]interior.east);
%           \filldraw [white, blur shadow={shadow opacity=60, shadow yshift=-.75ex}, rounded corners=2pt] (interior.north west) rectangle (interior.south east);
%         \end{scope}
%         \begin{scope}[gray!80!black]
%           \fill (dotA) circle (2pt);
%           \fill (dotB) circle (2pt);
%         \end{scope}
%       },
%     },
%     #1,
%   }

%   \newtcbtheorem{Question}{Question}{enhanced,
%     breakable,
%     colback=white,
%     colframe=myblue!80!black,
%     attach boxed title to top left={yshift*=-\tcboxedtitleheight},
%     fonttitle=\bfseries,
%     title=\textbf{Question:-},
%     boxed title size=title,
%     boxed title style={%
%       sharp corners,
%       rounded corners=northwest,
%       colback=tcbcolframe,
%       boxrule=0pt,
%     },
%     underlay boxed title={%
%       \path[fill=tcbcolframe] (title.south west)--(title.south east)
%       to[out=0, in=180] ([xshift=5mm]title.east)--
%       (title.center-|frame.east)
%       [rounded corners=\kvtcb@arc] |-
%       (frame.north) -| cycle;
%     },
%     #1
%   }{def}

%   \NewDocumentEnvironment{question}{O{}O{}}
%   {\begin{Question}{#1}{#2}}{\end{Question}}

%   \newtcolorbox{Solution}{enhanced,
%     breakable,
%     colback=white,
%     colframe=mygreen!80!black,
%     attach boxed title to top left={yshift*=-\tcboxedtitleheight},
%     title=\textbf{Solution:-},
%     boxed title size=title,
%     boxed title style={%
%       sharp corners,
%       rounded corners=northwest,
%       colback=tcbcolframe,
%       boxrule=0pt,
%     },
%     underlay boxed title={%
%       \path[fill=tcbcolframe] (title.south west)--(title.south east)
%       to[out=0, in=180] ([xshift=5mm]title.east)--
%       (title.center-|frame.east)
%       [rounded corners=\kvtcb@arc] |-
%       (frame.north) -| cycle;
%     },
%   }

%   \NewDocumentEnvironment{solution}{O{}O{}}
%   {\vspace{-10pt}\begin{Solution}{#1}{#2}}{\end{Solution}}
% }
% \makeatother


%%%%% END OF FANCY GRAPHICS %%%%%%%%%%



%%%%%%%%%%%%%%%%%%%%%%%%%%%%
%  Edit Proof Environment  %
%%%%%%%%%%%%%%%%%%%%%%%%%%%%

\renewenvironment{proof}[2][\proofname]{
    % \vspace{-12pt}
    \begin{replacementproof} [#2]
}{\end{replacementproof}}

\newenvironment{definition}[1]{
    \begin{definition0}[#1]

    \hfill
        
    \vspace{0.2cm}

}{

    \vspace{0.2cm}
    \end{definition0}
}


\theoremstyle{definition}

\newtheorem*{notation}{Notation}
\newtheorem*{previouslyseen}{As previously seen}
\newtheorem*{property}{Property}
% \newtheorem*{intuition}{Intuition}
% \newtheorem*{goal}{Goal}
% \newtheorem*{recall}{Recall}
% \newtheorem*{motivation}{Motivation}
% \newtheorem*{remark}{Remark}
% \newtheorem*{observe}{Observe}

\author{Cong Hung Le Tran}


%%%% MATH SHORTHANDS %%%%
%% blackboard bold math capitals
\newcommand{\bbf}{\mathbb{F}}
\newcommand{\bbn}{\mathbb{N}}
\newcommand{\bbq}{\mathbb{Q}}
\newcommand{\bbr}{\mathbb{R}}
\newcommand{\bbz}{\mathbb{Z}}
\newcommand{\bbc}{\mathbb{C}}
\newcommand{\bbk}{\mathbb{K}}
\newcommand{\bbm}{\mathbb{M}}
\renewcommand{\phi}{\varphi}
\newcommand{\st}{\;\text{such that}\;}


% MATH 20250 %
\newcommand{\Hom}{\mathrm{Hom}}
\newcommand{\im}{\mathrm{im}}

% https://tex.stackexchange.com/questions/438612/space-between-exists-and-forall
\let\oldforall\forall
\renewcommand{\forall}{\;\oldforall\; }
\let\oldexist\exists
\renewcommand{\exists}{\;\oldexist\; }
\newcommand\existu{\;\oldexist!\: }


\renewcommand{\_}[1]{\underline{ #1 }}
\DeclarePairedDelimiter{\abs}{\lvert}{\rvert}
\DeclarePairedDelimiter{\norm}{\lVert}{\rVert}
\setlength\parindent{0pt}
\setlength{\headheight}{12.0pt}
\addtolength{\topmargin}{-12.0pt}


% Default skipping, change if you want more spacing
% \thinmuskip=3mu
% \medmuskip=4mu plus 2mu minus 4mu
% \thickmuskip=5mu plus 5mu



% \DeclareMathOperator{\ext}{ext}
% \DeclareMathOperator{\bridge}{bridge}
\title{MATH 20800: Honors Analysis in Rn II \\ \large Problem Set 3}
\date{06 Feb 2024}
\author{Hung Le Tran}
\begin{document}
\maketitle
\setcounter{section}{3}
\textbf{Textbook:} Rudin, \textit{Principles of Mathematical Analysis} 

\textbf{Collaborators:} Duc Nguyen, Hung Pham, Otto Reed
\begin{problem} [8.11 \redtext{done}]
    Suppose $f$ Riemann integrable on $[0, A]$ for all $A < \infty$, and $f(x) \cvgx 1$. Prove that \begin{equation*}
    \lim_{t \to 0}t \int_{0}^{\infty} e^{-tx} f(x) \dx = 1 \quad (t > 0)
    \end{equation*}
\end{problem}
\begin{solution}
    Fix $\epsilon > 0$. Since $f(x) \cvgx 1$, there exists $N> 0$ such that $x \geq N \implies \abs{f(x) - 1} < \epsilon/2$.

    $f$ is then Riemann integrable on $[0, N]$, therefore exists $M = \sup_{[0, N]} \abs{f(x)} < \infty$.

    Denote $D_t(x) = te^{-tx}$ then $D_t(x) \geq 0$ and simple integration yields $\int_{0}^{\infty} D_t(x) = 1 \forall t > 0$, and $\int_{0}^{N} D_t(x) \dx = 1 - e^{-Nt}$

    We can now bound: 
    \begin{align*}
        \abs*{t \int_{0}^{\infty} e^{tx} f(x) \dx - 1} &= \abs*{\int_{0}^{\infty} D_t(x) f(x) \dx - 1}\\
        &= \abs*{\int_{0}^{\infty}D_t (x) f(x) \dx - \int_{0}^{\infty} D_t(x) \dx } \\
        &= \abs*{\int_{0}^{\infty} D_t(x) (f(x) - 1) \dx } \\
        &= \int_{0}^{N} D_t(x) \abs{f(x) - 1} \dx  + \int_{N}^{\infty} D_t(x) \abs{f(x) - 1} \dx \\
        &\leq \int_{0}^{N} D_t(x) (M + 1) \dx + 1 \times \epsilon/2 \\
        &= (M+1) (1 - e^{-Nt}) + \epsilon/2 \xrightarrow{t \to 0} 0
    \end{align*}
    since $e^{-Nt} \xrightarrow{t \to 0} 1$.
\end{solution}

\begin{problem} [8.12 \redtext{done}]
    Suppose $0 < \delta < \pi, f(x) = 1$ if $\abs{x} < \delta, f(x) = 0$ if $\delta < \abs{x} \leq \pi$, and $f$ is $2\pi$-periodic. 
    \begin{enumerate}
    \item Compute Fourier coefficients of $f$.
    \item Conclude that \begin{equation*}
    \sum_{n=1}^{\infty} \frac{\sin (n\delta)}{n} = \frac{n - \delta}{2}
    \end{equation*}
    \item Deduce from Parseval's theorem that \begin{equation*}
    \sum_{n=1}^{\infty} \frac{\sin^2(n\delta)}{n^2\delta} = \frac{n - \delta}{2}
    \end{equation*}
    \item Let $\delta > 0$ and prove that \begin{equation*}
    \int_{0}^{\infty} \left(\frac{\sin x}{x}\right)^2 \dx = \frac{\pi}{2}
    \end{equation*}
    \item Put $\delta = \pi/2$ in (c), what do you get?
    \end{enumerate}
\end{problem}
\begin{solution}
    \textbf{(a)} For $n = 0$: \begin{equation*}
    \hat{f}(0) = \frac{1}{2\pi} \int_{-\pi}^{\pi} f(x) e^{-inx}\dx =\frac{1}{2\pi} \int_{-\delta}^{\delta} 1 \times 1 \dx = \delta/\pi
    \end{equation*}
    and for $n \neq 0$: \begin{align*}
        \hat{f}(n) &= \frac{1}{2\pi} \int_{-\pi}^{\pi} f(x) e^{-inx} \dx \\
        &= \frac{1}{2\pi} \int_{-\delta}^{\delta} 1 e^{-inx} \dx\\
        &= \frac{1}{2\pi} \left[\frac{e^{-inx}}{-in}\right]^{\delta}_{-\delta} \\
        &= \frac{-1}{2\pi in} (e^{-in\delta} - e^{in\delta}) \\
        &= \frac{\sin(n\delta)}{n\pi}
    \end{align*}

    \textbf{(b)} $f$ is clearly locally Lipschitz at 0, since it is locally constant (on $(-\delta, +\delta)$ with Lipschitz constant 1). It follows that \begin{equation*}
    \lim_{N \to \infty} s_N(f, 0) = f(0)
    \end{equation*}
    We also realize that $\hat{f}(n) = \hat{f}(-n)$, since $\sin$ is odd.
    It then follows that: 
    \begin{align*}
        1 &= f(0) \\
        &= \sum_{-\infty}^{\infty} \hat{f}(n) e^{0in} \\
        &= \frac{\delta}{\pi} + 2\sum_{n=1}^{\infty} \hat{f}(n) \\
        \implies \frac{\pi - \delta}{2\pi} &= \sum_{n=1}^{\infty} \hat{f}(n) \\
        &= \sum_{n=1}^{\infty} \frac{\sin(n\delta)}{n\pi}\\
        \implies \sum_{n=1}^{\infty} \frac{\sin(n\delta)}{n} &= \frac{\pi-\delta}{2} 
    \end{align*}
    as required.

    \textbf{(c)} $f$ is discontinuous on a zero set ($\{-\delta, \delta\}$), hence it is Riemann integrable. Therefore we can apply Parseval's Theorem:
    \begin{align*}
    \sum_{-\infty}^{\infty} \abs{\hat{f}(n)}^2 &= \frac{1}{2\pi} \int_{-\pi}^{\pi} \abs{f(x)}^2 \dx \\
    \left(\frac{\delta}{\pi}\right) ^2 + 2 \sum_{n=1}^{\infty} \left(\frac{\sin(n\delta)}{n\pi}\right)^2 &= \frac{1}{2\pi}  \int_{-\delta}^{\delta} 1 \dx = \frac{\delta}{\pi} \\
    \implies \sum_{n=1}^{\infty} \frac{\sin^2 (n\delta)}{n^2} &= \frac{\pi \delta - \delta^2}{2} \\
    \implies \sum_{n=1}^{\infty} \frac{\sin^2 (n\delta)}{n^2 \delta} &= \frac{\pi - \delta}{2}
    \end{align*}
    as required.

    \textbf{(d)} Letting $\delta \to 0$, then
    \begin{align*}
        \sum_{n=1}^{\infty} \frac{\sin^2(n\delta)}{n^2\delta} &= \sum_{n=1}^{\infty} \left(\frac{\sin(n\delta)}{n\delta}\right)^2 \delta \\
        &\xrightarrow{\delta \to 0} \int_{0}^{\infty} \left(\frac{\sin (x)}{x}\right)^2
    \end{align*}
    since each $\sum_{n=1}^{\infty} \left(\frac{\sin(n\delta)}{n\delta}\right)^2 \delta$ is the Riemann sum of the integral.

    Then $RHS \xrightarrow{\delta \to 0} \pi/2$, hence \begin{equation*}
    \int_{0}^{\infty} \left(\frac{\sin x}{x}\right)^2 = \frac{\pi}{2}
    \end{equation*}

    \textbf{(e)} Putting $\delta = \pi/2$ in \textbf{(c)}. We have that $\sin(n\frac{\pi}{2})^2 = 1$ when $n$ odd, and $= 0$ when $n$ even, so
     \begin{equation*}
    \sum_{n \:\text{odd} \:} \frac{1}{n^2 \pi/2} = \frac{\pi}{4}
    \end{equation*}
    hence 
    \begin{equation*}
        \sum_{n \:\text{odd}\: } \frac{1}{n^2} = \frac{\pi^2}{8}
    \end{equation*}
\end{solution}

\begin{problem} [8.13 \redtext{done}]
    Put $f(x) = x$ if $0 \leq x < 2\pi$, and apply Parseval's theorem to conclude that \begin{equation*}
        \sum_{n=1}^{\infty} \frac{1}{n^2} = \frac{\pi^2}{6}
    \end{equation*}
\end{problem}
\begin{solution}
    $f(x) = x$, hence \begin{align*}
        \hat{f}(n) &= \frac{1}{2 \pi}\int_{0}^{2\pi} x e^{-inx} \dx \\
        &= \frac{1}{2 \pi}\left[\frac{(inx + 1)e^{-inx}}{n^2}\right]^{2\pi}_{0} \\
        &= \frac{1}{2 \pi}\frac{2\pi i}{n} = \frac{i}{n}
    \end{align*}
    with the exception of $n = 0$:
    \begin{align*}
        \hat{f}(0) &= \frac{1}{2\pi} \int_{0}^{2\pi} x \dx = \pi
    \end{align*}
    Also notice that $\abs{\hat{f}(n)}^2 = \abs{\hat{f}(-n)}^2 = \frac{1}{n^2}$, so 
    \begin{align*}
        \pi + 2\sum_{n=1}^{\infty} \frac{1}{n^2} &= \frac{1}{2\pi} \int_{0}^{2\pi} \abs{f(x)}^2 \dx \\
        &= \frac{1}{2\pi} \int_{0}^{2\pi} x^2 \dx \\
        &= \frac{4\pi^2}{3} \\
        \implies \sum_{n=1}^{\infty} \frac{1}{n^2} &= \frac{\pi^2}{6}
    \end{align*}
    as required.
\end{solution}

\begin{problem} [8.14 \redtext{done}]
    If $f(x) = (\pi - \abs{x})^2$ on $[-\pi, \pi]$, prove that \begin{equation*}
    f(x) = \frac{\pi^2}{3} + \sum_{n=1}^{\infty} \frac{4}{n^2} \cos (nx)
    \end{equation*}
    and deduce that \begin{equation*}
    \sum_{n=1}^{\infty}  \frac{1}{n^2} = \frac{\pi^2}{6} \qquad \sum_{n=1}^{\infty} \frac{1}{n^4} = \frac{\pi^4}{90}
    \end{equation*}
\end{problem}
\begin{solution}
    $f(x) = (\pi - \abs{x})^2$ on $[-\pi, \pi]$. We have that
    \begin{align*}
    \abs{f(x+t) - f(x)} &= \abs{\abs{x+t} - \abs{x}} \abs{2\pi - \abs{x+t} - \abs{x}} \\
    &\leq  \abs{t} 2\pi
    \end{align*}
    so $f$ is locally Lipschitz at all points with $M = 2\pi$, since locally $x+t$ and $x$ have the same sign.

    Therefore, it follows that $s_N(f, x) \xrightarrow{N \to \infty} f(x)$ on $[-\pi, \pi]$.

    We then calculate the Fourier coefficients:
    \begin{align*}
        \hat{f}(n) &= \frac{1}{2\pi} \int_{-\pi}^{\pi} (\pi - \abs{x})^2 e^{-inx} \dx \\
        &= \frac{1}{2\pi}\left( \int_{-\pi}^{0} (\pi + x)^2 e^{-inx} \dx + \int_{0}^{\pi} (\pi - x)^2 e^{-inx}\dx \right) \\
        &= \frac{2}{n^2} - \frac{2 \sin(\pi n)}{\pi n^3} = \frac{2}{n^2}
    \end{align*}
    with the exception of $n = 0$:
    \begin{equation*}
        \hat{f}(0) = \frac{1}{2\pi} \int_{-\pi}^{\pi} (\pi - \abs{x})^2 \dx = \frac{\pi^2}{3}   
    \end{equation*}
    Therefore \begin{align*}
    f(x) &= \sum_{-\infty}^{\infty} \hat{f}(n) e^{inx} \\
    &=  \frac{\pi^2}{3} + \sum_{n=1}^{\infty}\left( \frac{2}{n^2} e^{inx} + \frac{2}{n^2} e^{-inx} \right)\\
    &= \frac{\pi^2}{3} + \sum_{n=1}^{\infty} \frac{2}{n^2} 2 \cos(nx) \\
    &= \frac{\pi^2}{3} + \sum_{n=1}^{\infty} \frac{4}{n^2} \cos (nx)
    \end{align*}
    as required.

    In particular, when $x = 0$: \begin{align*}
        f(0) &= (\pi - 0)^2 = \pi^2 \\
        \implies \frac{\pi^2}{3} + \sum_{n=1}^{\infty} \frac{4}{n^2}     &= \pi^2 \\
        \implies \sum_{n=1}^{\infty} \frac{1}{n^2} &= (\pi^2 - \pi^2/3)/4 = \frac{\pi^2}{6}
    \end{align*}
    as required.

    $f$ is continuous, so it is Riemann integrable. We can therefore apply Parseval's: \begin{align*}
        \sum_{-\infty}^{\infty} \abs{\hat{f}(n)}^2 &= \frac{\pi^4}{9} + 2 \sum_{n=1}^{\infty} \left(\frac{4}{n^4}\right) \\
        \implies \frac{\pi^4}{9} + 2 \sum_{n=1}^{\infty} \left(\frac{4}{n^4}\right)  &= \frac{1}{2\pi} \int_{-\pi}^{\pi} (\pi - \abs{x})^4 \dx \\
        &= \frac{\pi^4}{5}   \\
        \implies \sum_{n=1}^{\infty} \frac{1}{n^4} &= (\pi^4/5 - \pi^4/9)/8 = \frac{\pi^4}{90}
    \end{align*}
    as required.
\end{solution}

\begin{problem} [8.15 \redtext{done}]
    With $D_n(x) = \sum_{k=-n}^{n} e^{inx}$, put \begin{equation*}
    K_N(x) = \frac{1}{N+1} \sum_{n=0}^{N} D_n(x)
    \end{equation*}
    Prove that \begin{equation*}
    K_N(x) = \frac{1}{N+1} \frac{1 - \cos (N+1) x}{1 - \cos x}
    \end{equation*}
    and that
    \begin{enumerate}
    \item $K_N \geq 0$
    \item $\frac{1}{2\pi} \int_{-\pi}^{\pi} K_N(x) \dx = 1$,
    \item $K_N(x) \leq \frac{1}{N+1} \frac{2}{1 -\cos \delta}$ if $0 < \delta \leq \abs{x} \leq \pi$
    \end{enumerate}

    If $s_N$ is the $N$th partial sum of the Fourier series of $f$, consider the arithmetic means \begin{equation*}
    \sigma_N = \frac{s_0 + \ldots + s_N}{N+1}
    \end{equation*}
    Prove that \begin{equation*}
    \sigma_N(f; x) = \frac{1}{2\pi} \int_{-\pi}^{\pi} f(x-t) K_N(t) \dt
    \end{equation*}
    and hence prove Fejer's theorem: ``If $f$ is continuous, $2\pi$-periodic, then $\sigma_N(f; x) \to f(x)$ uniformly on $[-\pi, \pi]$.''
\end{problem}
\begin{solution}
    We first rewrite 
    \begin{equation*}
        D_n(x) = \sum_{k=-n}^{n} e^{ikx} = \frac{e^{i(n+1)x} - e^{-inx}}{e^{ix} - 1}
    \end{equation*}
    therefore \begin{align*}
        K_n(x) &= \frac{1}{N+1 } \sum_{n=0}^{N} \frac{e^{i(n+1)x} - e^{-inx}}{e^{ix} - 1} \\
        &=\frac{1}{N+1 } \frac{1}{e^{ix} - 1} \left[\sum_{n=0}^{N} (e^{ix})^{(n+1)} - \sum_{n=0}^{N} (e^{-ix})^n\right] \\
        &=\frac{1}{N+1 } \frac{1}{e^{ix} - 1} \left(\frac{e^{i(N+2)x} - e^{ix}}{e^{ix} - 1} - \frac{e^{-i(N+1)x} - 1}{e^{-ix} - 1}\right) \\
        &=\frac{1}{N+1 } \frac{1}{e^{ix} - 1} \frac{1}{2 - 2\cos x} (e^{ix} - 1) (2 - e^{-i(N+1)x} - e^{i(N+1)x}) \\
        &=\frac{1}{N+1 } \frac{2 - 2 \cos (N+1) x}{2 - 2 \cos x} \\
        &=\frac{1}{N+1 } \frac{1 - \cos(N+1) x}{ 1 - \cos x}
    \end{align*}
    as required.

    \textbf{(a)} $\cos(N+1) x, \cos x \leq 1 \implies K_n(x) \geq 0$.

    \textbf{(b)} We know that $\frac{1}{2\pi}\int_{-\pi}^{\pi} D_n(x) \dx = 1$ hence \begin{equation*}
    \frac{1}{2\pi} \int_{-\pi}^{\pi} K_N(x) \dx = \frac{1}{N+1} \sum_{n=0}^{N} \int_{-\pi}^{\pi} D_n(x) \dx = \frac{1}{N+1} (N+1) = 1
    \end{equation*}

    \textbf{(c)} If $0 < \delta \leq \abs{x} \leq \pi$ then \begin{align*}
        K_N(x) = \frac{1}{N+1} \frac{1 - \cos(N+1)x}{1 - \cos x} \leq \frac{1}{N+1} \frac{1 - (-1)}{1 - \cos \delta} = \frac{1}{N+1} \frac{2}{1 - \cos \delta}
    \end{align*}

    We've therefore proven all properties of $K_N(x)$.

    Now, \begin{align*}
    \sigma_N(x) &= \frac{1}{N+1} \sum_{n=0}^{N}\\
    &= \frac{1}{N+1} \sum_{n=0}^{N} \frac{1}{2\pi} \int_{-\pi}^{\pi} f(x - t) D_n(t) \dt \\
    &= \frac{1}{2\pi} \int_{-\pi}^{\pi} f(x - t) \sum_{n=0}^{N} D_n(t) \dt \\
    &=\frac{1}{2\pi} \int_{-\pi}^{\pi} f(x-t) K_N(t) \dt
    \end{align*}
    
    We are now ready to prove Fejer's Theorem: Suppose that $f$ is continuous with period $2\pi$, then $\sigma_N(f, x) \cvgg{N \to \infty} f(x)$ uniformly on $[-\pi, \pi]$. 

    $f$ is continuous on $[-\pi, \pi]$ and is $2\pi$-periodic, hence there exists $M \geq \norm{f}$.

    For all $\epsilon > 0$, since $f$ is continuous on $[-\pi, \pi]$ and $2\pi$-periodic, it is uniformly continuous, hence there exists $\delta > 0$ such that \begin{equation*}
    \abs{ u -v} < \delta \implies \abs{fu - fv} < \epsilon/2
    \end{equation*}

    Then, we have that \begin{align*}
        \abs*{\sigma_N(f; x) - f(x)} &= \abs*{\frac{1}{2\pi} \int_{-\pi}^{\pi} f(x-t) K_N(t) \dt - f(x) \frac{1}{2\pi} \int_{-\pi}^{\pi} K_N(t) \dt} \\
        &= \frac{1}{2\pi}   \abs*{\int_{-\pi}^{\pi} K_N(t) (f(x-t) - f(x)) \dt} \\
        &\leq \frac{1}{2\pi} \abs*{\int_{\delta \leq \abs{t}} K_N(t) (f(x-t) - f(x)) \dt} \\
        &+ \frac{1}{2\pi}\abs*{\int_{-\delta}^{\delta} K_N(t) (f(x-t) - f(x)) \dt} \\
        & \leq \frac{1}{2\pi} 2M \int_{\delta \leq \abs{t}} {K_N(t)} \dt + \frac{1}{2\pi} \int_{-\delta}^{\delta} K_N(t) \epsilon/2 \dt \\
        &\leq \frac{1}{2\pi} 2M \frac{1}{N+1} \frac{2}{1 - \cos \delta} + \epsilon/2
    \end{align*}
    There then exists $N_1$ big enough such that $\frac{1}{2\pi} 2M \frac{1}{N_1 + 1} \frac{2}{1 - \cos \delta} < \epsilon/2$, then for $N \geq N_1$, $\norm{\sigma_N - f} < \epsilon$, hence the convergence is uniform.
\end{solution}

\begin{problem} [8.16 \redtext{done}]
    Prove a pointwise version of Fejer's theorem: If $f$ Riemann integrable and $f(x+), f(x-)$ exist for some $x$, then \begin{equation*}
    \lim_{N \to \infty} \sigma_N(f; x) = \frac{1}{2} [f(x+) + f(x-)]
    \end{equation*}
\end{problem}
\begin{solution}
    Note that $K_n(x) = \frac{1}{N+1} \frac{1 - \cos(N+1)x}{1 - \cos x}$ is even. Therefore $\int_{-\pi}^{0} K_N(t) \dt = \int_{0}^{\pi} K_N(t) \dt = \frac{1}{2}$.

    Since there exists $f(x+), f(x-)$, there exists $\delta_1, \delta_2 > 0$ such that $0 < t < \delta_1 \implies \abs{f(x+t) - f(x+)} < \epsilon/2$  and $0 < t < \delta_2 \implies \abs{f(x-t) - f(x-)} < \epsilon/2$. Take $\delta = \min\{\delta_1, \delta_2\}$.

    Then, \begin{align*}
        &\abs*{\sigma_N(f; x) - \frac{1}{2} (f(x+) + f(x-))} \\
        &= \frac{1}{2\pi} \abs*{\int_{-\pi}^{0}f(x-t) K_N(t) \dt - f(x+) \int_{-\pi}^{0} K_N(t) \dt + \int_{0}^{\pi} f(x-t) K_N(t) \dt - f(x-) \int_{0}^{\pi} K_N(t) \dt }\\
        &= \frac{1}{2\pi} \abs*{\int_{-\pi}^{0} (f(x-t) - f(x+)) K_N(t) \dt + \int_{0}^{\pi} (f(x-t) - f(x-)) K_N(t) \dt} \\
        &\leq \frac{1}{2\pi} \abs*{\int_{-\pi}^{-\delta} (f(x-t) - f(x+)) K_N(t) \dt} + \frac{1}{2\pi} \abs*{\int_{-\delta}^{0} (f(x-t) - f(x+)) K_N(t) \dt} \\
        &+ \frac{1}{2\pi} \abs*{\int_{0}^{\delta} (f(x-t) - f(x+)) K_N(t) \dt} + \frac{1}{2\pi} \abs*{\int_{\delta}^{\pi} (f(x-t) - f(x+)) K_N(t) \dt} \\
        &\leq \frac{1}{2\pi} \left(2M\pi \frac{1}{N+1} \frac{2}{1 - \cos \delta} + \pi \epsilon/2 + \pi \epsilon/2 + 2M \pi \frac{1}{N+1} \frac{2}{1 - \cos \delta}\right) < \epsilon
    \end{align*}
    for $N$ sufficiently large. The pointwise convergence is thus proven.
\end{solution}

\begin{problem} [8.17]
    Assume $f$ is bounded and monotonic on $[-\pi, \pi)$, with Fourier coefficients $c_n$.
    \begin{enumerate}
    \item Use Exercise 6.17 to prove that $\{nc_n\}$ is bounded.
    \item Combine (a) with Exercise 3.14(e) to conclude that \begin{equation*}
    \lim_{N \to \infty} s_N(f; x) = \frac{1}{2} [f(x+) + f(x-)]
    \end{equation*}
    for every $x$.
    \item Assume only that $f$ Riemann integrable on $[-\pi, \pi]$ and that $f$ is monotonic in some segment $(\alpha, \beta) \subset [-\pi, \pi]$. Prove that the conclusion of (b) holds for every $x \in (\alpha, \beta)$. (This is an application of the localization theorem.)
    \end{enumerate}
\end{problem}
\begin{solution}
    \textbf{(a)} Suppose $\norm{f} < M$. Then we have \begin{align*}
        c_n &= \frac{1}{2\pi} \int_{-\pi}^{\pi} f(x) e^{-inx} \dx  \\
        &= \frac{1}{2\pi} \left(e^{-i n \pi}/(-in) - e^{-in(-\pi)}/(-in) - \int_{-\pi}^{\pi} e^{-inx}/(-in) df\right) \\
        &= \frac{1}{2\pi} \int_{-\pi}^{\pi} e^{-inx}/(-in) df
    \end{align*}
    so \begin{align*}
        \abs{nc_n} &= \frac{1}{2\pi} \abs*{\int_{-\pi}^{\pi} e^{-inx} df} = \frac{1}{2\pi} \abs*{f(\pi)e^{-in\pi} - f(-\pi)e^{in\pi}} \leq M/\pi
    \end{align*}
    is therefore bounded.

    \textbf{(b)} The pointwise version of Fejer's Theorem implies that for all $x$ where the right and left limits exist,
    
    \begin{equation*}
    \lim_{N \to \infty} \sigma_N(f; x) = \frac{1}{2} [f(x+) + f(x-)]
    \end{equation*}

    Using Exercise 3.14(a), since $\abs{nc_n}$ is bounded, it follows that $\lim_{N \to \infty} s_N(f; x) = \lim_{N \to \infty} \sigma_N(f; x) = \frac{1}{2} [f(x+) + f(x-)]$

    \textbf{(c)} Define $g$ such that it agrees with $f$ on $(a, b)$: $g(x) = f(x)$ on $(\alpha, \beta)$, and $g(x) = f(\alpha)$ for $x \leq \alpha$ and $g(x) = f(\beta)$ for $x \geq \beta$. Then, $g$ is clearly monotonic in $[-\pi, \pi]$, so \begin{equation*}
    \lim_{N \to \infty} S_N(g; x) = \frac{1}{2} [g(x+) + g(x-)]
    \end{equation*}

    But on $(\alpha, \beta)$, $g(x+) = f(x+), g(x-) = f(x-)$ so on $(\alpha, \beta)$:
    
    \begin{equation*}
    \lim_{N \to \infty} S_N(g; x) = \frac{1}{2} [f(x+) + f(x-)]
    \end{equation*}

    $f$ and $g$ agree on $(\alpha, \beta)$, so by the localization theorem, it implies that \begin{equation*}
    \lim_{N \to \infty} S_N(f; x) - S_N(g; x) = 0
    \end{equation*}

    Therefore on $(\alpha, \beta), \lim_{N \to \infty} S_N(f; x) = \frac{1}{2} [f(x+) + f(x-)]$.
\end{solution}

\begin{problem} [8.19 \redtext{done}]
    Suppose that $f$ is a continuous, $2\pi$-periodic, real-valued function and some $\alpha$ such that $\alpha/\pi$ is irrational. Prove that \begin{equation*}
    \lim_{N \to \infty} \frac{1}{N} \sum_{n=1}^{N} f(x + n \alpha) = \frac{1}{2\pi} \int_{-\pi}^{\pi} f(t) \dt
    \end{equation*}
\end{problem}
\begin{solution}
We first prove that the proposition is true for $e^{ikx}$ for any $k \in \bbz$.

If $k = 0$ then \begin{align*}
    \rhs &= 1 \\
    \lhs &= \lim_{N \to \infty} \frac{1}{N} \sum_{n=1}^{N} 1 = 1
\end{align*}
else 
\begin{align*}
    \rhs &= \frac{1}{2\pi} \int_{-\pi}^{\pi} e^{ikx} \dx \\
    &= \eval{\frac{e^{ikx}}{ik}}{\pi}{-\pi}  = 0
\end{align*}
and \begin{align*}
    \lhs &= \lim_{N \to \infty} \frac{1}{N} \sum_{n=1}^{N} e^{ik(x + n \alpha)} \\
    &= e^{ikx + ik\alpha} \lim_{N \to \infty} \frac{e^{ikN\alpha} - 1}{N(e^{ik\alpha} - 1)} = 0
\end{align*}
Note that this computation does not into problem for all $k \in \bbz \backslash \{0\}$ (denominator being 0), since $\alpha / \pi $ is irrational so $e^{ik\alpha} \neq 1$ for all $k \in \bbz  \backslash \{0\}$.

Thus, indeed the proposition is true for $e^{ikx}$ for all $k \in \bbz$. A trigonometric polynomial is a linear combination of $e^{ikx}$ terms, and the proposition is linear in $f$, so the proposition is true for all trigonometric polynomial too.

Since $f$ is continuous and $2\pi$-periodic, given $\epsilon > 0$, by Stone-Weierstrass, there exists some trigonometric polynomial $P$ such that \begin{equation*}
\norm{f - P}_{\infty} < \epsilon/2
\end{equation*}
Note that per our remarks above, \begin{equation*}
    \lim_{N \to \infty} \frac{1}{N} \sum_{n=1}^{N} P(x + n \alpha)= \frac{1}{2\pi} \int_{-\pi}^{\pi} P(t) \dt  
\end{equation*}
It then follows that \begin{multline*}
    \abs*{\lim_{N \to \infty} \frac{1}{N} \sum_{n=1}^{N} f(x + n \alpha) - \frac{1}{2\pi} \int_{-\pi}^{\pi} f(t) \dt} \leq \\  \abs*{\lim_{N \to \infty} \frac{1}{N} \sum_{n=1}^{N} f(x + n \alpha) - \lim_{N \to \infty} \frac{1}{N} \sum_{n=1}^{N} P(x + n \alpha)} + \abs*{\frac{1}{2\pi} - \int_{-\pi}^{\pi} P(t) \dt   \frac{1}{2\pi} \int_{-\pi}^{\pi} f(t) \dt}
\end{multline*}
\begin{align*}
    &\leq \left(\lim_{N \to \infty} \frac{1}{N} \sum_{n=1}^{N} \abs*{f(x + n\alpha) - P(x + n\alpha)}\right) + \frac{1}{2\pi} \int_{-\pi}^{\pi} \abs*{P(t) - f(t)} \dt \\
    &< \left(\lim_{N \to \infty} \sum_{n=1}^{N} N\epsilon/2\right) + \frac{1}{2\pi} 2\pi \epsilon/2 \\
    &= \epsilon
\end{align*}
This is true for all $\epsilon$, hence the proposition holds for any continuous, $2\pi$-periodic $f$.
\end{solution}

\begin{problem} [8.21 \redtext{done}]
Let \begin{equation*}
L_n = \frac{1}{2\pi} \abs{D_n(t)} \dt 
\end{equation*}
Prove that there exists $C > 0$ such that \begin{equation*}
L_N > C \log n
\end{equation*}
or, more precisely, that the sequence \begin{equation*}
\left\{L_n - \frac{4}{\pi^2} \log n \right\}
\end{equation*}
is bounded.
\end{problem}
\begin{solution}
    We first show the lower bound:
    \begin{align*}
        L_n &= \frac{1}{2\pi} \int_{-\pi}^{\pi} \abs*{D_n(t)} \dt \\
        &= \frac{1}{\pi} \int_{0}^{\pi} \abs*{\frac{\sin (n+\frac{1}{2}) t}{\sin t/2}}\dt\\
        &= \frac{2}{\pi} \int_{0}^{\pi/2} \frac{\abs*{\sin (n + \frac{1}{2})(2t)}}{\sin t} \dt \\
        &> \frac{2}{\pi} \int_{0}^{\pi/2} \frac{\abs*{\sin (n + \frac{1}{2})(2t)}}{t} \dt \\
        &= \frac{2}{\pi} \int_{0}^{(n+ \frac{1}{2}) \pi} \frac{\abs*{\sin u}}{u} \du \\
        &> \frac{2}{\pi} \int_{0}^{n\pi}\frac{\abs*{\sin u}}{u} \du \\
        &> \frac{2}{\pi} \sum_{k=0}^{n-1} \int_{0}^{\pi} \frac{\sin u}{(k+1) \pi} \du \\
        &= \frac{2}{\pi^2} \left(\sum_{k=1}^{n} \frac{1}{k}\right) \eval{-\cos u}{\pi}{0} \\
        &= \frac{4}{\pi^2} \left(\sum_{k=1}^{n} \frac{1}{k}\right) \geq \frac{4}{\pi^2} \log n
    \end{align*}
    then the upper bound, where we first use a preliminary bound:
    \begin{align*}
        \abs*{\frac{\sin (2n+1) t}{\sin t}} &= \abs*{\frac{\sin (2nt) \cos t + \cos (2nt) \sin t}{\sin t}} \\
        &= \abs*{\frac{\sin (2nt)}{\tan t} + \cos (2nt)} \\
        &\leq \abs*{\frac{\sin(2nt)}{\tan t}} + 1
    \end{align*}
    so
    \begin{align*}
            L_n &= \frac{2}{\pi} \int_{0}^{\pi/2} \abs*{\frac{\sin (2n+1) t}{\sin t}} \dt \\
            &\leq \frac{2}{\pi} \int_{0}^{\pi/2} \abs*{\frac{\sin(2nt)}{\tan t}} + 1 \dt \\
            & = 1 + \frac{2}{\pi} \int_{0}^{\pi/2} \abs*{\frac{\sin(2nt)}{\tan t}} \dt \\
            &= 1 + \frac{2}{\pi} \int_{0}^{n\pi}\frac{\abs*{\sin u}}{u} \du \\
            &= 1 + \frac{2}{\pi} \int_{0}^{\pi} \sum_{k=0}^{n-1} \frac{\sin u}{u + k\pi} \du \\
            &= 1 + \frac{2}{\pi} \int_{0}^{\pi} \frac{\sin u}{u} + \frac{2}{\pi} \int_{0}^{\pi} \sum_{k=1}^{n-1} \frac{\sin u}{u + k\pi} \du \\
            &< C + \frac{2}{\pi}  \int_{0}^{\pi} \sum_{k=1}^{n-1} \frac{\sin u}{u + k\pi} \du \\
            &= C + \frac{2}{\pi} \int_{0}^{\pi} \sum_{k=1}^{n-1} \frac{\sin u}{k\pi} \du \\
            &= C + \frac{2}{\pi} \left(\sum_{k=1}^{n-1} \frac{1}{k}\right) \eval{-\cos u}{\pi}{0}  \\
            &< C + \frac{4}{\pi} (\log n + \gamma) = C + \frac{4}{\pi} \log n
    \end{align*}
    It follows that $\{L_n - \frac{4}{\pi^2} \log n\}$ is bounded as required.
\end{solution}

% \newpage
% It is true that \begin{multline*}
% \bbp(X_1 = H, X_2 = H, \ldots, X_{999} = H, X_{1000} = T)\\ = \bbp(X_1 = H, \ldots, X_{500} = H, X_{501} = T, X_{502} = T, \ldots, X_{1000} = T) = \frac{1}{2^{1000}} 
% \end{multline*}
% but if $A = \:\text{number of heads in 1000 flips}\: = \sum_{n=1}^{1000} 1\{X_n = H\}$
% where $1\{X_n = H\}$ just returns 1 when $X_n = H$ and 0 otherwise then
% \begin{align*}
% \bbp(A = 500) &= \binom{1000}{500} 2^{-1000}\\
% \bbp(A = 999) &= \binom{1000}{999} 2^{-1000} \\
% \implies \bbp(A = 500) &> \bbp (A = 999)
% \end{align*}
\end{document}
