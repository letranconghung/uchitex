\documentclass[a4paper, 12pt]{article}
%%%%%%%%%%%%%%%%%%%%%%%%%%%%%%%%%%%%%%%%%%%%%%%%%%%%%%%%%%%%%%%%%%%%%%%%%%%%%%%
%                                Basic Packages                               %
%%%%%%%%%%%%%%%%%%%%%%%%%%%%%%%%%%%%%%%%%%%%%%%%%%%%%%%%%%%%%%%%%%%%%%%%%%%%%%%

% Gives us multiple colors.
\usepackage[usenames,dvipsnames,pdftex]{xcolor}
% Lets us style link colors.
\usepackage{hyperref}
% Lets us import images and graphics.
\usepackage{graphicx}
% Lets us use figures in floating environments.
\usepackage{float}
% Lets us create multiple columns.
\usepackage{multicol}
% Gives us better math syntax.
\usepackage{amsmath,amsfonts,mathtools,amsthm,amssymb}
% Lets us strikethrough text.
\usepackage{cancel}
% Lets us edit the caption of a figure.
\usepackage{caption}
% Lets us import pdf directly in our tex code.
\usepackage{pdfpages}
% Lets us do algorithm stuff.
\usepackage[ruled,vlined,linesnumbered]{algorithm2e}
% Use a smiley face for our qed symbol.
\usepackage{tikzsymbols}
% \usepackage{fullpage} %%smaller margins
\usepackage[shortlabels]{enumitem}

\usepackage{setspace}
\usepackage[margin=1in, headsep=12pt]{geometry}
\usepackage{wrapfig}
\def\class{article}


%%%%%%%%%%%%%%%%%%%%%%%%%%%%%%%%%%%%%%%%%%%%%%%%%%%%%%%%%%%%%%%%%%%%%%%%%%%%%%%
%                                Basic Settings                               %
%%%%%%%%%%%%%%%%%%%%%%%%%%%%%%%%%%%%%%%%%%%%%%%%%%%%%%%%%%%%%%%%%%%%%%%%%%%%%%%

%%%%%%%%%%%%%
%  Symbols  %
%%%%%%%%%%%%%

\let\implies\Rightarrow
\let\impliedby\Leftarrow
\let\iff\Leftrightarrow
\let\epsilon\varepsilon
%%%%%%%%%%%%
%  Tables  %
%%%%%%%%%%%%

\setlength{\tabcolsep}{5pt}
\renewcommand\arraystretch{1.5}

%%%%%%%%%%%%%%
%  SI Unitx  %
%%%%%%%%%%%%%%

\usepackage{siunitx}
\sisetup{locale = FR}

%%%%%%%%%%
%  TikZ  %
%%%%%%%%%%

\usepackage[framemethod=TikZ]{mdframed}
\usepackage{tikz}
\usepackage{tikz-cd}
\usepackage{tikzsymbols}

\usetikzlibrary{intersections, angles, quotes, calc, positioning}
\usetikzlibrary{arrows.meta}

\tikzset{
    force/.style={thick, {Circle[length=2pt]}-stealth, shorten <=-1pt}
}

%%%%%%%%%%%%%%%
%  PGF Plots  %
%%%%%%%%%%%%%%%

\usepackage{pgfplots}
\pgfplotsset{width=10cm, compat=newest}

%%%%%%%%%%%%%%%%%%%%%%%
%  Center Title Page  %
%%%%%%%%%%%%%%%%%%%%%%%

\usepackage{titling}
\renewcommand\maketitlehooka{\null\mbox{}\vfill}
\renewcommand\maketitlehookd{\vfill\null}

%%%%%%%%%%%%%%%%%%%%%%%%%%%%%%%%%%%%%%%%%%%%%%%%%%%%%%%
%  Create a grey background in the middle of the PDF  %
%%%%%%%%%%%%%%%%%%%%%%%%%%%%%%%%%%%%%%%%%%%%%%%%%%%%%%%

\usepackage{eso-pic}
\newcommand\definegraybackground{
    \definecolor{reallylightgray}{HTML}{FAFAFA}
    \AddToShipoutPicture{
        \ifthenelse{\isodd{\thepage}}{
            \AtPageLowerLeft{
                \put(\LenToUnit{\dimexpr\paperwidth-222pt},0){
                    \color{reallylightgray}\rule{222pt}{297mm}
                }
            }
        }
        {
            \AtPageLowerLeft{
                \color{reallylightgray}\rule{222pt}{297mm}
            }
        }
    }
}

%%%%%%%%%%%%%%%%%%%%%%%%
%  Modify Links Color  %
%%%%%%%%%%%%%%%%%%%%%%%%

\hypersetup{
    % Enable highlighting links.
    colorlinks,
    % Change the color of links to blue.
    urlcolor=blue,
    % Change the color of citations to black.
    citecolor={black},
    % Change the color of url's to blue with some black.
    linkcolor={blue!80!black}
}

%%%%%%%%%%%%%%%%%%
% Fix WrapFigure %
%%%%%%%%%%%%%%%%%%

\newcommand{\wrapfill}{\par\ifnum\value{WF@wrappedlines}>0
        \parskip=0pt
        \addtocounter{WF@wrappedlines}{-1}%
        \null\vspace{\arabic{WF@wrappedlines}\baselineskip}%
        \WFclear
    \fi}

%%%%%%%%%%%%%%%%%
% Multi Columns %
%%%%%%%%%%%%%%%%%

\let\multicolmulticols\multicols
\let\endmulticolmulticols\endmulticols

\RenewDocumentEnvironment{multicols}{mO{}}
{%
    \ifnum#1=1
        #2%
    \else % More than 1 column
        \multicolmulticols{#1}[#2]
    \fi
}
{%
    \ifnum#1=1
    \else % More than 1 column
        \endmulticolmulticols
    \fi
}

\newlength{\thickarrayrulewidth}
\setlength{\thickarrayrulewidth}{5\arrayrulewidth}


%%%%%%%%%%%%%%%%%%%%%%%%%%%%%%%%%%%%%%%%%%%%%%%%%%%%%%%%%%%%%%%%%%%%%%%%%%%%%%%
%                           School Specific Commands                          %
%%%%%%%%%%%%%%%%%%%%%%%%%%%%%%%%%%%%%%%%%%%%%%%%%%%%%%%%%%%%%%%%%%%%%%%%%%%%%%%

%%%%%%%%%%%%%%%%%%%%%%%%%%%
%  Initiate New Counters  %
%%%%%%%%%%%%%%%%%%%%%%%%%%%

\newcounter{lecturecounter}

%%%%%%%%%%%%%%%%%%%%%%%%%%
%  Helpful New Commands  %
%%%%%%%%%%%%%%%%%%%%%%%%%%

\makeatletter

\newcommand\resetcounters{
    % Reset the counters for subsection, subsubsection and the definition
    % all the custom environments.
    \setcounter{subsection}{0}
    \setcounter{subsubsection}{0}
    \setcounter{paragraph}{0}
    \setcounter{subparagraph}{0}
    \setcounter{theorem}{0}
    \setcounter{claim}{0}
    \setcounter{corollary}{0}
    \setcounter{lemma}{0}
    \setcounter{exercise}{0}

    \@ifclasswith\class{nocolor}{
        \setcounter{definition}{0}
    }{}
}

%%%%%%%%%%%%%%%%%%%%%
%  Lecture Command  %
%%%%%%%%%%%%%%%%%%%%%

\usepackage{xifthen}

% EXAMPLE:
% 1. \lecture{Oct 17 2022 Mon (08:46:48)}{Lecture Title}
% 2. \lecture[4]{Oct 17 2022 Mon (08:46:48)}{Lecture Title}
% 3. \lecture{Oct 17 2022 Mon (08:46:48)}{}
% 4. \lecture[4]{Oct 17 2022 Mon (08:46:48)}{}
% Parameters:
% 1. (Optional) lecture number.
% 2. Time and date of lecture.
% 3. Lecture Title.
\def\@lecture{}
\def\@lectitle{}
\def\@leccount{}
\newcommand\lecture[3]{
    %   % Add 1 to the lecture counter.
    %   \addtocounter{lecturecounter}{1}

    % Set the section number to the lecture counter.
    \setcounter{section}{#1}
    \renewcommand\thesubsection{#1.\arabic{subsection}}

    % Reset the counters.
    \resetcounters

    % Check if user passed the lecture title or not.
    \def\@leccount{Lecture #1}
    \ifthenelse{\isempty{#3}}{
        \def\@lecture{Lecture #1}
    }{
        \def\@lecture{Lecture #1: #3}
        \def\@lectitle{#3}
    }

    % Display the information like the following:
    %                                                  Oct 17 2022 Mon (08:49:10)
    % ---------------------------------------------------------------------------
    % Lecture 1: Lecture Title
    \newpage
    \begin{mdframed}
        % \section*{\@lecture}
        \begin{center}
            \Large \textbf{\@leccount}

            \vspace*{0.2cm}

            \large\@lectitle

            \vspace*{0.2cm}
            \normalsize #2
        \end{center}
    \end{mdframed}
    \addcontentsline{toc}{section}{\@lecture}
}

%%%%%%%%%%%%%%%%%%%%
%  Import Figures  %
%%%%%%%%%%%%%%%%%%%%

\usepackage{import}
\pdfminorversion=7

% EXAMPLE:
% 1. \incfig{limit-graph}
% 2. \incfig[0.4]{limit-graph}
% Parameters:
% 1. The figure name. It should be located in figures/NAME.tex_pdf.
% 2. (Optional) The width of the figure. Example: 0.5, 0.35.
\newcommand\incfig[2][1]{%
    \def\svgwidth{#1\columnwidth}
    \import{./figures/}{#2.pdf_tex}
}

\begingroup\expandafter\expandafter\expandafter\endgroup
\expandafter\ifx\csname pdfsuppresswarningpagegroup\endcsname\relax
\else
    \pdfsuppresswarningpagegroup=1\relax
\fi

%%%%%%%%%%%%%%%%%
% Fancy Headers %
%%%%%%%%%%%%%%%%%

\usepackage{fancyhdr}

% Force a new page.
\newcommand\forcenewpage{\clearpage\mbox{~}\clearpage\newpage}

% This command makes it easier to manage my headers and footers.
\newcommand\createintro{
    % Use roman page numbers (e.g. i, v, vi, x, ...)
    \pagenumbering{roman}

    % Display the page style.
    \maketitle
    % Make the title pagestyle empty, meaning no fancy headers and footers.
    \thispagestyle{empty}
    % Create a newpage.
    \newpage

    % Input the intro.tex page if it exists.
    \IfFileExists{intro.tex}{ % If the intro.tex file exists.
        % Input the intro.tex file.
        \textbf{Course}: COURSE

\textbf{Section}: SECTION

\textbf{Professor}: PROFESSOR

\textbf{At}: The University of Chicago

\textbf{Quarter}: QUARTER

\textbf{Course materials}: COURSE_MATERIALS

\vspace{1cm}
\textbf{Disclaimer}: This document will inevitably contain some mistakes, both simple typos and serious logical and mathematical errors. Take what you read with a grain of salt as it is made by an undergraduate student going through the learning process himself. If you do find any error, I would really appreciate it if you can let me know by email at \href{mailto:conghungletran@gmail.com}{conghungletran@gmail.com}.

        % Make the pagestyle fancy for the intro.tex page.
        \pagestyle{fancy}

        % Remove the line for the header.
        \renewcommand\headrulewidth{0pt}

        % Remove all header stuff.
        \fancyhead{}

        % Add stuff for the footer in the center.
        % \fancyfoot[C]{
        %   \textit{For more notes like this, visit
        %   \href{\linktootherpages}{\shortlinkname}}. \\
        %   \vspace{0.1cm}
        %   \hrule
        %   \vspace{0.1cm}
        %   \@author, \\
        %   \term: \academicyear, \\
        %   Last Update: \@date, \\
        %   \faculty
        % }
    }{ % If the intro.tex file doesn't exist.
        % Force a \newpageage.
        \forcenewpage
    }

    % Create a new page.
    \newpage

    % Remove the center stuff we did above, and replace it with just the page
    % number, which is still in roman numerals.
    \fancyfoot[C]{\thepage}
    % Add the table of contents.
    \tableofcontents
    % Force a new page.
    \forcenewpage

    % Move the page numberings back to arabic, from roman numerals.
    \pagenumbering{arabic}
    % Set the page number to 1.
    \setcounter{page}{1}

    % Add the header line back.
    \renewcommand\headrulewidth{0.4pt}
    % In the top right, add the lecture title.
    \fancyhead[R]{\footnotesize \@lecture}
    % In the top left, add the author name.
    \fancyhead[L]{\footnotesize \@author}
    % In the bottom center, add the page.
    \fancyfoot[C]{\thepage}
    % Add a nice gray background in the middle of all the upcoming pages.
    % \definegraybackground
}

\makeatother


%%%%%%%%%%%%%%%%%%%%%%%%%%%%%%%%%%%%%%%%%%%%%%%%%%%%%%%%%%%%%%%%%%%%%%%%%%%%%%%
%                               Custom Commands                               %
%%%%%%%%%%%%%%%%%%%%%%%%%%%%%%%%%%%%%%%%%%%%%%%%%%%%%%%%%%%%%%%%%%%%%%%%%%%%%%%

%%%%%%%%%%%%
%  Circle  %
%%%%%%%%%%%%

\newcommand*\circled[1]{\tikz[baseline=(char.base)]{
        \node[shape=circle,draw,inner sep=1pt] (char) {#1};}
}

%%%%%%%%%%%%%%%%%%%
%  Todo Commands  %
%%%%%%%%%%%%%%%%%%%

% \usepackage{xargs}
% \usepackage[colorinlistoftodos]{todonotes}

% \makeatletter

% \@ifclasswith\class{working}{
%     \newcommandx\unsure[2][1=]{\todo[linecolor=red,backgroundcolor=red!25,bordercolor=red,#1]{#2}}
%     \newcommandx\change[2][1=]{\todo[linecolor=blue,backgroundcolor=blue!25,bordercolor=blue,#1]{#2}}
%     \newcommandx\info[2][1=]{\todo[linecolor=OliveGreen,backgroundcolor=OliveGreen!25,bordercolor=OliveGreen,#1]{#2}}
%     \newcommandx\improvement[2][1=]{\todo[linecolor=Plum,backgroundcolor=Plum!25,bordercolor=Plum,#1]{#2}}

%     \newcommand\listnotes{
%         \newpage
%         \listoftodos[Notes]
%     }
% }{
%     \newcommandx\unsure[2][1=]{}
%     \newcommandx\change[2][1=]{}
%     \newcommandx\info[2][1=]{}
%     \newcommandx\improvement[2][1=]{}

%     \newcommand\listnotes{}
% }

% \makeatother

%%%%%%%%%%%%%
%  Correct  %
%%%%%%%%%%%%%

% EXAMPLE:
% 1. \correct{INCORRECT}{CORRECT}
% Parameters:
% 1. The incorrect statement.
% 2. The correct statement.
\definecolor{correct}{HTML}{009900}
\newcommand\correct[2]{{\color{red}{#1 }}\ensuremath{\to}{\color{correct}{ #2}}}


%%%%%%%%%%%%%%%%%%%%%%%%%%%%%%%%%%%%%%%%%%%%%%%%%%%%%%%%%%%%%%%%%%%%%%%%%%%%%%%
%                                 Environments                                %
%%%%%%%%%%%%%%%%%%%%%%%%%%%%%%%%%%%%%%%%%%%%%%%%%%%%%%%%%%%%%%%%%%%%%%%%%%%%%%%

\usepackage{varwidth}
\usepackage{thmtools}
\usepackage[most,many,breakable]{tcolorbox}

\tcbuselibrary{theorems,skins,hooks}
\usetikzlibrary{arrows,calc,shadows.blur}

%%%%%%%%%%%%%%%%%%%
%  Define Colors  %
%%%%%%%%%%%%%%%%%%%

\definecolor{myblue}{RGB}{45, 111, 177}
\definecolor{mygreen}{RGB}{56, 140, 70}
\definecolor{myred}{RGB}{199, 68, 64}
\definecolor{mypurple}{RGB}{197, 92, 212}

% ESSENTIALS: 
\definecolor{definition_color}{HTML}{c74540}

\definecolor{theorem_color}{HTML}{00007B}
\colorlet{proof_color}{theorem_color}
\colorlet{prop_color}{theorem_color}

\colorlet{corollary_color}{mypurple!85!black}

\definecolor{lemma_color}{HTML}{983b0f}

\definecolor{example_color}{HTML}{2A7F7F}
\colorlet{exercise_color}{example_color}

\colorlet{claim_color}{mygreen!85!black}

% MISCS: 
%%%%%%%%%%%%%%%%%%%%%%%%%%%%%%%%%%%%%%%%%%%%%%%%%%%%%%%%%
%  Create Environments Styles Based on Given Parameter  %
%%%%%%%%%%%%%%%%%%%%%%%%%%%%%%%%%%%%%%%%%%%%%%%%%%%%%%%%%

\mdfsetup{skipabove=1em,skipbelow=0em}

%%%%%%%%%%%%%%%%%%%%%%
%  Helpful Commands  %
%%%%%%%%%%%%%%%%%%%%%%

% EXAMPLE:
% 1. \createnewtheoremstyle{thmdefinitionbox}{}{}
% 2. \createnewtheoremstyle{thmtheorembox}{}{}
% 3. \createnewtheoremstyle{thmproofbox}{qed=\qedsymbol}{
%       rightline=false, topline=false, bottomline=false
%    }
% Parameters:
% 1. Theorem name.
% 2. Any extra parameters to pass directly to declaretheoremstyle.
% 3. Any extra parameters to pass directly to mdframed.
\newcommand\createnewtheoremstyle[3]{
    \declaretheoremstyle[
        headfont=\bfseries\sffamily, bodyfont=\normalfont, #2,
        mdframed={
                #3,
            },
    ]{#1}
}

% EXAMPLE:
% 1. \createnewcoloredtheoremstyle{thmdefinitionbox}{definition}{}{}
% 2. \createnewcoloredtheoremstyle{thmexamplebox}{example}{}{
%       rightline=true, leftline=true, topline=true, bottomline=true
%     }
% 3. \createnewcoloredtheoremstyle{thmproofbox}{proof}{qed=\qedsymbol}{backgroundcolor=white}
% Parameters:
% 1. Theorem name.
% 2. Color of theorem.
% 3. Any extra parameters to pass directly to declaretheoremstyle.
% 4. Any extra parameters to pass directly to mdframed.

% change backgroundcolor to #2!5 if user wants a colored backdrop to theorem environments. It's a cool color theme, but there's too much going on in the page.
\newcommand\createnewcoloredtheoremstyle[4]{
    \declaretheoremstyle[
        headfont=\bfseries\sffamily\color{#2}, bodyfont=\normalfont, #3,
        mdframed={
                linewidth=2pt,
                rightline=false, leftline=true, topline=false, bottomline=false,
                linecolor=#2, backgroundcolor=white, #4,
            },
    ]{#1}
}



%%%%%%%%%%%%%%%%%%%%%%%%%%%%%%%%%%%
%  Create the Environment Styles  %
%%%%%%%%%%%%%%%%%%%%%%%%%%%%%%%%%%%

\makeatletter
\@ifclasswith\class{nocolor}{
    % Environments without color.

    % ESSENTIALS:
    \createnewtheoremstyle{thmdefinitionbox}{}{}
    \createnewtheoremstyle{thmtheorembox}{}{}
    \createnewtheoremstyle{thmproofbox}{qed=\qedsymbol}{}
    \createnewtheoremstyle{thmcorollarybox}{}{}
    \createnewtheoremstyle{thmlemmabox}{}{}
    \createnewtheoremstyle{thmclaimbox}{}{}
    \createnewtheoremstyle{thmexamplebox}{}{}

    % MISCS: 
    \createnewtheoremstyle{thmpropbox}{}{}
    \createnewtheoremstyle{thmexercisebox}{}{}
    \createnewtheoremstyle{thmexplanationbox}{}{}
    \createnewtheoremstyle{thmremarkbox}{}{}

    % STYLIZED MORE BELOW
    \createnewtheoremstyle{thmquestionbox}{}{}
    \createnewtheoremstyle{thmsolutionbox}{qed=\qedsymbol}{}
}{
    % Environments with color.

    % ESSENTIALS: definition, theorem, proof, corollary, lemma, claim, example
    \createnewcoloredtheoremstyle{thmdefinitionbox}{definition_color}{}{
        leftline=true
    }
    \createnewcoloredtheoremstyle{thmtheorembox}{theorem_color}{}{
        leftline=true
    }
    \createnewcoloredtheoremstyle{thmproofbox}{proof_color}{qed=\qedsymbol}{backgroundcolor=white}
    \createnewcoloredtheoremstyle{thmcorollarybox}{corollary_color}{}{backgroundcolor=white}
    \createnewcoloredtheoremstyle{thmlemmabox}{lemma_color}{}{backgroundcolor=white}
    \createnewcoloredtheoremstyle{thmclaimbox}{claim_color}{}{}
    \createnewcoloredtheoremstyle{thmexamplebox}{example_color}{}{backgroundcolor=white}
    \createnewcoloredtheoremstyle{thmexplanationbox}{example_color}{qed=\qedsymbol}{backgroundcolor=white}
    \createnewcoloredtheoremstyle{thmremarkbox}{theorem_color}{}{backgroundcolor=white}

    \createnewcoloredtheoremstyle{thmmiscbox}{black}{}{leftline=false,backgroundcolor=white}


    \createnewcoloredtheoremstyle{thmpropbox}{prop_color}{}{backgroundcolor=white}
    \createnewcoloredtheoremstyle{thmexercisebox}{exercise_color}{}{backgroundcolor=white}

    \createnewcoloredtheoremstyle{thmproblembox}{myred}{}{backgroundcolor=white}
    \createnewcoloredtheoremstyle{thmsolutionbox}{mygreen}{qed=\qedsymbol}{backgroundcolor=white}
}
\makeatother

%%%%%%%%%%%%%%%%%%%%%%%%%%%%%
%  Create the Environments  %
%%%%%%%%%%%%%%%%%%%%%%%%%%%%%
\declaretheorem[numberwithin=section, style=thmdefinitionbox,     name=Definition]{definition0}
\declaretheorem[numberwithin=section, style=thmtheorembox,     name=Theorem]{theorem}
\declaretheorem[numbered=no,          style=thmexamplebox,     name=Example]{example}
\declaretheorem[numberwithin=section, style=thmclaimbox,       name=Claim]{claim}
\declaretheorem[numberwithin=section, style=thmcorollarybox,   name=Corollary]{corollary}
\declaretheorem[numberwithin=section, style=thmpropbox,        name=Proposition]{proposition}
\declaretheorem[numberwithin=section, style=thmlemmabox,       name=Lemma]{lemma}
\declaretheorem[numberwithin=section, style=thmexercisebox,    name=Exercise]{exercise}
\declaretheorem[numbered=no,          style=thmproofbox,       name=Proof]{replacementproof}
\declaretheorem[numbered=no,          style=thmexplanationbox, name=Explanation]{explanation}
\declaretheorem[numbered=no,          style=thmsolutionbox,    name=Solution]{solution}
\declaretheorem[numberwithin=section,          style=thmproblembox,     name=Problem]{problem}
\declaretheorem[numbered=no,          style=thmmiscbox,    name=Intuition]{intuition}
\declaretheorem[numbered=no,          style=thmmiscbox,    name=Goal]{goal}
\declaretheorem[numbered=no,          style=thmmiscbox,    name=Recall]{recall}
\declaretheorem[numbered=no,          style=thmmiscbox,    name=Motivation]{motivation}
\declaretheorem[numbered=no,          style=thmmiscbox,    name=Remark]{remark}
\declaretheorem[numbered=no,          style=thmmiscbox,    name=Observe]{observe}




%%%% FANCY GRAPHICS:

% \makeatletter
% \@ifclasswith\class{nocolor}{
%   % Environments without color.

%   \newtheorem*{note}{Note}

%   \declaretheorem[numberwithin=section, style=thmdefinitionbox, name=Definition]{definition}
%   \declaretheorem[numberwithin=section, style=thmquestionbox,   name=Question]{question}
%   \declaretheorem[numberwithin=section, style=thmsolutionbox,   name=Solution]{solution}
% }{
%   % Environments with color.

%   \newtcbtheorem[number within=section]{Definition}{Definition}{
%     enhanced,
%     before skip=2mm,
%     after skip=2mm,
%     colback=red!5,
%     colframe=red!80!black,
%     colbacktitle=red!75!black,
%     boxrule=0.5mm,
%     attach boxed title to top left={
%       xshift=1cm,
%       yshift*=1mm-\tcboxedtitleheight
%     },
%     varwidth boxed title*=-3cm,
%     boxed title style={
%       interior engine=empty,
%       frame code={
%         \path[fill=tcbcolback]
%         ([yshift=-1mm,xshift=-1mm]frame.north west)
%         arc[start angle=0,end angle=180,radius=1mm]
%         ([yshift=-1mm,xshift=1mm]frame.north east)
%         arc[start angle=180,end angle=0,radius=1mm];
%         \path[left color=tcbcolback!60!black,right color=tcbcolback!60!black,
%         middle color=tcbcolback!80!black]
%         ([xshift=-2mm]frame.north west) -- ([xshift=2mm]frame.north east)
%         [rounded corners=1mm]-- ([xshift=1mm,yshift=-1mm]frame.north east)
%         -- (frame.south east) -- (frame.south west)
%         -- ([xshift=-1mm,yshift=-1mm]frame.north west)
%         [sharp corners]-- cycle;
%       },
%     },
%     fonttitle=\bfseries,
%     title={#2},
%     #1
%   }{def}

%   \NewDocumentEnvironment{definition}{O{}O{}}
%     {\begin{Definition}{#1}{#2}}{\end{Definition}}

%   \newtcolorbox{note}[1][]{%
%     enhanced jigsaw,
%     colback=gray!20!white,%
%     colframe=gray!80!black,
%     size=small,
%     boxrule=1pt,
%     title=\textbf{Note:-},
%     halign title=flush center,
%     coltitle=black,
%     breakable,
%     drop shadow=black!50!white,
%     attach boxed title to top left={xshift=1cm,yshift=-\tcboxedtitleheight/2,yshifttext=-\tcboxedtitleheight/2},
%     minipage boxed title=1.5cm,
%     boxed title style={%
%       colback=white,
%       size=fbox,
%       boxrule=1pt,
%       boxsep=2pt,
%       underlay={%
%         \coordinate (dotA) at ($(interior.west) + (-0.5pt,0)$);
%         \coordinate (dotB) at ($(interior.east) + (0.5pt,0)$);
%         \begin{scope}
%           \clip (interior.north west) rectangle ([xshift=3ex]interior.east);
%           \filldraw [white, blur shadow={shadow opacity=60, shadow yshift=-.75ex}, rounded corners=2pt] (interior.north west) rectangle (interior.south east);
%         \end{scope}
%         \begin{scope}[gray!80!black]
%           \fill (dotA) circle (2pt);
%           \fill (dotB) circle (2pt);
%         \end{scope}
%       },
%     },
%     #1,
%   }

%   \newtcbtheorem{Question}{Question}{enhanced,
%     breakable,
%     colback=white,
%     colframe=myblue!80!black,
%     attach boxed title to top left={yshift*=-\tcboxedtitleheight},
%     fonttitle=\bfseries,
%     title=\textbf{Question:-},
%     boxed title size=title,
%     boxed title style={%
%       sharp corners,
%       rounded corners=northwest,
%       colback=tcbcolframe,
%       boxrule=0pt,
%     },
%     underlay boxed title={%
%       \path[fill=tcbcolframe] (title.south west)--(title.south east)
%       to[out=0, in=180] ([xshift=5mm]title.east)--
%       (title.center-|frame.east)
%       [rounded corners=\kvtcb@arc] |-
%       (frame.north) -| cycle;
%     },
%     #1
%   }{def}

%   \NewDocumentEnvironment{question}{O{}O{}}
%   {\begin{Question}{#1}{#2}}{\end{Question}}

%   \newtcolorbox{Solution}{enhanced,
%     breakable,
%     colback=white,
%     colframe=mygreen!80!black,
%     attach boxed title to top left={yshift*=-\tcboxedtitleheight},
%     title=\textbf{Solution:-},
%     boxed title size=title,
%     boxed title style={%
%       sharp corners,
%       rounded corners=northwest,
%       colback=tcbcolframe,
%       boxrule=0pt,
%     },
%     underlay boxed title={%
%       \path[fill=tcbcolframe] (title.south west)--(title.south east)
%       to[out=0, in=180] ([xshift=5mm]title.east)--
%       (title.center-|frame.east)
%       [rounded corners=\kvtcb@arc] |-
%       (frame.north) -| cycle;
%     },
%   }

%   \NewDocumentEnvironment{solution}{O{}O{}}
%   {\vspace{-10pt}\begin{Solution}{#1}{#2}}{\end{Solution}}
% }
% \makeatother


%%%%% END OF FANCY GRAPHICS %%%%%%%%%%



%%%%%%%%%%%%%%%%%%%%%%%%%%%%
%  Edit Proof Environment  %
%%%%%%%%%%%%%%%%%%%%%%%%%%%%

\renewenvironment{proof}[2][\proofname]{
    % \vspace{-12pt}
    \begin{replacementproof} [#2]
}{\end{replacementproof}}

\newenvironment{definition}[1]{
    \begin{definition0}[#1]

    \hfill
        
    \vspace{0.2cm}

}{

    \vspace{0.2cm}
    \end{definition0}
}


\theoremstyle{definition}

\newtheorem*{notation}{Notation}
\newtheorem*{previouslyseen}{As previously seen}
\newtheorem*{property}{Property}
% \newtheorem*{intuition}{Intuition}
% \newtheorem*{goal}{Goal}
% \newtheorem*{recall}{Recall}
% \newtheorem*{motivation}{Motivation}
% \newtheorem*{remark}{Remark}
% \newtheorem*{observe}{Observe}

\author{Cong Hung Le Tran}


%%%% MATH SHORTHANDS %%%%
%% blackboard bold math capitals
\newcommand{\bbf}{\mathbb{F}}
\newcommand{\bbn}{\mathbb{N}}
\newcommand{\bbq}{\mathbb{Q}}
\newcommand{\bbr}{\mathbb{R}}
\newcommand{\bbz}{\mathbb{Z}}
\newcommand{\bbc}{\mathbb{C}}
\newcommand{\bbk}{\mathbb{K}}
\newcommand{\bbm}{\mathbb{M}}
\renewcommand{\phi}{\varphi}
\newcommand{\st}{\;\text{such that}\;}


% MATH 20250 %
\newcommand{\Hom}{\mathrm{Hom}}
\newcommand{\im}{\mathrm{im}}

% https://tex.stackexchange.com/questions/438612/space-between-exists-and-forall
\let\oldforall\forall
\renewcommand{\forall}{\;\oldforall\; }
\let\oldexist\exists
\renewcommand{\exists}{\;\oldexist\; }
\newcommand\existu{\;\oldexist!\: }


\renewcommand{\_}[1]{\underline{ #1 }}
\DeclarePairedDelimiter{\abs}{\lvert}{\rvert}
\DeclarePairedDelimiter{\norm}{\lVert}{\rVert}
\setlength\parindent{0pt}
\setlength{\headheight}{12.0pt}
\addtolength{\topmargin}{-12.0pt}


% Default skipping, change if you want more spacing
% \thinmuskip=3mu
% \medmuskip=4mu plus 2mu minus 4mu
% \thickmuskip=5mu plus 5mu



% \DeclareMathOperator{\ext}{ext}
% \DeclareMathOperator{\bridge}{bridge}
\title{MATH 20800: Honors Analysis in Rn II \\ \large Problem Set 3}
\date{26 Jan 2024}
\author{Hung Le Tran}
\begin{document}
\maketitle
\setcounter{section}{3}
\textbf{Textbook: Pugh's Real Mathematical Analysis, Rudin's Principles of Mathematical Analysis}

\textit{Collaborators: Duc Nguyen, Hung Pham}
\begin{problem} [Rudin 7.20 \redtext{done}]
    If $f$ is continuous on $[0, 1]$, and $\int_{0}^{1} f(x) x^n \dx = 0$ for all $n \geq 0$, prove that $f(x) = 0$ on $[0, 1]$.
\end{problem}
\begin{solution}
    Let $P(x) = \sum_{k=0}^{N} c_kx^k$ be any polynomial in $x$, then
    \begin{align*}
        \int_{0}^{1} f(x) P(x) &= \int_{0}^{1} f(x) \sum_{k=0}^{N} c_k x^k\\
        &= \sum_{k=0}^{n} c_k \int_{0}^1 f(x) x^k = 0
    \end{align*}

    From Weierstrass, we know that there exists $\{P_n\} \subset C_0([0, 1], \bbr)$ such that $P_n \unicvg f$, i.e., that \begin{equation*}
    d_{sup}(f, P_n) \cvgn 0.
    \end{equation*}
    
    From the above observation, it follows that: \begin{equation*}
    \int_{0}^{1} f(x) P_n(x) \dx = 0.
    \end{equation*}

    Also, since $f$ is continuous on compact $[0, 1]$, there exists $M \geq 0$ satisfying $\abs{f} \leq M$ on $[0, 1]$.
    
    Therefore, \begin{align*}
        \abs*{\int_{0}^{1} f(x)^2 \dx} &= \abs*{\int_{0}^{1} f(x)[f(x) - P_n(x)] \dx} \\
        &\leq \abs*{\int_{0}^{1} \abs{f} \abs{f(x) - P_n(x)} \dx} \\
        &\leq 1 \times M \times d_{sup}(f, P_n) = M d_{sup}(f, P_n)
    \end{align*}
    gets arbitrarily small. $\abs*{\int_{0}^{1} f(x)^2 \dx}  \geq 0$ so $\abs*{\int_{0}^{1} f(x)^2 \dx}  = 0$. $f^2$ is continuous and non-negative. It must therefore be concluded that $f^2 \equiv 0$ on $[0, 1]$ (otherwise, if $f^2(x_0) = c > 0$ for some $x_0$, then there is a neighborhood of size $\delta$, within which the infimum is $\geq c/2$, making the Riemann integral positive).

    Therefore $f \equiv 0$.
\end{solution}

\begin{problem} [Rudin 7.21 \redtext{done}]
Let $K$ be the unit circle in the complex plane, i.e., $\{z \in \bbc : \abs{z} = 1\}$. Consider the algebra $\cala$ of all functions of the form \begin{equation*}
f(e^{i\theta}) = \sum_{n=0}^{N}c_n e^{in\theta} \quad (\theta \in \bbr)
\end{equation*}
Show that $\cala$ separates points on $K$ and $\cala$ vanishes at no point of $K$, but nevertheless there are continuous functions on $K$ which are not in the uniform closure of $\cala$.
\end{problem}
\begin{solution}
    \textbf{1.} $\cala$ separates points on $K$: Given $e^{i\theta_1} \neq e^{i\theta_2}$, then we have $f = id, f(e^{i\theta}) = e^{i\theta}$ that trivially separates them.

    \textbf{2.} $\cala$ vanishes at no point of $K$: Given any $e^{i\theta} \in K$, $f = id$ trivally does not vanish there.

    \textbf{3.} WTS \begin{equation*}
    \int_{0}^{2\pi} f(e^{i\theta}) e^{i\theta} d\theta = 0
    \end{equation*}
    for every $f \in \cala$ and for every $g$ in the uniform closure of $\cala$.

    \textbf{3.1.} Take any $f(e^{i\theta}) = \sum_{n=0}^{N}c_n e^{in\theta} \in \cala$. Then \begin{align*}
        \int_{0}^{2\pi} f(e^{i\theta}) e^{i\theta} d\theta &= \int_{0}^{2\pi} \sum_{n=0}^{N}c_n e^{in\theta} e^{i\theta} d\theta \\
        &= \sum_{n=0}^{N} c_n \int_{0}^{2\pi} e^{i(n+1)\theta} d\theta\\
        &= \sum_{n=0}^{N} c_n \left[\frac{e^{i(n+1)\theta}}{i(n+1)}\right]^{2\pi}_0 \\
        &= 0
    \end{align*}

    \textbf{3.2.} Then, take $g$ in the uniform closure of $\cala$, i.e., for every $\epsilon > 0$, there exists $f \in A$ such that $d_{sup}(f, g) < \epsilon$.

    Accordingly, \begin{align*}
        \abs*{\int_{0}^{2\pi} g(e^{i\theta}) e^{i\theta} d\theta} &= \abs*{\int_{0}^{2\pi}[g(e^{i\theta}) - f(e^{i\theta})] e^{i\theta} d\theta} \\
        &\leq \int_{0}^{2\pi} \abs*{g(e^{i\theta}) - f(e^{i\theta})} \abs*{e^{i\theta}} d\theta \\
        &\leq 2\pi \times 1 \times d_{sup}(f, g) = 2\pi d_{sup}(f, g)
    \end{align*}
    that gets arbitrariliy small. It follows that for any $g$ in the uniform closure of $\cala$, \begin{equation*}
        \int_{0}^{2\pi} g(e^{i\theta}) e^{i\theta} d\theta = 0.
    \end{equation*}

    However, \begin{equation*}
    h(e^{i\theta}) = e^{-i\theta},
    \end{equation*}
    the complex conjugate function, which is trivially a continuous function on $K$, has \begin{equation*}
        \int_{0}^{2\pi} h(e^{i\theta}) e^{i\theta} d\theta = \int_{0}^{2\pi} 1  = 2\pi \neq 0,
    \end{equation*}
    so $h$ is not in the uniform closure of $\cala$.
\end{solution}

\begin{problem} [Rudin 7.23 \redtext{done}]
Let $P_0 = 0$ and define, for $n = 0, 1, 2, \ldots$, \begin{equation*}
P_{n+1}(x) = P_n(x) + \frac{x^2 - P_n^2(x)}{2}
\end{equation*}
Prove that $\lim_{n \to \infty} P_n(x) = \abs{x}$ uniformly on $[-1, 1]$.
\end{problem}

\begin{solution}
Observe that
\begin{align*}
    \abs{x} - P_{n+1}(x) &= \abs{x} - P_n(x) + \frac{P_n^2(x) - x^2}{2} \\
    &= \left(\abs{x} - P_n(x)\right)\left(1 - \frac{\abs{x} + P_n(x)}{2}\right) \\
\end{align*}
We will now use induction to prove that $0 \leq P_n(x) \leq P_{n+1}(x) \leq \abs{x}$ for $\abs{x} \leq 1$.

It is true for $n = 0$: $P_0(x) = 0 \leq \abs{x}, 0 \geq 0$. Suppose it is also true for $n = k$. Then
\begin{align*}
    \abs{x} - P_{k+1}(x) = \left(\abs{x} - P_k(x)\right)\left(1 - \frac{\abs{x} + P_k(x)}{2}\right)
\end{align*}
Then: 
\begin{equation*}
    1 - \frac{\abs{x} + P_k(x)}{2} \geq 1 - \frac{\abs{x} + \abs{x}}{2} = 1 - \abs{x} \geq 0,  \abs{x} - P_k(x) \geq 0
\end{equation*}
and
\begin{equation*}
    1 - \frac{\abs{x} + P_k(x)}{2}  \leq 1 - 0 = 1
\end{equation*}
so \begin{equation*}
    \abs{x} - P_{k+1}(x) = \left(\abs{x} - P_k(x)\right)\left(1 - \frac{\abs{x} + P_k(x)}{2}\right) \leq \abs{x} - P_k(x), \geq 0.
\end{equation*}
so it follows that $0 \leq P_k(x) \leq P_{k+1}(x) \leq \abs{x}$. By induction, $0 \leq P_n(x) \leq P_{n+1}(x) \leq \abs{x}$ is true for all $n \in \bbn$ (on $[-1, 1]$).

Then, we can apply \begin{align*}
\abs{x} - P_{n+1}(x) &= \left(\abs{x} - P_n(x)\right)\left(1 - \frac{\abs{x} + P_n(x)}{2}\right) \\
&\leq \left(\abs{x} - P_n(x)\right) \left(1 - \frac{\abs{x}}{2}\right)
\end{align*}
iteratively to get \begin{equation*}
    0 \leq \abs{x} - P_{n}(x) \leq (\abs{x} - P_0(x)) \left(1 - \frac{\abs{x}}{2}\right)^n = \abs{x} \left(1 - \frac{\abs{x}}{2}\right)^n
\end{equation*}
Then, for all $\epsilon > 0$, for $\abs{x} < \epsilon$, we have that for all $n \in \bbn$ that $\abs{x} - P_{n}(x) \leq \abs{x} \times 1 < \epsilon$.

For $\abs{x} \geq \epsilon$, then \begin{equation*}
\abs{x} - P_n(x) \leq \abs{x}\left(1 - \epsilon/2\right)^n\leq \left(1 - \epsilon/2\right)^n
\end{equation*}
can get uniformly arbitrarily small, since $1 - \frac{\epsilon}{2} < 1$. 

It follows that the convergence is uniform on $[-1, 1]$.
\end{solution}

\begin{problem} [Pugh 4.55 \redtext{done}]
    Let $f$ be a real valued continuous function on the compact interval $[a, b]$. Given $\epsilon > 0$, show that there is a polynomial $p$ such that
    \begin{align*}
        p(a) &= f(a), \\
        p'(a) &= 0,\\
        \abs{p(x) - f(x)} &< \epsilon
    \end{align*}
    for all $x \in [a, b]$.
\end{problem}
\begin{solution}
    WLOG, $[a, b] = [0, 1]$, $f(a = 0) = 0$ (can always scale and translate). Our goal is now to find polynomial $p$ such that $p(0) = p'(0) = 0, d_{sup}(p, f) < \epsilon$ ($d_{sup}$ on $[0, 1]$).

    
    Since $f \in C^0([0, 1], \bbr)$. Fix $\epsilon > 0$. By Weierstrass, we know that there exists polynomial $g = \sum_{k=0}^{N} a_k x^k$ such that $d_{sup}(f, g) < \epsilon/3$. In particular, $\epsilon / 3 > d_{sup}(f, g) \geq \abs{f(0) - g(0)} = \abs{a_0}$.
    
    From the previous problem, we know that there exists polynomials $P_n(x) \unicvg \abs{x}$ on $[-1, 1]$. Restrict this to $[0, 1]$, then $P_n(x) \unicvg x$ on $[0, 1]$; and notice that in the recursive definition of $P_n(x)$, its lowest degree of $x$ is 2.

    Choose $M \in \bbn$ such that $d_{sup}(P_M, x) < \frac{\epsilon}{3\abs{a_1}}$. Let $P_M(x) = \sum_{k=1}^{L} b_k x^{k}$.
     Then construct \begin{equation*}
    p(x) = a_1P_M(x) + \sum_{k=2}^{N} a_k x^k = \sum_{k=2}^{\max\{N, L\}} c_k x^k
    \end{equation*}
    Then \begin{align*}
        d_{sup}(p, f) &\leq d_{sup}(p, g) + d_{sup}(g, f) \\
        & < \sup\{\abs*{a_0 - a_1 (P_M(x) - x)}\} + \epsilon/3 \\
        & \leq \sup\{\abs{a_0}\} + \sup\{\abs{a_1 (P_m(x) - x)}\} + \epsilon / 3 \\
        & < \epsilon/3 + \epsilon/3 + \epsilon/3 = \epsilon
    \end{align*}
    so this $p$ satisfies the third condition. How about the first 2?
    \begin{align*}
        p(0) &= \sum_{k=2}^{\max\{N, L\}} c_k 0^k = 0 \\
        p'(0) &= \sum_{k=2}^{\max\{N, L\}} kc_k0^{k-1} = 0
    \end{align*}
    And we are done.
\end{solution}

\begin{problem} [4.53 \redtext{done}]
Let $f$ be a $C^2$ function on the real line. Assume that $f$ is bounded with bounded second derivative. Let $A = \sup_x \abs{f(x)}$ and $B = \sup_x \abs{f''(x)}$. Prove that \begin{equation*}
\sup_x \abs{f'(x)} \leq 2\sqrt{AB}
\end{equation*}
\end{problem}
\begin{solution}
    Take any $x_0$. WLOG, $M = f'(x_0) > 0$. Therefore, for $t > 0$, $\abs{f'(x_0 + t) - f'(x_0) } = \abs{\int_{x_0}^{x_0+t} f''(s) ds} \leq tB$. It follows that \begin{align*}
        f'(x_0 + t) \geq f'(x_0) - tB = M - tB
    \end{align*}

    Therefore, \begin{align*}
        f(x_0 + M/B) - f(x_0) &= \int_{x_0}^{x_0 + M/B} f'(t) \dt \\
                            &\geq \int_{0}^{M/B} (M-tB) \dt \\
                            &= M^2/B - B(M/B)^2/2 = \frac{M^2}{2B}
    \end{align*}
    Therefore \begin{equation*}
        \frac{M^2}{2B} \leq f(x_0 + M/B) - f(x_0) \leq \abs{f(x_0 + M/B) - f(x_0)} \leq 2A \implies M \leq 2\sqrt{AB}
    \end{equation*}

    Since $f'(x_0) \leq 2\sqrt{AB}$ for all $x_0 \in \bbr$, it follows that $\sup f' \leq 2\sqrt{AB}$. 
\end{solution}

\begin{problem} [4.54 \redtext{done}]
Let $f$ be continuous on $\bbr$ and let \begin{equation*}
f_n(x) = \frac{1}{n} \sum_{k=0}^{n-1}f\left(x + \frac{k}{n}\right)
\end{equation*}
Prove that $f_n(x)$ converges uniformly to a limit on every finite interval $[a, b]$.
\end{problem}
\begin{solution}
    Define $g(x) = \int_{0}^{1} f(x + t) \dt$. $f$ is continuous on $\bbr$ so $g$ is well-defined, and continuous. WTS for every $[a, b]$, $f_n \unicvg g$.

    Fix $[a, b]$ and $\epsilon > 0$. $f$ is continuous on compact interval, so is uniformly continuous. Thus there exists $\delta > 0$ such that $\abs{u - v} < \delta \implies \abs{fu - fv} < \epsilon$.

    Take $N$ large enough so that $N \delta > 1$. Then for any $n \geq N$ (and thus $1/ n \leq 1/N < \delta$), we have for any $x$, \begin{align*}
    \abs{f_n(x) - g(x)} &= \abs*{\frac{1}{n} \sum_{k=0}^{n-1}f\left(x + \frac{k}{n}\right) - \int_{0}^{1} f(x + t) \dt} \\
    &\leq \sum_{k=0}^{n-1} \abs*{\frac{1}{n} f\left( x + \frac{k}{n} \right) - \int_{k/n}^{(k+1)/n} f(x + t) \dt } \\
    & \leq \sum_{k=0}^{n-1}  \abs*{\int_{k/n}^{(k+1)/n} \left(f\left( x + \frac{k}{n} \right) - f(x + t) \right) \dt} \\
    &< \sum_{k=0}^{n-1} \epsilon/n = \epsilon
    \end{align*}
    and therefore $f_n \unicvg g$ on $[a, b]$.
\end{solution}

\begin{problem} [Pugh 4.57 \redtext{done}]
Let $f$ and $f_n$ be functions from $\bbr$ to $\bbr$. Assume that $f_n(x_n) \to f(x)$ as $n \to \infty$ whenever $x_n \cvgn x$. Prove that $f$ is continuous. (Note: the functions $f_n$ are not assumed to be continuous.)
\end{problem}
\begin{solution}
    Suppose not. Then there exists $x_n \cvgn x$ such that $f(x_n) \not \to f(x)$, i.e., that there exists some $\epsilon > 0$ such that for all $N \in \bbn$, there exists some $m \geq N$ such that $\abs{f(x_m) - f(x)} \geq \epsilon$.

    For each $x_n$, take the sequence $(y_k)_{k \in \bbn} \coloneqq (y_k = x_n)_{k \in \bbn}$. Trivially, $y_k \cvgk x_n$. Therefore $f_k(x_n) = f_k(y_k) \cvgk f(x_n)$. In short, we have pointwise convergence of $\{f_k\}$ on each $x_n$. This implies there exists $M_n$ such that $k \geq M_n \implies \abs{f_k(x_n) - f(x_n)} < \epsilon/2$. We iteratively choose $M_1, M_2, \ldots$ such that they are in strict increasing order (can always take $M_{n+1} > \max\{M_1, \ldots, M_n\}$).

    Then, define a new sequence $(z_l)_{l \in \bbn}$ as follows: 
    \begin{align*}
        z_0 = \ldots = z_{M_1 - 1} &= 0 \\
        z_{M_1} = z_{M_1 + 1} = \ldots = z_{M_2 - 1} &= x_1 \\
        z_{M_2} = z_{M_2 + 1} = \ldots = z_{M_3 - 1} &= x_2 \\
        \ldots
    \end{align*}
    where $z_l = x_j$ iff $M_j \leq l < M_{j+1}$.

    From definition, notice that for $l \geq M_1$,  $\abs{f_l(z_l) - f(z_l)} < \epsilon / 2$, since their index, $l$, satisfies the pointwise condition above.

    Furthermore, $z_l \to x$. So $f_l(z_l) \to f(x)$ by hypothesis, which means there exists some $L'$ such that $\forall l \geq L'$, $\abs{f_l(z_l) - f(x)} < \epsilon/2$. Choose $L = \max\{L', M_1\}$. Then $L \leq M_N$ for some $N$. Then, for all $m \geq N$ ($\implies M_m \geq M_N \geq L$), we can pick some $z_l$ such that $z_l = x_m$, which implies, $l \geq M_m \geq L$,
    and can bound
    \begin{align*}
        \abs{f(x_m) - f(x)} & \leq \abs{f(z_l) - f_l(z_l)} + \abs{f_l(z_l) - f(x)} \\
                    &<  \epsilon/2 \:\text{(by design of sequence, $l \geq M_1$)}\:  + \epsilon/2 \:\text{($l \geq L$)}\:  \\
                    &= \epsilon
    \end{align*}
    We have therefore found $N$ that was supposed to be impossible to find from the start, \contra.

    It follows that $f$ must be continuous.
\end{solution}

\begin{problem} [Pugh 4.58 \redtext{done}]
Let $f(x), 0 \leq x \leq 1$, be a continuous real function with continuous derivative $f'(x)$. Let $M = \sup_{x \in [0, 1]} \abs{f'(x)}$. Prove, for $n = 1, 2, \ldots$, \begin{equation*}
\abs*{\frac{1}{n}\sum_{k=0}^{n-1} f\left(\frac{k}{n}\right) - \int_{0}^{1} f(x) \dx} \leq \frac{M}{2n}
\end{equation*}
\end{problem}
\begin{solution}
    We have:
    \begin{align*}
        \abs*{\frac{1}{n}\sum_{k=0}^{n-1} f\left(\frac{k}{n}\right) - \int_{0}^{1} f(x) \dx} &= \abs*{\sum_{k=0}^{n-1} \frac{1}{n} f\left(\frac{k}{n}\right) - \sum_{k=0}^{n-1} \int_{k/n}^{(k+1)/n} f(x) \dx} \\
        &\leq \sum_{k=0}^{n-1} \int_{k/n}^{(k+1)/n} \abs*{f(x) - f\left(\frac{k}{n}\right)} \dx\\
        &\leq \sum_{k=0}^{n-1} \int_{k/n}^{(k+1)/n} M\left(x - \frac{k}{n}\right) \dx \\
        &=\sum_{k=0}^{n-1} \int_{0}^{1/n} M t \dt \\
        &= \sum_{k=0}^{n-1} \frac{M}{2n^2} = \frac{M}{2n}
    \end{align*}
    as required.
\end{solution}

\begin{problem} [Pugh 4.60 \redtext{done}]
Let $f$ be a continuous real-valued function on $[0, \infty)$ such that \begin{equation*}
\lim_{x \to \infty} \left(f(x) + \int_{0}^{x} f(t) \dt \right)
\end{equation*}
exists (and is finite). Prove that $\lim_{x \to \infty} f(x) = 0$.
\end{problem}
\begin{solution}
    Notice that:
    \begin{equation*}
    f(x) + \int_{0}^{x} f(t) \dt = \frac{1}{e^x} \ddx  \left(e^x \int_{0}^{x} f(t) \dt \right) = \frac{\ddx  \left(e^x \int_{0}^{x} f(t) \dt \right)}{\ddx e^x}
   \end{equation*}

   Let $g = e^x \int_{0}^{x} f(t) \dt, h = e^x$ then we have that \begin{equation*}
   \lim_{x \to \infty} \frac{g'}{h'} = \lim_{x \to \infty}  f(x) + \int_{0}^{x} f(t) \dt = L < \infty
   \end{equation*}

   We want to show that $\lim_{x \to \infty} \frac{g}{h} = L$ too. (Technically we're simply proving L'Hopital rule, but we have to be explicitly clear here, since it is not trivially clear that $g \xrightarrow{x \to \infty} \pm \infty$.)

   Take any $\epsilon > 0$. Since $\lim_{x \to \infty} \frac{g'}{h'} = L$, there exists $X_1$ such that $x \geq X_1 \implies \abs*{\frac{g'}{h'} - L} < \frac{\epsilon}{2}$.

%    Since $h(x) \xrightarrow{x \to \infty} +\infty$, there exists $X_2 > X_1$ and $X_3 > X_2$ such that $h(X_2) > 0$ and $x \geq X_3 \implies h(x) > h(X_2) > 0$.

   By construction, it follows that for all $x \geq X_1$, \begin{equation*}
   \abs*{\frac{g(x) - g(X_1)}{h(x) - h(X_1)} - L} = \abs*{\frac{g'(\theta)}{h'(\theta)} - L} < \epsilon/2
   \end{equation*}
   since $\theta \in (X_1, x) \implies \theta > X_1 \implies\abs*{\frac{g'(\theta)}{h'(\theta)} - L} < \epsilon/2$.
   
   Then we can estimate for all $x \geq X_1$:
   \begin{align*}
    \abs*{\frac{g(x)}{h(x)} - L} &= \abs*{\frac{g(x) - g(X_1)}{h(x) - h(X_1)} \frac{h(x) - h(X_1)}{h(x)} + \frac{g(X_1)}{h(x)} - L} \\
    &\leq \abs*{\frac{g(x) - g(X_1)}{h(x) - h(X_1)} - L} + \abs*{\frac{g(x) - g(X_1)}{h(x) - h(X_1)}\frac{h(X_1)}{h(x)}} + \abs*{\frac{g(X_1)}{h(x)}} \\
    &\leq \frac{\epsilon}{2} + \left(\abs{L} + \frac{\epsilon}{2}\right) \abs*{\frac{h(X_1)}{h(x)}} + \abs*{\frac{g(X_1)}{h(x)}}
   \end{align*}

   goes arbitrarily small as $h(X_1), g(X_1)$ are fixed, and $h(x) \xrightarrow{x \to \infty} +\infty$.

   All in all, it follows that \begin{equation*}
   \lim_{x \to \infty} \frac{g(x)}{h(x)} = L
   \end{equation*}
   Therefore \begin{align*}
    L &= \lim_{x \to \infty} \int_{0}^{x} f(t) \dt \\
    \implies \lim_{x \to \infty} f(x) &= \lim_{x \to \infty}(f(x) + \int_{0}^{x} f(t) \dt) - \lim_{x \to \infty} \int_{0}^{x} f(t) \dt \\
    &= L - L = 0
   \end{align*}
   as required.
\end{solution}

\begin{problem} [Pugh 4.65 \redtext{done}]
Let $f$ be a continuous, strictly increasing function from $[0, \infty)$ onto $[0, \infty)$ and let $g = f^{-1}$ (the inverse, not the reciprocal). Prove that \begin{equation*}
\int_{0}^{a} f(x) \dx + \int_{0}^{b} g(y) \dy \geq ab
\end{equation*}
\end{problem}
\begin{solution}
    Fix any $c \geq 0$. Then $g$ is strictly increasing on $[0, c]$, so it is integrable on $[0, c]$. $\int_{0}^{b} g(y) \dy$ is then well-defined.

    We use the following Lemma:
    \begin{psetlemma}
    For $a \geq 0$, \begin{equation*}
    \int_{0}^{a} f(x) \dx + \int_{0}^{f(a)} g(y) \dy = af(a)
    \end{equation*}
    \end{psetlemma}

    \begin{proof} [Lemma]
        Take any partition $P = \{x_0 = 0, x_1, \ldots, x_n = a\}$ of $[0, a]$. Then, since $f$ is a bijection and strictly increasing, $Q_P \coloneqq \{f(x_0) = f(0) = 0, f(x_1), \ldots, f(x_n) = f(a)\}$ is a partition of $[0, f(a)]$. In fact, it is clear that $P \mapsto Q_P$ is a bijective map between the set of partitions on $[0, a]$ and $[0, f(a)]$.

        Then, we have that:
        \begin{align*}
            L(f, P) + U(g, Q_P) &= \sum_{i=1}^{n} \inf_{t \in [x_{i-1}, x_i]} f(t) \Delta x_i + \sup_{s \in [f(x_{i-1}), f(x_i)]} g(s) \Delta f(x_i) \\
            &= \sum_{i=1}^{n} f(x_{i-1}) (x_i - x_{i-1}) + x_i (f(x_i) - f(x_{i-1})) \\
            &= \sum_{i=1}^{n} x_i f(x_i) - x_{i-1} f(x_{i-1}) \\
            &= x_n f(x_n) - x_0 f(x_0) = a f(a)
        \end{align*}
        Similarly, \begin{equation*}
            U(f, P) + L(g, Q_P) = a f(a)
        \end{equation*}

        It follows that \begin{equation*}
            L(f, P) + U(g, Q_P) + U(f, P) + L(g, Q_P) = 2af(a)
        \end{equation*}
        The equality holds for all $P$ and corresponding $Q_P$. Fix some $\epsilon > 0$. Since $f, g$ are integrable, there exists some $P, Q$ such that
        \begin{align*}
             2\int_{0}^{a} f(x) \dx - \epsilon &\leq L(f, P) + U(f, P) &\leq 2\int_{0}^{a} f(x) \dx + \epsilon \\
            2\int_{0}^{f(a)} g(y) \dy - \epsilon &\leq L(g, Q) + U(g, Q) &\leq 2\int_{0}^{f(a)} g(y) \dy + \epsilon
        \end{align*}
        Then we can define $P'$ as the refinement of $P$ and $f^{-1}(Q) = g(Q) = \{g(y_i) = f^{-1}(y_i) : y_i \in Q\}$ on $[0, a]$, then the bound remains the same, with $P$ replaced by $P'$ and $Q$ replaced by $Q_{P'}$. It then follows that \begin{equation*}
            2 \left(\int_{0}^{a} f(x) \dx + \int_{0}^{f(a)} g(y) \dy\right) - 2\epsilon \leq 2a f(a) \leq  2 \left(\int_{0}^{a} f(x) \dx + \int_{0}^{f(a)} g(y) \dy\right) + 2\epsilon
        \end{equation*}
        And this holds for all $\epsilon$ so it must be the case that \begin{equation*}
            \int_{0}^{a} f(x) \dx + \int_{0}^{f(a)} g(y) \dy = af(a)
        \end{equation*}
        as required.
    \end{proof}

    Now that the lemma is proven, we use it for our problem.

    \textbf{Case 1:} $b \leq f(a)$. Let $a' = g(b) \leq a$. Then
    \begin{align*}
        \int_{0}^{a} f(x) \dx + \int_{0}^{b} g(y) \dy &= \int_{a'}^{a} f(x) \dx + \left(\int_{0}^{a'} f(x) \dx + \int_{0}^{f(a')} g(y) \dy\right) \\
        &= \int_{a'}^{a} f(x) \dx + a'b \geq (a - a') f(a') + a'b = (a - a')b - a'b = ab
    \end{align*}

    \textbf{Case 2:} $b \geq f(a)$. Similarly, let $a' = g(b) \geq a$. Then
    \begin{align*}
        \int_{0}^{a} f(x) \dx + \int_{0}^{b} g(y) \dy &= \left(\int_{0}^{a} f(x) \dx + \int_{0}^{f(a)} g(y) \dy \right) + \int_{f(a)}^{b} g(y) \dy \\
        &\geq af(a) + (b - f(a)) a = ab
    \end{align*}

    From 2 cases, we are done.
\end{solution}
\end{document}