\documentclass[a4paper, 12pt]{article}
%%%%%%%%%%%%%%%%%%%%%%%%%%%%%%%%%%%%%%%%%%%%%%%%%%%%%%%%%%%%%%%%%%%%%%%%%%%%%%%
%                                Basic Packages                               %
%%%%%%%%%%%%%%%%%%%%%%%%%%%%%%%%%%%%%%%%%%%%%%%%%%%%%%%%%%%%%%%%%%%%%%%%%%%%%%%

% Gives us multiple colors.
\usepackage[usenames,dvipsnames,pdftex]{xcolor}
% Lets us style link colors.
\usepackage{hyperref}
% Lets us import images and graphics.
\usepackage{graphicx}
% Lets us use figures in floating environments.
\usepackage{float}
% Lets us create multiple columns.
\usepackage{multicol}
% Gives us better math syntax.
\usepackage{amsmath,amsfonts,mathtools,amsthm,amssymb}
% Lets us strikethrough text.
\usepackage{cancel}
% Lets us edit the caption of a figure.
\usepackage{caption}
% Lets us import pdf directly in our tex code.
\usepackage{pdfpages}
% Lets us do algorithm stuff.
\usepackage[ruled,vlined,linesnumbered]{algorithm2e}
% Use a smiley face for our qed symbol.
\usepackage{tikzsymbols}
% \usepackage{fullpage} %%smaller margins
\usepackage[shortlabels]{enumitem}

\setlist[enumerate]{font={\bfseries}} % global settings, for all lists

\usepackage{setspace}
\usepackage[margin=1in, headsep=12pt]{geometry}
\usepackage{wrapfig}
\usepackage{listings}
\usepackage{parskip}

\definecolor{codegreen}{rgb}{0,0.6,0}
\definecolor{codegray}{rgb}{0.5,0.5,0.5}
\definecolor{codepurple}{rgb}{0.58,0,0.82}
\definecolor{backcolour}{rgb}{0.95,0.95,0.95}

\lstdefinestyle{mystyle}{
    backgroundcolor=\color{backcolour},   
    commentstyle=\color{codegreen},
    keywordstyle=\color{magenta},
    numberstyle=\tiny\color{codegray},
    stringstyle=\color{codepurple},
    basicstyle=\ttfamily\footnotesize,
    breakatwhitespace=false,         
    breaklines=true,                 
    captionpos=b,                    
    keepspaces=true,                 
    numbers=left,                    
    numbersep=5pt,                  
    showspaces=false,                
    showstringspaces=false,
    showtabs=false,                  
    tabsize=2,
    numbers=none
}

\lstset{style=mystyle}
\def\class{article}


%%%%%%%%%%%%%%%%%%%%%%%%%%%%%%%%%%%%%%%%%%%%%%%%%%%%%%%%%%%%%%%%%%%%%%%%%%%%%%%
%                                Basic Settings                               %
%%%%%%%%%%%%%%%%%%%%%%%%%%%%%%%%%%%%%%%%%%%%%%%%%%%%%%%%%%%%%%%%%%%%%%%%%%%%%%%

%%%%%%%%%%%%%
%  Symbols  %
%%%%%%%%%%%%%

\let\implies\Rightarrow
\let\impliedby\Leftarrow
\let\iff\Leftrightarrow
\let\epsilon\varepsilon
%%%%%%%%%%%%
%  Tables  %
%%%%%%%%%%%%

\setlength{\tabcolsep}{5pt}
\renewcommand\arraystretch{1.5}

%%%%%%%%%%%%%%
%  SI Unitx  %
%%%%%%%%%%%%%%

\usepackage{siunitx}
\sisetup{locale = FR}

%%%%%%%%%%
%  TikZ  %
%%%%%%%%%%

\usepackage[framemethod=TikZ]{mdframed}
\usepackage{tikz}
\usepackage{tikz-cd}
\usepackage{tikzsymbols}

\usetikzlibrary{intersections, angles, quotes, calc, positioning}
\usetikzlibrary{arrows.meta}

\tikzset{
    force/.style={thick, {Circle[length=2pt]}-stealth, shorten <=-1pt}
}

%%%%%%%%%%%%%%%
%  PGF Plots  %
%%%%%%%%%%%%%%%

\usepackage{pgfplots}
\pgfplotsset{width=10cm, compat=newest}

%%%%%%%%%%%%%%%%%%%%%%%
%  Center Title Page  %
%%%%%%%%%%%%%%%%%%%%%%%

\usepackage{titling}
\renewcommand\maketitlehooka{\null\mbox{}\vfill}
\renewcommand\maketitlehookd{\vfill\null}

%%%%%%%%%%%%%%%%%%%%%%%%%%%%%%%%%%%%%%%%%%%%%%%%%%%%%%%
%  Create a grey background in the middle of the PDF  %
%%%%%%%%%%%%%%%%%%%%%%%%%%%%%%%%%%%%%%%%%%%%%%%%%%%%%%%

\usepackage{eso-pic}
\newcommand\definegraybackground{
    \definecolor{reallylightgray}{HTML}{FAFAFA}
    \AddToShipoutPicture{
        \ifthenelse{\isodd{\thepage}}{
            \AtPageLowerLeft{
                \put(\LenToUnit{\dimexpr\paperwidth-222pt},0){
                    \color{reallylightgray}\rule{222pt}{297mm}
                }
            }
        }
        {
            \AtPageLowerLeft{
                \color{reallylightgray}\rule{222pt}{297mm}
            }
        }
    }
}

%%%%%%%%%%%%%%%%%%%%%%%%
%  Modify Links Color  %
%%%%%%%%%%%%%%%%%%%%%%%%

\hypersetup{
    % Enable highlighting links.
    colorlinks,
    % Change the color of links to blue.
    urlcolor=blue,
    % Change the color of citations to black.
    citecolor={black},
    % Change the color of url's to blue with some black.
    linkcolor={blue!80!black}
}

%%%%%%%%%%%%%%%%%%
% Fix WrapFigure %
%%%%%%%%%%%%%%%%%%

\newcommand{\wrapfill}{\par\ifnum\value{WF@wrappedlines}>0
        \parskip=0pt
        \addtocounter{WF@wrappedlines}{-1}%
        \null\vspace{\arabic{WF@wrappedlines}\baselineskip}%
        \WFclear
    \fi}

%%%%%%%%%%%%%%%%%
% Multi Columns %
%%%%%%%%%%%%%%%%%

\let\multicolmulticols\multicols
\let\endmulticolmulticols\endmulticols

\RenewDocumentEnvironment{multicols}{mO{}}
{%
    \ifnum#1=1
        #2%
    \else % More than 1 column
        \multicolmulticols{#1}[#2]
    \fi
}
{%
    \ifnum#1=1
    \else % More than 1 column
        \endmulticolmulticols
    \fi
}

\newlength{\thickarrayrulewidth}
\setlength{\thickarrayrulewidth}{5\arrayrulewidth}


%%%%%%%%%%%%%%%%%%%%%%%%%%%%%%%%%%%%%%%%%%%%%%%%%%%%%%%%%%%%%%%%%%%%%%%%%%%%%%%
%                           School Specific Commands                          %
%%%%%%%%%%%%%%%%%%%%%%%%%%%%%%%%%%%%%%%%%%%%%%%%%%%%%%%%%%%%%%%%%%%%%%%%%%%%%%%

%%%%%%%%%%%%%%%%%%%%%%%%%%%
%  Initiate New Counters  %
%%%%%%%%%%%%%%%%%%%%%%%%%%%

\newcounter{lecturecounter}

%%%%%%%%%%%%%%%%%%%%%%%%%%
%  Helpful New Commands  %
%%%%%%%%%%%%%%%%%%%%%%%%%%

\makeatletter

\newcommand\resetcounters{
    % Reset the counters for subsection, subsubsection and the definition
    % all the custom environments.
    \setcounter{subsection}{0}
    \setcounter{subsubsection}{0}
    \setcounter{definition0}{0}
    \setcounter{paragraph}{0}
    \setcounter{theorem}{0}
    \setcounter{claim}{0}
    \setcounter{corollary}{0}
    \setcounter{proposition}{0}
    \setcounter{lemma}{0}
    \setcounter{exercise}{0}
    \setcounter{problem}{0}
    
    \setcounter{subparagraph}{0}
    % \@ifclasswith\class{nocolor}{
    %     \setcounter{definition}{0}
    % }{}
}

%%%%%%%%%%%%%%%%%%%%%
%  Lecture Command  %
%%%%%%%%%%%%%%%%%%%%%

\usepackage{xifthen}

% EXAMPLE:
% 1. \lecture{Oct 17 2022 Mon (08:46:48)}{Lecture Title}
% 2. \lecture[4]{Oct 17 2022 Mon (08:46:48)}{Lecture Title}
% 3. \lecture{Oct 17 2022 Mon (08:46:48)}{}
% 4. \lecture[4]{Oct 17 2022 Mon (08:46:48)}{}
% Parameters:
% 1. (Optional) lecture number.
% 2. Time and date of lecture.
% 3. Lecture Title.
\def\@lecture{}
\def\@lectitle{}
\def\@leccount{}
\newcommand\lecture[3]{
    \newpage

    % Check if user passed the lecture title or not.
    \def\@leccount{Lecture #1}
    \ifthenelse{\isempty{#3}}{
        \def\@lecture{Lecture #1}
        \def\@lectitle{Lecture #1}
    }{
        \def\@lecture{Lecture #1: #3}
        \def\@lectitle{#3}
    }

    \setcounter{section}{#1}
    \renewcommand\thesubsection{#1.\arabic{subsection}}
    
    \phantomsection
    \addcontentsline{toc}{section}{\@lecture}
    \resetcounters

    \begin{mdframed}
        \begin{center}
            \Large \textbf{\@leccount}
            
            \vspace*{0.2cm}
            
            \large \@lectitle
            
            
            \vspace*{0.2cm}

            \normalsize #2
        \end{center}
    \end{mdframed}

}

%%%%%%%%%%%%%%%%%%%%
%  Import Figures  %
%%%%%%%%%%%%%%%%%%%%

\usepackage{import}
\pdfminorversion=7

% EXAMPLE:
% 1. \incfig{limit-graph}
% 2. \incfig[0.4]{limit-graph}
% Parameters:
% 1. The figure name. It should be located in figures/NAME.tex_pdf.
% 2. (Optional) The width of the figure. Example: 0.5, 0.35.
\newcommand\incfig[2][1]{%
    \def\svgwidth{#1\columnwidth}
    \import{./figures/}{#2.pdf_tex}
}

\begingroup\expandafter\expandafter\expandafter\endgroup
\expandafter\ifx\csname pdfsuppresswarningpagegroup\endcsname\relax
\else
    \pdfsuppresswarningpagegroup=1\relax
\fi

%%%%%%%%%%%%%%%%%
% Fancy Headers %
%%%%%%%%%%%%%%%%%

\usepackage{fancyhdr}

% Force a new page.
\newcommand\forcenewpage{\clearpage\mbox{~}\clearpage\newpage}

% This command makes it easier to manage my headers and footers.
\newcommand\createintro{
    % Use roman page numbers (e.g. i, v, vi, x, ...)
    \pagenumbering{roman}

    % Display the page style.
    \maketitle
    % Make the title pagestyle empty, meaning no fancy headers and footers.
    \thispagestyle{empty}
    % Create a newpage.
    \newpage

    % Input the intro.tex page if it exists.
    \IfFileExists{intro.tex}{ % If the intro.tex file exists.
        % Input the intro.tex file.
        \textbf{Course}: MATH 16300: Honors Calculus III

\textbf{Section}: 43

\textbf{Professor}: Minjae Park

\textbf{At}: The University of Chicago

\textbf{Quarter}: Spring 2023

\textbf{Course materials}: Calculus by Spivak (4th Edition), Calculus On Manifolds by Spivak

\vspace{1cm}
\textbf{Disclaimer}: This document will inevitably contain some mistakes, both simple typos and serious logical and mathematical errors. Take what you read with a grain of salt as it is made by an undergraduate student going through the learning process himself. If you do find any error, I would really appreciate it if you can let me know by email at \href{mailto:conghungletran@gmail.com}{conghungletran@gmail.com}.

        % Make the pagestyle fancy for the intro.tex page.
        \pagestyle{fancy}

        % Remove the line for the header.
        \renewcommand\headrulewidth{0pt}

        % Remove all header stuff.
        \fancyhead{}

        % Add stuff for the footer in the center.
        % \fancyfoot[C]{
        %   \textit{For more notes like this, visit
        %   \href{\linktootherpages}{\shortlinkname}}. \\
        %   \vspace{0.1cm}
        %   \hrule
        %   \vspace{0.1cm}
        %   \@author, \\
        %   \term: \academicyear, \\
        %   Last Update: \@date, \\
        %   \faculty
        % }

        \newpage
    }{ % If the intro.tex file doesn't exist.
        % Force a \newpageage.
        % \forcenewpage
        \newpage
    }

    % Remove the center stuff we did above, and replace it with just the page
    % number, which is still in roman numerals.
    \fancyfoot[C]{\thepage}
    % Add the table of contents.
    \tableofcontents
    % Force a new page.
    \newpage

    % Move the page numberings back to arabic, from roman numerals.
    \pagenumbering{arabic}
    % Set the page number to 1.
    \setcounter{page}{1}

    % Add the header line back.
    \renewcommand\headrulewidth{0.4pt}
    % In the top right, add the lecture title.
    \fancyhead[R]{\footnotesize \@lecture}
    % In the top left, add the author name.
    \fancyhead[L]{\footnotesize \@author}
    % In the bottom center, add the page.
    \fancyfoot[C]{\thepage}
    % Add a nice gray background in the middle of all the upcoming pages.
    % \definegraybackground
}

\makeatother


%%%%%%%%%%%%%%%%%%%%%%%%%%%%%%%%%%%%%%%%%%%%%%%%%%%%%%%%%%%%%%%%%%%%%%%%%%%%%%%
%                               Custom Commands                               %
%%%%%%%%%%%%%%%%%%%%%%%%%%%%%%%%%%%%%%%%%%%%%%%%%%%%%%%%%%%%%%%%%%%%%%%%%%%%%%%

%%%%%%%%%%%%
%  Circle  %
%%%%%%%%%%%%

\newcommand*\circled[1]{\tikz[baseline= (char.base)]{
        \node[shape=circle,draw,inner sep=1pt] (char) {#1};}
}

%%%%%%%%%%%%%%%%%%%
%  Todo Commands  %
%%%%%%%%%%%%%%%%%%%

% \usepackage{xargs}
% \usepackage[colorinlistoftodos]{todonotes}

% \makeatletter

% \@ifclasswith\class{working}{
%     \newcommandx\unsure[2][1=]{\todo[linecolor=red,backgroundcolor=red!25,bordercolor=red,#1]{#2}}
%     \newcommandx\change[2][1=]{\todo[linecolor=blue,backgroundcolor=blue!25,bordercolor=blue,#1]{#2}}
%     \newcommandx\info[2][1=]{\todo[linecolor=OliveGreen,backgroundcolor=OliveGreen!25,bordercolor=OliveGreen,#1]{#2}}
%     \newcommandx\improvement[2][1=]{\todo[linecolor=Plum,backgroundcolor=Plum!25,bordercolor=Plum,#1]{#2}}

%     \newcommand\listnotes{
%         \newpage
%         \listoftodos[Notes]
%     }
% }{
%     \newcommandx\unsure[2][1=]{}
%     \newcommandx\change[2][1=]{}
%     \newcommandx\info[2][1=]{}
%     \newcommandx\improvement[2][1=]{}

%     \newcommand\listnotes{}
% }

% \makeatother

%%%%%%%%%%%%%
%  Correct  %
%%%%%%%%%%%%%

% EXAMPLE:
% 1. \correct{INCORRECT}{CORRECT}
% Parameters:
% 1. The incorrect statement.
% 2. The correct statement.
\definecolor{correct}{HTML}{009900}
\newcommand\correct[2]{{\color{red}{#1 }}\ensuremath{\to}{\color{correct}{ #2}}}


%%%%%%%%%%%%%%%%%%%%%%%%%%%%%%%%%%%%%%%%%%%%%%%%%%%%%%%%%%%%%%%%%%%%%%%%%%%%%%%
%                                 Environments                                %
%%%%%%%%%%%%%%%%%%%%%%%%%%%%%%%%%%%%%%%%%%%%%%%%%%%%%%%%%%%%%%%%%%%%%%%%%%%%%%%

\usepackage{varwidth}
\usepackage{thmtools}
\usepackage[most,many,breakable]{tcolorbox}

\tcbuselibrary{theorems,skins,hooks}
\usetikzlibrary{arrows,calc,shadows.blur}

%%%%%%%%%%%%%%%%%%%
%  Define Colors  %
%%%%%%%%%%%%%%%%%%%

% color prototype
% \definecolor{color}{RGB}{45, 111, 177}

% ESSENTIALS: 
\definecolor{myred}{HTML}{c74540}
\definecolor{myblue}{HTML}{072b85}
\definecolor{mygreen}{HTML}{388c46}
\definecolor{myblack}{HTML}{000000}

\colorlet{definition_color}{myred}

\colorlet{theorem_color}{myblue}
\colorlet{lemma_color}{myblue}
\colorlet{prop_color}{myblue}
\colorlet{corollary_color}{myblue}
\colorlet{claim_color}{myblue}

\colorlet{proof_color}{myblack}
\colorlet{example_color}{myblack}
\colorlet{exercise_color}{myblack}

% MISCS: 
%%%%%%%%%%%%%%%%%%%%%%%%%%%%%%%%%%%%%%%%%%%%%%%%%%%%%%%%%
%  Create Environments Styles Based on Given Parameter  %
%%%%%%%%%%%%%%%%%%%%%%%%%%%%%%%%%%%%%%%%%%%%%%%%%%%%%%%%%

% \mdfsetup{skipabove=1em,skipbelow=0em}

%%%%%%%%%%%%%%%%%%%%%%
%  Helpful Commands  %
%%%%%%%%%%%%%%%%%%%%%%

% EXAMPLE:
% 1. \createnewtheoremstyle{thmdefinitionbox}{}{}
% 2. \createnewtheoremstyle{thmtheorembox}{}{}
% 3. \createnewtheoremstyle{thmproofbox}{qed=\qedsymbol}{
%       rightline=false, topline=false, bottomline=false
%    }
% Parameters:
% 1. Theorem name.
% 2. Any extra parameters to pass directly to declaretheoremstyle.
% 3. Any extra parameters to pass directly to mdframed.
\newcommand\createnewtheoremstyle[3]{
    \declaretheoremstyle[
        headfont=\bfseries\sffamily, bodyfont=\normalfont, #2,
        mdframed={
                #3,
            },
    ]{#1}
}

% EXAMPLE:
% 1. \createnewcoloredtheoremstyle{thmdefinitionbox}{definition}{}{}
% 2. \createnewcoloredtheoremstyle{thmexamplebox}{example}{}{
%       rightline=true, leftline=true, topline=true, bottomline=true
%     }
% 3. \createnewcoloredtheoremstyle{thmproofbox}{proof}{qed=\qedsymbol}{backgroundcolor=white}
% Parameters:
% 1. Theorem name.
% 2. Color of theorem.
% 3. Any extra parameters to pass directly to declaretheoremstyle.
% 4. Any extra parameters to pass directly to mdframed.

% change backgroundcolor to #2!5 if user wants a colored backdrop to theorem environments. It's a cool color theme, but there's too much going on in the page.
\newcommand\createnewcoloredtheoremstyle[4]{
    \declaretheoremstyle[
        headfont=\bfseries\sffamily\color{#2},
        bodyfont=\normalfont,
        headpunct=,
        headformat = \NAME~\NUMBER\NOTE \hfill\smallskip\linebreak,
        #3,
        mdframed={
                outerlinewidth=0.75pt,
                rightline=false,
                leftline=false,
                topline=false,
                bottomline=false,
                backgroundcolor=white,
                skipabove = 5pt,
                skipbelow = 0pt,
                linecolor=#2,
                innertopmargin = 0pt,
                innerbottommargin = 0pt,
                innerrightmargin = 4pt,
                innerleftmargin= 6pt,
                leftmargin = -6pt,
                #4,
            },
    ]{#1}
}



%%%%%%%%%%%%%%%%%%%%%%%%%%%%%%%%%%%
%  Create the Environment Styles  %
%%%%%%%%%%%%%%%%%%%%%%%%%%%%%%%%%%%

\makeatletter
\@ifclasswith\class{nocolor}{
    % Environments without color.

    % ESSENTIALS:
    \createnewtheoremstyle{thmdefinitionbox}{}{}
    \createnewtheoremstyle{thmtheorembox}{}{}
    \createnewtheoremstyle{thmproofbox}{qed=\qedsymbol}{}
    \createnewtheoremstyle{thmcorollarybox}{}{}
    \createnewtheoremstyle{thmlemmabox}{}{}
    \createnewtheoremstyle{thmclaimbox}{}{}
    \createnewtheoremstyle{thmexamplebox}{}{}

    % MISCS: 
    \createnewtheoremstyle{thmpropbox}{}{}
    \createnewtheoremstyle{thmexercisebox}{}{}
    \createnewtheoremstyle{thmexplanationbox}{}{}
    \createnewtheoremstyle{thmremarkbox}{}{}
    
    % STYLIZED MORE BELOW
    \createnewtheoremstyle{thmquestionbox}{}{}
    \createnewtheoremstyle{thmsolutionbox}{qed=\qedsymbol}{}
}{
    % Environments with color.

    % ESSENTIALS: definition, theorem, proof, corollary, lemma, claim, example
    \createnewcoloredtheoremstyle{thmdefinitionbox}{definition_color}{}{leftline=false}
    \createnewcoloredtheoremstyle{thmtheorembox}{theorem_color}{}{leftline=false}
    \createnewcoloredtheoremstyle{thmproofbox}{proof_color}{qed=\qedsymbol}{}
    \createnewcoloredtheoremstyle{thmcorollarybox}{corollary_color}{}{leftline=false}
    \createnewcoloredtheoremstyle{thmlemmabox}{lemma_color}{}{leftline=false}
    \createnewcoloredtheoremstyle{thmpropbox}{prop_color}{}{leftline=false}
    \createnewcoloredtheoremstyle{thmclaimbox}{claim_color}{}{leftline=false}
    \createnewcoloredtheoremstyle{thmexamplebox}{example_color}{}{}
    \createnewcoloredtheoremstyle{thmexplanationbox}{example_color}{qed=\qedsymbol}{}
    \createnewcoloredtheoremstyle{thmremarkbox}{theorem_color}{}{}

    \createnewcoloredtheoremstyle{thmmiscbox}{black}{}{}

    \createnewcoloredtheoremstyle{thmexercisebox}{exercise_color}{}{}
    \createnewcoloredtheoremstyle{thmproblembox}{theorem_color}{}{leftline=false}
    \createnewcoloredtheoremstyle{thmsolutionbox}{mygreen}{qed=\qedsymbol}{}
}
\makeatother

%%%%%%%%%%%%%%%%%%%%%%%%%%%%%
%  Create the Environments  %
%%%%%%%%%%%%%%%%%%%%%%%%%%%%%
\declaretheorem[numberwithin=section, style=thmdefinitionbox,     name=Definition]{definition}
\declaretheorem[numberwithin=section, style=thmtheorembox,     name=Theorem]{theorem}
\declaretheorem[numbered=no,          style=thmexamplebox,     name=Example]{example}
\declaretheorem[numberwithin=section, style=thmtheorembox,       name=Claim]{claim}
\declaretheorem[numberwithin=section, style=thmcorollarybox,   name=Corollary]{corollary}
\declaretheorem[numberwithin=section, style=thmpropbox,        name=Proposition]{proposition}
\declaretheorem[numberwithin=section, style=thmlemmabox,       name=Lemma]{lemma}
\declaretheorem[numberwithin=section, style=thmexercisebox,    name=Exercise]{exercise}
\declaretheorem[numbered=no,          style=thmproofbox,       name=Proof]{proof0}
\declaretheorem[numbered=no,          style=thmexplanationbox, name=Explanation]{explanation}
\declaretheorem[numbered=no,          style=thmsolutionbox,    name=Solution]{solution}
\declaretheorem[numberwithin=section,          style=thmproblembox,     name=Problem]{problem}
\declaretheorem[numbered=no,          style=thmmiscbox,    name=Intuition]{intuition}
\declaretheorem[numbered=no,          style=thmmiscbox,    name=Goal]{goal}
\declaretheorem[numbered=no,          style=thmmiscbox,    name=Recall]{recall}
\declaretheorem[numbered=no,          style=thmmiscbox,    name=Motivation]{motivation}
\declaretheorem[numbered=no,          style=thmmiscbox,    name=Remark]{remark}
\declaretheorem[numbered=no,          style=thmmiscbox,    name=Observe]{observe}
\declaretheorem[numbered=no,          style=thmmiscbox,    name=Question]{question}


%%%%%%%%%%%%%%%%%%%%%%%%%%%%
%  Edit Proof Environment  %
%%%%%%%%%%%%%%%%%%%%%%%%%%%%

\renewenvironment{proof}[2][\proofname]{
    % \vspace{-12pt}
    \begin{proof0} [#2]
        }{\end{proof0}}

\theoremstyle{definition}

\newtheorem*{notation}{Notation}
\newtheorem*{previouslyseen}{As previously seen}
\newtheorem*{property}{Property}
% \newtheorem*{intuition}{Intuition}
% \newtheorem*{goal}{Goal}
% \newtheorem*{recall}{Recall}
% \newtheorem*{motivation}{Motivation}
% \newtheorem*{remark}{Remark}
% \newtheorem*{observe}{Observe}

\author{Hung C. Le Tran}


%%%% MATH SHORTHANDS %%%%
%% blackboard bold math capitals
\DeclareMathOperator*{\esssup}{ess\,sup}
\DeclareMathOperator*{\Hom}{Hom}
\newcommand{\bbf}{\mathbb{F}}
\newcommand{\bbn}{\mathbb{N}}
\newcommand{\bbq}{\mathbb{Q}}
\newcommand{\bbr}{\mathbb{R}}
\newcommand{\bbz}{\mathbb{Z}}
\newcommand{\bbc}{\mathbb{C}}
\newcommand{\bbk}{\mathbb{K}}
\newcommand{\bbm}{\mathbb{M}}
\newcommand{\bbp}{\mathbb{P}}
\newcommand{\bbe}{\mathbb{E}}

\newcommand{\bfw}{\mathbf{w}}
\newcommand{\bfx}{\mathbf{x}}
\newcommand{\bfX}{\mathbf{X}}
\newcommand{\bfy}{\mathbf{y}}
\newcommand{\bfyhat}{\mathbf{\hat{y}}}

\newcommand{\calb}{\mathcal{B}}
\newcommand{\calf}{\mathcal{F}}
\newcommand{\calt}{\mathcal{T}}
\newcommand{\call}{\mathcal{L}}
\renewcommand{\phi}{\varphi}

% Universal Math Shortcuts
\newcommand{\st}{\hspace*{2pt}\text{s.t.}\hspace*{2pt}}
\newcommand{\pffwd}{\hspace*{2pt}\fbox{\(\Rightarrow\)}\hspace*{10pt}}
\newcommand{\pfbwd}{\hspace*{2pt}\fbox{\(\Leftarrow\)}\hspace*{10pt}}
\newcommand{\contra}{\ensuremath{\Rightarrow\Leftarrow}}
\newcommand{\cvgn}{\xrightarrow{n \to \infty}}
\newcommand{\cvgj}{\xrightarrow{j \to \infty}}

\newcommand{\im}{\mathrm{im}}
\newcommand{\innerproduct}[2]{\langle #1, #2 \rangle}
\newcommand*{\conj}[1]{\overline{#1}}

% https://tex.stackexchange.com/questions/438612/space-between-exists-and-forall
% https://tex.stackexchange.com/questions/22798/nice-looking-empty-set
\let\oldforall\forall
\renewcommand{\forall}{\;\oldforall\; }
\let\oldexist\exists
\renewcommand{\exists}{\;\oldexist\; }
\newcommand\existu{\;\oldexist!\: }
\let\oldemptyset\emptyset
\let\emptyset\varnothing


\renewcommand{\_}[1]{\underline{#1}}
\DeclarePairedDelimiter{\abs}{\lvert}{\rvert}
\DeclarePairedDelimiter{\norm}{\lVert}{\rVert}
\DeclarePairedDelimiter\ceil{\lceil}{\rceil}
\DeclarePairedDelimiter\floor{\lfloor}{\rfloor}
\setlength\parindent{0pt}
\setlength{\headheight}{12.0pt}
\addtolength{\topmargin}{-12.0pt}


% Default skipping, change if you want more spacing
% \thinmuskip=3mu
% \medmuskip=4mu plus 2mu minus 4mu
% \thickmuskip=5mu plus 5mu

% \DeclareMathOperator{\ext}{ext}
% \DeclareMathOperator{\bridge}{bridge}
\title{MATH 20800: Honors Analysis in Rn II \\ \large Problem Set 1}
\date{07 Jan 2024}
\author{Hung Le Tran}
\begin{document}
\maketitle
\setcounter{section}{1}
\textbf{Textbook:} Pugh, \textit{Real Mathematical Analysis}
\begin{problem} [5.57 \redtext{done}]
    Show that $\rmd: \Omega^k \to \Omega^{k+1}$ is a linear vector space homomorphism.
\end{problem}
\begin{solution}
    Let $\alpha, \beta \in \Omega^k; c \in \bbr$. WLOG, $\alpha = f \rmd x_{I}, \beta = g \rmd x_{J}$ where $I, J$ are increasing $k-$tuples. The linearity of $\rmd$ for simple forms trivially implies linearity for general forms.f

    To show that $\rmd$ is a linear transformation, WTS $\rmd (\alpha + c \beta) = \rmd \alpha + c \rmd \beta$.

    If $I, J$ are the same tuple, then: \begin{align*} 
        \rmd(\alpha + c \beta) &= \rmd ((f + cg) \rmd x_I) \\
        &= \rmd (f + cg) \wedge \rmd x_I \\
        &= \left(\sum_{i=1}^{n} \frac{\partial (f + cg)}{\partial x_i}(x) \rmd x_i\right) \wedge \rmd x_I \\
        &= \left(\sum_{i=1}^{n} \frac{\partial f}{\partial x_i}(x) \rmd x_i + \sum_{i=1}^{n} c \frac{\partial g}{\partial x_i}(x) \rmd x_i\right) \wedge \rmd x_I \\
        &= \left(\sum_{i=1}^{n} \frac{\partial f}{\partial x_i}(x) \rmd x_i \right) \wedge dx_I + \left(\sum_{i=1}^{n} c \frac{\partial g}{\partial x_i}(x) \rmd x_i\right) \wedge \rmd x_I \quad \:\text{(wedge product distributes)}\: \\
        &= \left(\sum_{i=1}^{n} \frac{\partial f}{\partial x_i}(x) \rmd x_i \right) \wedge dx_I + c\left(\sum_{i=1}^{n}\frac{\partial g}{\partial x_i}(x) \rmd x_i\right) \wedge \rmd x_I \\
        &= \rmd f \wedge \rmd x_I + c \rmd g \wedge \rmd x_I \\
        &= \rmd \alpha + c \rmd \beta
    \end{align*}
    Otherwise,
    \begin{align*}
        \rmd (\alpha + c \beta ) &= \rmd (f \rmd x_I + cg \rmd x_J) \\
        &= \rmd f \wedge \rmd x_I + \rmd (cg) \wedge \rmd x_J \\
        &= \rmd \alpha + \left( \sum_{i=1}^{n} \frac{\partial (cg)}{\partial x_i} (x) \rmd x_i \right) \wedge \rmd x_J \\
        &= \rmd \alpha + \left(\sum_{i=1}^{n} c \frac{\partial g}{\partial x_i}(x) \rmd x_i \right) \wedge \rmd x_J \\
        &= \rmd \alpha + c \left(\sum_{i=1}^{n} \frac{\partial g}{\partial x_i}(x) \rmd x_i \right) \wedge \rmd x_J \\
        &= \rmd \alpha + c \rmd g \wedge \rmd x_J \\
        &= \rmd \alpha + c \rmd \beta
    \end{align*}
\end{solution}

\begin{problem} [5.61 \redtext{done}]
Assume $\rmd^2 f = 0$ for all smooth functions $f$, and prove that $\rmd^2 \omega = 0$ for all smooth $k$-forms $\omega$.
\end{problem}
\begin{solution}
    Let $\omega$ be any smooth $k-$form. If $k = 0$, then $\omega$ is a smooth function so $\rmd^2 \omega = 0$ by assumption. Otherwise, WLOG, let $\omega = f \rmd x_I$ for some increasing $k$-tuple $I$. Then 
    \begin{align*}
        \rmd^2 \omega = \rmd (\rmd \omega) &= \rmd (\rmd f \wedge \rmd x_I) \\
        &= \rmd^2 f  \wedge \rmd x_I + (-1)^{l+1} \rmd f \wedge \rmd (\rmd x_I) \\
        &= (-1)^{l+1} \rmd f \wedge \rmd (\rmd x_I)
    \end{align*}

    We compute \begin{equation*}
        \rmd (\rmd x_I) = \left(\sum_{i=1}^{n} \frac{\partial (1)}{\partial x_i}(x) \rmd x_i\right) \wedge \rmd x_I = 0
    \end{equation*}

    Therefore $\rmd^2 \omega = (-1)^{l+1} \rmd f \wedge 0 = 0$ as required.
\end{solution}

\begin{problem} [5.62 \redtext{done}]
Does there exist a continuous mapping from the circle to itself that has no fixed-point? What about the 2-torus? The 2-sphere?
\end{problem}
\begin{solution}
    Yes. We note that there exists a center for each of the 3 shapes. Taking each point to its reflection across that center is a continuous mapping, and there is no fixed point for that mapping.
\end{solution}

\begin{problem} [5.63 \redtext{done}]
Show that a smooth map $T: U \to V$ induces a linear map of cohomology groups $H^k (V) \to H^k (U)$ defined by \[
 T^*: [\omega] \mapsto [T^* \omega]
 \]

Here, $[\omega]$ denotes the equivalence class of $\omega \in Z^k (V)$ in $H^k(V)$. The question amounts to showing that the pullback of a closed form $\omega$ is closed and that its cohomology class depends only on the cohomology class of $\omega$.
\end{problem}
\begin{solution}
    \textbf{1.} WTS if $\omega \in Z^k(V)$ then $T^* \omega \in Z^k(U)$, so that the mapping $T^*$ is indeed $H^k(V) \to H^k(U)$ and thus the cohomology class $[T^* \omega]$ is well-defined.

    Let $\omega \in Z^k(V)$, which means $\rmd \omega = 0$.

    Then $\rmd T^* \omega = T^* \rmd \omega = T^* 0 = 0$ so $\rmd T^* \omega \in Z^k(U)$ as required.

    \textbf{2.} WTS if $\omega_1 \in [\omega]$ then $T^* \omega_1 \in [T^* \omega]$, i.e., that the mapping between the equivalence classes is independent of representative.

    Since $\omega_1 \in [\omega] \in H^k(V) = B^k(V)/Z_k(V)$, there exists $\lambda \in \Omega^{k-1} (V)$ such that \begin{equation*}
    \omega_1 = \omega + \rmd \lambda
    \end{equation*}

    Then \begin{align*}
        T^* \omega_1 &= T^* (\omega + \rmd \lambda) \\
        &= T^* \omega + T^* \rmd \lambda \\
        &= T^* \omega + d T^* \lambda
    \end{align*}
    But $T^* \omega \in Z^k(U), \rmd T^* \lambda \in B^k(U)$ so $T^* \omega + \rmd T^* \lambda \in [T^* \omega]$. Thus $T^* \omega_1 \in [T^* \omega]$ as required.
    
    The linearity of $T^*$ on cohomology classes is trivial from the linearity of $T^*$ on forms.
\end{solution}

\begin{problem} [Problem 1 \redtext{done}]
If $\omega$ and $\lambda$ are $k$- and $m$-forms respectively, prove that \[
\omega \wedge \lambda = (-1)^{km} \lambda \wedge \omega
\]
\end{problem}
\begin{solution}
    WLOG, let $\omega = f \rmd x_I, \lambda = g \rmd x_J$ where $I$ and $J$ are $k$- and $m$-tuples respectively.

    Then \begin{align*}
        \omega \wedge \lambda &= (f \rmd x_I) \wedge (g \rmd x_J) \\
        &= (fg) \rmd x_{i_1} \wedge \rmd x_{i_2} \ldots \wedge \rmd x_{i_k} \wedge \rmd x_{j_1} \wedge \rmd x_{j_2} \ldots \wedge \rmd x_{j_m} \\
        &= (fg) (-1)^m \rmd x_{i_1} \wedge \rmd x_{i_2} \ldots \wedge \rmd x_{i_k-1} \wedge \rmd x_{j_1} \wedge \rmd x_{j_2} \ldots \wedge \rmd x_{j_m} \wedge \rmd x_{i_k} \\
        & \:\text{(perform the same permutation $(k-1)$ more times)}\: \\
        &= (fg)(-1)^{km} \rmd x_J \wedge \rmd x_I\\
        &= (-1)^{km} (g \rmd x_J) \wedge (f \rmd x_I) = (-1)^{km} \lambda \wedge \omega
    \end{align*}

    The linearity of wedge product promotes the result for simple forms to general forms.
\end{solution}

\begin{problem} [Problem 2 \redtext{done}]
Consider the 1-form $\eta = \frac{x \rmd y - y \rmd x}{x^2 + y^2}$ on $\bbr^2 \backslash \{0\}$.

\begin{enumerate} [(a)]
\item Prove that $\rmd \eta = 0$.
\item Let $\gamma = (r \cos t, r \sin t)$ for some $r > 0$, and let $\Gamma$ be $C^1$-curve in $\bbr^2 \backslash \{0\}$ with parameter interval $[0, 2\pi]$ and $\Gamma(0) = \Gamma(2 \pi)$, such that the intervals $[\gamma(t), \Gamma(t)]$ do not contain $(0, 0)$ for any $t \in [0, 2\pi]$.

Prove that \[
\int_{\Gamma} \eta = 2\pi
\]

(Hint: For $t \in [0, 2\pi], u \in [0, 1]$, define $\Phi(t, u) = (1-u) \Gamma(t) + u \gamma(t)$. Then $\Phi$ is a 2-surface in $\bbr^2 \backslash \{0\}$ with domain $[0, 2\pi] \times [0, 1]$. Show $\partial \Phi = \Gamma - \gamma$. Deduce that $\int_{\Gamma} \eta = \int_{\gamma} \eta$ and compute $\int_{\gamma} \eta$.)
\end{enumerate}
\end{problem}
\begin{solution}
    \textbf{(a)} Let $f(x, y) = \frac{-y}{x^2 + y^2}, g(x, y) = \frac{x}{x^2 + y^2}$ then $\eta = f \rmd x + g \rmd y$.

    Then $f_y = g_x = \frac{y^2 - x^2}{(x^2 + y^2)^2}$, so 
    \begin{align*}
        \rmd \eta &= \rmd (f \rmd x + g \rmd y) \\
        &= \rmd f \wedge \rmd x + \rmd g \wedge \rmd y \\
        &= \left(f_x \rmd x + f_y \rmd y\right) \wedge \rmd x + \left(g_x \rmd x + g_y \rmd y\right) \wedge \rmd y \\
        &= f_y \rmd y \wedge \rmd x + g_x \rmd x \wedge \rmd y \\
        &= (g_x - f_y) \rmd x \wedge \rmd y = 0 \qed
    \end{align*}

    \textbf{(b)}
    $\partial \Phi$ is the 1-surface:
    \begin{align*}
        \partial \Phi (x) &= \delta^1 \Phi(x) - \delta^2 \Phi (x) \\
        &= ((1-x) \Gamma(2\pi) + x \gamma(2\pi)) - ((1-x) \Gamma(0) + x \gamma(0)) \\
        &- (\gamma(x) - \Gamma(x)) \\
        &= (1-x) (\Gamma(2\pi) - \Gamma(0)) + x(\gamma(2\pi) - \gamma(0)) + \Gamma(x) - \gamma(x) \\
        &= \Gamma(x) - \gamma(x)
    \end{align*}

    By Stokes' Theorem, since $\rmd \eta = 0$, we therefore get: 
    \begin{equation*}
        0 = \int_{\Phi} \rmd \eta = \int_{\partial \Phi} \eta = \int_{\Gamma - \gamma} \eta = \int_{\Gamma} \eta - \int_{\gamma} \eta
    \end{equation*}
    which implies \begin{equation*}
        \int_{\Gamma} \eta = \int_{\gamma} \eta
    \end{equation*}
We can compute: \begin{align*}
        \int_{\gamma} \eta &= \int_{0}^{2\pi} \frac{-r \sin t}{r^2} (-r \sin t) + \frac{r \cos t}{r^2} (r \cos t) \dt \\
        &= \int_{0}^{2\pi}1 \dt = 2\pi
    \end{align*}
    Therefore \begin{equation*}
        \int_{\Gamma} \eta = \int_{\gamma} \eta = 2\pi
    \end{equation*}
    as required.
\end{solution}

\begin{problem} [Problem 3 \redtext{done}]
Define $\zeta$ on $\bbr^3 \backslash \{0\}$ by \begin{equation*}
\zeta = \frac{x \rmd y \wedge \rmd z + y \rmd z \wedge \rmd x + z \rmd x \wedge \rmd y}{r^3} \quad (r = (x^2 + y^2 + z^2)^{1/2}),
\end{equation*}

Let $D = [0, \pi] \times [0, 2\pi]$ and let $\Sigma$ be the 2-surface in $\bbr^3$ defined on $D$ given by \begin{equation*}
x = \sin u \cos v, y = \sin u \sin v, z = \cos u \quad (u \in [0, \pi], v \in [0, 2\pi])
\end{equation*}

\begin{enumerate} [(a)]
\item Prove $\rmd \zeta = 0$ in $\bbr^3 \backslash \{0\}$
\item Let $S$ denote the restriction of $\Sigma$ to $E \subset D$. Prove \begin{equation*}
\int_S \zeta = \int_E \sin u \rmd u \rmd v
\end{equation*}
\item Suppose $g, h_1, h_2, h_3$ are $C^2-$functions on $[0, 1]$ and let $(x, y, z) = \Phi(s, t)$ be the 2-surface $x = g(t)h_1(s), y = g(t) h_2(s), z = g(t) h_3(s)$. Prove, using the definition of forms, that \begin{equation*}
\int_{\Phi} \zeta = 0
\end{equation*}
\end{enumerate}
\end{problem}
\begin{solution}
\textbf{(a)}
We note that all basic 3-forms with overlapping indices are 0, so the only relevant basic 3-forms in this computation are those in which all 3 indices appear. 
Preliminarily, 
\begin{equation*}
    \frac{\partial}{\partial x}\left( \frac{x}{(x^2 + y^2 + z^2)^{3/2}} \right)= \frac{y^2 + z^2 - 2x^2}{r^2}
\end{equation*}
and similarly for $y, z$. Then,
\begin{align*}
\rmd \zeta &= \frac{y^2 + z^2 - 2x^2}{r^2} \dx \wedge \dy \wedge \dz + \frac{z^2 + x^2 - 2y^2}{r^2} \dy \wedge \dz \wedge \dx \\
&+ \frac{x^2 + y^2 - 2z^2}{r^2} \dz \wedge \dx \wedge \dy \\
&= \left(\frac{y^2 + z^2 - 2x^2}{r^2} + \frac{z^2 + x^2 - 2y^2}{r^2} + \frac{x^2 + y^2 - 2z^2}{r^2}\right) \dx \wedge \dy \wedge \dz \\
&= 0 \qed
\end{align*}

\textbf{(b)}
Since $x = \sin u \cos v, y = \sin u \sin v, z = \cos u$, we can preliminarily compute:
\begin{align*}
    \frac{\partial \Sigma_{(2, 3)}}{\partial (u, v)} &= \det \begin{bmatrix}
        \cos u \sin v & \sin u \cos v \\
        -\sin u & 0
    \end{bmatrix} = \sin^2 u \cos v \\
    \frac{\partial \Sigma_{(3, 1)}}{\partial (u, v)} &= \det \begin{bmatrix}
        -\sin u & 0 \\
        \cos u \cos v & -\sin u \sin v
    \end{bmatrix} = \sin^2 u \sin v \\
    \frac{\partial \Sigma_{(1, 2)}}{\partial (u, v)} &= \det \begin{bmatrix}
        \cos u \cos v & -\sin u \sin v \\
        \cos u \sin v & \sin u \cos v
    \end{bmatrix} = \sin u \cos u (\cos^2 v + \sin^2 v) = \sin u \cos u
\end{align*}
therefore
\begin{align*}
    \int_S \zeta &= \int_S \frac{x}{r^3}\dy \wedge \dz + \frac{y}{r^3} \dz \wedge \dx + \frac{z}{r^3} \dx \wedge \dy \\
    &= \int_E \left(\sin u \cos v \frac{\partial \Sigma_{(2, 3)}}{\partial (u, v)} + \sin u \sin v \frac{\partial \Sigma_{(3, 1)}}{\partial (u, v)} + \cos u \frac{\partial \Sigma_{(1, 2)}}{\partial (u, v)} \right) \du \dv\\
    &= \int_E \left(\sin u \cos v \sin^2 u \cos v + \sin u \sin v \sin^2 u \sin v + \cos u \sin u \cos v\right) \du \dv\\
    &= \int_E (\sin^3 u (\cos^2 v + \sin^2 v) + \cos^2 u \sin u) \du \dv \\
    &= \int_E (\sin u (\sin^2 u + \cos^2 u)) \du \dv \\
    &= \int_E \sin u \du \dv
\end{align*}
as required. \qed 

\textbf{(c)}
Restate that $\Phi$ is the 2-surface in $\bbr^3$, mapping $(s, t) \mapsto (g(t)h_1(s), g(t) h_2(s), g(t) h_3(s))$.

First computing the Jacobians (abuse of notation: suppressing the arguments for conciseness):
\begin{align*}
    \frac{\partial \Phi_{(2, 3)}}{\partial (u, v)} &= \det \begin{bmatrix}
        g h_2' & g' h_2 \\
        g h_3' & g' h_3
    \end{bmatrix} = gg' (h_3 h_2' - h_2 h_3')\\
    \frac{\partial \Phi_{(3, 1)}}{\partial (u, v)} &= \det \begin{bmatrix}
        g h_3' & g' h_3 \\
        g h_1' & g' h_1
    \end{bmatrix}  = gg'(h_1 h_3' - h_3 h_1') \\
    \frac{\partial \Phi_{(1, 2)}}{\partial (u, v)} &= \det \begin{bmatrix}
        g h_1' & g' h_1 \\
        g h_2' & g' h_2
    \end{bmatrix}  = gg' (h_2 h_1' - h_1 h_2') \\
    r^3 &= (g^2(h_1^2 + h_2^2 + h_3^2))^{3/2}
\end{align*}
therefore
\begin{align*}
    \int_\Phi \zeta &= \int_{I^2} \frac{1}{(g^2(h_1^2 + h_2^2 + h_3^2))^{3/2}}\left(gh_1 \frac{\partial \Phi_{(2, 3)}}{\partial (u, v)} + gh_2 \frac{\partial \Phi_{(3, 1)}}{\partial (u, v)} + gh_3 \frac{\partial \Phi_{(1, 2)}}{\partial (u, v)}\right) \du \dv \\
    &= \int_{I^2} \frac{gh_1  gg' (h_3 h_2' - h_2 h_3') + gh_2gg'(h_1 h_3' - h_3 h_1') + gh_3 gg' (h_2 h_1' - h_1 h_2')}{(g^2(h_1^2 + h_2^2 + h_3^2))^{3/2}} \du \dv \\
    &= \int_{I^2} \frac{g^2 g' (h_1h_3h_2' - h_1h_2h_3' + h_2h_1h_3' - h_2h_3h_1' + h_3h_2h_1' - h_3 h_1 h_2')}{(g^2(h_1^2 + h_2^2 + h_3^2))^{3/2}} \du \dv \\
    &= \int_{I^2} 0 \du \dv = 0
\end{align*}
\end{solution}
\end{document}