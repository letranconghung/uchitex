\documentclass[a4paper, 10pt]{article}
%%%%%%%%%%%%%%%%%%%%%%%%%%%%%%%%%%%%%%%%%%%%%%%%%%%%%%%%%%%%%%%%%%%%%%%%%%%%%%%
%                                Basic Packages                               %
%%%%%%%%%%%%%%%%%%%%%%%%%%%%%%%%%%%%%%%%%%%%%%%%%%%%%%%%%%%%%%%%%%%%%%%%%%%%%%%

% Gives us multiple colors.
\usepackage[usenames,dvipsnames,pdftex]{xcolor}
% Lets us style link colors.
\usepackage{hyperref}
% Lets us import images and graphics.
\usepackage{graphicx}
% Lets us use figures in floating environments.
\usepackage{float}
% Lets us create multiple columns.
\usepackage{multicol}
% Gives us better math syntax.
\usepackage{amsmath,amsfonts,mathtools,amsthm,amssymb}
% Lets us strikethrough text.
\usepackage{cancel}
% Lets us edit the caption of a figure.
\usepackage{caption}
% Lets us import pdf directly in our tex code.
\usepackage{pdfpages}
% Lets us do algorithm stuff.
\usepackage[ruled,vlined,linesnumbered]{algorithm2e}
% Use a smiley face for our qed symbol.
\usepackage{tikzsymbols}
% \usepackage{fullpage} %%smaller margins
\usepackage[shortlabels]{enumitem}

\setlist[enumerate]{font={\bfseries}} % global settings, for all lists

\usepackage{setspace}
\usepackage[margin=1in, headsep=12pt]{geometry}
\usepackage{wrapfig}
\usepackage{listings}
\usepackage{parskip}

\definecolor{codegreen}{rgb}{0,0.6,0}
\definecolor{codegray}{rgb}{0.5,0.5,0.5}
\definecolor{codepurple}{rgb}{0.58,0,0.82}
\definecolor{backcolour}{rgb}{0.95,0.95,0.95}

\lstdefinestyle{mystyle}{
    backgroundcolor=\color{backcolour},   
    commentstyle=\color{codegreen},
    keywordstyle=\color{magenta},
    numberstyle=\tiny\color{codegray},
    stringstyle=\color{codepurple},
    basicstyle=\ttfamily\footnotesize,
    breakatwhitespace=false,         
    breaklines=true,                 
    captionpos=b,                    
    keepspaces=true,                 
    numbers=left,                    
    numbersep=5pt,                  
    showspaces=false,                
    showstringspaces=false,
    showtabs=false,                  
    tabsize=2,
    numbers=none
}

\lstset{style=mystyle}
\def\class{article}


%%%%%%%%%%%%%%%%%%%%%%%%%%%%%%%%%%%%%%%%%%%%%%%%%%%%%%%%%%%%%%%%%%%%%%%%%%%%%%%
%                                Basic Settings                               %
%%%%%%%%%%%%%%%%%%%%%%%%%%%%%%%%%%%%%%%%%%%%%%%%%%%%%%%%%%%%%%%%%%%%%%%%%%%%%%%

%%%%%%%%%%%%%
%  Symbols  %
%%%%%%%%%%%%%

\let\implies\Rightarrow
\let\impliedby\Leftarrow
\let\iff\Leftrightarrow
\let\epsilon\varepsilon
%%%%%%%%%%%%
%  Tables  %
%%%%%%%%%%%%

\setlength{\tabcolsep}{5pt}
\renewcommand\arraystretch{1.5}

%%%%%%%%%%%%%%
%  SI Unitx  %
%%%%%%%%%%%%%%

\usepackage{siunitx}
\sisetup{locale = FR}

%%%%%%%%%%
%  TikZ  %
%%%%%%%%%%

\usepackage[framemethod=TikZ]{mdframed}
\usepackage{tikz}
\usepackage{tikz-cd}
\usepackage{tikzsymbols}

\usetikzlibrary{intersections, angles, quotes, calc, positioning}
\usetikzlibrary{arrows.meta}

\tikzset{
    force/.style={thick, {Circle[length=2pt]}-stealth, shorten <=-1pt}
}

%%%%%%%%%%%%%%%
%  PGF Plots  %
%%%%%%%%%%%%%%%

\usepackage{pgfplots}
\pgfplotsset{width=10cm, compat=newest}

%%%%%%%%%%%%%%%%%%%%%%%
%  Center Title Page  %
%%%%%%%%%%%%%%%%%%%%%%%

\usepackage{titling}
\renewcommand\maketitlehooka{\null\mbox{}\vfill}
\renewcommand\maketitlehookd{\vfill\null}

%%%%%%%%%%%%%%%%%%%%%%%%%%%%%%%%%%%%%%%%%%%%%%%%%%%%%%%
%  Create a grey background in the middle of the PDF  %
%%%%%%%%%%%%%%%%%%%%%%%%%%%%%%%%%%%%%%%%%%%%%%%%%%%%%%%

\usepackage{eso-pic}
\newcommand\definegraybackground{
    \definecolor{reallylightgray}{HTML}{FAFAFA}
    \AddToShipoutPicture{
        \ifthenelse{\isodd{\thepage}}{
            \AtPageLowerLeft{
                \put(\LenToUnit{\dimexpr\paperwidth-222pt},0){
                    \color{reallylightgray}\rule{222pt}{297mm}
                }
            }
        }
        {
            \AtPageLowerLeft{
                \color{reallylightgray}\rule{222pt}{297mm}
            }
        }
    }
}

%%%%%%%%%%%%%%%%%%%%%%%%
%  Modify Links Color  %
%%%%%%%%%%%%%%%%%%%%%%%%

\hypersetup{
    % Enable highlighting links.
    colorlinks,
    % Change the color of links to blue.
    urlcolor=blue,
    % Change the color of citations to black.
    citecolor={black},
    % Change the color of url's to blue with some black.
    linkcolor={blue!80!black}
}

%%%%%%%%%%%%%%%%%%
% Fix WrapFigure %
%%%%%%%%%%%%%%%%%%

\newcommand{\wrapfill}{\par\ifnum\value{WF@wrappedlines}>0
        \parskip=0pt
        \addtocounter{WF@wrappedlines}{-1}%
        \null\vspace{\arabic{WF@wrappedlines}\baselineskip}%
        \WFclear
    \fi}

%%%%%%%%%%%%%%%%%
% Multi Columns %
%%%%%%%%%%%%%%%%%

\let\multicolmulticols\multicols
\let\endmulticolmulticols\endmulticols

\RenewDocumentEnvironment{multicols}{mO{}}
{%
    \ifnum#1=1
        #2%
    \else % More than 1 column
        \multicolmulticols{#1}[#2]
    \fi
}
{%
    \ifnum#1=1
    \else % More than 1 column
        \endmulticolmulticols
    \fi
}

\newlength{\thickarrayrulewidth}
\setlength{\thickarrayrulewidth}{5\arrayrulewidth}


%%%%%%%%%%%%%%%%%%%%%%%%%%%%%%%%%%%%%%%%%%%%%%%%%%%%%%%%%%%%%%%%%%%%%%%%%%%%%%%
%                           School Specific Commands                          %
%%%%%%%%%%%%%%%%%%%%%%%%%%%%%%%%%%%%%%%%%%%%%%%%%%%%%%%%%%%%%%%%%%%%%%%%%%%%%%%

%%%%%%%%%%%%%%%%%%%%%%%%%%%
%  Initiate New Counters  %
%%%%%%%%%%%%%%%%%%%%%%%%%%%

\newcounter{lecturecounter}

%%%%%%%%%%%%%%%%%%%%%%%%%%
%  Helpful New Commands  %
%%%%%%%%%%%%%%%%%%%%%%%%%%

\makeatletter

\newcommand\resetcounters{
    % Reset the counters for subsection, subsubsection and the definition
    % all the custom environments.
    \setcounter{subsection}{0}
    \setcounter{subsubsection}{0}
    \setcounter{definition0}{0}
    \setcounter{paragraph}{0}
    \setcounter{theorem}{0}
    \setcounter{claim}{0}
    \setcounter{corollary}{0}
    \setcounter{proposition}{0}
    \setcounter{lemma}{0}
    \setcounter{exercise}{0}
    \setcounter{problem}{0}
    
    \setcounter{subparagraph}{0}
    % \@ifclasswith\class{nocolor}{
    %     \setcounter{definition}{0}
    % }{}
}

%%%%%%%%%%%%%%%%%%%%%
%  Lecture Command  %
%%%%%%%%%%%%%%%%%%%%%

\usepackage{xifthen}

% EXAMPLE:
% 1. \lecture{Oct 17 2022 Mon (08:46:48)}{Lecture Title}
% 2. \lecture[4]{Oct 17 2022 Mon (08:46:48)}{Lecture Title}
% 3. \lecture{Oct 17 2022 Mon (08:46:48)}{}
% 4. \lecture[4]{Oct 17 2022 Mon (08:46:48)}{}
% Parameters:
% 1. (Optional) lecture number.
% 2. Time and date of lecture.
% 3. Lecture Title.
\def\@lecture{}
\def\@lectitle{}
\def\@leccount{}
\newcommand\lecture[3]{
    \newpage

    % Check if user passed the lecture title or not.
    \def\@leccount{Lecture #1}
    \ifthenelse{\isempty{#3}}{
        \def\@lecture{Lecture #1}
        \def\@lectitle{Lecture #1}
    }{
        \def\@lecture{Lecture #1: #3}
        \def\@lectitle{#3}
    }

    \setcounter{section}{#1}
    \renewcommand\thesubsection{#1.\arabic{subsection}}
    
    \phantomsection
    \addcontentsline{toc}{section}{\@lecture}
    \resetcounters

    \begin{mdframed}
        \begin{center}
            \Large \textbf{\@leccount}
            
            \vspace*{0.2cm}
            
            \large \@lectitle
            
            
            \vspace*{0.2cm}

            \normalsize #2
        \end{center}
    \end{mdframed}

}

%%%%%%%%%%%%%%%%%%%%
%  Import Figures  %
%%%%%%%%%%%%%%%%%%%%

\usepackage{import}
\pdfminorversion=7

% EXAMPLE:
% 1. \incfig{limit-graph}
% 2. \incfig[0.4]{limit-graph}
% Parameters:
% 1. The figure name. It should be located in figures/NAME.tex_pdf.
% 2. (Optional) The width of the figure. Example: 0.5, 0.35.
\newcommand\incfig[2][1]{%
    \def\svgwidth{#1\columnwidth}
    \import{./figures/}{#2.pdf_tex}
}

\begingroup\expandafter\expandafter\expandafter\endgroup
\expandafter\ifx\csname pdfsuppresswarningpagegroup\endcsname\relax
\else
    \pdfsuppresswarningpagegroup=1\relax
\fi

%%%%%%%%%%%%%%%%%
% Fancy Headers %
%%%%%%%%%%%%%%%%%

\usepackage{fancyhdr}

% Force a new page.
\newcommand\forcenewpage{\clearpage\mbox{~}\clearpage\newpage}

% This command makes it easier to manage my headers and footers.
\newcommand\createintro{
    % Use roman page numbers (e.g. i, v, vi, x, ...)
    \pagenumbering{roman}

    % Display the page style.
    \maketitle
    % Make the title pagestyle empty, meaning no fancy headers and footers.
    \thispagestyle{empty}
    % Create a newpage.
    \newpage

    % Input the intro.tex page if it exists.
    \IfFileExists{intro.tex}{ % If the intro.tex file exists.
        % Input the intro.tex file.
        \textbf{Course}: MATH 16300: Honors Calculus III

\textbf{Section}: 43

\textbf{Professor}: Minjae Park

\textbf{At}: The University of Chicago

\textbf{Quarter}: Spring 2023

\textbf{Course materials}: Calculus by Spivak (4th Edition), Calculus On Manifolds by Spivak

\vspace{1cm}
\textbf{Disclaimer}: This document will inevitably contain some mistakes, both simple typos and serious logical and mathematical errors. Take what you read with a grain of salt as it is made by an undergraduate student going through the learning process himself. If you do find any error, I would really appreciate it if you can let me know by email at \href{mailto:conghungletran@gmail.com}{conghungletran@gmail.com}.

        % Make the pagestyle fancy for the intro.tex page.
        \pagestyle{fancy}

        % Remove the line for the header.
        \renewcommand\headrulewidth{0pt}

        % Remove all header stuff.
        \fancyhead{}

        % Add stuff for the footer in the center.
        % \fancyfoot[C]{
        %   \textit{For more notes like this, visit
        %   \href{\linktootherpages}{\shortlinkname}}. \\
        %   \vspace{0.1cm}
        %   \hrule
        %   \vspace{0.1cm}
        %   \@author, \\
        %   \term: \academicyear, \\
        %   Last Update: \@date, \\
        %   \faculty
        % }

        \newpage
    }{ % If the intro.tex file doesn't exist.
        % Force a \newpageage.
        % \forcenewpage
        \newpage
    }

    % Remove the center stuff we did above, and replace it with just the page
    % number, which is still in roman numerals.
    \fancyfoot[C]{\thepage}
    % Add the table of contents.
    \tableofcontents
    % Force a new page.
    \newpage

    % Move the page numberings back to arabic, from roman numerals.
    \pagenumbering{arabic}
    % Set the page number to 1.
    \setcounter{page}{1}

    % Add the header line back.
    \renewcommand\headrulewidth{0.4pt}
    % In the top right, add the lecture title.
    \fancyhead[R]{\footnotesize \@lecture}
    % In the top left, add the author name.
    \fancyhead[L]{\footnotesize \@author}
    % In the bottom center, add the page.
    \fancyfoot[C]{\thepage}
    % Add a nice gray background in the middle of all the upcoming pages.
    % \definegraybackground
}

\makeatother


%%%%%%%%%%%%%%%%%%%%%%%%%%%%%%%%%%%%%%%%%%%%%%%%%%%%%%%%%%%%%%%%%%%%%%%%%%%%%%%
%                               Custom Commands                               %
%%%%%%%%%%%%%%%%%%%%%%%%%%%%%%%%%%%%%%%%%%%%%%%%%%%%%%%%%%%%%%%%%%%%%%%%%%%%%%%

%%%%%%%%%%%%
%  Circle  %
%%%%%%%%%%%%

\newcommand*\circled[1]{\tikz[baseline= (char.base)]{
        \node[shape=circle,draw,inner sep=1pt] (char) {#1};}
}

%%%%%%%%%%%%%%%%%%%
%  Todo Commands  %
%%%%%%%%%%%%%%%%%%%

% \usepackage{xargs}
% \usepackage[colorinlistoftodos]{todonotes}

% \makeatletter

% \@ifclasswith\class{working}{
%     \newcommandx\unsure[2][1=]{\todo[linecolor=red,backgroundcolor=red!25,bordercolor=red,#1]{#2}}
%     \newcommandx\change[2][1=]{\todo[linecolor=blue,backgroundcolor=blue!25,bordercolor=blue,#1]{#2}}
%     \newcommandx\info[2][1=]{\todo[linecolor=OliveGreen,backgroundcolor=OliveGreen!25,bordercolor=OliveGreen,#1]{#2}}
%     \newcommandx\improvement[2][1=]{\todo[linecolor=Plum,backgroundcolor=Plum!25,bordercolor=Plum,#1]{#2}}

%     \newcommand\listnotes{
%         \newpage
%         \listoftodos[Notes]
%     }
% }{
%     \newcommandx\unsure[2][1=]{}
%     \newcommandx\change[2][1=]{}
%     \newcommandx\info[2][1=]{}
%     \newcommandx\improvement[2][1=]{}

%     \newcommand\listnotes{}
% }

% \makeatother

%%%%%%%%%%%%%
%  Correct  %
%%%%%%%%%%%%%

% EXAMPLE:
% 1. \correct{INCORRECT}{CORRECT}
% Parameters:
% 1. The incorrect statement.
% 2. The correct statement.
\definecolor{correct}{HTML}{009900}
\newcommand\correct[2]{{\color{red}{#1 }}\ensuremath{\to}{\color{correct}{ #2}}}


%%%%%%%%%%%%%%%%%%%%%%%%%%%%%%%%%%%%%%%%%%%%%%%%%%%%%%%%%%%%%%%%%%%%%%%%%%%%%%%
%                                 Environments                                %
%%%%%%%%%%%%%%%%%%%%%%%%%%%%%%%%%%%%%%%%%%%%%%%%%%%%%%%%%%%%%%%%%%%%%%%%%%%%%%%

\usepackage{varwidth}
\usepackage{thmtools}
\usepackage[most,many,breakable]{tcolorbox}

\tcbuselibrary{theorems,skins,hooks}
\usetikzlibrary{arrows,calc,shadows.blur}

%%%%%%%%%%%%%%%%%%%
%  Define Colors  %
%%%%%%%%%%%%%%%%%%%

% color prototype
% \definecolor{color}{RGB}{45, 111, 177}

% ESSENTIALS: 
\definecolor{myred}{HTML}{c74540}
\definecolor{myblue}{HTML}{072b85}
\definecolor{mygreen}{HTML}{388c46}
\definecolor{myblack}{HTML}{000000}

\colorlet{definition_color}{myred}

\colorlet{theorem_color}{myblue}
\colorlet{lemma_color}{myblue}
\colorlet{prop_color}{myblue}
\colorlet{corollary_color}{myblue}
\colorlet{claim_color}{myblue}

\colorlet{proof_color}{myblack}
\colorlet{example_color}{myblack}
\colorlet{exercise_color}{myblack}

% MISCS: 
%%%%%%%%%%%%%%%%%%%%%%%%%%%%%%%%%%%%%%%%%%%%%%%%%%%%%%%%%
%  Create Environments Styles Based on Given Parameter  %
%%%%%%%%%%%%%%%%%%%%%%%%%%%%%%%%%%%%%%%%%%%%%%%%%%%%%%%%%

% \mdfsetup{skipabove=1em,skipbelow=0em}

%%%%%%%%%%%%%%%%%%%%%%
%  Helpful Commands  %
%%%%%%%%%%%%%%%%%%%%%%

% EXAMPLE:
% 1. \createnewtheoremstyle{thmdefinitionbox}{}{}
% 2. \createnewtheoremstyle{thmtheorembox}{}{}
% 3. \createnewtheoremstyle{thmproofbox}{qed=\qedsymbol}{
%       rightline=false, topline=false, bottomline=false
%    }
% Parameters:
% 1. Theorem name.
% 2. Any extra parameters to pass directly to declaretheoremstyle.
% 3. Any extra parameters to pass directly to mdframed.
\newcommand\createnewtheoremstyle[3]{
    \declaretheoremstyle[
        headfont=\bfseries\sffamily, bodyfont=\normalfont, #2,
        mdframed={
                #3,
            },
    ]{#1}
}

% EXAMPLE:
% 1. \createnewcoloredtheoremstyle{thmdefinitionbox}{definition}{}{}
% 2. \createnewcoloredtheoremstyle{thmexamplebox}{example}{}{
%       rightline=true, leftline=true, topline=true, bottomline=true
%     }
% 3. \createnewcoloredtheoremstyle{thmproofbox}{proof}{qed=\qedsymbol}{backgroundcolor=white}
% Parameters:
% 1. Theorem name.
% 2. Color of theorem.
% 3. Any extra parameters to pass directly to declaretheoremstyle.
% 4. Any extra parameters to pass directly to mdframed.

% change backgroundcolor to #2!5 if user wants a colored backdrop to theorem environments. It's a cool color theme, but there's too much going on in the page.
\newcommand\createnewcoloredtheoremstyle[4]{
    \declaretheoremstyle[
        headfont=\bfseries\sffamily\color{#2},
        bodyfont=\normalfont,
        headpunct=,
        headformat = \NAME~\NUMBER\NOTE \hfill\smallskip\linebreak,
        #3,
        mdframed={
                outerlinewidth=0.75pt,
                rightline=false,
                leftline=false,
                topline=false,
                bottomline=false,
                backgroundcolor=white,
                skipabove = 5pt,
                skipbelow = 0pt,
                linecolor=#2,
                innertopmargin = 0pt,
                innerbottommargin = 0pt,
                innerrightmargin = 4pt,
                innerleftmargin= 6pt,
                leftmargin = -6pt,
                #4,
            },
    ]{#1}
}



%%%%%%%%%%%%%%%%%%%%%%%%%%%%%%%%%%%
%  Create the Environment Styles  %
%%%%%%%%%%%%%%%%%%%%%%%%%%%%%%%%%%%

\makeatletter
\@ifclasswith\class{nocolor}{
    % Environments without color.

    % ESSENTIALS:
    \createnewtheoremstyle{thmdefinitionbox}{}{}
    \createnewtheoremstyle{thmtheorembox}{}{}
    \createnewtheoremstyle{thmproofbox}{qed=\qedsymbol}{}
    \createnewtheoremstyle{thmcorollarybox}{}{}
    \createnewtheoremstyle{thmlemmabox}{}{}
    \createnewtheoremstyle{thmclaimbox}{}{}
    \createnewtheoremstyle{thmexamplebox}{}{}

    % MISCS: 
    \createnewtheoremstyle{thmpropbox}{}{}
    \createnewtheoremstyle{thmexercisebox}{}{}
    \createnewtheoremstyle{thmexplanationbox}{}{}
    \createnewtheoremstyle{thmremarkbox}{}{}
    
    % STYLIZED MORE BELOW
    \createnewtheoremstyle{thmquestionbox}{}{}
    \createnewtheoremstyle{thmsolutionbox}{qed=\qedsymbol}{}
}{
    % Environments with color.

    % ESSENTIALS: definition, theorem, proof, corollary, lemma, claim, example
    \createnewcoloredtheoremstyle{thmdefinitionbox}{definition_color}{}{leftline=false}
    \createnewcoloredtheoremstyle{thmtheorembox}{theorem_color}{}{leftline=false}
    \createnewcoloredtheoremstyle{thmproofbox}{proof_color}{qed=\qedsymbol}{}
    \createnewcoloredtheoremstyle{thmcorollarybox}{corollary_color}{}{leftline=false}
    \createnewcoloredtheoremstyle{thmlemmabox}{lemma_color}{}{leftline=false}
    \createnewcoloredtheoremstyle{thmpropbox}{prop_color}{}{leftline=false}
    \createnewcoloredtheoremstyle{thmclaimbox}{claim_color}{}{leftline=false}
    \createnewcoloredtheoremstyle{thmexamplebox}{example_color}{}{}
    \createnewcoloredtheoremstyle{thmexplanationbox}{example_color}{qed=\qedsymbol}{}
    \createnewcoloredtheoremstyle{thmremarkbox}{theorem_color}{}{}

    \createnewcoloredtheoremstyle{thmmiscbox}{black}{}{}

    \createnewcoloredtheoremstyle{thmexercisebox}{exercise_color}{}{}
    \createnewcoloredtheoremstyle{thmproblembox}{theorem_color}{}{leftline=false}
    \createnewcoloredtheoremstyle{thmsolutionbox}{mygreen}{qed=\qedsymbol}{}
}
\makeatother

%%%%%%%%%%%%%%%%%%%%%%%%%%%%%
%  Create the Environments  %
%%%%%%%%%%%%%%%%%%%%%%%%%%%%%
\declaretheorem[numberwithin=section, style=thmdefinitionbox,     name=Definition]{definition}
\declaretheorem[numberwithin=section, style=thmtheorembox,     name=Theorem]{theorem}
\declaretheorem[numbered=no,          style=thmexamplebox,     name=Example]{example}
\declaretheorem[numberwithin=section, style=thmtheorembox,       name=Claim]{claim}
\declaretheorem[numberwithin=section, style=thmcorollarybox,   name=Corollary]{corollary}
\declaretheorem[numberwithin=section, style=thmpropbox,        name=Proposition]{proposition}
\declaretheorem[numberwithin=section, style=thmlemmabox,       name=Lemma]{lemma}
\declaretheorem[numberwithin=section, style=thmexercisebox,    name=Exercise]{exercise}
\declaretheorem[numbered=no,          style=thmproofbox,       name=Proof]{proof0}
\declaretheorem[numbered=no,          style=thmexplanationbox, name=Explanation]{explanation}
\declaretheorem[numbered=no,          style=thmsolutionbox,    name=Solution]{solution}
\declaretheorem[numberwithin=section,          style=thmproblembox,     name=Problem]{problem}
\declaretheorem[numbered=no,          style=thmmiscbox,    name=Intuition]{intuition}
\declaretheorem[numbered=no,          style=thmmiscbox,    name=Goal]{goal}
\declaretheorem[numbered=no,          style=thmmiscbox,    name=Recall]{recall}
\declaretheorem[numbered=no,          style=thmmiscbox,    name=Motivation]{motivation}
\declaretheorem[numbered=no,          style=thmmiscbox,    name=Remark]{remark}
\declaretheorem[numbered=no,          style=thmmiscbox,    name=Observe]{observe}
\declaretheorem[numbered=no,          style=thmmiscbox,    name=Question]{question}


%%%%%%%%%%%%%%%%%%%%%%%%%%%%
%  Edit Proof Environment  %
%%%%%%%%%%%%%%%%%%%%%%%%%%%%

\renewenvironment{proof}[2][\proofname]{
    % \vspace{-12pt}
    \begin{proof0} [#2]
        }{\end{proof0}}

\theoremstyle{definition}

\newtheorem*{notation}{Notation}
\newtheorem*{previouslyseen}{As previously seen}
\newtheorem*{property}{Property}
% \newtheorem*{intuition}{Intuition}
% \newtheorem*{goal}{Goal}
% \newtheorem*{recall}{Recall}
% \newtheorem*{motivation}{Motivation}
% \newtheorem*{remark}{Remark}
% \newtheorem*{observe}{Observe}

\author{Hung C. Le Tran}


%%%% MATH SHORTHANDS %%%%
%% blackboard bold math capitals
\DeclareMathOperator*{\esssup}{ess\,sup}
\DeclareMathOperator*{\Hom}{Hom}
\newcommand{\bbf}{\mathbb{F}}
\newcommand{\bbn}{\mathbb{N}}
\newcommand{\bbq}{\mathbb{Q}}
\newcommand{\bbr}{\mathbb{R}}
\newcommand{\bbz}{\mathbb{Z}}
\newcommand{\bbc}{\mathbb{C}}
\newcommand{\bbk}{\mathbb{K}}
\newcommand{\bbm}{\mathbb{M}}
\newcommand{\bbp}{\mathbb{P}}
\newcommand{\bbe}{\mathbb{E}}

\newcommand{\bfw}{\mathbf{w}}
\newcommand{\bfx}{\mathbf{x}}
\newcommand{\bfX}{\mathbf{X}}
\newcommand{\bfy}{\mathbf{y}}
\newcommand{\bfyhat}{\mathbf{\hat{y}}}

\newcommand{\calb}{\mathcal{B}}
\newcommand{\calf}{\mathcal{F}}
\newcommand{\calt}{\mathcal{T}}
\newcommand{\call}{\mathcal{L}}
\renewcommand{\phi}{\varphi}

% Universal Math Shortcuts
\newcommand{\st}{\hspace*{2pt}\text{s.t.}\hspace*{2pt}}
\newcommand{\pffwd}{\hspace*{2pt}\fbox{\(\Rightarrow\)}\hspace*{10pt}}
\newcommand{\pfbwd}{\hspace*{2pt}\fbox{\(\Leftarrow\)}\hspace*{10pt}}
\newcommand{\contra}{\ensuremath{\Rightarrow\Leftarrow}}
\newcommand{\cvgn}{\xrightarrow{n \to \infty}}
\newcommand{\cvgj}{\xrightarrow{j \to \infty}}

\newcommand{\im}{\mathrm{im}}
\newcommand{\innerproduct}[2]{\langle #1, #2 \rangle}
\newcommand*{\conj}[1]{\overline{#1}}

% https://tex.stackexchange.com/questions/438612/space-between-exists-and-forall
% https://tex.stackexchange.com/questions/22798/nice-looking-empty-set
\let\oldforall\forall
\renewcommand{\forall}{\;\oldforall\; }
\let\oldexist\exists
\renewcommand{\exists}{\;\oldexist\; }
\newcommand\existu{\;\oldexist!\: }
\let\oldemptyset\emptyset
\let\emptyset\varnothing


\renewcommand{\_}[1]{\underline{#1}}
\DeclarePairedDelimiter{\abs}{\lvert}{\rvert}
\DeclarePairedDelimiter{\norm}{\lVert}{\rVert}
\DeclarePairedDelimiter\ceil{\lceil}{\rceil}
\DeclarePairedDelimiter\floor{\lfloor}{\rfloor}
\setlength\parindent{0pt}
\setlength{\headheight}{12.0pt}
\addtolength{\topmargin}{-12.0pt}


% Default skipping, change if you want more spacing
% \thinmuskip=3mu
% \medmuskip=4mu plus 2mu minus 4mu
% \thickmuskip=5mu plus 5mu

% \DeclareMathOperator{\ext}{ext}
% \DeclareMathOperator{\bridge}{bridge}
\title{MATH 20800: Honors Analysis in Rn II \\ \large Problem Set 5}
\date{25 Feb 2024}
\author{Hung Le Tran}
\begin{document}
\maketitle
\setcounter{section}{5}
\begin{notation}
    Let $V, W$ be normed vector spaces. Then $\calb(V, W) = \call(V, W)$ is the space of bounded (hence continuous) linear operators from $V$ to $W$.
\end{notation}
\begin{problem} [Typed Problem 1 \done]
    Let $B$ be a Banach space. \begin{enumerate}
    \item Prove that if $T \in \calb(B, B)$ and $\norm{I - T} < 1$ where $I$ is the identity operator, then $T$ is invertible and in fact $\sum_{n=0}^{\infty} (I - T)^n$ converges in $\calb(B, B)$ to $\inv{T}$.
    \item Prove that the set of invertible operators is open in $\calb(B, B)$.
    \end{enumerate}
\end{problem}
\begin{solution}
    \textbf{(a)} Write the operator \begin{equation*}
    Q \coloneqq \sum_{n=0}^{\infty} (I-T)^n : B \to B
    \end{equation*}

    A priori, $Q$ might not be well-defined pointwise at all, not even considering if $Q \in \calb(B, B)$. We intend to show at once that $Q \in \calb(B, B)$.

    First, we use the fact that if $T \in \calb(B, B)$ then $\norm{T^n} \leq \norm{T}^n$ for any $n \in \bbn$. This is true by the submultiplicativity of operator norms. It follows that 
     \begin{equation*}
        \norm{Q} \leq \sum_{n= 0}^{\infty} \norm{(I-T)^n} \leq \sum_{n= 0}^{\infty} \norm{I-T}^n < \infty
    \end{equation*}
    since $\norm{I-T} < 1$. Therefore $Q \in \calb(B, B)$.

    We now want to show that in fact $T^{-1} = Q$, by showing that $T \circ Q = Q \circ T = I$. Apply linearity:
    \begin{align*}
        T \circ Q &= \left((I - (I-T)) \circ \sum_{n=0}^{\infty} (I-T)^n\right) \\
        &= \sum_{n=0}^{\infty} (I-T)^n - \sum_{n=1}^{\infty}(I-T)^n = I
    \end{align*}
    and similarly
    \begin{align*}
        Q \circ T &= \left(\sum_{n=0}^{\infty} (I-T)^n \circ (I - (I-T))\right) \\
        &= \sum_{n=0}^{\infty} (I-T)^n - \sum_{n=1}^{\infty}(I-T)^n = I
    \end{align*}
    It follows that $T$ is invertible; its inverse is $Q \in \calb(B, B)$.

    \textbf{(b)} Let $S$ be the set of invertible operators in $\calb(B, B)$, take $T \in S$.

    Consider $\norm{\inv{T}}$. If $\norm{\inv{T}} = 0 \implies \inv{T} = 0 \implies \inv{T}T \neq I, \contra$. So $\norm{\inv{T}} > 0$. 
    
    Then for all operator $Q \in B(T, 1/\norm{\inv{T}})$ we have that
    \begin{equation*}
    \norm{T - Q} < 1/\norm{\inv{T}} \implies \norm{\inv{T}(T - Q)} < 1 \implies \norm{I - \inv{T}Q} < 1
    \end{equation*}
    which implies $\inv{T}Q$ is invertible. Composition of invertible maps is invertible. So $Q = T \circ \inv{T}Q$ is invertible.

    So we've found ball $B(T, 1/\norm{\inv{T}})$ around $T$ that is inside $S$, so $S$ is open as required.
\end{solution}

\begin{problem} [Typed Problem 2 \done]
    Let $V$ be a normed vector space and $W \subset V$ a proper closed subspace.
    \begin{enumerate}
    \item Prove that $\norm{v + W} \coloneqq \inf_{w \in W} \norm{v + w}$ is a norm on $V / W$.
    \item Prove that for any $\epsilon > 0$ there exists $v \in V$ such that $\norm{v} = 1$ and $\norm{v + W} \geq 1 - \epsilon$.
    
    Hint: Let $u \in V\backslash W$. Then $\norm{u + W} > 0$ and there exists $w \in W$ such that $\norm{u + W} \leq \norm{u + w}$ and \begin{equation*}
    \norm{u + w} \leq \norm{u + W} + \epsilon \norm{u + W}.
    \end{equation*}

    Now consider $\frac{u + w}{\norm{u + w}}$.
    \end{enumerate}
\end{problem}
\begin{solution}
    \textbf{(a)} We first have to check that this is a well-defined function, i.e., that for $\norm{v + W} = \inf_{w \in W} \norm{v + w}$, it does not matter which representative $v$ we pick on the RHS. Indeed, for any $v_1, v_2 \in v + W \implies v_1 = v_2 + w'$ for some $w' \in W$. It then follows that \begin{equation*}
    \inf_{w \in W} \norm{v_1 + w} = \inf_{w \in W} \norm{v_2 + w' + w} = \inf_{w' + w \in W} \norm{v_2 + w' + w}
    \end{equation*}
    since $W$ is a subspace of $V$. So indeed it does not matter which representative we pick.


    We check through requirements of a norm:
    \begin{itemize}
        \item $\norm{v + W}$ is infimum of non negative things, so it is nonnegative
        \item If $\norm{v + W} = 0$, fix representative $v \in v + W$, then for all $\epsilon > 0$, there exists $w \in W$ such that $\epsilon > \norm{v + w} = \norm{v - (-w)}$. Since $(-w) \in W$, it follows that $v \in \cl{W}$. $\cl{W} = W$ is $W$ is closed, so $v \in W$.
        \item $\norm{\lambda v + W} = \inf_{w \in W} \norm{\lambda v + w} = \inf_{w \in W} \norm{\lambda(v + w/\lambda)} = \abs{\lambda} \inf_{w/\lambda \in W} \norm{v + w/\lambda} = \abs{\lambda} \norm{v + W}$.
        \item For any $v_1, v_2 \in V; w \in W$, we have that  $\norm{v_1 + v_2+ w} \leq \norm{v_1 + w/2} + \norm{v_2 + w/2} \implies \norm{v_1 + v_2 + W} \leq \norm{v_1 + W} + \norm{v_2 + W}$.
    \end{itemize}
    Thus $\norm{v + W}$ is a norm.

    \textbf{(b)} Fix $\epsilon > 0$. Let $u \in V \backslash W$. Then $u + W \neq 0 \implies \norm{u + W} > 0$.

    Since $\norm{u + W} = \inf_{w \in W} \norm{u + w}$, there exists $w \in W$ such that $\norm{u + W} \leq \norm{u + w} \leq \norm{u + W} + \epsilon \norm{u + W} = (1 + \epsilon) \norm{u + W}$. It then follows that if we define $v = \frac{u+w}{\norm{u+w}}$ then clearly $\norm{v} = 1$ and \begin{equation*}
    \norm{v + W} = \norm*{\frac{u + w}{\norm{u + w}} + W} = \frac{1}{\norm{u + w}}\norm{u + W} \geq \frac{1}{1 + \epsilon} > \frac{1-\epsilon^2}{1 + \epsilon} = 1 - \epsilon
    \end{equation*}
    as required.
\end{solution}

\begin{problem} [Typed Problem 3 \done]
    Let $V$ be a Banach space and $W \subset V$ a proper closed subspace. Prove that $V / W$ with the norm defined in problem 2 is a Banach space.

    Hint: Suppose that the series $\sum_{n} (v_n + W)$ is absolutely summable, i.e., $\sum_n \norm{v_n + W}$ converges. We wish to prove that $\sum_{n} (v_n + W)$ converges in $V / W$. For each $n \in \bbn$, there exists $w_n \in W$ such that \begin{equation*}
    \norm{v_n + w_n} \leq \norm{v_n + W} + 2^{-n}
    \end{equation*}
    Then $\sum_{n} (v_n + w_n)$ is absolutely summable, and since $V$ is a Banach space, there exists $v \in V$ such that $v = \sum_{n} (v_n + w_n)$. Prove that $v + W = \sum_{n} (v_n + W)$, i.e., \begin{equation*}
    \lim_{N \to \infty} v + W - \sum_{n=1}^{N} (v_n + W) = 0
    \end{equation*}
\end{problem}
\begin{solution}
    To show that $V/W$ is Banach, we show that every absolutely summable series is summable, i.e., suppose $\sum_{n = 1}^\infty \norm{v_n + W}$ converges, WTS $\sum_{n=1}^{\infty} (v_n + W)$ converges in $V / W$.

    Indeed, by definition of $\norm{v_n + W}$, there exists some $w_n$ such that \begin{equation*}
    \norm{v_n + W} \leq \norm{v_n + w_n} \leq \norm{v_n + W} + 2^{-n} 
    \end{equation*}

    It then follows that \begin{equation*}
    \sum_{n =1}^{\infty} \norm{v_n + w_n} \leq \sum_{n=1}^{\infty} \norm{v_n + W} + \sum_{n=1}^{\infty} 2^{-n}  = \sum_{n=1}^{\infty} \norm{v_n + W} + 1
    \end{equation*}
    so $\sum_{n} \norm{v_n + w_n}$ is absolutely summable. They are points in Banach $V$, so $\sum_{n} (v_n + w_n)$ is summable in $V$. Set $v \coloneqq \sum_{n=1}^{\infty} (v_n + w_n) \in V$.

    We now WTS the partial sum $\sum_{n=1}^{N} (v_n + W) \cvgg{N \to \infty} v + w$ in $V / W$. Indeed, \begin{align*}
    \norm*{v + W - \sum_{n=1}^{N} (v_n + W)} &= \norm*{v - \sum_{n=1}^{N} (v_n + w_n)} \\
    &= \norm*{\sum_{n = N}^{\infty} (v_n + w_n)} \leq \sum_{n=N}^{\infty} \norm*{v_n + w_n} \cvgg{N \to \infty} 0
    \end{align*}
    since $\sum_{n} \norm{v_n + w_n}$ is absolutely summable.

    Therefore, it follows that $\sum_{n = 1}^{\infty} (v_n + W)$ is summable, its limit is $v + W$, so $V/W$ is Banach.
\end{solution}

\begin{problem} [Typed Problem 4 \done]
    Suppose $V$ and $W$ are Banach spaces, $T \in \calb(V, W)$ and recall the following subspaces \begin{equation*}
    \ker(T) = \{v \in V : Tv = 0\}, \quad \range(T) = \{Tv \in W : v \in V\}.
    \end{equation*}
    \begin{enumerate}
    \item Prove that $\ker(T)$ is a closed subspace of $V$.
    \item If $V_1$ and $V_2$ are normed linear spaces, we say a bijective linear operator $S: V_1 \to V_2$ is an isomorphism if $S \in \calb(V_1, V_2)$ and $\inv{S} \in \calb(V_2, V_1)$. We say $V_1$ and $V_2$ are \textit{isomorphic} if there exists an isomorphism $S: V_1 \to V_2$.
    
    Prove that $V/\ker(T)$ is isomorphic to $\range(T)$ if and only if $\range(T)$ is closed.

    Hint: Consider the map $S: V/\ker(T) \to \range(T)$ given by \begin{equation*}
    S(v + \ker(T)) = Tv,
    \end{equation*}
    and first show that $S$ is a well-defined, bijective bounded linear operator.
    \end{enumerate}
\end{problem}
\begin{solution}
    \textbf{(a)} $T \in \calb(V, W)$ so $\norm{T} < \infty$. Take sequence $(v_n)_{n \in \bbn}$ in $\ker(T)$ such that $v_n \cvgn v \in V$. WTS $v \in \ker(T)$.

    Indeed, \begin{align*}
        \norm{Tv} &=\norm{Tv - Tv_n} \\
        &= \norm{T (v - v_n)} \\
        &\leq \norm{T} \norm{v - v_n} \cvgn 0
    \end{align*}
    so $\norm{Tv} = 0 \implies Tv = 0 \implies v \in \ker(T)$. \qed

    \textbf{(b)} 

    \pffwd Suppose that $V/\ker(T)$ is isomorphic to $\range(T)$, then there exists some isomorphism $S: V/\ker(T) \to \range(T)$. Then take sequence of points $(Tv_n)_{n \in \bbn}$ in $\range(T)$ such that $Tv_n \cvgn w \in W$. WTS $w \in \range(T)$.

    Denote $w_n = Tv_n \in W$. So $(w_n)$ is Cauchy, which implies $(\inv{S}w_n)$ is Cauchy since $\norm{\inv{S}} < \infty$. $(\inv{S}w_n)$ is Cauchy in Banach $V /\ker(T)$, so converges to some $v' + \ker(T) \in V/\ker(T)$. But $S$ is also continuous, so $S(\inv{S}w_n) = w_n \cvgn S(v' + \ker(T)) \in \range(T)$. So $w = S(v' + \ker(T)) \in \range(T)$, so $\range(T)$ is closed. \qed

    \pfbwd Suppose that $\range(T)$ is closed. $\range(T)$ is closed in Banach $W$ so it is Banach. Consider the mapping \begin{align*}
    S : V / \ker(T) & \to \range(T) \\
    v + \ker(T) &\mapsto Tv
    \end{align*}
    \begin{enumerate} [1.]
    
    \item WTS this mapping is an isomorphism between $V /\ker(T)$ and $\range(T)$.

    First, it is well-defined, in the sense that it does not matter which representative of $v + \ker(T)$ we choose. Indeed, if $v_1, v_2 \in v + \ker(T)$, which means $v_1 = v_2 + u$ for some $u$ such that $Tu = 0$, then $Tv_1 = T(v_2 + u) = Tv_2 + Tu = Tv_2$.

    \item WTS $S$ is linear. Linearity of $S$ is clear from linearity of $T$.

    WTS it is bounded. For any $v$ such that $\norm{v + \ker(T)} = 1$, there exists some $u \in \ker(T)$ such that $\norm{v + u} < 1 + 1 = 2$. Hence
    \begin{equation*}
        \norm{Tv} = \norm{Tv + Tu} \leq \norm{T} \norm{v + u}  < 2 \norm{T}
    \end{equation*}

    It follows that $\norm{S} = \sup_{\norm{v + \ker(T)} = 1} \norm{Tv} \leq 2 \norm{T} < \infty$ so it is indeed a bounded map.

   \item WTS it is injective. Indeed, if $v_1 + \ker(T) \neq v_2 + \ker(T)$ then $S(v_1 + \ker(T)) - S(v_2 + \ker(T)) = T(v_1 - v_2) \neq 0$.
   \item WTS it is surjective. Take any $Tv \in \range(T)$. Then $Tv = S(v + \ker(T))$.
   \item From all above, it can be concluded that $S$ is a bijective, bounded linear operator from Banach $V/\ker(T)$ to Banach $\range(T)$. Applying Open Mapping Theorem, it can be concluded that $S$ is an isomorphism between $V/\ker(T)$ and $\range(T)$. Thus they are isomorphic as required.
\end{enumerate}
\end{solution}

\begin{problem} [Typed Problem 5 \done]
    The following exercise shows we cannot drop certain hypotheses in the closed graph theorem and open mapping theorem. Let \begin{equation*}
    W = \left\{a = \{a_k\}_k : \sum_{k} k \abs{a_k} < \infty\right\},
    \end{equation*}
    equipped with the $\ell^1$ norm.
    \begin{enumerate}
    \item Prove that $W$ is a proper, dense subspace of $\ell^1$ (hence, $W$ is not complete).

    Hint: Show that if $b = \{b_k\}_{k} \in \ell^1$ and $\epsilon > 0$, then there exists $N \in \bbn$ such that if \begin{equation*}
    a\coloneqq \{b_1, b_2, \ldots, b_N, 0, 0, \ldots\}   \in W.
    \end{equation*}
    then $\norm{a - b}_1 < \epsilon$.
    \item Define $T: W \to \ell^1$ by $(Ta)_k = ka_k$. Prove that the graph of $T$ is closed but $T$ is not bounded.
    \item Let $S = \inv{T} : \ell^1 \to W$. Prove that $S$ is bounded and surjective but is not an open mapping.
    \end{enumerate}
\end{problem}
\begin{solution}
    \textbf{(a)} Take $a = \{a_k\}_k$ where $a_k = \frac{1}{k^2}$ then $\norm{a}_1 = \frac{\pi^2}{6}$ so $a \in \ell^1$, but then $\sum_{k} k \norm{a_k} = \sum_{k} \frac{1}{k} = \infty$ so $a \not \in W$. So $W$ is a proper subset of $\ell^1$.

    To show that it is dense, pick any $b = \{b_k\}_k \in\ell^1$ and $\epsilon > 0$. WTS there exists some $a \in W$ such that $\norm{a - b}_1 < \epsilon$.

    Since $\norm{b}_1 < \infty$, there exists $N \in \bbn$ such that $\sum_{k = N}^{\infty} \abs{b_k} < \epsilon$. Hence if we define $a \coloneqq \{b_1, b_2, \ldots, b_N, 0, 0, \ldots\}$ then \begin{equation*}
    \sum_{k} k \abs{a_k} < \infty
    \end{equation*}
     trivially so $a \in W$, while \begin{equation*}
     \norm{a - b}_1 = \sum_{k=N}^{\infty} \abs{b_k} < \epsilon
    \end{equation*}
    
    It follows that $W$ is indeed dense in $\ell^1$.

    It also follows that $W$ is not complete; for if it was complete then since every point of $\ell^1$ is a limit point of $W$, it would also be in $W$, making $W$ not proper.

    \textbf{(b)} Take $(a^{(n)}, Ta^{(n)}) \cvgn (u, z) \in W \times \ell^1$, which means $a^{(n)} \cvgn u, Ta^{(n)} \cvgn z$, both with respect to $\ell^1$ norm.
    
    WTS $z = Tu$, i.e., $z_k = ku_k$.

    Fix $k \in \bbn$. Then for any $\epsilon > 0$, there exists some $N \in \bbn$ such that \begin{equation*}
    \epsilon/(2k) > \norm{a^{(N)} - u}, \quad \epsilon/2 > \norm{Ta^{(N)} - z}
    \end{equation*}
    It then follows that \begin{align*}
        \abs{z_k - ku_k} &\leq \abs{z_k - ka^{(N)}_k} + \abs{ka^{(N)}_k - ku_k} \\
        &\leq \norm{Ta^{(N)} - z} + k \norm{a^{(N)} - u} \\
        &< \epsilon/2 + k\epsilon/(2k) = \epsilon
    \end{align*}
    This is true for all $\epsilon$, so $z_k = ku_k$. It follows that $z = Tu$ and $\Gamma(T)$ is closed.

    However, $T$ is not bounded. Given any $M > 0$, construct $a = \{a_k\}_{k \in\bbn}$ where $a_k = 1$ if $k = \ceil{M}$ and $0$ otherwise. Then $\norm{a}_1 = 1$ but $\norm{Ta} = \ceil{M} \geq M$.

    (Therefore closed graph theorem doesn't work if one of the spaces is not Banach.)

    \textbf{(c)} We write out the explicit definition for $S$: 
    \begin{align*}
    S: \ell^1 &\to W \\
    b = \{b_k\} & \mapsto a = \left\{a_k = \frac{b_k}{k}\right\}
    \end{align*}

    \begin{enumerate} [1.]
    \item WTS it is well-defined. If $b \in \ell^1$ then \begin{equation*}
    \sum_{k} \abs{a_k} = \sum_{k} \abs{b_k/k} \leq \sum_k \abs{b_k} < \infty
    \end{equation*}
    so indeed $a \in W$.
    \item WTS it is bounded. If $\norm{b}_1 = 1$ then \begin{equation*}
    \norm{S(b)}_1 = \sum_{k} \abs*{\frac{b_k}{k}} \leq \sum_{k} \abs{b_k} = 1
    \end{equation*}
    so $\norm{S} \leq 1 < \infty$.

    \item WTS it is surjective. Take $a \in W$. Then define $b$ such that $b_k = ka_k$. Then $\sum_{k} \abs{b_k} = \sum_{k} \abs{ka_k} < \infty$ since $a \in W$, so $b \in \ell^1$, and clearly $S(b) = a$. 
    \item WTS it is not an open mapping. Consider $U = B_{\ell^1}(0, 1)$. WTS $S(U)$ not open by demonstrating that we can't draw a ball (wrt $\ell^1$) around $f(0) = 0$. Suppose, for sake of contradiction, that we can draw a ball $B_{\ell^1}(0, 2r) \cap W$ that is open in $W$. Then define $M = \ceil*{\frac{1}{r}}$, and define $a$ such that $a_k = r$ when $k = M$ and $0$ otherwise. Clearly, $a \in B_{\ell^1}(0, 2r) \cap W$.
    
    However if there exists some $b \in U$ such that $S(b) = a$ then that means $b_M = rM > 1 \implies \norm{b}_1 > 1 \implies b \not \in U, \contra$.

    It follows that one can't draw a ball around $S(0)$, thus $S(U)$ is not open.
\end{enumerate}
\end{solution}

\begin{problem} [Written Problem 1, Brezis 1.3 \done]
    Let $E = \{u \in C([0, 1], \bbr) : u(0) = 0\}$ with the norm 
    \begin{equation*}    
        \norm{u} = \max_{t \in [0, 1]} \abs{u(t)}. 
    \end{equation*}
    Consider the linear functional \begin{equation*}
    f: u \in E \mapsto f(u) = \int_{0}^{1} u(t) \dt
    \end{equation*}
    \begin{enumerate}
    \item Show that $f \in E^*$ and compute $\norm{f}_{E^*}$.
    \item Can one find some $u \in E$ such that $\norm{u} = 1$ and $f(u) = \norm{f}_{E^*}$?
    \end{enumerate}
\end{problem}
\begin{solution}
    \textbf{(a)} To show that $f \in E^*$, WTS it is linear and bounded.
    \begin{itemize}
        \item Linearity of $f$ is clear from linearity of integrals.
        \item It is also bounded:
        \begin{equation*}
            \norm{f} = \sup_{\norm{u} = 1} \abs{f(u)} = \abs*{\int_{0}^{1} u(t) \dt} \leq \norm{u} = 1 < \infty
        \end{equation*}
    \end{itemize}

    Therefore $f \in E^*$.

    We then compute $\norm{f}_{E^*} = \sup_{\norm{u} = 1} \abs{f(u)}$. We know that $\norm{f} \leq 1$. WTS $\norm{f} = 1$, i.e., for any $ 1 \gg \epsilon > 0$ there exists some $\norm{u} = 1$ such that $\abs{f(u)} = 1 - \epsilon$. Indeed, for fixed $\epsilon > 0$, construct $u$ that linearly interpolates $(0, 0) \to (2\epsilon, 1) \to (1, 1)$. Then $\abs{f(u)} = 1 - \epsilon$ as required, while $\norm{u} = 1$ clearly. It follows that $\norm{f}_{E^*} = 1$.

    \textbf{(b)} Take any $u \in E$ such that $\norm{u} = 1$. Fix $\epsilon = 1/2$. Since $u$ continuous at $0$, there exists $\delta > 0$ such that $t < \delta \implies \abs{u(t) - u(0)} < \frac{1}{2}$, that is, $\abs{u(t)} < \frac{1}{2}$. Let $\delta' = \min\{\delta, 1\}$, then
    \begin{align*}
    \int_{0}^{1} u(t) \dt &\leq \int_{0}^{\delta'} u(t) \dt + (1 - \delta') \norm{u} \\
    &\leq \frac{1}{2} \delta' + (1 - \delta') < 1 = \norm{f}_{E^*}
    \end{align*}

    So there doesn't exist any $u$ with such conditions.
\end{solution}

\begin{problem} [Written Problem 2, Brezis 1.4 \done]
    Consider the space $E = c_0$ with its usual norm $\ell^\infty$ norm. For every element $u = (u_1, u_2, u_3, \ldots)$ in $E$ define \begin{equation*}
    f(u) = \sum_{n=1}^{\infty} \frac{1}{2^n} u_n.
    \end{equation*}
    \begin{enumerate}
    \item Check that $f$ is a continuous linear functional on $E$ and compute $\norm{f}_{E^*}$.
    \item Can one find some $u \in E$ such that $\norm{u} = 1$ and $f(u) = \norm{f}_{E^*}$?
    \end{enumerate}
\end{problem}
\begin{solution}
    \textbf{(a)} Recall that \begin{equation*}
    c_0 = \left\{u = (u_i)_{i \in \bbn} : u_i \cvgi 0\right\}
    \end{equation*}
    endowed with $\norm{u}_{\infty} = \sup_{i \in \bbn} \abs{u_i}$.

    We first check that $f(u)$ is well-defined for each $u \in E$.
    
    Indeed, for all $\epsilon > 0$, since $u_i \cvgi 0$, there exists some $N \in \bbn$ such that $i \geq N \implies \abs{u_i} < \epsilon$. Then, for all $m, n \geq N$, we have 
    \begin{align*}
    \abs*{\sum_{i=n+1}^{m} \frac{1}{2^i}u_i} &\leq \sum_{i=n+1}^{m} \frac{1}{2^i}\epsilon\\
    \leq \frac{1}{2^N} \epsilon \leq \epsilon
    \end{align*}
    so the partial sums are Cauchy in complete $\bbc$, so the series $\sum_{i=1}^{\infty} \frac{1}{2^i} u_i$ indeed converges. $f(u)$ is thus well-defined.

    Now WTS $f$ is linear and bounded.
    \begin{itemize}
        \item It is linear:
        \begin{equation*}
            f(u + \lambda v) = \sum_{i=1}^{\infty} \frac{1}{2^i} (u_i + \lambda v_i) = f(u) + \lambda f(v)
        \end{equation*}
        \item It is bounded:
        \begin{align*}
        \norm{f} &= \sup_{\norm{u} = 1} \abs{f(u)} \\
        &= \sup_{\norm{u} = 1} \abs*{\sum_{i=1}^{\infty} \frac{1}{2^i} u_i} \\
        &\leq \sup_{\norm{u} = 1} \sum_{i=1}^{\infty} \frac{1}{2^i} \norm{u_i} \\
        &\leq \sup_{\norm{u} = 1} \norm{u} = 1  < \infty
        \end{align*}
    \end{itemize}
    
    So $f \in E^*$. We know that $\norm{f} \leq 1$, WTS for any $\epsilon > 0$, there exists some $u$ such that $f(u) \geq 1 - \epsilon$. Let $N \in \bbn$ large such that $\frac{1}{2^N} < \epsilon$. Then consider the sequence $u \in c_0$ defined by \begin{equation*}
    u_i = \begin{cases}
        1 & \:\text{for}\: i \leq N+1 \\
        0 & \:\text{for}\:  i > N+1
    \end{cases}
    \end{equation*}

    Then it follows that \begin{equation*}
    f(u) = \sum_{i=1}^{\infty} \frac{u_i}{2^i} = \sum_{i=1}^{N+1} \frac{1}{2^i} = 1 - \frac{1}{2^{N}} \geq 1 - \epsilon
    \end{equation*}
    while $u_i \cvgi 0$ clearly. Therefore $\norm{f}_{E^*} = 1$.

    \textbf{(b)} Take any $u \in c_0$ such that $\norm{u}_{\infty} = 1$. Then there exists $N$ such that $i \geq N \implies \abs{u_i} < 1/3$. It follows that \begin{align*}
        f(u) &= \sum_{i=1}^{\infty} \frac{u_i}{2^i} \\
        &= \sum_{i=1}^{N-1} \frac{u_i}{2^i} + \sum_{i=N}^{\infty} \frac{u_i}{2^i} \\
        &\leq 1 - 2^{-N} + \frac{1}{3} \frac{1}{2^{N-1}} < 1 =\norm{f}
    \end{align*}

    So there doesn't exist any $u$ with such conditions.
\end{solution}

\begin{problem} [Written Problem 3, Brezis 1.8 \done]
    Let $E$ be a normed vector space with norm $\norm{\cdot}$. Let $C \subset E$ be an open convex set such that $0 \in C$. Let $p$ denote the gauge of $C$.
    \begin{enumerate}
    \item Assume $C$ is symmetric (i.e., $-C = C$) and $C$ is bounded, prove that $p$ is a norm which is equivalent to $\norm{\cdot}$.
    \item Let $E = C([0, 1], \bbr)$ with its usual norm \begin{equation*}
    \norm{u} = \max_{t \in [0, 1]} \abs{u(t)}.
    \end{equation*}
    Let \begin{equation*}
    C = \left\{u \in E, \int_{0}^{1} \abs{u(t)}^2 \dt < 1\right\}
    \end{equation*}
    Check that $C$ is convex and symmetric and that $0 \in C$. Is $C$ bounded in $E$? Compute the gauge $p$ of $C$ and show that $p$ is a norm on $E$. Is $p$ equivalent to $\norm{\cdot}$?
    \end{enumerate}
\end{problem}
\begin{solution}
    \textbf{(a)} Recall that the gauge of $C$ is \begin{equation*}
    p(x) = \inf \{\alpha > 0 : \inv{\alpha} x  \in C\}
    \end{equation*}
    \textbf{1.} We first show that it is indeed a norm.
    \begin{itemize}
    \item $p$ is the infimum of positive numbers so it is nonnegative.
    \item When $\lambda \geq 0, p(\lambda x) = \inf \{\alpha > 0 : \inv{\alpha}(\lambda x) \in C\} = \lambda \inf \{\alpha > 0 : \inv{\alpha} (x) \in C\} = \lambda p(x) = \abs{\lambda} p(x)$.
    \item When $\lambda < 0, p(\lambda x) = \inf\{\alpha > 0 : \inv{\alpha} (\lambda x) \in C\} = \inf \{\alpha > 0 : \abs{\lambda}\inv{\alpha}(-x) \in C\}  = \abs{\lambda} \inf \{\alpha > 0 : \abs{\lambda}\inv{\alpha}(x) \in C\} = \abs{\lambda} p(x)$, since $C = -C$.
    \item Therefore $p(\lambda x) = \abs{\lambda} p(x) \forall \lambda \in \bbr$.
    \item We've proven triangle inequality for $p$ in Brezis.
    \item It remains to show that if $p(x) = 0$ then $x = 0$. Suppose we have that $p(a) = 0$ and $a \neq 0 \implies \norm{a} > 0$. Then since $C$ is bounded, there exists some $M$ such that $C \subset B(0, M)$. Since $0 = p(a) = \inf\{\alpha > 0 : \inv{\alpha} a \in C\}$, there exists some $\alpha < \frac{\norm{a}}{M}$ such that $\inv{\alpha}a \in C$. But then \begin{equation*}
    \norm{\inv{\alpha} a} > \norm*{\frac{M}{\norm{a}}a} = M \implies \inv{\alpha}a \not \in C
    \end{equation*}
    It follows that $a = 0$ necessarily.
    \end{itemize}
    
    So $p$ is a norm. 
    
    \textbf{2.} We now need to show that it is comparable to $\norm{\cdot}$. 
    
    Since $0 \in C$ and $C$ is open, there exists $r > 0$ such that $B(0, r) \subset C$. It then follows that $p(x) \leq \frac{1}{r} \norm{x}$.

    Since $C$ is bounded, there exists $M > 0$ such that $C \subset B(0, M)$. Then \begin{equation*}
    p(x) = \inf \{\alpha > 0 : \inv{\alpha}x \in C\} \geq \inf \{\alpha > 0 : \inv{\alpha} x \in B(0, M)\} = \frac{1}{M} \norm{x}   
    \end{equation*}

    Hence $p$ and $\norm{\cdot}$ are comparable.

    \textbf{(b)} \textbf{1.} WTS $C$ is convex, symmetric and $0 \in C$.

    \begin{itemize}
        \item $0 \in C$ trivially.
        \item Take $u, v \in C$ and some $\lambda \in [0, 1]$. Then
        \begin{align*}
        \int_{0}^{1} \abs{\lambda u(t) + (1-\lambda)v(t)}^2 \dt &= \lambda^2 \int_{0}^{1} \abs{u(t)}^2 \dt + (1-\lambda)^2 \int_{0}^{1} \abs{v(t)}^2 \dt + 2\lambda(1-\lambda)\int_{0}^{1} u(t) v(t) \dt \\
        &< \lambda^2 + (1 - \lambda)^2 + 2\lambda (1 - \lambda)\left(\int_{0}^{1} \abs{u(t)}^2 \dt\right)\left(\int_{0}^{1} \abs{v(t)}^2 \dt\right) \\
        &< \lambda^2 + (1 - \lambda)^2 + 2\lambda (1 - \lambda) = 1 \\
        \implies \lambda u + (1 - \lambda)v \in C
        \end{align*}

        So $C$ is convex.

        \item If $u \in C$ then $-u \in C$ since $\abs{u(t)} = \abs((-u)(t))$. So $C$ is symmetric.
    \end{itemize}

    \textbf{2.} WTS $C$ is NOT bounded in $E$.

    For any $M > 0$, construct $u_M(t) = \begin{cases}
    \sqrt{M - M^2t} & \:\text{if}\: t \in [0, \frac{1}{M}] \\
    0 & \:\text{if}\: t \in [\frac{1}{M}, 1]
    \end{cases}$,
    which is the square root of the function that linearly interpolates $(0, M) \to (\frac{1}{M}, 0) \to (1, 0)$. Then $\int_{0}^{1} \abs{u_M(t)}^2 \dt = \frac{1}{2} < 1 $ so $u_M \in C$. But $\norm{u_M} = M$ is unbounded. Hence $C$ is unbounded in $E$.

    \textbf{3.} We now compute the gauge $p$ of $C$. WTS it is also a norm on $E$, but this norm is not equivalent to $\norm{\cdot}$.

    Recall that \begin{equation*}
    p(u) = \inf \{\alpha > 0 : \inv{\alpha} u \in C\} = \inf \{\alpha > 0 : \inv{\alpha}\int_{0}^{1}\abs{u(t)}^2 \dt < 1\}
    \end{equation*}

    Then it's clear that $p(u) = \int_{0}^{1} \abs{u(t)}^2 \dt = \norm{u}_{L^2([0, 1])}$. It is the $L^2([0, 1])$ norm so it is a norm.

    This norm is not equivalent to $\norm{\cdot}$, since suppose not, that there exists some $M$ such that \begin{equation*}
    \norm{u} \leq M \norm{u}_{L^2([0, 1])} \forall u \in C([0, 1], \bbr)
    \end{equation*}
    Then one can construct function \begin{equation*}
    u_M \equiv \frac{1}{2M}
    \end{equation*}
    Then $\norm{u} = \frac{1}{2M}$ and $M\norm{u}_{L^2([0, 1])} = \frac{M}{4M^2} = \frac{1}{4M}$. So $\norm{u} > M \norm{u}_{L^2([0, 1])}, \contra$

    Hence they are not equivalent.
\end{solution}

\begin{problem} [Written Problem 4, Brezis 2.3 \done]
    Let $E, F$ be Banach spaces and $(T_n)_{n \in \bbn}$ a sequence in $\call(E, F)$. Assume that for every $x \in E$, $T_n x$ converges as $n \to \infty$ to a limit denoted by $Tx$. Show that if $x_n \cvgn x$ in $E$, then $T_n x_n \cvgn Tx$ in $F$.
\end{problem}
\begin{solution}
    From Banach-Steinhaus and its Corollary 2.3, we find that there exists some $C$ such that $\norm{T_n} \leq C \forall n$ and $T \in \call(E, F)$.

    By hypothesis, $x_n \cvgn x$ in $E$.

    Fix $\epsilon > 0$. Since $x_n \cvgn x$, there exists some $N_1$ such that $n \geq N_1 \implies \norm{x_n - x} < \epsilon/(2C)$.

    Furthermore, since $T_n x \cvgn Tx$, there exists some $N_2$ such that $n \geq N_2 \implies \norm{T_nx - Tx} < \epsilon/2$.

    It then follows that for $n \geq N_1 + N_2$, we have \begin{equation*}
    \norm{Tx - T_n x_n} \leq \norm{Tx - T_nx} + \norm{T_n x - T_n x_n} < \epsilon/2 + C \epsilon/(2C) = \epsilon
    \end{equation*}
    Hence $T_n x_n \cvgn Tx$.
\end{solution}

\begin{problem} [Written Problem 5, Brezis 2.17 \done]
    Let $E = C([0, 1])$ with its usual norm. Consider the operator $A: D(A) \subset E \to E$ defined by \begin{equation*}
    D(A) = C^1([0, 1]) \quad \:\text{and}\: \quad Au = u' = \frac{\du}{\dt}
    \end{equation*}
    \begin{enumerate}
    \item Check that $\cl{D(A)} = E$.
    \item Is $A$ closed?
    \item Consider the operator $B: D(B) \subset E \to E$ defined by
    \begin{equation*}
        D(B) = C^2([0, 1]) \quad \:\text{and}\: \quad  Bu = u' = \frac{\du}{\dt}
    \end{equation*}
    Is $B$ closed?
    \end{enumerate}
\end{problem}
\begin{solution}
    \textbf{(a)} Clearly, $D(A) \subset E$. It remains for us to show that every $u \in E$ is the limit of some sequence in $D(A) = C^1([0, 1])$.
    
    Fix $u \in E = C([0, 1])$. Then the standard mollification gives that $u$ is the (uniform) limit of $\{u * \eta_{1/n}\}_{n \in \bbn}$. Each $u * \eta_{1/n} \in C^\infty \subset C^1([0, 1])$, so indeed $u$ is a limit point of $C^1([0, 1])$. 

    Hence $\cl{D(A)} = E$ as required.

    \textbf{(b)} $A$ is closed iff it maps closed sets to closed sets.

    Consider $K = \{u_n(t) \in E\}_{n \in \bbn}$ where $u_n(t) = n + \frac{t}{n}$. Then clearly $K \subset D(A)$, with $u_n' \equiv \frac{1}{n}$.

    We show that $K$ is closed by showing that it is not Cauchy in any rearrangement, hence does not have a limit point. Indeed, take any $m \neq n$, WLOG, $m > n$, then \begin{equation*}
        \norm{u_m - u_n} = \sup_{t \in [0, 1]} \abs{(m-n) + t\left(\frac{1}{m} - \frac{1}{n}\right)} \geq \abs{(m-n)} \geq 1
    \end{equation*}

    So $K$ is closed.

    But it is clear that $u'_n \equiv \frac{1}{n} \cvgn u'_0 \equiv 0$ (uniformly), but $u'_0 \not \in A(K)$. So $A(K)$ is not closed.

    Hence $A$ is not a closed map.

    \textbf{(c)} The same example demonstrates the same point. So $B$ is not closed.

\end{solution}
\end{document}