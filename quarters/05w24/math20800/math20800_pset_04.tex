\documentclass[a4paper, 10pt]{article}
%%%%%%%%%%%%%%%%%%%%%%%%%%%%%%%%%%%%%%%%%%%%%%%%%%%%%%%%%%%%%%%%%%%%%%%%%%%%%%%
%                                Basic Packages                               %
%%%%%%%%%%%%%%%%%%%%%%%%%%%%%%%%%%%%%%%%%%%%%%%%%%%%%%%%%%%%%%%%%%%%%%%%%%%%%%%

% Gives us multiple colors.
\usepackage[usenames,dvipsnames,pdftex]{xcolor}
% Lets us style link colors.
\usepackage{hyperref}
% Lets us import images and graphics.
\usepackage{graphicx}
% Lets us use figures in floating environments.
\usepackage{float}
% Lets us create multiple columns.
\usepackage{multicol}
% Gives us better math syntax.
\usepackage{amsmath,amsfonts,mathtools,amsthm,amssymb}
% Lets us strikethrough text.
\usepackage{cancel}
% Lets us edit the caption of a figure.
\usepackage{caption}
% Lets us import pdf directly in our tex code.
\usepackage{pdfpages}
% Lets us do algorithm stuff.
\usepackage[ruled,vlined,linesnumbered]{algorithm2e}
% Use a smiley face for our qed symbol.
\usepackage{tikzsymbols}
% \usepackage{fullpage} %%smaller margins
\usepackage[shortlabels]{enumitem}

\setlist[enumerate]{font={\bfseries}} % global settings, for all lists

\usepackage{setspace}
\usepackage[margin=1in, headsep=12pt]{geometry}
\usepackage{wrapfig}
\usepackage{listings}
\usepackage{parskip}

\definecolor{codegreen}{rgb}{0,0.6,0}
\definecolor{codegray}{rgb}{0.5,0.5,0.5}
\definecolor{codepurple}{rgb}{0.58,0,0.82}
\definecolor{backcolour}{rgb}{0.95,0.95,0.95}

\lstdefinestyle{mystyle}{
    backgroundcolor=\color{backcolour},   
    commentstyle=\color{codegreen},
    keywordstyle=\color{magenta},
    numberstyle=\tiny\color{codegray},
    stringstyle=\color{codepurple},
    basicstyle=\ttfamily\footnotesize,
    breakatwhitespace=false,         
    breaklines=true,                 
    captionpos=b,                    
    keepspaces=true,                 
    numbers=left,                    
    numbersep=5pt,                  
    showspaces=false,                
    showstringspaces=false,
    showtabs=false,                  
    tabsize=2,
    numbers=none
}

\lstset{style=mystyle}
\def\class{article}


%%%%%%%%%%%%%%%%%%%%%%%%%%%%%%%%%%%%%%%%%%%%%%%%%%%%%%%%%%%%%%%%%%%%%%%%%%%%%%%
%                                Basic Settings                               %
%%%%%%%%%%%%%%%%%%%%%%%%%%%%%%%%%%%%%%%%%%%%%%%%%%%%%%%%%%%%%%%%%%%%%%%%%%%%%%%

%%%%%%%%%%%%%
%  Symbols  %
%%%%%%%%%%%%%

\let\implies\Rightarrow
\let\impliedby\Leftarrow
\let\iff\Leftrightarrow
\let\epsilon\varepsilon
%%%%%%%%%%%%
%  Tables  %
%%%%%%%%%%%%

\setlength{\tabcolsep}{5pt}
\renewcommand\arraystretch{1.5}

%%%%%%%%%%%%%%
%  SI Unitx  %
%%%%%%%%%%%%%%

\usepackage{siunitx}
\sisetup{locale = FR}

%%%%%%%%%%
%  TikZ  %
%%%%%%%%%%

\usepackage[framemethod=TikZ]{mdframed}
\usepackage{tikz}
\usepackage{tikz-cd}
\usepackage{tikzsymbols}

\usetikzlibrary{intersections, angles, quotes, calc, positioning}
\usetikzlibrary{arrows.meta}

\tikzset{
    force/.style={thick, {Circle[length=2pt]}-stealth, shorten <=-1pt}
}

%%%%%%%%%%%%%%%
%  PGF Plots  %
%%%%%%%%%%%%%%%

\usepackage{pgfplots}
\pgfplotsset{width=10cm, compat=newest}

%%%%%%%%%%%%%%%%%%%%%%%
%  Center Title Page  %
%%%%%%%%%%%%%%%%%%%%%%%

\usepackage{titling}
\renewcommand\maketitlehooka{\null\mbox{}\vfill}
\renewcommand\maketitlehookd{\vfill\null}

%%%%%%%%%%%%%%%%%%%%%%%%%%%%%%%%%%%%%%%%%%%%%%%%%%%%%%%
%  Create a grey background in the middle of the PDF  %
%%%%%%%%%%%%%%%%%%%%%%%%%%%%%%%%%%%%%%%%%%%%%%%%%%%%%%%

\usepackage{eso-pic}
\newcommand\definegraybackground{
    \definecolor{reallylightgray}{HTML}{FAFAFA}
    \AddToShipoutPicture{
        \ifthenelse{\isodd{\thepage}}{
            \AtPageLowerLeft{
                \put(\LenToUnit{\dimexpr\paperwidth-222pt},0){
                    \color{reallylightgray}\rule{222pt}{297mm}
                }
            }
        }
        {
            \AtPageLowerLeft{
                \color{reallylightgray}\rule{222pt}{297mm}
            }
        }
    }
}

%%%%%%%%%%%%%%%%%%%%%%%%
%  Modify Links Color  %
%%%%%%%%%%%%%%%%%%%%%%%%

\hypersetup{
    % Enable highlighting links.
    colorlinks,
    % Change the color of links to blue.
    urlcolor=blue,
    % Change the color of citations to black.
    citecolor={black},
    % Change the color of url's to blue with some black.
    linkcolor={blue!80!black}
}

%%%%%%%%%%%%%%%%%%
% Fix WrapFigure %
%%%%%%%%%%%%%%%%%%

\newcommand{\wrapfill}{\par\ifnum\value{WF@wrappedlines}>0
        \parskip=0pt
        \addtocounter{WF@wrappedlines}{-1}%
        \null\vspace{\arabic{WF@wrappedlines}\baselineskip}%
        \WFclear
    \fi}

%%%%%%%%%%%%%%%%%
% Multi Columns %
%%%%%%%%%%%%%%%%%

\let\multicolmulticols\multicols
\let\endmulticolmulticols\endmulticols

\RenewDocumentEnvironment{multicols}{mO{}}
{%
    \ifnum#1=1
        #2%
    \else % More than 1 column
        \multicolmulticols{#1}[#2]
    \fi
}
{%
    \ifnum#1=1
    \else % More than 1 column
        \endmulticolmulticols
    \fi
}

\newlength{\thickarrayrulewidth}
\setlength{\thickarrayrulewidth}{5\arrayrulewidth}


%%%%%%%%%%%%%%%%%%%%%%%%%%%%%%%%%%%%%%%%%%%%%%%%%%%%%%%%%%%%%%%%%%%%%%%%%%%%%%%
%                           School Specific Commands                          %
%%%%%%%%%%%%%%%%%%%%%%%%%%%%%%%%%%%%%%%%%%%%%%%%%%%%%%%%%%%%%%%%%%%%%%%%%%%%%%%

%%%%%%%%%%%%%%%%%%%%%%%%%%%
%  Initiate New Counters  %
%%%%%%%%%%%%%%%%%%%%%%%%%%%

\newcounter{lecturecounter}

%%%%%%%%%%%%%%%%%%%%%%%%%%
%  Helpful New Commands  %
%%%%%%%%%%%%%%%%%%%%%%%%%%

\makeatletter

\newcommand\resetcounters{
    % Reset the counters for subsection, subsubsection and the definition
    % all the custom environments.
    \setcounter{subsection}{0}
    \setcounter{subsubsection}{0}
    \setcounter{definition0}{0}
    \setcounter{paragraph}{0}
    \setcounter{theorem}{0}
    \setcounter{claim}{0}
    \setcounter{corollary}{0}
    \setcounter{proposition}{0}
    \setcounter{lemma}{0}
    \setcounter{exercise}{0}
    \setcounter{problem}{0}
    
    \setcounter{subparagraph}{0}
    % \@ifclasswith\class{nocolor}{
    %     \setcounter{definition}{0}
    % }{}
}

%%%%%%%%%%%%%%%%%%%%%
%  Lecture Command  %
%%%%%%%%%%%%%%%%%%%%%

\usepackage{xifthen}

% EXAMPLE:
% 1. \lecture{Oct 17 2022 Mon (08:46:48)}{Lecture Title}
% 2. \lecture[4]{Oct 17 2022 Mon (08:46:48)}{Lecture Title}
% 3. \lecture{Oct 17 2022 Mon (08:46:48)}{}
% 4. \lecture[4]{Oct 17 2022 Mon (08:46:48)}{}
% Parameters:
% 1. (Optional) lecture number.
% 2. Time and date of lecture.
% 3. Lecture Title.
\def\@lecture{}
\def\@lectitle{}
\def\@leccount{}
\newcommand\lecture[3]{
    \newpage

    % Check if user passed the lecture title or not.
    \def\@leccount{Lecture #1}
    \ifthenelse{\isempty{#3}}{
        \def\@lecture{Lecture #1}
        \def\@lectitle{Lecture #1}
    }{
        \def\@lecture{Lecture #1: #3}
        \def\@lectitle{#3}
    }

    \setcounter{section}{#1}
    \renewcommand\thesubsection{#1.\arabic{subsection}}
    
    \phantomsection
    \addcontentsline{toc}{section}{\@lecture}
    \resetcounters

    \begin{mdframed}
        \begin{center}
            \Large \textbf{\@leccount}
            
            \vspace*{0.2cm}
            
            \large \@lectitle
            
            
            \vspace*{0.2cm}

            \normalsize #2
        \end{center}
    \end{mdframed}

}

%%%%%%%%%%%%%%%%%%%%
%  Import Figures  %
%%%%%%%%%%%%%%%%%%%%

\usepackage{import}
\pdfminorversion=7

% EXAMPLE:
% 1. \incfig{limit-graph}
% 2. \incfig[0.4]{limit-graph}
% Parameters:
% 1. The figure name. It should be located in figures/NAME.tex_pdf.
% 2. (Optional) The width of the figure. Example: 0.5, 0.35.
\newcommand\incfig[2][1]{%
    \def\svgwidth{#1\columnwidth}
    \import{./figures/}{#2.pdf_tex}
}

\begingroup\expandafter\expandafter\expandafter\endgroup
\expandafter\ifx\csname pdfsuppresswarningpagegroup\endcsname\relax
\else
    \pdfsuppresswarningpagegroup=1\relax
\fi

%%%%%%%%%%%%%%%%%
% Fancy Headers %
%%%%%%%%%%%%%%%%%

\usepackage{fancyhdr}

% Force a new page.
\newcommand\forcenewpage{\clearpage\mbox{~}\clearpage\newpage}

% This command makes it easier to manage my headers and footers.
\newcommand\createintro{
    % Use roman page numbers (e.g. i, v, vi, x, ...)
    \pagenumbering{roman}

    % Display the page style.
    \maketitle
    % Make the title pagestyle empty, meaning no fancy headers and footers.
    \thispagestyle{empty}
    % Create a newpage.
    \newpage

    % Input the intro.tex page if it exists.
    \IfFileExists{intro.tex}{ % If the intro.tex file exists.
        % Input the intro.tex file.
        \textbf{Course}: MATH 16300: Honors Calculus III

\textbf{Section}: 43

\textbf{Professor}: Minjae Park

\textbf{At}: The University of Chicago

\textbf{Quarter}: Spring 2023

\textbf{Course materials}: Calculus by Spivak (4th Edition), Calculus On Manifolds by Spivak

\vspace{1cm}
\textbf{Disclaimer}: This document will inevitably contain some mistakes, both simple typos and serious logical and mathematical errors. Take what you read with a grain of salt as it is made by an undergraduate student going through the learning process himself. If you do find any error, I would really appreciate it if you can let me know by email at \href{mailto:conghungletran@gmail.com}{conghungletran@gmail.com}.

        % Make the pagestyle fancy for the intro.tex page.
        \pagestyle{fancy}

        % Remove the line for the header.
        \renewcommand\headrulewidth{0pt}

        % Remove all header stuff.
        \fancyhead{}

        % Add stuff for the footer in the center.
        % \fancyfoot[C]{
        %   \textit{For more notes like this, visit
        %   \href{\linktootherpages}{\shortlinkname}}. \\
        %   \vspace{0.1cm}
        %   \hrule
        %   \vspace{0.1cm}
        %   \@author, \\
        %   \term: \academicyear, \\
        %   Last Update: \@date, \\
        %   \faculty
        % }

        \newpage
    }{ % If the intro.tex file doesn't exist.
        % Force a \newpageage.
        % \forcenewpage
        \newpage
    }

    % Remove the center stuff we did above, and replace it with just the page
    % number, which is still in roman numerals.
    \fancyfoot[C]{\thepage}
    % Add the table of contents.
    \tableofcontents
    % Force a new page.
    \newpage

    % Move the page numberings back to arabic, from roman numerals.
    \pagenumbering{arabic}
    % Set the page number to 1.
    \setcounter{page}{1}

    % Add the header line back.
    \renewcommand\headrulewidth{0.4pt}
    % In the top right, add the lecture title.
    \fancyhead[R]{\footnotesize \@lecture}
    % In the top left, add the author name.
    \fancyhead[L]{\footnotesize \@author}
    % In the bottom center, add the page.
    \fancyfoot[C]{\thepage}
    % Add a nice gray background in the middle of all the upcoming pages.
    % \definegraybackground
}

\makeatother


%%%%%%%%%%%%%%%%%%%%%%%%%%%%%%%%%%%%%%%%%%%%%%%%%%%%%%%%%%%%%%%%%%%%%%%%%%%%%%%
%                               Custom Commands                               %
%%%%%%%%%%%%%%%%%%%%%%%%%%%%%%%%%%%%%%%%%%%%%%%%%%%%%%%%%%%%%%%%%%%%%%%%%%%%%%%

%%%%%%%%%%%%
%  Circle  %
%%%%%%%%%%%%

\newcommand*\circled[1]{\tikz[baseline= (char.base)]{
        \node[shape=circle,draw,inner sep=1pt] (char) {#1};}
}

%%%%%%%%%%%%%%%%%%%
%  Todo Commands  %
%%%%%%%%%%%%%%%%%%%

% \usepackage{xargs}
% \usepackage[colorinlistoftodos]{todonotes}

% \makeatletter

% \@ifclasswith\class{working}{
%     \newcommandx\unsure[2][1=]{\todo[linecolor=red,backgroundcolor=red!25,bordercolor=red,#1]{#2}}
%     \newcommandx\change[2][1=]{\todo[linecolor=blue,backgroundcolor=blue!25,bordercolor=blue,#1]{#2}}
%     \newcommandx\info[2][1=]{\todo[linecolor=OliveGreen,backgroundcolor=OliveGreen!25,bordercolor=OliveGreen,#1]{#2}}
%     \newcommandx\improvement[2][1=]{\todo[linecolor=Plum,backgroundcolor=Plum!25,bordercolor=Plum,#1]{#2}}

%     \newcommand\listnotes{
%         \newpage
%         \listoftodos[Notes]
%     }
% }{
%     \newcommandx\unsure[2][1=]{}
%     \newcommandx\change[2][1=]{}
%     \newcommandx\info[2][1=]{}
%     \newcommandx\improvement[2][1=]{}

%     \newcommand\listnotes{}
% }

% \makeatother

%%%%%%%%%%%%%
%  Correct  %
%%%%%%%%%%%%%

% EXAMPLE:
% 1. \correct{INCORRECT}{CORRECT}
% Parameters:
% 1. The incorrect statement.
% 2. The correct statement.
\definecolor{correct}{HTML}{009900}
\newcommand\correct[2]{{\color{red}{#1 }}\ensuremath{\to}{\color{correct}{ #2}}}


%%%%%%%%%%%%%%%%%%%%%%%%%%%%%%%%%%%%%%%%%%%%%%%%%%%%%%%%%%%%%%%%%%%%%%%%%%%%%%%
%                                 Environments                                %
%%%%%%%%%%%%%%%%%%%%%%%%%%%%%%%%%%%%%%%%%%%%%%%%%%%%%%%%%%%%%%%%%%%%%%%%%%%%%%%

\usepackage{varwidth}
\usepackage{thmtools}
\usepackage[most,many,breakable]{tcolorbox}

\tcbuselibrary{theorems,skins,hooks}
\usetikzlibrary{arrows,calc,shadows.blur}

%%%%%%%%%%%%%%%%%%%
%  Define Colors  %
%%%%%%%%%%%%%%%%%%%

% color prototype
% \definecolor{color}{RGB}{45, 111, 177}

% ESSENTIALS: 
\definecolor{myred}{HTML}{c74540}
\definecolor{myblue}{HTML}{072b85}
\definecolor{mygreen}{HTML}{388c46}
\definecolor{myblack}{HTML}{000000}

\colorlet{definition_color}{myred}

\colorlet{theorem_color}{myblue}
\colorlet{lemma_color}{myblue}
\colorlet{prop_color}{myblue}
\colorlet{corollary_color}{myblue}
\colorlet{claim_color}{myblue}

\colorlet{proof_color}{myblack}
\colorlet{example_color}{myblack}
\colorlet{exercise_color}{myblack}

% MISCS: 
%%%%%%%%%%%%%%%%%%%%%%%%%%%%%%%%%%%%%%%%%%%%%%%%%%%%%%%%%
%  Create Environments Styles Based on Given Parameter  %
%%%%%%%%%%%%%%%%%%%%%%%%%%%%%%%%%%%%%%%%%%%%%%%%%%%%%%%%%

% \mdfsetup{skipabove=1em,skipbelow=0em}

%%%%%%%%%%%%%%%%%%%%%%
%  Helpful Commands  %
%%%%%%%%%%%%%%%%%%%%%%

% EXAMPLE:
% 1. \createnewtheoremstyle{thmdefinitionbox}{}{}
% 2. \createnewtheoremstyle{thmtheorembox}{}{}
% 3. \createnewtheoremstyle{thmproofbox}{qed=\qedsymbol}{
%       rightline=false, topline=false, bottomline=false
%    }
% Parameters:
% 1. Theorem name.
% 2. Any extra parameters to pass directly to declaretheoremstyle.
% 3. Any extra parameters to pass directly to mdframed.
\newcommand\createnewtheoremstyle[3]{
    \declaretheoremstyle[
        headfont=\bfseries\sffamily, bodyfont=\normalfont, #2,
        mdframed={
                #3,
            },
    ]{#1}
}

% EXAMPLE:
% 1. \createnewcoloredtheoremstyle{thmdefinitionbox}{definition}{}{}
% 2. \createnewcoloredtheoremstyle{thmexamplebox}{example}{}{
%       rightline=true, leftline=true, topline=true, bottomline=true
%     }
% 3. \createnewcoloredtheoremstyle{thmproofbox}{proof}{qed=\qedsymbol}{backgroundcolor=white}
% Parameters:
% 1. Theorem name.
% 2. Color of theorem.
% 3. Any extra parameters to pass directly to declaretheoremstyle.
% 4. Any extra parameters to pass directly to mdframed.

% change backgroundcolor to #2!5 if user wants a colored backdrop to theorem environments. It's a cool color theme, but there's too much going on in the page.
\newcommand\createnewcoloredtheoremstyle[4]{
    \declaretheoremstyle[
        headfont=\bfseries\sffamily\color{#2},
        bodyfont=\normalfont,
        headpunct=,
        headformat = \NAME~\NUMBER\NOTE \hfill\smallskip\linebreak,
        #3,
        mdframed={
                outerlinewidth=0.75pt,
                rightline=false,
                leftline=false,
                topline=false,
                bottomline=false,
                backgroundcolor=white,
                skipabove = 5pt,
                skipbelow = 0pt,
                linecolor=#2,
                innertopmargin = 0pt,
                innerbottommargin = 0pt,
                innerrightmargin = 4pt,
                innerleftmargin= 6pt,
                leftmargin = -6pt,
                #4,
            },
    ]{#1}
}



%%%%%%%%%%%%%%%%%%%%%%%%%%%%%%%%%%%
%  Create the Environment Styles  %
%%%%%%%%%%%%%%%%%%%%%%%%%%%%%%%%%%%

\makeatletter
\@ifclasswith\class{nocolor}{
    % Environments without color.

    % ESSENTIALS:
    \createnewtheoremstyle{thmdefinitionbox}{}{}
    \createnewtheoremstyle{thmtheorembox}{}{}
    \createnewtheoremstyle{thmproofbox}{qed=\qedsymbol}{}
    \createnewtheoremstyle{thmcorollarybox}{}{}
    \createnewtheoremstyle{thmlemmabox}{}{}
    \createnewtheoremstyle{thmclaimbox}{}{}
    \createnewtheoremstyle{thmexamplebox}{}{}

    % MISCS: 
    \createnewtheoremstyle{thmpropbox}{}{}
    \createnewtheoremstyle{thmexercisebox}{}{}
    \createnewtheoremstyle{thmexplanationbox}{}{}
    \createnewtheoremstyle{thmremarkbox}{}{}
    
    % STYLIZED MORE BELOW
    \createnewtheoremstyle{thmquestionbox}{}{}
    \createnewtheoremstyle{thmsolutionbox}{qed=\qedsymbol}{}
}{
    % Environments with color.

    % ESSENTIALS: definition, theorem, proof, corollary, lemma, claim, example
    \createnewcoloredtheoremstyle{thmdefinitionbox}{definition_color}{}{leftline=false}
    \createnewcoloredtheoremstyle{thmtheorembox}{theorem_color}{}{leftline=false}
    \createnewcoloredtheoremstyle{thmproofbox}{proof_color}{qed=\qedsymbol}{}
    \createnewcoloredtheoremstyle{thmcorollarybox}{corollary_color}{}{leftline=false}
    \createnewcoloredtheoremstyle{thmlemmabox}{lemma_color}{}{leftline=false}
    \createnewcoloredtheoremstyle{thmpropbox}{prop_color}{}{leftline=false}
    \createnewcoloredtheoremstyle{thmclaimbox}{claim_color}{}{leftline=false}
    \createnewcoloredtheoremstyle{thmexamplebox}{example_color}{}{}
    \createnewcoloredtheoremstyle{thmexplanationbox}{example_color}{qed=\qedsymbol}{}
    \createnewcoloredtheoremstyle{thmremarkbox}{theorem_color}{}{}

    \createnewcoloredtheoremstyle{thmmiscbox}{black}{}{}

    \createnewcoloredtheoremstyle{thmexercisebox}{exercise_color}{}{}
    \createnewcoloredtheoremstyle{thmproblembox}{theorem_color}{}{leftline=false}
    \createnewcoloredtheoremstyle{thmsolutionbox}{mygreen}{qed=\qedsymbol}{}
}
\makeatother

%%%%%%%%%%%%%%%%%%%%%%%%%%%%%
%  Create the Environments  %
%%%%%%%%%%%%%%%%%%%%%%%%%%%%%
\declaretheorem[numberwithin=section, style=thmdefinitionbox,     name=Definition]{definition}
\declaretheorem[numberwithin=section, style=thmtheorembox,     name=Theorem]{theorem}
\declaretheorem[numbered=no,          style=thmexamplebox,     name=Example]{example}
\declaretheorem[numberwithin=section, style=thmtheorembox,       name=Claim]{claim}
\declaretheorem[numberwithin=section, style=thmcorollarybox,   name=Corollary]{corollary}
\declaretheorem[numberwithin=section, style=thmpropbox,        name=Proposition]{proposition}
\declaretheorem[numberwithin=section, style=thmlemmabox,       name=Lemma]{lemma}
\declaretheorem[numberwithin=section, style=thmexercisebox,    name=Exercise]{exercise}
\declaretheorem[numbered=no,          style=thmproofbox,       name=Proof]{proof0}
\declaretheorem[numbered=no,          style=thmexplanationbox, name=Explanation]{explanation}
\declaretheorem[numbered=no,          style=thmsolutionbox,    name=Solution]{solution}
\declaretheorem[numberwithin=section,          style=thmproblembox,     name=Problem]{problem}
\declaretheorem[numbered=no,          style=thmmiscbox,    name=Intuition]{intuition}
\declaretheorem[numbered=no,          style=thmmiscbox,    name=Goal]{goal}
\declaretheorem[numbered=no,          style=thmmiscbox,    name=Recall]{recall}
\declaretheorem[numbered=no,          style=thmmiscbox,    name=Motivation]{motivation}
\declaretheorem[numbered=no,          style=thmmiscbox,    name=Remark]{remark}
\declaretheorem[numbered=no,          style=thmmiscbox,    name=Observe]{observe}
\declaretheorem[numbered=no,          style=thmmiscbox,    name=Question]{question}


%%%%%%%%%%%%%%%%%%%%%%%%%%%%
%  Edit Proof Environment  %
%%%%%%%%%%%%%%%%%%%%%%%%%%%%

\renewenvironment{proof}[2][\proofname]{
    % \vspace{-12pt}
    \begin{proof0} [#2]
        }{\end{proof0}}

\theoremstyle{definition}

\newtheorem*{notation}{Notation}
\newtheorem*{previouslyseen}{As previously seen}
\newtheorem*{property}{Property}
% \newtheorem*{intuition}{Intuition}
% \newtheorem*{goal}{Goal}
% \newtheorem*{recall}{Recall}
% \newtheorem*{motivation}{Motivation}
% \newtheorem*{remark}{Remark}
% \newtheorem*{observe}{Observe}

\author{Hung C. Le Tran}


%%%% MATH SHORTHANDS %%%%
%% blackboard bold math capitals
\DeclareMathOperator*{\esssup}{ess\,sup}
\DeclareMathOperator*{\Hom}{Hom}
\newcommand{\bbf}{\mathbb{F}}
\newcommand{\bbn}{\mathbb{N}}
\newcommand{\bbq}{\mathbb{Q}}
\newcommand{\bbr}{\mathbb{R}}
\newcommand{\bbz}{\mathbb{Z}}
\newcommand{\bbc}{\mathbb{C}}
\newcommand{\bbk}{\mathbb{K}}
\newcommand{\bbm}{\mathbb{M}}
\newcommand{\bbp}{\mathbb{P}}
\newcommand{\bbe}{\mathbb{E}}

\newcommand{\bfw}{\mathbf{w}}
\newcommand{\bfx}{\mathbf{x}}
\newcommand{\bfX}{\mathbf{X}}
\newcommand{\bfy}{\mathbf{y}}
\newcommand{\bfyhat}{\mathbf{\hat{y}}}

\newcommand{\calb}{\mathcal{B}}
\newcommand{\calf}{\mathcal{F}}
\newcommand{\calt}{\mathcal{T}}
\newcommand{\call}{\mathcal{L}}
\renewcommand{\phi}{\varphi}

% Universal Math Shortcuts
\newcommand{\st}{\hspace*{2pt}\text{s.t.}\hspace*{2pt}}
\newcommand{\pffwd}{\hspace*{2pt}\fbox{\(\Rightarrow\)}\hspace*{10pt}}
\newcommand{\pfbwd}{\hspace*{2pt}\fbox{\(\Leftarrow\)}\hspace*{10pt}}
\newcommand{\contra}{\ensuremath{\Rightarrow\Leftarrow}}
\newcommand{\cvgn}{\xrightarrow{n \to \infty}}
\newcommand{\cvgj}{\xrightarrow{j \to \infty}}

\newcommand{\im}{\mathrm{im}}
\newcommand{\innerproduct}[2]{\langle #1, #2 \rangle}
\newcommand*{\conj}[1]{\overline{#1}}

% https://tex.stackexchange.com/questions/438612/space-between-exists-and-forall
% https://tex.stackexchange.com/questions/22798/nice-looking-empty-set
\let\oldforall\forall
\renewcommand{\forall}{\;\oldforall\; }
\let\oldexist\exists
\renewcommand{\exists}{\;\oldexist\; }
\newcommand\existu{\;\oldexist!\: }
\let\oldemptyset\emptyset
\let\emptyset\varnothing


\renewcommand{\_}[1]{\underline{#1}}
\DeclarePairedDelimiter{\abs}{\lvert}{\rvert}
\DeclarePairedDelimiter{\norm}{\lVert}{\rVert}
\DeclarePairedDelimiter\ceil{\lceil}{\rceil}
\DeclarePairedDelimiter\floor{\lfloor}{\rfloor}
\setlength\parindent{0pt}
\setlength{\headheight}{12.0pt}
\addtolength{\topmargin}{-12.0pt}


% Default skipping, change if you want more spacing
% \thinmuskip=3mu
% \medmuskip=4mu plus 2mu minus 4mu
% \thickmuskip=5mu plus 5mu

% \DeclareMathOperator{\ext}{ext}
% \DeclareMathOperator{\bridge}{bridge}

\title{MATH 20800: Honors Analysis in Rn II \\ \large Problem Set 4}
\date{14 Feb 2024}
\author{Hung Le Tran}
\begin{document}
\maketitle
\setcounter{section}{4}
\begin{problem} [\done]
    \begin{enumerate}
        \item[]
        \item (H\"{o}lder's inequality) Suppose that $n \in \bbn$, and let $a_k, b_k \in \bbn, 1 \leq k \leq n$. Prove that if $1 < p < \infty$ and $1/p + 1/q = 1$ then \begin{equation*}
            \sum_{k=1}^{n} \abs{a_k b_k} \leq \left[\sum_{k=1}^{n} \abs{a_k}^p\right]^{1/p} \left[\sum_{k=1}^{n} \abs{b_k}^q\right]^{1/q}
        \end{equation*}

        Hint: Prove that if $A, B > 0$ and $t \in (0, 1)$ then $A^t B^{1-t} \leq tA + (1-t) B$ by showing the function \begin{equation*}
        f(x) \coloneqq tx + (1-t)B - x^t B^{1-t}, \quad x > 0
        \end{equation*}
        has a minimum at $x = B$.

        \item (Minkowski's inequality) Suppose that $n \in \bbn$, and let $a_k, b_k \in \bbr, 1 \leq k \leq n.$ Prove that if $1 \leq p \leq \infty$ then \begin{equation*}
        \left[\sum_{k=1}^{n} \abs{a_k + b_k}^{p}\right]^{1/p} \leq \left[\sum_{k=1}^{n} \abs{a_k}^p\right]^{1/p} + \left[\sum_{k=1}^{n} \abs{b_k}^p\right]^{1/p}
        \end{equation*}
        
        Hint: By the triangle inequality \begin{equation*}
        \sum_{k=1}^{n} \abs{a_k + b_k}^p \leq \sum_{k=1}^{n} \abs{a_k} \abs{a_k + b_k}^{p-1} + \sum_{k=1}^{n} \abs{b_k} \abs{a_k + b_k}^{p-1}
        \end{equation*}

        Now apply H\"{o}lder's inequality.
    \end{enumerate}
\end{problem}
\begin{solution}
    \textbf{(a)} We first prove that for $A, B > 0$ and $t \in (0, 1)$, we have
    \begin{equation*}
        A^t B^{1-t} \leq tA + (1-t) B
        \end{equation*}
        Indeed, if we define $f(x) \coloneqq tx + (1-t) B - x^t B^{1-t}$ on $(0, \infty)$ then we find the critical point $x_0$:
        \begin{align*}
            f'(x_0) &=t - tx_0^{t-1}B^{1-t} \\
            \implies x_0 &= B
        \end{align*}
        and it is a minimum point since
        \begin{equation*}
        f''(x_0) = -t(t-1)B^{t-2}B^{1-t} > 0
        \end{equation*}

        Therefore for any $A > 0$, we have that $f(A) \geq f(B) = 0 \implies A^t B^{1-t} \leq tA + (1-t) B$.

        To prove Holder's inequality, take divide both sides by the \rhs, then we have:
        \begin{align*}
            \sum_{k=1}^{n} \left[\frac{\abs{a_k}^p}{\sum_{k=1}^{n} \abs{a_k}^p}\right]^{1/p}\left[\frac{\abs{b_k}^q}{\sum_{k=1}^{n} \abs{b_k}^q}\right]^{1/q} & \leq \sum_{k=1}^{n} \frac{1}{p} \left[\frac{\abs{a_k}^p}{\sum_{k=1}^{n} \abs{a_k}^p}\right] + \frac{1}{q} \left[\frac{\abs{b_k}^q}{\sum_{k=1}^{n} \abs{b_k}^q}\right] \\
            &= \frac{1}{p} \sum_{k=1}^{n} \left[\frac{\abs{a_k}^p}{\sum_{k=1}^{n} \abs{a_k}^p}\right] + \frac{1}{q} \sum_{k=1}^{n} \left[\frac{\abs{b_k}^q}{\sum_{k=1}^{n} \abs{b_k}^q}\right] \\
            &= 1
        \end{align*}
        hence $\lhs \leq \rhs$ as required.

        \textbf{(b)} Set $q = \frac{p}{p-1}$ then $1/p + 1/q = 1$.

        We have, first by triangle inequality then Holder's on $\frac{1}{p} + \frac{1}{q} = 1$ then
        \begin{align*}
            \sum_{k=1}^{n} \abs{a_k + b_k}^p &\leq \sum_{k=1}^{n} \abs{a_k} \abs{a_k + b_k}^{p-1} + \sum_{k=1}^{n} \abs{b_k} \abs{a_k + b_k}^{p-1} \\
            &\leq \left(\left[\sum_{k=1}^{n} \abs{a_k}^p\right]^{1/p} + \left[\sum_{k=1}^{n} \abs{b_k}^p\right]^{1/p}\right) \left[\sum_{k=1}^{n} \abs{a_k + b_k}^{(p-1)q}\right]^{1/q} \\
            &\leq \left(\left[\sum_{k=1}^{n} \abs{a_k}^p\right]^{1/p} + \left[\sum_{k=1}^{n} \abs{b_k}^p\right]^{1/p}\right) \left[\sum_{k=1}^{n} \abs{a_k + b_k}^p\right]^{1 - 1/p} \\
            \implies \left[\sum_{k=1}^{n} \abs{a_k + b_k}^p\right]^{1/p} &\leq \left[\sum_{k=1}^{n} \abs{a_k}^p\right]^{1/p} + \left[\sum_{k=1}^{n} \abs{b_k}^p\right]^{1/p}
        \end{align*}
        as required.
\end{solution}

\begin{problem} [\done]
    Prove that if $1 \leq p < \infty$, then $\ell^p$ is a Banach space (you must show it is a normed space and it is complete)
\end{problem}
\begin{solution}
    Let us first have the definition of the $\ell^p$ space: \begin{equation*}
    \ell^p = \left\{a = (a_1, a_2, \ldots) \mid a_k \in \bbc, \left(\sum_{k=1}^{\infty} \abs{a_k}^p\right)^{1/p} < \infty\right\}
    \end{equation*}

    Define the norm $\norm{a} = \norm{a}_p = \left(\sum_{k=1}^{\infty} \abs{a_k}^p\right)^{1/p}$.
    
    To show that $\ell^p$ is a normed space, we have to show that it is first a vector space (over $\bbc$), and the norm $\norm{\cdot}$ as above is indeed a norm. 
    \begin{itemize}
        \item On this space, define addition and scalar multiplication as pointwise addition and pointwise scalar multiplication. Then $(0) = (0, \cdots) \in \ell^p$ is trivially the identity.
        \item If $a, b \in \ell^p; \lambda \in \bbc$ then $a + \lambda b = (a_1 + \lambda b_1, \cdots)$ has:
        \begin{equation*}
            \left[\sum_{k=1}^{\infty}\abs{a_k + \lambda b_k}^p\right]^{1/p} \leq \left[\sum_{k=1}^{\infty} \abs{a_k}^p \right]^{1/p} + \left[\sum_{k=1}^{\infty} \abs{\lambda b_k}^p\right]^{1/p} < \infty
        \end{equation*}
        by Minkowski's (apply for $n$, then $n \to \infty$ implies LHS converges  and is thus well-defined) hence $a + \lambda b \in \ell^p$ too.
        \item $\abs{a_k} \geq 0 \forall k \implies \norm{a} \geq 0 \forall a$
        \item $\norm{a} = 0 \implies \sum_{k=1}^{\infty} \abs{a_k}^p = 0 \implies a_k = 0 \forall k \implies a = 0$
        \item Triangle inequality: Minkowski's tells us that \begin{equation*}
        \left[\sum_{k=1}^{n} \abs{a_k + b_k}^p\right]^{1/p} \leq \left[\sum_{k=1}^{n} \abs{a_k}^p\right]^{1/p}  + \left[\sum_{k=1}^{n} \abs{b_k}^p\right]^{1/p} \leq \left[\sum_{k=1}^{\infty} \abs{a_k}^p\right]^{1/p}  + \left[\sum_{k=1}^{\infty} \abs{b_k}^p\right]^{1/p} = \norm{a} + \norm{b} < \infty
        \end{equation*}
        The series on the LHS then is monotonic increasing in $n$ and bounded above and so converges, so $\norm{a + b}$. When taken to the limit, the inequality still holds, so $\norm{a + b} \leq \norm{a} + \norm{b}$. 
    \end{itemize}
    It remains to show that that $\ell^p$ is complete (with respect to norm $\norm{\cdot}$).

    Take a Cauchy sequence $\{a^{(i)}\}_{i \in \bbn} \subset \ell^p$. Fix $\epsilon > 0$, then there exists $N = N_\epsilon \in \bbn$ such that $i, j \geq N$ implies \begin{equation*}
    \norm{a^{(i)} - a^{(j)}} < \epsilon
    \end{equation*}

    Write out $a^{(i)} - a^{(j)} = (a^{(i)}_1 - a^{(j)}_1, a^{(i)}_2 - a^{(j)}_2, \ldots)$ then it follows that for all $i, j \geq N; k \in \bbn$, we have \begin{equation*}
        \abs{a^{(i)}_k - a^{(j)}_k} \leq \norm{a^{(i)} - a^{(j)}} < \epsilon
    \end{equation*}
    so for each $k \in \bbn$, the sequence $\{a^{(i)}_k\}_{i \in \bbn} \subset \bbc$ is Cauchy. $\bbc$ is complete, so $a^{(i)}_k \cvgg{i \to \infty} b_k \in \bbc$. Define $b = (b_k)_{k \in \bbn}$. Then WTS $b \in \ell^p$ and $a^{(i)} \cvgg{i \to \infty} b$.

    We first show that $\norm{a^{(i)} - b}_p \cvgg{i \to \infty} 0$. A priori, this ``norm'' might not exist, but by showing that it gets arbitrarily small, we in the process also show that it is well-defined.

    We know that $\{a^{(i)}\}_{i \in \bbn}$ in Cauchy wrt $\norm{}_p$, so for any $n \in \bbn$, we have that for all $i, j \geq N$,
    \begin{equation*}
    \sum_{k=1}^{n} \abs{a^{(i)}_k - a^{(j)}_k}^p < \epsilon^p
    \end{equation*}

    Let $j \to \infty$, then \begin{equation*}
    \sum_{k=1}^{n} \abs{a^{(i)}_k - b_k}^p \leq \epsilon^p
    \end{equation*}

    This holds for all $n \in \bbn$, so it follows that \begin{equation*}
    \sum_{k=1}^{\infty} \abs{a^{(i)}_k - b_k}^p \leq \epsilon^p
    \end{equation*}
    since the sequence of partial sums is increasing and bounded. It follows that for all $i \geq N$, \begin{equation*}
    \norm{a^{(i)} - b}_p \cvgg{i \to \infty} 0
    \end{equation*}

    It remains to show that $b \in \ell^p$. The triangle inequality then implies that \begin{equation*}
    \norm{b}_p \leq \norm{a^{(i)}}_p +  \norm{a^{(i)} - b}_p < \infty
    \end{equation*}
    for $i$ sufficiently large ($\geq N$), so $b \in \ell^p$.

    Hence $a^{(i)} \cvgg{i \to \infty} b \in \ell^p$, so $\ell^p$ is indeed a complete normed vector space, i.e., a Banach space.
\end{solution}

\begin{problem} [\done]
    The set of all bounded sequences, $\ell^{\infty}$, can be identified with $C_{\infty}(\bbn)$, the set of all bounded continuous functions on the metric space $(\bbn, d_{disc})$    where $d_{disc}$    is the discrete metric. Thus, $\ell^{\infty}$ is a Banach space. Prove that \begin{equation*}
    c_0 \coloneqq \{\{a_k\}_{k} \in \ell^{\infty} \mid \lim_{k \to \infty} a_k =0\} 
    \end{equation*}
    is a closed subspace of $\ell^{\infty}$ (and is thus, a Banach space).
\end{problem}
\begin{solution}
    Take sequence $\{a^{(i)} \}_{i \in \bbn} \subset c_0$ such that $a^{(i)} \cvgg{i \to \infty} b \in \ell^{\infty}$. WTS $b \in c_0$.

    To show that $b \in c_0$, we show that 
    \begin{equation*}
    \lim_{k \to \infty} b_k = 0
    \end{equation*}

    Fix $\epsilon > 0$. Since $a^{(i)} \cvgg{i \to \infty} b$, there exists $N = N_\epsilon \in \bbn$ such that $i \geq N$ implies \begin{equation*}
    \norm{a^{(i)} - b}_\infty < \epsilon/2
    \end{equation*}
    In particular, we have that \begin{equation*}
        \norm{a^{(N)} - b}_\infty < \epsilon/2
    \end{equation*}
    Since $a^{(N)} \in c_0$, there exists $K = K_{N, \epsilon} = K_\epsilon$ such that \begin{equation*}
    k \geq K \implies \abs{a^{(N)}_k} < \epsilon/2
    \end{equation*}

    It then follows that for $k \geq K$, we have \begin{equation*}
    \abs{b_k} \leq \abs{b_k - a^{(N)}_k} + \abs{a^{(N)}_k} \leq \norm{a^{(N)} - b}_\infty + \abs{a^{(N)}_k} < \epsilon
    \end{equation*}
    hence $\lim_{ k \to \infty} b_k = 0$ as required.

    It follows that $c_0$ is closed.
\end{solution}

\begin{problem} [\done]
    Let $1 \leq p \leq \infty$ and \begin{equation*}
    S \coloneqq \{a = \{a_k\}_{k} \in \ell^p \mid \norm{a}_p = 1    \}
    \end{equation*}

    \begin{enumerate}
    \item Prove that $S$ is a closed subset of $\ell^p$.
    \item Prove that $S$ is not compact. Hint: Let $e_n \coloneqq \{\delta_{kn}\}_k \in S$ where $\delta_{kn}$ is the Kronecker delta. Show that $\{e_n\}_n$ does not have a convergent subsequence in $S$.
    \end{enumerate}
\end{problem}
\begin{solution}
    \textbf{(a)} Note that the norm as a function from a normed vector space to $\bbr$  is always continuous, since it is 1-Lipschitz.

    In this case, $\norm{\cdot}_p: \ell^p \to \bbr$ is therefore continuous. It follows that \begin{equation*}
    S = \norm{\cdot}_p^{-1}(\{1\})
    \end{equation*}
    is closed in $\ell^p$ since $\{1\}$ is closed in $\bbr$.

    \textbf{(b)} To show that $S$ is not compact, we demonstrate a sequence in $S$ that has does not have a convergent subsequence in $S$.

    For each $n \in \bbn$, let $e_n = \{\delta_{kn}\}_{k \in \bbn}$, i.e., $e_n$ is the sequence of all zeros except for 1 at its $n$th index. Clearly, $e_n \in S \forall n \in \bbn$.

     Suppose that $\{e_n\}_{n \in \bbn}$ has a convergent subsequence $\{e_{n_j}\}_{j \in \bbn}$ that converges to some $a = \{a_k\}_{k \in \bbn} \in S$.
    
    For $p = \infty$, then the limit is the pointwise limit, so, $a_k = \lim_{j \to \infty} e_{n_j}[k] = 0$. But then $\norm{a} = 0 \neq 1, \contra$.

    We then now consider only $1 \leq p < \infty$. Then for $\epsilon = 0.1$, there exists some $N$ such that $j \geq J$ implies \begin{equation*}
     \norm{e_{n_j} - a}_p < \epsilon
    \end{equation*}
    Then for all $j \geq J$,
    \begin{align*}
        \epsilon^p > \lhs^p &= \sum_{k=1}^{\infty} \abs{a_k - \delta_{kn_j}}^p \\
        &= \norm{a}_p^p + (\abs{a_{n_j} - 1}^p - \abs{a_{n_j}}^p) \\
        &= 1 + \abs{a_{n_j} - 1}^p - \abs{a_{n_j}}^p \\
    \end{align*}

    It follows that \begin{equation*}
        \abs{a_{n_j}}^p -  \abs{a_{n_j} - 1}^p > 1 - \epsilon^p > 0 \implies \abs{a_{n_j}} > \abs{a_{n_j} - 1} \geq 1 - \abs{a_{n_j}}
    \end{equation*}
    therefore \begin{equation*}
    \abs{a_{n_j}} \geq 1/2
    \end{equation*}
    This is true for all $j \geq J$, so \begin{equation*}
    1 = \norm{a}_p^p \geq \sum_{j = J}^{J + \ceil{3^p}} \abs{a_{n_j}}^p \geq 3^p \frac{1}{2^p} > 1, \contra
    \end{equation*}

    Therefore, for both cases of $p = \infty$ and $1 \leq p < \infty$, there exists a sequence in $S$ that does not have a convergent subsequence. So $S$ is not compact.
\end{solution}

\begin{problem} [\done]
    Let $1 \leq p < \infty$ and $1/p + 1/q = 1$. \begin{enumerate}
    \item Prove that if $a = \{a_k\}_{k} \in \ell^p$ and $b = \{b_k\}_k \in \ell^q$ then \begin{equation*}
    \sum_{k=1}^{\infty} \abs{a_k b_k} \leq \norm{a}_p \norm{b}_q
    \end{equation*}
    \item Let $b \in \ell^q$. Prove that $F_b: \ell^p \to \bbc$ defined via
    \begin{equation*}
    F_b(a) \coloneqq \sum_{k=1}^{\infty} a_k b_k, \quad a\in \ell^p,
    \end{equation*}
    is an element of $(\ell^p)^*$, the dual space of $\ell^p$, and $\norm{F_b} = \norm{b}_{\ell^q}$.
    \item Prove that $F: \ell^q \to {(\ell^p)}^*, b \mapsto F_b$ is a bijective bounded linear operator.
    \end{enumerate}
\end{problem}
\begin{solution}
    \textbf{(a)} From Holder's, we know that \begin{equation*}
        \sum_{k=1}^{n} \abs{a_k b_k} \leq \left[\sum_{k=1}^{n} \abs{a_k}^p \right]^{1/p}\left[\sum_{k=1}^{n} \abs{b_k}^q \right]^{1/q} \leq \norm{a}_p \norm{b}_q
    \end{equation*}
    The partial sums are monotonically increasing and bounded above, so they converge and the limit is bounded by the same upper bound, hence\begin{equation*}
        \sum_{k=1}^{\infty} \abs{a_k b_k} \leq \norm{a}_p \norm{b}_q
        \end{equation*}
        as required. \qed

    \textbf{(b)} We have $F_b: \ell^p \to \bbc$ with definition
    \begin{equation*}
        F_b(a) \coloneqq \sum_{k=1}^{\infty} a_k b_k
        \end{equation*}
    To show that $F_b \in (\ell^p)^*$, we have to show that it is a bounded linear functional on $\ell^p$.

    To show linearity, take any $\alpha, \beta \in \ell^p, \lambda \in \bbc$ then
    \begin{align*}
        F_b(\alpha + \lambda \beta) &= \sum_{k=1}^{\infty} ((\alpha + \lambda \beta)_k b_k) \\
        &= \sum_{k=1}^{\infty} (\alpha_k + \lambda \beta_k) b_k \\
        &= F_b(\alpha) + \lambda F_b(\beta)
    \end{align*}
    so it is indeed linear. It is also bounded, since \begin{align*}
        \abs{F_b(a)} \leq \sum_{k=1}^{\infty} \abs{a_k b_k} \leq \norm{a}_p \norm{b}_q
    \end{align*}
    so $\norm{F_b} \leq \norm{b}_q < \infty$. It follows that $F_b \in (\ell^p)^*$.

    We've shown that $\norm{F_b} \leq \norm{b}_q$. To show equality, we exhibit a particular $a$ such that $\abs{F_b(a)} = \norm{a}_p \norm{b}_q$.

    The crux lies in that we construct $a$ such that the equality in Holder's inequality holds:
    \begin{equation*}
        \forall k \in \bbn, \frac{\abs{a_k}^p}{\norm{a}_p^p} = \frac{\abs{b_k}^q}{\norm{b}_q^q}
    \end{equation*}
    so that \begin{equation*}
        \sum_{k=1}^{\infty} \abs{a_k b_k} = \norm{a}_p \norm{b}_q
    \end{equation*}

    We also want $\abs{F_b(a)} = \abs{\sum_{k=1}^{\infty} a_k b_k} = \sum_{k=1}^{\infty} \abs{a_k b_k}$, so we choose $a_k = c_k \overline{b_k}$ where $\overline{b_k}$ is the complex conjugate of $b_k$, and $c_k$ real, nonnegative. It would then follow that 
\begin{equation*}
    \abs{F_b(a)} = \abs*{\sum_{k=1}^{\infty} c_k \abs{b_k}^2} = \sum_{k=1}^{\infty} c_k \abs{b_k}^2 = \sum_{k=1}^{\infty} \abs{a_k b_k}
\end{equation*}
so that $\abs{F_b(a)} = \norm{a}_p \norm{b}_q$, forcing $\norm{F_b} = \norm{b}_q$.

It remains for us to show a choice of $\{c_k\}$ so that $a \in \ell^p$ and satisfies the equal conditions of Holder's inequality so that all statements (especially those regarding convergence) are valid. Indeed, if \begin{equation*}
c_k \coloneqq \abs{b_k}^{(q-p)/p} \geq 0
\end{equation*}
then \begin{align*}
\norm{a}_p^p &= \sum_{k=1}^{\infty} \left(\abs{b_k}^{(q-p)} \abs{\overline{b_k}}^p\right) \\
&= \sum_{k=1}^{\infty} \abs{b_k}^q = \norm{b}_q^q
\end{align*}
so $a \in \ell^p$ and for all $k \in \bbn$:\begin{equation*}
    \frac{\abs{a_k}^p}{\norm{a}_p^p} = \frac{\abs{b_k}^{q-p} \abs{\overline{b_k}}^p}{\norm{b}_q^q} = \frac{\abs{b_k}^q}{\norm{b}_q^q}
\end{equation*}
as required, and we're done. \qed


    \textbf{(c)} Consider $F: \ell^q \to (\ell^p)^*, b \mapsto F(b) = F_b$.

    We proved from above that $F_b \in (\ell^p)^*$ for all $b$. So this is clearly a bijection.

    It is linear, since for all $\alpha, \beta \in \ell^q; \lambda \in \bbc$ and $a \in \ell^p$, we have
    \begin{align*}
    F (\alpha + \lambda \beta) (a) &= \sum_{k=1}^{\infty} a_k (\alpha + \lambda \beta)_k \\
    &= \sum_{k=1}^{\infty} a_k \alpha_k + \sum_{k=1}^{\infty} \lambda a_k \beta_k \\
    &= F(\alpha)(a) + \lambda F(\beta)(a)
    \end{align*}
    so $F(\alpha + \lambda \beta) = F(\alpha) + \lambda F(\beta)$ (expanding series of sum as sum of series makes sense, since we know a priori that each component series converges).

    It remains to show that $F$ is bounded.

    From (b), we saw that $\norm{F_b} = \norm{b}_q$, i.e., $\norm{F(b)} = \norm{b}$. Theefore, $\norm{F} \leq 1$, so it is a bounded linear operator as required.
\end{solution}
\end{document}