\documentclass[a4paper, 10pt]{article}
%%%%%%%%%%%%%%%%%%%%%%%%%%%%%%%%%%%%%%%%%%%%%%%%%%%%%%%%%%%%%%%%%%%%%%%%%%%%%%%
%                                Basic Packages                               %
%%%%%%%%%%%%%%%%%%%%%%%%%%%%%%%%%%%%%%%%%%%%%%%%%%%%%%%%%%%%%%%%%%%%%%%%%%%%%%%

% Gives us multiple colors.
\usepackage[usenames,dvipsnames,pdftex]{xcolor}
% Lets us style link colors.
\usepackage{hyperref}
% Lets us import images and graphics.
\usepackage{graphicx}
% Lets us use figures in floating environments.
\usepackage{float}
% Lets us create multiple columns.
\usepackage{multicol}
% Gives us better math syntax.
\usepackage{amsmath,amsfonts,mathtools,amsthm,amssymb}
% Lets us strikethrough text.
\usepackage{cancel}
% Lets us edit the caption of a figure.
\usepackage{caption}
% Lets us import pdf directly in our tex code.
\usepackage{pdfpages}
% Lets us do algorithm stuff.
\usepackage[ruled,vlined,linesnumbered]{algorithm2e}
% Use a smiley face for our qed symbol.
\usepackage{tikzsymbols}
% \usepackage{fullpage} %%smaller margins
\usepackage[shortlabels]{enumitem}

\setlist[enumerate]{font={\bfseries}} % global settings, for all lists

\usepackage{setspace}
\usepackage[margin=1in, headsep=12pt]{geometry}
\usepackage{wrapfig}
\usepackage{listings}
\usepackage{parskip}

\definecolor{codegreen}{rgb}{0,0.6,0}
\definecolor{codegray}{rgb}{0.5,0.5,0.5}
\definecolor{codepurple}{rgb}{0.58,0,0.82}
\definecolor{backcolour}{rgb}{0.95,0.95,0.95}

\lstdefinestyle{mystyle}{
    backgroundcolor=\color{backcolour},   
    commentstyle=\color{codegreen},
    keywordstyle=\color{magenta},
    numberstyle=\tiny\color{codegray},
    stringstyle=\color{codepurple},
    basicstyle=\ttfamily\footnotesize,
    breakatwhitespace=false,         
    breaklines=true,                 
    captionpos=b,                    
    keepspaces=true,                 
    numbers=left,                    
    numbersep=5pt,                  
    showspaces=false,                
    showstringspaces=false,
    showtabs=false,                  
    tabsize=2,
    numbers=none
}

\lstset{style=mystyle}
\def\class{article}


%%%%%%%%%%%%%%%%%%%%%%%%%%%%%%%%%%%%%%%%%%%%%%%%%%%%%%%%%%%%%%%%%%%%%%%%%%%%%%%
%                                Basic Settings                               %
%%%%%%%%%%%%%%%%%%%%%%%%%%%%%%%%%%%%%%%%%%%%%%%%%%%%%%%%%%%%%%%%%%%%%%%%%%%%%%%

%%%%%%%%%%%%%
%  Symbols  %
%%%%%%%%%%%%%

\let\implies\Rightarrow
\let\impliedby\Leftarrow
\let\iff\Leftrightarrow
\let\epsilon\varepsilon
%%%%%%%%%%%%
%  Tables  %
%%%%%%%%%%%%

\setlength{\tabcolsep}{5pt}
\renewcommand\arraystretch{1.5}

%%%%%%%%%%%%%%
%  SI Unitx  %
%%%%%%%%%%%%%%

\usepackage{siunitx}
\sisetup{locale = FR}

%%%%%%%%%%
%  TikZ  %
%%%%%%%%%%

\usepackage[framemethod=TikZ]{mdframed}
\usepackage{tikz}
\usepackage{tikz-cd}
\usepackage{tikzsymbols}

\usetikzlibrary{intersections, angles, quotes, calc, positioning}
\usetikzlibrary{arrows.meta}

\tikzset{
    force/.style={thick, {Circle[length=2pt]}-stealth, shorten <=-1pt}
}

%%%%%%%%%%%%%%%
%  PGF Plots  %
%%%%%%%%%%%%%%%

\usepackage{pgfplots}
\pgfplotsset{width=10cm, compat=newest}

%%%%%%%%%%%%%%%%%%%%%%%
%  Center Title Page  %
%%%%%%%%%%%%%%%%%%%%%%%

\usepackage{titling}
\renewcommand\maketitlehooka{\null\mbox{}\vfill}
\renewcommand\maketitlehookd{\vfill\null}

%%%%%%%%%%%%%%%%%%%%%%%%%%%%%%%%%%%%%%%%%%%%%%%%%%%%%%%
%  Create a grey background in the middle of the PDF  %
%%%%%%%%%%%%%%%%%%%%%%%%%%%%%%%%%%%%%%%%%%%%%%%%%%%%%%%

\usepackage{eso-pic}
\newcommand\definegraybackground{
    \definecolor{reallylightgray}{HTML}{FAFAFA}
    \AddToShipoutPicture{
        \ifthenelse{\isodd{\thepage}}{
            \AtPageLowerLeft{
                \put(\LenToUnit{\dimexpr\paperwidth-222pt},0){
                    \color{reallylightgray}\rule{222pt}{297mm}
                }
            }
        }
        {
            \AtPageLowerLeft{
                \color{reallylightgray}\rule{222pt}{297mm}
            }
        }
    }
}

%%%%%%%%%%%%%%%%%%%%%%%%
%  Modify Links Color  %
%%%%%%%%%%%%%%%%%%%%%%%%

\hypersetup{
    % Enable highlighting links.
    colorlinks,
    % Change the color of links to blue.
    urlcolor=blue,
    % Change the color of citations to black.
    citecolor={black},
    % Change the color of url's to blue with some black.
    linkcolor={blue!80!black}
}

%%%%%%%%%%%%%%%%%%
% Fix WrapFigure %
%%%%%%%%%%%%%%%%%%

\newcommand{\wrapfill}{\par\ifnum\value{WF@wrappedlines}>0
        \parskip=0pt
        \addtocounter{WF@wrappedlines}{-1}%
        \null\vspace{\arabic{WF@wrappedlines}\baselineskip}%
        \WFclear
    \fi}

%%%%%%%%%%%%%%%%%
% Multi Columns %
%%%%%%%%%%%%%%%%%

\let\multicolmulticols\multicols
\let\endmulticolmulticols\endmulticols

\RenewDocumentEnvironment{multicols}{mO{}}
{%
    \ifnum#1=1
        #2%
    \else % More than 1 column
        \multicolmulticols{#1}[#2]
    \fi
}
{%
    \ifnum#1=1
    \else % More than 1 column
        \endmulticolmulticols
    \fi
}

\newlength{\thickarrayrulewidth}
\setlength{\thickarrayrulewidth}{5\arrayrulewidth}


%%%%%%%%%%%%%%%%%%%%%%%%%%%%%%%%%%%%%%%%%%%%%%%%%%%%%%%%%%%%%%%%%%%%%%%%%%%%%%%
%                           School Specific Commands                          %
%%%%%%%%%%%%%%%%%%%%%%%%%%%%%%%%%%%%%%%%%%%%%%%%%%%%%%%%%%%%%%%%%%%%%%%%%%%%%%%

%%%%%%%%%%%%%%%%%%%%%%%%%%%
%  Initiate New Counters  %
%%%%%%%%%%%%%%%%%%%%%%%%%%%

\newcounter{lecturecounter}

%%%%%%%%%%%%%%%%%%%%%%%%%%
%  Helpful New Commands  %
%%%%%%%%%%%%%%%%%%%%%%%%%%

\makeatletter

\newcommand\resetcounters{
    % Reset the counters for subsection, subsubsection and the definition
    % all the custom environments.
    \setcounter{subsection}{0}
    \setcounter{subsubsection}{0}
    \setcounter{definition0}{0}
    \setcounter{paragraph}{0}
    \setcounter{theorem}{0}
    \setcounter{claim}{0}
    \setcounter{corollary}{0}
    \setcounter{proposition}{0}
    \setcounter{lemma}{0}
    \setcounter{exercise}{0}
    \setcounter{problem}{0}
    
    \setcounter{subparagraph}{0}
    % \@ifclasswith\class{nocolor}{
    %     \setcounter{definition}{0}
    % }{}
}

%%%%%%%%%%%%%%%%%%%%%
%  Lecture Command  %
%%%%%%%%%%%%%%%%%%%%%

\usepackage{xifthen}

% EXAMPLE:
% 1. \lecture{Oct 17 2022 Mon (08:46:48)}{Lecture Title}
% 2. \lecture[4]{Oct 17 2022 Mon (08:46:48)}{Lecture Title}
% 3. \lecture{Oct 17 2022 Mon (08:46:48)}{}
% 4. \lecture[4]{Oct 17 2022 Mon (08:46:48)}{}
% Parameters:
% 1. (Optional) lecture number.
% 2. Time and date of lecture.
% 3. Lecture Title.
\def\@lecture{}
\def\@lectitle{}
\def\@leccount{}
\newcommand\lecture[3]{
    \newpage

    % Check if user passed the lecture title or not.
    \def\@leccount{Lecture #1}
    \ifthenelse{\isempty{#3}}{
        \def\@lecture{Lecture #1}
        \def\@lectitle{Lecture #1}
    }{
        \def\@lecture{Lecture #1: #3}
        \def\@lectitle{#3}
    }

    \setcounter{section}{#1}
    \renewcommand\thesubsection{#1.\arabic{subsection}}
    
    \phantomsection
    \addcontentsline{toc}{section}{\@lecture}
    \resetcounters

    \begin{mdframed}
        \begin{center}
            \Large \textbf{\@leccount}
            
            \vspace*{0.2cm}
            
            \large \@lectitle
            
            
            \vspace*{0.2cm}

            \normalsize #2
        \end{center}
    \end{mdframed}

}

%%%%%%%%%%%%%%%%%%%%
%  Import Figures  %
%%%%%%%%%%%%%%%%%%%%

\usepackage{import}
\pdfminorversion=7

% EXAMPLE:
% 1. \incfig{limit-graph}
% 2. \incfig[0.4]{limit-graph}
% Parameters:
% 1. The figure name. It should be located in figures/NAME.tex_pdf.
% 2. (Optional) The width of the figure. Example: 0.5, 0.35.
\newcommand\incfig[2][1]{%
    \def\svgwidth{#1\columnwidth}
    \import{./figures/}{#2.pdf_tex}
}

\begingroup\expandafter\expandafter\expandafter\endgroup
\expandafter\ifx\csname pdfsuppresswarningpagegroup\endcsname\relax
\else
    \pdfsuppresswarningpagegroup=1\relax
\fi

%%%%%%%%%%%%%%%%%
% Fancy Headers %
%%%%%%%%%%%%%%%%%

\usepackage{fancyhdr}

% Force a new page.
\newcommand\forcenewpage{\clearpage\mbox{~}\clearpage\newpage}

% This command makes it easier to manage my headers and footers.
\newcommand\createintro{
    % Use roman page numbers (e.g. i, v, vi, x, ...)
    \pagenumbering{roman}

    % Display the page style.
    \maketitle
    % Make the title pagestyle empty, meaning no fancy headers and footers.
    \thispagestyle{empty}
    % Create a newpage.
    \newpage

    % Input the intro.tex page if it exists.
    \IfFileExists{intro.tex}{ % If the intro.tex file exists.
        % Input the intro.tex file.
        \textbf{Course}: MATH 16300: Honors Calculus III

\textbf{Section}: 43

\textbf{Professor}: Minjae Park

\textbf{At}: The University of Chicago

\textbf{Quarter}: Spring 2023

\textbf{Course materials}: Calculus by Spivak (4th Edition), Calculus On Manifolds by Spivak

\vspace{1cm}
\textbf{Disclaimer}: This document will inevitably contain some mistakes, both simple typos and serious logical and mathematical errors. Take what you read with a grain of salt as it is made by an undergraduate student going through the learning process himself. If you do find any error, I would really appreciate it if you can let me know by email at \href{mailto:conghungletran@gmail.com}{conghungletran@gmail.com}.

        % Make the pagestyle fancy for the intro.tex page.
        \pagestyle{fancy}

        % Remove the line for the header.
        \renewcommand\headrulewidth{0pt}

        % Remove all header stuff.
        \fancyhead{}

        % Add stuff for the footer in the center.
        % \fancyfoot[C]{
        %   \textit{For more notes like this, visit
        %   \href{\linktootherpages}{\shortlinkname}}. \\
        %   \vspace{0.1cm}
        %   \hrule
        %   \vspace{0.1cm}
        %   \@author, \\
        %   \term: \academicyear, \\
        %   Last Update: \@date, \\
        %   \faculty
        % }

        \newpage
    }{ % If the intro.tex file doesn't exist.
        % Force a \newpageage.
        % \forcenewpage
        \newpage
    }

    % Remove the center stuff we did above, and replace it with just the page
    % number, which is still in roman numerals.
    \fancyfoot[C]{\thepage}
    % Add the table of contents.
    \tableofcontents
    % Force a new page.
    \newpage

    % Move the page numberings back to arabic, from roman numerals.
    \pagenumbering{arabic}
    % Set the page number to 1.
    \setcounter{page}{1}

    % Add the header line back.
    \renewcommand\headrulewidth{0.4pt}
    % In the top right, add the lecture title.
    \fancyhead[R]{\footnotesize \@lecture}
    % In the top left, add the author name.
    \fancyhead[L]{\footnotesize \@author}
    % In the bottom center, add the page.
    \fancyfoot[C]{\thepage}
    % Add a nice gray background in the middle of all the upcoming pages.
    % \definegraybackground
}

\makeatother


%%%%%%%%%%%%%%%%%%%%%%%%%%%%%%%%%%%%%%%%%%%%%%%%%%%%%%%%%%%%%%%%%%%%%%%%%%%%%%%
%                               Custom Commands                               %
%%%%%%%%%%%%%%%%%%%%%%%%%%%%%%%%%%%%%%%%%%%%%%%%%%%%%%%%%%%%%%%%%%%%%%%%%%%%%%%

%%%%%%%%%%%%
%  Circle  %
%%%%%%%%%%%%

\newcommand*\circled[1]{\tikz[baseline= (char.base)]{
        \node[shape=circle,draw,inner sep=1pt] (char) {#1};}
}

%%%%%%%%%%%%%%%%%%%
%  Todo Commands  %
%%%%%%%%%%%%%%%%%%%

% \usepackage{xargs}
% \usepackage[colorinlistoftodos]{todonotes}

% \makeatletter

% \@ifclasswith\class{working}{
%     \newcommandx\unsure[2][1=]{\todo[linecolor=red,backgroundcolor=red!25,bordercolor=red,#1]{#2}}
%     \newcommandx\change[2][1=]{\todo[linecolor=blue,backgroundcolor=blue!25,bordercolor=blue,#1]{#2}}
%     \newcommandx\info[2][1=]{\todo[linecolor=OliveGreen,backgroundcolor=OliveGreen!25,bordercolor=OliveGreen,#1]{#2}}
%     \newcommandx\improvement[2][1=]{\todo[linecolor=Plum,backgroundcolor=Plum!25,bordercolor=Plum,#1]{#2}}

%     \newcommand\listnotes{
%         \newpage
%         \listoftodos[Notes]
%     }
% }{
%     \newcommandx\unsure[2][1=]{}
%     \newcommandx\change[2][1=]{}
%     \newcommandx\info[2][1=]{}
%     \newcommandx\improvement[2][1=]{}

%     \newcommand\listnotes{}
% }

% \makeatother

%%%%%%%%%%%%%
%  Correct  %
%%%%%%%%%%%%%

% EXAMPLE:
% 1. \correct{INCORRECT}{CORRECT}
% Parameters:
% 1. The incorrect statement.
% 2. The correct statement.
\definecolor{correct}{HTML}{009900}
\newcommand\correct[2]{{\color{red}{#1 }}\ensuremath{\to}{\color{correct}{ #2}}}


%%%%%%%%%%%%%%%%%%%%%%%%%%%%%%%%%%%%%%%%%%%%%%%%%%%%%%%%%%%%%%%%%%%%%%%%%%%%%%%
%                                 Environments                                %
%%%%%%%%%%%%%%%%%%%%%%%%%%%%%%%%%%%%%%%%%%%%%%%%%%%%%%%%%%%%%%%%%%%%%%%%%%%%%%%

\usepackage{varwidth}
\usepackage{thmtools}
\usepackage[most,many,breakable]{tcolorbox}

\tcbuselibrary{theorems,skins,hooks}
\usetikzlibrary{arrows,calc,shadows.blur}

%%%%%%%%%%%%%%%%%%%
%  Define Colors  %
%%%%%%%%%%%%%%%%%%%

% color prototype
% \definecolor{color}{RGB}{45, 111, 177}

% ESSENTIALS: 
\definecolor{myred}{HTML}{c74540}
\definecolor{myblue}{HTML}{072b85}
\definecolor{mygreen}{HTML}{388c46}
\definecolor{myblack}{HTML}{000000}

\colorlet{definition_color}{myred}

\colorlet{theorem_color}{myblue}
\colorlet{lemma_color}{myblue}
\colorlet{prop_color}{myblue}
\colorlet{corollary_color}{myblue}
\colorlet{claim_color}{myblue}

\colorlet{proof_color}{myblack}
\colorlet{example_color}{myblack}
\colorlet{exercise_color}{myblack}

% MISCS: 
%%%%%%%%%%%%%%%%%%%%%%%%%%%%%%%%%%%%%%%%%%%%%%%%%%%%%%%%%
%  Create Environments Styles Based on Given Parameter  %
%%%%%%%%%%%%%%%%%%%%%%%%%%%%%%%%%%%%%%%%%%%%%%%%%%%%%%%%%

% \mdfsetup{skipabove=1em,skipbelow=0em}

%%%%%%%%%%%%%%%%%%%%%%
%  Helpful Commands  %
%%%%%%%%%%%%%%%%%%%%%%

% EXAMPLE:
% 1. \createnewtheoremstyle{thmdefinitionbox}{}{}
% 2. \createnewtheoremstyle{thmtheorembox}{}{}
% 3. \createnewtheoremstyle{thmproofbox}{qed=\qedsymbol}{
%       rightline=false, topline=false, bottomline=false
%    }
% Parameters:
% 1. Theorem name.
% 2. Any extra parameters to pass directly to declaretheoremstyle.
% 3. Any extra parameters to pass directly to mdframed.
\newcommand\createnewtheoremstyle[3]{
    \declaretheoremstyle[
        headfont=\bfseries\sffamily, bodyfont=\normalfont, #2,
        mdframed={
                #3,
            },
    ]{#1}
}

% EXAMPLE:
% 1. \createnewcoloredtheoremstyle{thmdefinitionbox}{definition}{}{}
% 2. \createnewcoloredtheoremstyle{thmexamplebox}{example}{}{
%       rightline=true, leftline=true, topline=true, bottomline=true
%     }
% 3. \createnewcoloredtheoremstyle{thmproofbox}{proof}{qed=\qedsymbol}{backgroundcolor=white}
% Parameters:
% 1. Theorem name.
% 2. Color of theorem.
% 3. Any extra parameters to pass directly to declaretheoremstyle.
% 4. Any extra parameters to pass directly to mdframed.

% change backgroundcolor to #2!5 if user wants a colored backdrop to theorem environments. It's a cool color theme, but there's too much going on in the page.
\newcommand\createnewcoloredtheoremstyle[4]{
    \declaretheoremstyle[
        headfont=\bfseries\sffamily\color{#2},
        bodyfont=\normalfont,
        headpunct=,
        headformat = \NAME~\NUMBER\NOTE \hfill\smallskip\linebreak,
        #3,
        mdframed={
                outerlinewidth=0.75pt,
                rightline=false,
                leftline=false,
                topline=false,
                bottomline=false,
                backgroundcolor=white,
                skipabove = 5pt,
                skipbelow = 0pt,
                linecolor=#2,
                innertopmargin = 0pt,
                innerbottommargin = 0pt,
                innerrightmargin = 4pt,
                innerleftmargin= 6pt,
                leftmargin = -6pt,
                #4,
            },
    ]{#1}
}



%%%%%%%%%%%%%%%%%%%%%%%%%%%%%%%%%%%
%  Create the Environment Styles  %
%%%%%%%%%%%%%%%%%%%%%%%%%%%%%%%%%%%

\makeatletter
\@ifclasswith\class{nocolor}{
    % Environments without color.

    % ESSENTIALS:
    \createnewtheoremstyle{thmdefinitionbox}{}{}
    \createnewtheoremstyle{thmtheorembox}{}{}
    \createnewtheoremstyle{thmproofbox}{qed=\qedsymbol}{}
    \createnewtheoremstyle{thmcorollarybox}{}{}
    \createnewtheoremstyle{thmlemmabox}{}{}
    \createnewtheoremstyle{thmclaimbox}{}{}
    \createnewtheoremstyle{thmexamplebox}{}{}

    % MISCS: 
    \createnewtheoremstyle{thmpropbox}{}{}
    \createnewtheoremstyle{thmexercisebox}{}{}
    \createnewtheoremstyle{thmexplanationbox}{}{}
    \createnewtheoremstyle{thmremarkbox}{}{}
    
    % STYLIZED MORE BELOW
    \createnewtheoremstyle{thmquestionbox}{}{}
    \createnewtheoremstyle{thmsolutionbox}{qed=\qedsymbol}{}
}{
    % Environments with color.

    % ESSENTIALS: definition, theorem, proof, corollary, lemma, claim, example
    \createnewcoloredtheoremstyle{thmdefinitionbox}{definition_color}{}{leftline=false}
    \createnewcoloredtheoremstyle{thmtheorembox}{theorem_color}{}{leftline=false}
    \createnewcoloredtheoremstyle{thmproofbox}{proof_color}{qed=\qedsymbol}{}
    \createnewcoloredtheoremstyle{thmcorollarybox}{corollary_color}{}{leftline=false}
    \createnewcoloredtheoremstyle{thmlemmabox}{lemma_color}{}{leftline=false}
    \createnewcoloredtheoremstyle{thmpropbox}{prop_color}{}{leftline=false}
    \createnewcoloredtheoremstyle{thmclaimbox}{claim_color}{}{leftline=false}
    \createnewcoloredtheoremstyle{thmexamplebox}{example_color}{}{}
    \createnewcoloredtheoremstyle{thmexplanationbox}{example_color}{qed=\qedsymbol}{}
    \createnewcoloredtheoremstyle{thmremarkbox}{theorem_color}{}{}

    \createnewcoloredtheoremstyle{thmmiscbox}{black}{}{}

    \createnewcoloredtheoremstyle{thmexercisebox}{exercise_color}{}{}
    \createnewcoloredtheoremstyle{thmproblembox}{theorem_color}{}{leftline=false}
    \createnewcoloredtheoremstyle{thmsolutionbox}{mygreen}{qed=\qedsymbol}{}
}
\makeatother

%%%%%%%%%%%%%%%%%%%%%%%%%%%%%
%  Create the Environments  %
%%%%%%%%%%%%%%%%%%%%%%%%%%%%%
\declaretheorem[numberwithin=section, style=thmdefinitionbox,     name=Definition]{definition}
\declaretheorem[numberwithin=section, style=thmtheorembox,     name=Theorem]{theorem}
\declaretheorem[numbered=no,          style=thmexamplebox,     name=Example]{example}
\declaretheorem[numberwithin=section, style=thmtheorembox,       name=Claim]{claim}
\declaretheorem[numberwithin=section, style=thmcorollarybox,   name=Corollary]{corollary}
\declaretheorem[numberwithin=section, style=thmpropbox,        name=Proposition]{proposition}
\declaretheorem[numberwithin=section, style=thmlemmabox,       name=Lemma]{lemma}
\declaretheorem[numberwithin=section, style=thmexercisebox,    name=Exercise]{exercise}
\declaretheorem[numbered=no,          style=thmproofbox,       name=Proof]{proof0}
\declaretheorem[numbered=no,          style=thmexplanationbox, name=Explanation]{explanation}
\declaretheorem[numbered=no,          style=thmsolutionbox,    name=Solution]{solution}
\declaretheorem[numberwithin=section,          style=thmproblembox,     name=Problem]{problem}
\declaretheorem[numbered=no,          style=thmmiscbox,    name=Intuition]{intuition}
\declaretheorem[numbered=no,          style=thmmiscbox,    name=Goal]{goal}
\declaretheorem[numbered=no,          style=thmmiscbox,    name=Recall]{recall}
\declaretheorem[numbered=no,          style=thmmiscbox,    name=Motivation]{motivation}
\declaretheorem[numbered=no,          style=thmmiscbox,    name=Remark]{remark}
\declaretheorem[numbered=no,          style=thmmiscbox,    name=Observe]{observe}
\declaretheorem[numbered=no,          style=thmmiscbox,    name=Question]{question}


%%%%%%%%%%%%%%%%%%%%%%%%%%%%
%  Edit Proof Environment  %
%%%%%%%%%%%%%%%%%%%%%%%%%%%%

\renewenvironment{proof}[2][\proofname]{
    % \vspace{-12pt}
    \begin{proof0} [#2]
        }{\end{proof0}}

\theoremstyle{definition}

\newtheorem*{notation}{Notation}
\newtheorem*{previouslyseen}{As previously seen}
\newtheorem*{property}{Property}
% \newtheorem*{intuition}{Intuition}
% \newtheorem*{goal}{Goal}
% \newtheorem*{recall}{Recall}
% \newtheorem*{motivation}{Motivation}
% \newtheorem*{remark}{Remark}
% \newtheorem*{observe}{Observe}

\author{Hung C. Le Tran}


%%%% MATH SHORTHANDS %%%%
%% blackboard bold math capitals
\DeclareMathOperator*{\esssup}{ess\,sup}
\DeclareMathOperator*{\Hom}{Hom}
\newcommand{\bbf}{\mathbb{F}}
\newcommand{\bbn}{\mathbb{N}}
\newcommand{\bbq}{\mathbb{Q}}
\newcommand{\bbr}{\mathbb{R}}
\newcommand{\bbz}{\mathbb{Z}}
\newcommand{\bbc}{\mathbb{C}}
\newcommand{\bbk}{\mathbb{K}}
\newcommand{\bbm}{\mathbb{M}}
\newcommand{\bbp}{\mathbb{P}}
\newcommand{\bbe}{\mathbb{E}}

\newcommand{\bfw}{\mathbf{w}}
\newcommand{\bfx}{\mathbf{x}}
\newcommand{\bfX}{\mathbf{X}}
\newcommand{\bfy}{\mathbf{y}}
\newcommand{\bfyhat}{\mathbf{\hat{y}}}

\newcommand{\calb}{\mathcal{B}}
\newcommand{\calf}{\mathcal{F}}
\newcommand{\calt}{\mathcal{T}}
\newcommand{\call}{\mathcal{L}}
\renewcommand{\phi}{\varphi}

% Universal Math Shortcuts
\newcommand{\st}{\hspace*{2pt}\text{s.t.}\hspace*{2pt}}
\newcommand{\pffwd}{\hspace*{2pt}\fbox{\(\Rightarrow\)}\hspace*{10pt}}
\newcommand{\pfbwd}{\hspace*{2pt}\fbox{\(\Leftarrow\)}\hspace*{10pt}}
\newcommand{\contra}{\ensuremath{\Rightarrow\Leftarrow}}
\newcommand{\cvgn}{\xrightarrow{n \to \infty}}
\newcommand{\cvgj}{\xrightarrow{j \to \infty}}

\newcommand{\im}{\mathrm{im}}
\newcommand{\innerproduct}[2]{\langle #1, #2 \rangle}
\newcommand*{\conj}[1]{\overline{#1}}

% https://tex.stackexchange.com/questions/438612/space-between-exists-and-forall
% https://tex.stackexchange.com/questions/22798/nice-looking-empty-set
\let\oldforall\forall
\renewcommand{\forall}{\;\oldforall\; }
\let\oldexist\exists
\renewcommand{\exists}{\;\oldexist\; }
\newcommand\existu{\;\oldexist!\: }
\let\oldemptyset\emptyset
\let\emptyset\varnothing


\renewcommand{\_}[1]{\underline{#1}}
\DeclarePairedDelimiter{\abs}{\lvert}{\rvert}
\DeclarePairedDelimiter{\norm}{\lVert}{\rVert}
\DeclarePairedDelimiter\ceil{\lceil}{\rceil}
\DeclarePairedDelimiter\floor{\lfloor}{\rfloor}
\setlength\parindent{0pt}
\setlength{\headheight}{12.0pt}
\addtolength{\topmargin}{-12.0pt}


% Default skipping, change if you want more spacing
% \thinmuskip=3mu
% \medmuskip=4mu plus 2mu minus 4mu
% \thickmuskip=5mu plus 5mu

% \DeclareMathOperator{\ext}{ext}
% \DeclareMathOperator{\bridge}{bridge}
\title{MATH 26200: Point-Set Topology \\ \large Problem Set 6}
\date{22 Feb 2024}
\author{Hung Le Tran}
\begin{document}
\maketitle
\setcounter{section}{6}
\textbf{Textbook:} Munkres, \textit{Topology}
\begin{problem} [28.1 \done]
    Give $[0, 1]^{\omega}$ the uniform topology. Find an infinite subset of this space that has no limit point.
\end{problem}
\begin{solution}
    Let \begin{equation*}
    x^{(n)} \coloneqq (0, \ldots, 0, 1, 0, \ldots) \in [0, 1]^\omega
    \end{equation*}
    where $1$ is at the $n$-th index.

    Consider set $A = \{x^{(n)} : n \in \bbn\}$. Since $[0, 1]^\omega$ is endowed with the uniform metric, if $A$ has some limit point $x \in [0, 1]^\omega$ then there must necessarily be a sequence $\{x^{(n_j)}\}_{j \in \bbn}$ such that $x_{n_j} \cvgj x$ (note that $n_j$ here is not required to be increasing in $j$).

    This implies for all $\epsilon > 0$, there exists $J \in \bbn$ such that for all $j \geq J$, \begin{equation*}
    \bar{\rho}(x^{(n_j)}, x) < \epsilon
    \end{equation*}
    Pick $\epsilon = 1/2$. Then for $J$, we have that \begin{equation*}
        \epsilon > \bar{\rho}(x^{(n_J)}, x) \geq \bar{d}(1, x_{n_J}) = \abs{x_{n_J} - 1}
    \end{equation*}
    hence $x_{n_J}> 1 - \epsilon$. It then follows that
    \begin{equation*}
    \bar{\rho}(x^{(n_{J+1})}, x) \geq \bar{d}(x^{(n_{J+1})}_{n_{J}}, x_{n_J}) = 1 - \epsilon = \epsilon, \contra
    \end{equation*}

    It follows that $A$ has no limit point.
\end{solution}

\begin{problem} [43.4 \done]
    Show that the metric space $(X, d)$ is complete if and only if for every nested sequence $A_1 \supset A_2 \supset \cdots$ of nonempty closed set of $X$ such that $\diam A_n \to 0$, the intersection of the sets $A_n$ is nonempty.
\end{problem}
\begin{solution}
\pffwd By hypothesis, $(X, d)$ is complete.

Take sequence $A_1 \supset A_2 \supset \cdots$ of nonempty closed set of $X$ such that $\diam A_n \to 0$, and let $A = \bigcap_{n \in \bbn} A_n$. WTS $A \neq \emptyset$.

For every $n \in \bbn$, since $A_n \neq \emptyset$, take $x_n \in A_n$. Consider the sequence $(x_n)_{n \in \bbn}$. Since $A_n \subset A_{n-1} \subset \cdots \subset A_1$, it follows that for all $n$, $x_n \in A_m$ for all $m \leq n$. It then follows that $d(x_n, x_m) \leq \diam A_m$. Take $m \leq n$ to $\infty$, since $\diam A_m \cvgg{m \to \infty} 0$, it follows that $(x_n)_{n \in \bbn}$ is Cauchy.

Since $X$ is complete, it follows that $x_n \cvgn x \in X$. WTS $x \in \bigcap_{n \in \bbn} A_n$.

Indeed, for each $A_m$, since it is closed in complete $X$, it is also complete. Then consider the cutoff-ed sequence $(y_n = x_{n + m})_{n \in \bbn}$. It is a sequence in $A_m$, and is Cauchy as we noted. It follows that $y_n \cvgn x' \in A_m$. But then $(y_n)$ is just a subsequence of the original $(x_n)$, hence $x' = x$. It then follows that $x \in A_m \forall m \in \bbn \implies x \in \bigcap_{m \in \bbn} A_m \implies A \neq \emptyset$ as required. \qed

\pfbwd Take Cauchy sequence $(x_n)_{n \in \bbn}$ in $X$. WTS $x_n \cvgn x \in X$.

Let $A_n = \{x_k : k \geq n \} = \{x_n, x_{n+1}, \ldots\}$. Then since $(x_n)$ is Cauchy, it follows that $\diam A_n \cvgn 0$. Consider $\diam \cl{A_n}$: then every point in the closure is well approximated by points in $A_n$, therefore \begin{equation*}
\diam \cl{A_n} \cvgn 0
\end{equation*}

By construction, we have that $A_1 \supset A_2 \supset \ldots \implies \cl{A_1} \supset \cl{A_2} \supset \ldots$, with $\diam \cl{A_n} \cvgn 0$. By hypothesis, therefore, $\bigcap_{n \in \bbn} \cl{A_n} \neq \emptyset$. Take $x \in \bigcap_{n \in \bbn} \cl{A_n} \implies \forall n \in \bbn, x \in \cl{A_n}$. But $x_n \in \cl{A_n}$ and $\diam \cl{A_n} \cvgn 0 \implies x_n \cvgn x$. $x \in X$, so $X$ is complete.
\end{solution}

\begin{problem} [43.8 \done]
    If $X$ and $Y$ are spaces, define \begin{equation*}
    e: X \times \calc(X, Y) \to Y
    \end{equation*}
    by the equation $e(x, f) = f(x)$; the map $e$ is called the \textbf{evaluation map}. Show that if $d$ is a metric for $Y$ and $\calc(X, Y)$ has the corresponding uniform topology, then $e$ is continuous.
\end{problem}
\begin{solution}
    Recall that the uniform topology on $\calc(X, Y)$ is the topology induced by the metric $\bar{\rho}$, where for $f, g \in \calc(X, Y)$,
    \begin{equation*}
    \bar{\rho}(f, g) = \sup \{\bar{d}(f( \alpha), g(\alpha)) : \alpha \in X\}
    \end{equation*}

    Take typical basis element $B(y, \epsilon)$ of $Y$. WTS $e^{-1}(B(y, \epsilon))$ is open in $X \times \calc(X, Y)$. Take any $x \times f \in e^{-1}(B(y, \epsilon))$. WTS there exists an open neighborhood of $x \times f$ that is contained in $e^{-1}(B(y, \epsilon))$. Indeed, consider \begin{equation*}
    U \coloneqq f^{-1}(fx, \epsilon/2) \times B(f, \epsilon / 2)
    \end{equation*}
    It is clear that $x \times f \in U$. Then, take any $x', f' \in U$. Then \begin{equation*}
    d(f'x', y) \leq d(f'x', fx') + d(fx', fx) < \epsilon/2 + \epsilon/2 = \epsilon
    \end{equation*}
    so it follows that $x \times f \in U \subset e^{-1}(B(y, \epsilon))$ as required.
\end{solution}

\begin{problem} [45.7 \done]
    Let $(X, d)$ be a metric space. If $A \subset X$ and $\epsilon > 0$, let $U(A, \epsilon)$ be the $\epsilon$-neighborhood of $A$. Let $\calh$ be the collection of all (nonempty) closed, bounded, subsets of $X$. If $A, B \in \calh$, define \begin{equation*}
    D(A, B) = \inf \{\epsilon \mid A \subset U(B, \epsilon) \:\text{and}\: B \subset U(A, \epsilon)\}
    \end{equation*}
    \begin{enumerate}
    \item Show that $D$ is a metric on $\calh$, it is called the \textbf{Hausdorff metric}.
    \item Show that if $(X, d)$ is complete, so is $(\calh, D)$. [Hint: Let $A_n$ be a Cauchy sequence in $\calh$, by passing to a subsequence, assume $D(A_n, A_{n+1}) < \frac{1}{2^n}$. Define $A$ to be the set of all points $x$ that are the limits of sequences $x_1, x_2, \ldots$ such that $x_i \in A_i$ for each $i$, and $d(x_i, x_{i+1}) < \frac{1}{2^i}$. Show $A_n \to \cl{A}$.]   
    \item Show that if $(X, d)$ is totally bounded, so is $(\calh, D)$. [Hint: Given $\epsilon$, choose $\delta < \epsilon$ and let $S$ be a finite subset of $X$ such that the collection $\{B_d(x, \delta) \mid x \in S\}$ covers $X$. Let $\cala$ be the collection of all nonempty subsets of $S$; show that $\{B_D(A, \epsilon) \mid A \in \cala\}$ covers $\calh$.]
    \item Theorem: If $X$ is compact in the metric $d$, then the space $\calh$ is compact in the Hausdorff metric $D$.
    \end{enumerate}
\end{problem}
\begin{solution}
    \textbf{(a)} WTS $D$ is a metric.

    \begin{itemize}
    \item $D(A, B) \geq 0$ since it is infimum of set of positive numbers.
    \item If $D(A, B) = 0$ then for all $\epsilon> 0$, $A \subset U(B, \epsilon), B \subset U(A, \epsilon)$. $A \subset U(B, \epsilon)$ means that for every point $a \in A, a \in U(B, \epsilon)$, i.e., $B(a, \epsilon) \cap B \neq \emptyset$. So every point $a \in A$ is in the closure of $B$. $B$ is closed so $\cl{B} = B$. Therefore $A \subset B$. Similarly, $B \subset A$. Hence $A = B$ as required.
    \item By definition, $D(A, B) = D(B, A)$ trivially.
    \item WTS $D(A, B) + D(B, C) \geq D(A, C)$. Take $\epsilon_{AB}, \epsilon_{BC}$ such that $A \subset U(B, \epsilon_{AB}), B \subset U(A, \epsilon{AB})$ and $B \subset U(C, \epsilon_{BC}), C \subset U(B, \epsilon_{BC})$. From $A \subset U(B, \epsilon_{AB})$ and $B \subset U(C, \epsilon_{BC})$ it follows that $A \subset U(C, \epsilon_{AB} + \epsilon_{BC})$ by triangle inequality. Vice versa, $C \subset U(A, \epsilon_{AB} + \epsilon_{BC})$. Taking infs, it then follows that $D(A, B) + D(B, C) \geq D(A, C)$.
    \end{itemize}
    Hence $D$ is indeed a metric. \qed

    \textbf{(b)} $(X, d)$ is complete.

    Take $(A_n)_{n \in \bbn}$  be a Cauchy sequence in $(\calh, D)$. WTS $A_n$ converges to some set in $\calh$.

    Since $(A_n)$ is Cauchy, it follows that for all $\epsilon > 0$, there exists $N_{\epsilon}$ such that for all $n, m \geq N_\epsilon$, we have \begin{equation*}
    D(A_n, A_m) < \epsilon
    \end{equation*}

    Then, for all $j$, set $\epsilon = \frac{1}{2^j}$. Then there exists some $N_j$ such that for all $n, m \geq N_j$, we have $D(A_n, A_m) < \frac{1}{2^j}$. WLOG, set the $N_j$ be increasing $j$. 

    Define a new sequence (as a subsequence of $A_n$) $B_j \coloneqq A_{N_j}$, then since $N_j$ is increasing in $j$, it follows that \begin{equation*}
    D(B_n, B_{n+1}) < \frac{1}{2^n}
    \end{equation*}

    Define $A$ to be the set of all points $x$ that are the limits of sequences $(x_1, x_2, \ldots)$ such that for all $i$, $x_i \in B_i$ and $d(x_i, x_{i+1}) < \frac{1}{2^i}$. We want to show that $B_n \cvgn \cl{A}$.

    \textbf{0.} We first show that $A$ is non-empty. Take any $x_1 \in B_1$. Then since $B_1 \subset U(B_2, \frac{1}{2})$, it follows that there exists some $x_2 \in B_2$ such that $d(x_1, x_2) < \frac{1}{2}$. Perform this inductively, then we have a sequence $(x_n)$ with $x_n \in B_n$ such that $d(x_n, x_{n+1}) < \frac{1}{2^n}$. So $(x_n)$ is Cauchy (the sum $\sum_{k = n}^{m} \frac{1}{2^k} \cvgn 0$, so by triangle inequality, it is Cauchy) in complete $X$, so $x_n \cvgn x \in X$. So $A$ is nonempty, so $\cl{A}$ is nonempty. We pay special attention to how a ``valid'' sequence can be ``generated'' from a point.
    
    \textbf{1.} It is trivial that if $A_n$ is bounded then $\cl{A}$ is also bounded.

    \textbf{2.} Fix $N$. We now want to show $B_N \subset U(\cl{A}, \frac{1}{2^{N-1}})$.
    
    Take some $b_N \in B_N$. Then by the aforementioned construction, we can form sequence $(b_n)_{n \geq N}$ such that $d(b_n, b_{n+1}) < \frac{1}{2^n}$, which is Cauchy, and since $X$ is complete, converges to some $b \in X$. Since $B_{n+1} \subset U(B_n, \frac{1}{2^n})$, we can also pad in front of this $(b_n)_{n \geq N}$ to form $(b_n)_{n \in \bbn}$ such that $d(b_n, b_{n+1}) < \frac{1}{2^n}$ still, while the limit point stays the same. It then follows that in fact $b \in A$.

    Then \begin{equation*}
    d(b_N, b) \leq d(b_N, b_{N_\epsilon}) + \epsilon \leq \sum_{k = N}^{\infty} \frac{1}{2^k} + \epsilon \leq \frac{1}{2^{N-1}} + \epsilon
    \end{equation*}
    for all $\epsilon$, hence $d(b_N, b) \leq \frac{1}{2^{N-1}}$. Therefore $B_N \subset U(A, \frac{1}{2^{N-1}}) \implies B_N \subset U(\cl{A}, \frac{1}{2^{N-1}})$.

    \textbf{3.} We now want to show $\cl{A} \subset U(B_N, \frac{1}{2^{N-1}})$.

    Take $a \in \cl{A}$. Then for all $\epsilon > 0$, there exists some $a_\epsilon \in A$ such that $d(a, a_\epsilon) < \epsilon$. Since $a_\epsilon \in A$, there exists some sequence $(b_{n, \epsilon})$ such that $b_{n, \epsilon} \in B_n$ and $d(b_{n, \epsilon}, b_{n+1, \epsilon}) < \frac{1}{2^n}$ for all $n$. Similar to above, we have that \begin{equation*}
    d(a_\epsilon, b_{n, \epsilon}) \leq \frac{1}{2^{n-1}} 
    \end{equation*}
    It then follows that there for every $n, \epsilon$, there exists $b_{n, \epsilon}$ such that \begin{equation*}
    d(a, b_{n, \epsilon}) \leq d(a, a_{\epsilon}) + d(a_\epsilon, b_{n, \epsilon}) < \frac{1}{2^{n-1}} + \epsilon
    \end{equation*}
    hence $d(a, b_{n, \epsilon}) \leq \frac{1}{2^{n-1}}$. It then follows that $\cl{A} \subset U(B_n, \frac{1}{2^{n-1}})$ for all $n$.

    \textbf{4.} From all the above, it then follows $\cl{A} \in \calh$ and so it is well-defined to conclude $D(\cl{A}, B_n) < \frac{1}{2^{n-1}}$ for all $n$. Hence $B_n \cvgn \cl{A}$. \qed

    \textbf{(c)} $(X, d)$ is totally bounded. We want to show that for all $\epsilon > 0$, we can cover $\calh$ by finitely many $\epsilon$-balls. Select $\delta = \epsilon/2$. Then since $X$ is totally bounded, there exists a finite cover of $X$ using $\delta$-balls, namely, $\{B(x_i, \delta) : x_i \in X\}_{i \leq N}$.

    Let $S = \{x_i : i \leq N\}$ and take $\cala \subset 2^S$ to be the collection of nonempty subsets of $S$. Since $\abs{S} \leq N \implies \abs{\cala} \leq \abs{2^S} \leq 2^N$.
    
    We therefore WTS $\{B_D(A, \epsilon) \mid A \in \cala\}$ covers $\calh$ to demonstrate a finite $\epsilon$-balls cover of $\calh$, i.e., that for any $C \in \calh$, then there exists some $A \in \cala$ such that $C \in B_D(A, \epsilon)$.

    Now, $C$ is a nonempty, closed, bounded subset of $X$. Define $A \coloneqq \{x_i : i \leq N, B(x_i, \delta) \cap C \neq \emptyset\} \in \cala$. Then WTS $C \in B_D(A, \epsilon) \Leftrightarrow C \subset U(A, \epsilon), A \subset U(C, \epsilon)$.

    Since $\{B(x_i, \delta) : i \leq N\}$ cover $X$, and $B(x_i, \delta) \cap C = \emptyset$ if $x_i \not \in A$, it then follows that $\{B(x_i, \delta) : x_i \in A\}$ covers $C$. But then since $\delta < \epsilon$, it then follows that $C \subset U(A, \epsilon)$.
    
    Furthermore, since for all $x_i \in A$, $B(x_i, \delta) \cap C \neq \emptyset$ and $\delta < \epsilon$, it then follows $A \subset U(C, \epsilon)$.

    From the 2 paragraphs above, we can therefore conclude that $C \in B_D(A, \epsilon)$ as required. \qed

    \textbf{(d)} $(X, d)$ compact implies $(X, d)$ complete and totally bounded implies $(\calh, D)$ complete and totally bounded implies $(\calh, D)$ compact.

\end{solution}

\begin{problem} [46.6 \done]
    Show that in the compact-open topology, $\calc(X, Y)$ is Hausdorff if $Y$ is Hausdorff, and regular if $Y$ is regular. [Hint: If $\cl{U} \subset V$ then $\cl{S(C, U)} \subset S(C, V)$]
\end{problem}
\begin{solution}
    \textbf{1.} $Y$ is Hausdorff.

    Take $f \neq g \in \calc(X, Y)$. Since $f \neq g$, it implies that there exists $a \in X$ such that $f(a) \neq g(a)$. Since $Y$ is Hausdorff, there exists $U_f, U_g$ disjoint and open in $Y$ such that $f(a) \in U_f, g(a) \in U_g$. Then $S(\{a\}, U_f) \ni f, S(\{a\}, U_g) \ni g$ and they are disjoint, hence $\calc(X, Y)$ is also Hausdorff.

    \textbf{2.} $Y$ is regular.

    Recall regularity is equivalent to saying that given an open neighborhood of a point, one can always find a smaller neighborhood around that point whose closure is contained in the first open neighborhood.

    To prove this, we shall only do this for the subbasis that generates the compact-open topology, since closure of the intersection is a subset of the intersection of closures.

    Take $f \in S(C, U)$ where $f \in \calc(X, Y)$, $C \subset X$ is compact and $U \subset Y$ open. Since $f \in S(C, U)$, it follows that $f(C) \subset U$.

    Since $C$ is compact and $f$ is continuous, $f(C)$ is compact. For each $y \in f(C) \subset U$, then, since $Y$ is regular, there exists some $V_y$ open such that $y \in V_y \subset \cl{V_y} \subset U$. Then $\{V_y : y \in f(C)\}$ is an open cover of $f(C)$ and hence reduces to some finite subcover $\{V_{y_i}\}_{i \leq N}$. Let $V = \cup_{i \leq N} V_{y_i}$, then $V$ is open and finiteness gives us $\cl{V} = \cup_{i \leq N} \cl{V_{y_i}} \subset Y$, and also that $f(C) \subset V$, i.e., $f \in S(C, V)$.

    WTS $\cl{S(C, V)} \subset S(C, U)$ by showing that $\cl{S(C, V)} \subset \{f \in \calc(X, Y) \mid f(C) \subset \cl{V}\}$. Suppose $f \in \calc(X, Y) \backslash \{f \in \calc (X, Y) \mid f(C) \subset \cl{V}\}$. This implies there exists some $x_0 \in C$ such that $f(x_0) \not \in \cl{V}$. Then there exists some $W \ni f(x_0)$ open such that $W \cap \cl{V} = \emptyset$. It then follows that $f \in S(\{x_0\}, W)$ and $S(\{x_0\}, W) \cap \{f \in \calc(X, Y) \mid f(C) \subset \cl{V}\} = \emptyset$. It follows that $\{f \in \calc(X, Y) \mid f(C) \subset \cl{V}\}$ is closed.

    By definition, $S(C, V) \subset \{f \in \calc(X, Y) \mid f(C) \subset \cl{V}\} \implies \cl{S(C, V)} \subset \cl{\{f \in \calc(X, Y) \mid f(C) \subset \cl{V}\}} = \{f \in \calc(X, Y) \mid f(C) \subset \cl{V}\} \subset S(C, U)$ as required.
\end{solution}

\begin{problem} [47.5]
    Let $(Y, d)$ be a metric space; let $f_n: X \to Y$ be a sequence of continuous functions; let $f: X \to Y$ be a function (not necessarily continuous). Suppose $f_n$ converges to $f$ is the topology of pointwise convergence. Show that if $\{f_n\}$ is equicontinuous, then $f$ is continuous and $f_n$ converges to $f$ in the topology of compact convergence.
\end{problem}
\begin{solution}
    Consider $\calf = \{f_n : n \in \bbn\}$ is a family of functions in $\calc(X, Y)$. WTS $\calf_a \coloneqq \{f_n(a) : n \in \bbn\}$ has compact closure for each $a \in X$.

    Since $f_n \cvgn f$ in the topology of pointwise convergence, $f_n(a) \cvgn f(a)$. It then follows that $\cl{\calf_a}$ is sequentially compact. It is sequentially compact in the metric space $(Y, d)$, so it is compact.

    Apply Ascoli's Theorem, it then follows that $\cl{\calf}$ is compact in $\calc(X, Y)$ in the topology of compact convergence, which implies that $f_n \cvgn f$ in this topology. $\calf$ is compact in Hausdorff $\calc(X, Y)$, hence is closed, so $f \in C(X, Y)$ as required.
\end{solution}
\end{document}