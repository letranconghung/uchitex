\documentclass[a4paper, 10pt]{article}
%%%%%%%%%%%%%%%%%%%%%%%%%%%%%%%%%%%%%%%%%%%%%%%%%%%%%%%%%%%%%%%%%%%%%%%%%%%%%%%
%                                Basic Packages                               %
%%%%%%%%%%%%%%%%%%%%%%%%%%%%%%%%%%%%%%%%%%%%%%%%%%%%%%%%%%%%%%%%%%%%%%%%%%%%%%%

% Gives us multiple colors.
\usepackage[usenames,dvipsnames,pdftex]{xcolor}
% Lets us style link colors.
\usepackage{hyperref}
% Lets us import images and graphics.
\usepackage{graphicx}
% Lets us use figures in floating environments.
\usepackage{float}
% Lets us create multiple columns.
\usepackage{multicol}
% Gives us better math syntax.
\usepackage{amsmath,amsfonts,mathtools,amsthm,amssymb}
% Lets us strikethrough text.
\usepackage{cancel}
% Lets us edit the caption of a figure.
\usepackage{caption}
% Lets us import pdf directly in our tex code.
\usepackage{pdfpages}
% Lets us do algorithm stuff.
\usepackage[ruled,vlined,linesnumbered]{algorithm2e}
% Use a smiley face for our qed symbol.
\usepackage{tikzsymbols}
% \usepackage{fullpage} %%smaller margins
\usepackage[shortlabels]{enumitem}

\setlist[enumerate]{font={\bfseries}} % global settings, for all lists

\usepackage{setspace}
\usepackage[margin=1in, headsep=12pt]{geometry}
\usepackage{wrapfig}
\usepackage{listings}
\usepackage{parskip}

\definecolor{codegreen}{rgb}{0,0.6,0}
\definecolor{codegray}{rgb}{0.5,0.5,0.5}
\definecolor{codepurple}{rgb}{0.58,0,0.82}
\definecolor{backcolour}{rgb}{0.95,0.95,0.95}

\lstdefinestyle{mystyle}{
    backgroundcolor=\color{backcolour},   
    commentstyle=\color{codegreen},
    keywordstyle=\color{magenta},
    numberstyle=\tiny\color{codegray},
    stringstyle=\color{codepurple},
    basicstyle=\ttfamily\footnotesize,
    breakatwhitespace=false,         
    breaklines=true,                 
    captionpos=b,                    
    keepspaces=true,                 
    numbers=left,                    
    numbersep=5pt,                  
    showspaces=false,                
    showstringspaces=false,
    showtabs=false,                  
    tabsize=2,
    numbers=none
}

\lstset{style=mystyle}
\def\class{article}


%%%%%%%%%%%%%%%%%%%%%%%%%%%%%%%%%%%%%%%%%%%%%%%%%%%%%%%%%%%%%%%%%%%%%%%%%%%%%%%
%                                Basic Settings                               %
%%%%%%%%%%%%%%%%%%%%%%%%%%%%%%%%%%%%%%%%%%%%%%%%%%%%%%%%%%%%%%%%%%%%%%%%%%%%%%%

%%%%%%%%%%%%%
%  Symbols  %
%%%%%%%%%%%%%

\let\implies\Rightarrow
\let\impliedby\Leftarrow
\let\iff\Leftrightarrow
\let\epsilon\varepsilon
%%%%%%%%%%%%
%  Tables  %
%%%%%%%%%%%%

\setlength{\tabcolsep}{5pt}
\renewcommand\arraystretch{1.5}

%%%%%%%%%%%%%%
%  SI Unitx  %
%%%%%%%%%%%%%%

\usepackage{siunitx}
\sisetup{locale = FR}

%%%%%%%%%%
%  TikZ  %
%%%%%%%%%%

\usepackage[framemethod=TikZ]{mdframed}
\usepackage{tikz}
\usepackage{tikz-cd}
\usepackage{tikzsymbols}

\usetikzlibrary{intersections, angles, quotes, calc, positioning}
\usetikzlibrary{arrows.meta}

\tikzset{
    force/.style={thick, {Circle[length=2pt]}-stealth, shorten <=-1pt}
}

%%%%%%%%%%%%%%%
%  PGF Plots  %
%%%%%%%%%%%%%%%

\usepackage{pgfplots}
\pgfplotsset{width=10cm, compat=newest}

%%%%%%%%%%%%%%%%%%%%%%%
%  Center Title Page  %
%%%%%%%%%%%%%%%%%%%%%%%

\usepackage{titling}
\renewcommand\maketitlehooka{\null\mbox{}\vfill}
\renewcommand\maketitlehookd{\vfill\null}

%%%%%%%%%%%%%%%%%%%%%%%%%%%%%%%%%%%%%%%%%%%%%%%%%%%%%%%
%  Create a grey background in the middle of the PDF  %
%%%%%%%%%%%%%%%%%%%%%%%%%%%%%%%%%%%%%%%%%%%%%%%%%%%%%%%

\usepackage{eso-pic}
\newcommand\definegraybackground{
    \definecolor{reallylightgray}{HTML}{FAFAFA}
    \AddToShipoutPicture{
        \ifthenelse{\isodd{\thepage}}{
            \AtPageLowerLeft{
                \put(\LenToUnit{\dimexpr\paperwidth-222pt},0){
                    \color{reallylightgray}\rule{222pt}{297mm}
                }
            }
        }
        {
            \AtPageLowerLeft{
                \color{reallylightgray}\rule{222pt}{297mm}
            }
        }
    }
}

%%%%%%%%%%%%%%%%%%%%%%%%
%  Modify Links Color  %
%%%%%%%%%%%%%%%%%%%%%%%%

\hypersetup{
    % Enable highlighting links.
    colorlinks,
    % Change the color of links to blue.
    urlcolor=blue,
    % Change the color of citations to black.
    citecolor={black},
    % Change the color of url's to blue with some black.
    linkcolor={blue!80!black}
}

%%%%%%%%%%%%%%%%%%
% Fix WrapFigure %
%%%%%%%%%%%%%%%%%%

\newcommand{\wrapfill}{\par\ifnum\value{WF@wrappedlines}>0
        \parskip=0pt
        \addtocounter{WF@wrappedlines}{-1}%
        \null\vspace{\arabic{WF@wrappedlines}\baselineskip}%
        \WFclear
    \fi}

%%%%%%%%%%%%%%%%%
% Multi Columns %
%%%%%%%%%%%%%%%%%

\let\multicolmulticols\multicols
\let\endmulticolmulticols\endmulticols

\RenewDocumentEnvironment{multicols}{mO{}}
{%
    \ifnum#1=1
        #2%
    \else % More than 1 column
        \multicolmulticols{#1}[#2]
    \fi
}
{%
    \ifnum#1=1
    \else % More than 1 column
        \endmulticolmulticols
    \fi
}

\newlength{\thickarrayrulewidth}
\setlength{\thickarrayrulewidth}{5\arrayrulewidth}


%%%%%%%%%%%%%%%%%%%%%%%%%%%%%%%%%%%%%%%%%%%%%%%%%%%%%%%%%%%%%%%%%%%%%%%%%%%%%%%
%                           School Specific Commands                          %
%%%%%%%%%%%%%%%%%%%%%%%%%%%%%%%%%%%%%%%%%%%%%%%%%%%%%%%%%%%%%%%%%%%%%%%%%%%%%%%

%%%%%%%%%%%%%%%%%%%%%%%%%%%
%  Initiate New Counters  %
%%%%%%%%%%%%%%%%%%%%%%%%%%%

\newcounter{lecturecounter}

%%%%%%%%%%%%%%%%%%%%%%%%%%
%  Helpful New Commands  %
%%%%%%%%%%%%%%%%%%%%%%%%%%

\makeatletter

\newcommand\resetcounters{
    % Reset the counters for subsection, subsubsection and the definition
    % all the custom environments.
    \setcounter{subsection}{0}
    \setcounter{subsubsection}{0}
    \setcounter{definition0}{0}
    \setcounter{paragraph}{0}
    \setcounter{theorem}{0}
    \setcounter{claim}{0}
    \setcounter{corollary}{0}
    \setcounter{proposition}{0}
    \setcounter{lemma}{0}
    \setcounter{exercise}{0}
    \setcounter{problem}{0}
    
    \setcounter{subparagraph}{0}
    % \@ifclasswith\class{nocolor}{
    %     \setcounter{definition}{0}
    % }{}
}

%%%%%%%%%%%%%%%%%%%%%
%  Lecture Command  %
%%%%%%%%%%%%%%%%%%%%%

\usepackage{xifthen}

% EXAMPLE:
% 1. \lecture{Oct 17 2022 Mon (08:46:48)}{Lecture Title}
% 2. \lecture[4]{Oct 17 2022 Mon (08:46:48)}{Lecture Title}
% 3. \lecture{Oct 17 2022 Mon (08:46:48)}{}
% 4. \lecture[4]{Oct 17 2022 Mon (08:46:48)}{}
% Parameters:
% 1. (Optional) lecture number.
% 2. Time and date of lecture.
% 3. Lecture Title.
\def\@lecture{}
\def\@lectitle{}
\def\@leccount{}
\newcommand\lecture[3]{
    \newpage

    % Check if user passed the lecture title or not.
    \def\@leccount{Lecture #1}
    \ifthenelse{\isempty{#3}}{
        \def\@lecture{Lecture #1}
        \def\@lectitle{Lecture #1}
    }{
        \def\@lecture{Lecture #1: #3}
        \def\@lectitle{#3}
    }

    \setcounter{section}{#1}
    \renewcommand\thesubsection{#1.\arabic{subsection}}
    
    \phantomsection
    \addcontentsline{toc}{section}{\@lecture}
    \resetcounters

    \begin{mdframed}
        \begin{center}
            \Large \textbf{\@leccount}
            
            \vspace*{0.2cm}
            
            \large \@lectitle
            
            
            \vspace*{0.2cm}

            \normalsize #2
        \end{center}
    \end{mdframed}

}

%%%%%%%%%%%%%%%%%%%%
%  Import Figures  %
%%%%%%%%%%%%%%%%%%%%

\usepackage{import}
\pdfminorversion=7

% EXAMPLE:
% 1. \incfig{limit-graph}
% 2. \incfig[0.4]{limit-graph}
% Parameters:
% 1. The figure name. It should be located in figures/NAME.tex_pdf.
% 2. (Optional) The width of the figure. Example: 0.5, 0.35.
\newcommand\incfig[2][1]{%
    \def\svgwidth{#1\columnwidth}
    \import{./figures/}{#2.pdf_tex}
}

\begingroup\expandafter\expandafter\expandafter\endgroup
\expandafter\ifx\csname pdfsuppresswarningpagegroup\endcsname\relax
\else
    \pdfsuppresswarningpagegroup=1\relax
\fi

%%%%%%%%%%%%%%%%%
% Fancy Headers %
%%%%%%%%%%%%%%%%%

\usepackage{fancyhdr}

% Force a new page.
\newcommand\forcenewpage{\clearpage\mbox{~}\clearpage\newpage}

% This command makes it easier to manage my headers and footers.
\newcommand\createintro{
    % Use roman page numbers (e.g. i, v, vi, x, ...)
    \pagenumbering{roman}

    % Display the page style.
    \maketitle
    % Make the title pagestyle empty, meaning no fancy headers and footers.
    \thispagestyle{empty}
    % Create a newpage.
    \newpage

    % Input the intro.tex page if it exists.
    \IfFileExists{intro.tex}{ % If the intro.tex file exists.
        % Input the intro.tex file.
        \textbf{Course}: MATH 16300: Honors Calculus III

\textbf{Section}: 43

\textbf{Professor}: Minjae Park

\textbf{At}: The University of Chicago

\textbf{Quarter}: Spring 2023

\textbf{Course materials}: Calculus by Spivak (4th Edition), Calculus On Manifolds by Spivak

\vspace{1cm}
\textbf{Disclaimer}: This document will inevitably contain some mistakes, both simple typos and serious logical and mathematical errors. Take what you read with a grain of salt as it is made by an undergraduate student going through the learning process himself. If you do find any error, I would really appreciate it if you can let me know by email at \href{mailto:conghungletran@gmail.com}{conghungletran@gmail.com}.

        % Make the pagestyle fancy for the intro.tex page.
        \pagestyle{fancy}

        % Remove the line for the header.
        \renewcommand\headrulewidth{0pt}

        % Remove all header stuff.
        \fancyhead{}

        % Add stuff for the footer in the center.
        % \fancyfoot[C]{
        %   \textit{For more notes like this, visit
        %   \href{\linktootherpages}{\shortlinkname}}. \\
        %   \vspace{0.1cm}
        %   \hrule
        %   \vspace{0.1cm}
        %   \@author, \\
        %   \term: \academicyear, \\
        %   Last Update: \@date, \\
        %   \faculty
        % }

        \newpage
    }{ % If the intro.tex file doesn't exist.
        % Force a \newpageage.
        % \forcenewpage
        \newpage
    }

    % Remove the center stuff we did above, and replace it with just the page
    % number, which is still in roman numerals.
    \fancyfoot[C]{\thepage}
    % Add the table of contents.
    \tableofcontents
    % Force a new page.
    \newpage

    % Move the page numberings back to arabic, from roman numerals.
    \pagenumbering{arabic}
    % Set the page number to 1.
    \setcounter{page}{1}

    % Add the header line back.
    \renewcommand\headrulewidth{0.4pt}
    % In the top right, add the lecture title.
    \fancyhead[R]{\footnotesize \@lecture}
    % In the top left, add the author name.
    \fancyhead[L]{\footnotesize \@author}
    % In the bottom center, add the page.
    \fancyfoot[C]{\thepage}
    % Add a nice gray background in the middle of all the upcoming pages.
    % \definegraybackground
}

\makeatother


%%%%%%%%%%%%%%%%%%%%%%%%%%%%%%%%%%%%%%%%%%%%%%%%%%%%%%%%%%%%%%%%%%%%%%%%%%%%%%%
%                               Custom Commands                               %
%%%%%%%%%%%%%%%%%%%%%%%%%%%%%%%%%%%%%%%%%%%%%%%%%%%%%%%%%%%%%%%%%%%%%%%%%%%%%%%

%%%%%%%%%%%%
%  Circle  %
%%%%%%%%%%%%

\newcommand*\circled[1]{\tikz[baseline= (char.base)]{
        \node[shape=circle,draw,inner sep=1pt] (char) {#1};}
}

%%%%%%%%%%%%%%%%%%%
%  Todo Commands  %
%%%%%%%%%%%%%%%%%%%

% \usepackage{xargs}
% \usepackage[colorinlistoftodos]{todonotes}

% \makeatletter

% \@ifclasswith\class{working}{
%     \newcommandx\unsure[2][1=]{\todo[linecolor=red,backgroundcolor=red!25,bordercolor=red,#1]{#2}}
%     \newcommandx\change[2][1=]{\todo[linecolor=blue,backgroundcolor=blue!25,bordercolor=blue,#1]{#2}}
%     \newcommandx\info[2][1=]{\todo[linecolor=OliveGreen,backgroundcolor=OliveGreen!25,bordercolor=OliveGreen,#1]{#2}}
%     \newcommandx\improvement[2][1=]{\todo[linecolor=Plum,backgroundcolor=Plum!25,bordercolor=Plum,#1]{#2}}

%     \newcommand\listnotes{
%         \newpage
%         \listoftodos[Notes]
%     }
% }{
%     \newcommandx\unsure[2][1=]{}
%     \newcommandx\change[2][1=]{}
%     \newcommandx\info[2][1=]{}
%     \newcommandx\improvement[2][1=]{}

%     \newcommand\listnotes{}
% }

% \makeatother

%%%%%%%%%%%%%
%  Correct  %
%%%%%%%%%%%%%

% EXAMPLE:
% 1. \correct{INCORRECT}{CORRECT}
% Parameters:
% 1. The incorrect statement.
% 2. The correct statement.
\definecolor{correct}{HTML}{009900}
\newcommand\correct[2]{{\color{red}{#1 }}\ensuremath{\to}{\color{correct}{ #2}}}


%%%%%%%%%%%%%%%%%%%%%%%%%%%%%%%%%%%%%%%%%%%%%%%%%%%%%%%%%%%%%%%%%%%%%%%%%%%%%%%
%                                 Environments                                %
%%%%%%%%%%%%%%%%%%%%%%%%%%%%%%%%%%%%%%%%%%%%%%%%%%%%%%%%%%%%%%%%%%%%%%%%%%%%%%%

\usepackage{varwidth}
\usepackage{thmtools}
\usepackage[most,many,breakable]{tcolorbox}

\tcbuselibrary{theorems,skins,hooks}
\usetikzlibrary{arrows,calc,shadows.blur}

%%%%%%%%%%%%%%%%%%%
%  Define Colors  %
%%%%%%%%%%%%%%%%%%%

% color prototype
% \definecolor{color}{RGB}{45, 111, 177}

% ESSENTIALS: 
\definecolor{myred}{HTML}{c74540}
\definecolor{myblue}{HTML}{072b85}
\definecolor{mygreen}{HTML}{388c46}
\definecolor{myblack}{HTML}{000000}

\colorlet{definition_color}{myred}

\colorlet{theorem_color}{myblue}
\colorlet{lemma_color}{myblue}
\colorlet{prop_color}{myblue}
\colorlet{corollary_color}{myblue}
\colorlet{claim_color}{myblue}

\colorlet{proof_color}{myblack}
\colorlet{example_color}{myblack}
\colorlet{exercise_color}{myblack}

% MISCS: 
%%%%%%%%%%%%%%%%%%%%%%%%%%%%%%%%%%%%%%%%%%%%%%%%%%%%%%%%%
%  Create Environments Styles Based on Given Parameter  %
%%%%%%%%%%%%%%%%%%%%%%%%%%%%%%%%%%%%%%%%%%%%%%%%%%%%%%%%%

% \mdfsetup{skipabove=1em,skipbelow=0em}

%%%%%%%%%%%%%%%%%%%%%%
%  Helpful Commands  %
%%%%%%%%%%%%%%%%%%%%%%

% EXAMPLE:
% 1. \createnewtheoremstyle{thmdefinitionbox}{}{}
% 2. \createnewtheoremstyle{thmtheorembox}{}{}
% 3. \createnewtheoremstyle{thmproofbox}{qed=\qedsymbol}{
%       rightline=false, topline=false, bottomline=false
%    }
% Parameters:
% 1. Theorem name.
% 2. Any extra parameters to pass directly to declaretheoremstyle.
% 3. Any extra parameters to pass directly to mdframed.
\newcommand\createnewtheoremstyle[3]{
    \declaretheoremstyle[
        headfont=\bfseries\sffamily, bodyfont=\normalfont, #2,
        mdframed={
                #3,
            },
    ]{#1}
}

% EXAMPLE:
% 1. \createnewcoloredtheoremstyle{thmdefinitionbox}{definition}{}{}
% 2. \createnewcoloredtheoremstyle{thmexamplebox}{example}{}{
%       rightline=true, leftline=true, topline=true, bottomline=true
%     }
% 3. \createnewcoloredtheoremstyle{thmproofbox}{proof}{qed=\qedsymbol}{backgroundcolor=white}
% Parameters:
% 1. Theorem name.
% 2. Color of theorem.
% 3. Any extra parameters to pass directly to declaretheoremstyle.
% 4. Any extra parameters to pass directly to mdframed.

% change backgroundcolor to #2!5 if user wants a colored backdrop to theorem environments. It's a cool color theme, but there's too much going on in the page.
\newcommand\createnewcoloredtheoremstyle[4]{
    \declaretheoremstyle[
        headfont=\bfseries\sffamily\color{#2},
        bodyfont=\normalfont,
        headpunct=,
        headformat = \NAME~\NUMBER\NOTE \hfill\smallskip\linebreak,
        #3,
        mdframed={
                outerlinewidth=0.75pt,
                rightline=false,
                leftline=false,
                topline=false,
                bottomline=false,
                backgroundcolor=white,
                skipabove = 5pt,
                skipbelow = 0pt,
                linecolor=#2,
                innertopmargin = 0pt,
                innerbottommargin = 0pt,
                innerrightmargin = 4pt,
                innerleftmargin= 6pt,
                leftmargin = -6pt,
                #4,
            },
    ]{#1}
}



%%%%%%%%%%%%%%%%%%%%%%%%%%%%%%%%%%%
%  Create the Environment Styles  %
%%%%%%%%%%%%%%%%%%%%%%%%%%%%%%%%%%%

\makeatletter
\@ifclasswith\class{nocolor}{
    % Environments without color.

    % ESSENTIALS:
    \createnewtheoremstyle{thmdefinitionbox}{}{}
    \createnewtheoremstyle{thmtheorembox}{}{}
    \createnewtheoremstyle{thmproofbox}{qed=\qedsymbol}{}
    \createnewtheoremstyle{thmcorollarybox}{}{}
    \createnewtheoremstyle{thmlemmabox}{}{}
    \createnewtheoremstyle{thmclaimbox}{}{}
    \createnewtheoremstyle{thmexamplebox}{}{}

    % MISCS: 
    \createnewtheoremstyle{thmpropbox}{}{}
    \createnewtheoremstyle{thmexercisebox}{}{}
    \createnewtheoremstyle{thmexplanationbox}{}{}
    \createnewtheoremstyle{thmremarkbox}{}{}
    
    % STYLIZED MORE BELOW
    \createnewtheoremstyle{thmquestionbox}{}{}
    \createnewtheoremstyle{thmsolutionbox}{qed=\qedsymbol}{}
}{
    % Environments with color.

    % ESSENTIALS: definition, theorem, proof, corollary, lemma, claim, example
    \createnewcoloredtheoremstyle{thmdefinitionbox}{definition_color}{}{leftline=false}
    \createnewcoloredtheoremstyle{thmtheorembox}{theorem_color}{}{leftline=false}
    \createnewcoloredtheoremstyle{thmproofbox}{proof_color}{qed=\qedsymbol}{}
    \createnewcoloredtheoremstyle{thmcorollarybox}{corollary_color}{}{leftline=false}
    \createnewcoloredtheoremstyle{thmlemmabox}{lemma_color}{}{leftline=false}
    \createnewcoloredtheoremstyle{thmpropbox}{prop_color}{}{leftline=false}
    \createnewcoloredtheoremstyle{thmclaimbox}{claim_color}{}{leftline=false}
    \createnewcoloredtheoremstyle{thmexamplebox}{example_color}{}{}
    \createnewcoloredtheoremstyle{thmexplanationbox}{example_color}{qed=\qedsymbol}{}
    \createnewcoloredtheoremstyle{thmremarkbox}{theorem_color}{}{}

    \createnewcoloredtheoremstyle{thmmiscbox}{black}{}{}

    \createnewcoloredtheoremstyle{thmexercisebox}{exercise_color}{}{}
    \createnewcoloredtheoremstyle{thmproblembox}{theorem_color}{}{leftline=false}
    \createnewcoloredtheoremstyle{thmsolutionbox}{mygreen}{qed=\qedsymbol}{}
}
\makeatother

%%%%%%%%%%%%%%%%%%%%%%%%%%%%%
%  Create the Environments  %
%%%%%%%%%%%%%%%%%%%%%%%%%%%%%
\declaretheorem[numberwithin=section, style=thmdefinitionbox,     name=Definition]{definition}
\declaretheorem[numberwithin=section, style=thmtheorembox,     name=Theorem]{theorem}
\declaretheorem[numbered=no,          style=thmexamplebox,     name=Example]{example}
\declaretheorem[numberwithin=section, style=thmtheorembox,       name=Claim]{claim}
\declaretheorem[numberwithin=section, style=thmcorollarybox,   name=Corollary]{corollary}
\declaretheorem[numberwithin=section, style=thmpropbox,        name=Proposition]{proposition}
\declaretheorem[numberwithin=section, style=thmlemmabox,       name=Lemma]{lemma}
\declaretheorem[numberwithin=section, style=thmexercisebox,    name=Exercise]{exercise}
\declaretheorem[numbered=no,          style=thmproofbox,       name=Proof]{proof0}
\declaretheorem[numbered=no,          style=thmexplanationbox, name=Explanation]{explanation}
\declaretheorem[numbered=no,          style=thmsolutionbox,    name=Solution]{solution}
\declaretheorem[numberwithin=section,          style=thmproblembox,     name=Problem]{problem}
\declaretheorem[numbered=no,          style=thmmiscbox,    name=Intuition]{intuition}
\declaretheorem[numbered=no,          style=thmmiscbox,    name=Goal]{goal}
\declaretheorem[numbered=no,          style=thmmiscbox,    name=Recall]{recall}
\declaretheorem[numbered=no,          style=thmmiscbox,    name=Motivation]{motivation}
\declaretheorem[numbered=no,          style=thmmiscbox,    name=Remark]{remark}
\declaretheorem[numbered=no,          style=thmmiscbox,    name=Observe]{observe}
\declaretheorem[numbered=no,          style=thmmiscbox,    name=Question]{question}


%%%%%%%%%%%%%%%%%%%%%%%%%%%%
%  Edit Proof Environment  %
%%%%%%%%%%%%%%%%%%%%%%%%%%%%

\renewenvironment{proof}[2][\proofname]{
    % \vspace{-12pt}
    \begin{proof0} [#2]
        }{\end{proof0}}

\theoremstyle{definition}

\newtheorem*{notation}{Notation}
\newtheorem*{previouslyseen}{As previously seen}
\newtheorem*{property}{Property}
% \newtheorem*{intuition}{Intuition}
% \newtheorem*{goal}{Goal}
% \newtheorem*{recall}{Recall}
% \newtheorem*{motivation}{Motivation}
% \newtheorem*{remark}{Remark}
% \newtheorem*{observe}{Observe}

\author{Hung C. Le Tran}


%%%% MATH SHORTHANDS %%%%
%% blackboard bold math capitals
\DeclareMathOperator*{\esssup}{ess\,sup}
\DeclareMathOperator*{\Hom}{Hom}
\newcommand{\bbf}{\mathbb{F}}
\newcommand{\bbn}{\mathbb{N}}
\newcommand{\bbq}{\mathbb{Q}}
\newcommand{\bbr}{\mathbb{R}}
\newcommand{\bbz}{\mathbb{Z}}
\newcommand{\bbc}{\mathbb{C}}
\newcommand{\bbk}{\mathbb{K}}
\newcommand{\bbm}{\mathbb{M}}
\newcommand{\bbp}{\mathbb{P}}
\newcommand{\bbe}{\mathbb{E}}

\newcommand{\bfw}{\mathbf{w}}
\newcommand{\bfx}{\mathbf{x}}
\newcommand{\bfX}{\mathbf{X}}
\newcommand{\bfy}{\mathbf{y}}
\newcommand{\bfyhat}{\mathbf{\hat{y}}}

\newcommand{\calb}{\mathcal{B}}
\newcommand{\calf}{\mathcal{F}}
\newcommand{\calt}{\mathcal{T}}
\newcommand{\call}{\mathcal{L}}
\renewcommand{\phi}{\varphi}

% Universal Math Shortcuts
\newcommand{\st}{\hspace*{2pt}\text{s.t.}\hspace*{2pt}}
\newcommand{\pffwd}{\hspace*{2pt}\fbox{\(\Rightarrow\)}\hspace*{10pt}}
\newcommand{\pfbwd}{\hspace*{2pt}\fbox{\(\Leftarrow\)}\hspace*{10pt}}
\newcommand{\contra}{\ensuremath{\Rightarrow\Leftarrow}}
\newcommand{\cvgn}{\xrightarrow{n \to \infty}}
\newcommand{\cvgj}{\xrightarrow{j \to \infty}}

\newcommand{\im}{\mathrm{im}}
\newcommand{\innerproduct}[2]{\langle #1, #2 \rangle}
\newcommand*{\conj}[1]{\overline{#1}}

% https://tex.stackexchange.com/questions/438612/space-between-exists-and-forall
% https://tex.stackexchange.com/questions/22798/nice-looking-empty-set
\let\oldforall\forall
\renewcommand{\forall}{\;\oldforall\; }
\let\oldexist\exists
\renewcommand{\exists}{\;\oldexist\; }
\newcommand\existu{\;\oldexist!\: }
\let\oldemptyset\emptyset
\let\emptyset\varnothing


\renewcommand{\_}[1]{\underline{#1}}
\DeclarePairedDelimiter{\abs}{\lvert}{\rvert}
\DeclarePairedDelimiter{\norm}{\lVert}{\rVert}
\DeclarePairedDelimiter\ceil{\lceil}{\rceil}
\DeclarePairedDelimiter\floor{\lfloor}{\rfloor}
\setlength\parindent{0pt}
\setlength{\headheight}{12.0pt}
\addtolength{\topmargin}{-12.0pt}


% Default skipping, change if you want more spacing
% \thinmuskip=3mu
% \medmuskip=4mu plus 2mu minus 4mu
% \thickmuskip=5mu plus 5mu

% \DeclareMathOperator{\ext}{ext}
% \DeclareMathOperator{\bridge}{bridge}
\title{MATH 26200: Point-Set Topology \\ \large Problem Set 5}
\date{16 Feb 2024}
\author{Hung Le Tran}
\begin{document}
\maketitle
\setcounter{section}{5}
\begin{problem} [Problem 1 \done]
    Let $X$ be a set. Let $\calc$ be a nonempty family of subsets of $X$ with the finite intersection property. Prove that $\calc$ is contained in some filter, and (using Zorn's lemma) that every filter is contained in some maximal filter. Finally, prove that a filter is maximal if and only if it is an ultrafilter. (Please write out full proofs of these facts, just from the definitions of filter and ultrafilter).
\end{problem}
\begin{solution}
    \textbf{1.} WTS $\calc$ is in some filter.
    $\calc$ is nonempty family of subsets of $X$ with fip. We use $\calc$ to generate a filter as follows:
    \begin{equation*}
    \calf = \{B \subset X \mid \exists C_1, \ldots, C_n \:\text{such that}\: \bigcap_{k \in [n]} C_k \subset B\}
    \end{equation*}

    Clearly, $\calf$ is a family of subsets of $X$. It remains to show that $\calf$ satifies the conditions of a filter.

    \begin{itemize}
    \item $\emptyset \not \in \calf$, since $\calc$ has fip, so $\bigcap_{k \in [n]} C_k \neq \emptyset \implies \not \subset \emptyset$.
    \item $X \in \calf$, since $C_1 \subset X$.
    \item If $B_1, B_2 \in \calf$, i.e., $B_1 \supset \bigcap_{k \in [n]} C^{(1)}_k,  B_2 \supset \bigcap_{j \in [m]} C^{(2)}_j$ then \begin{equation*}
    B_1 \cap B_2 \supset \bigcap_{k \in [n]} C^{(1)}_k \cap \bigcap_{j \in [m]} C^{(2)}_j
    \end{equation*}
    so $B_1 \cap B_2 \in \calf$ too.
    \item If $B \in \calf$ and $B \subset A$ then there exists \begin{equation*}
    \bigcap_{k \in [n]} C_k \subset B \subset A \implies A \in \calf
    \end{equation*}
    \end{itemize}
    Therefore $\calf$ is indeed a filter.

    \textbf{2.} WTS every filter is contained in some maximal filter.

    Zorn's lemma states that a partially ordered set such that every of its totally ordered subsets has an upper bound then it has at least one maximal element.

    Take some filter $\calf_0$. Then let's look at set $P = \{\:\text{filter}\: \calf' \mid \calf_0 \subset \calf'\}$, with the order that $\calf_1 < \calf_2$ if $\calf_1 \subset \calf_2$.

    Then any totally ordered subset of $P$, i.e., any chain in $P$, $Q = \{\calf_{\alpha}\}_{\alpha \in A}$, has an upper bound in $P$, namely, \begin{equation*}
    \calf = \bigcup_{\alpha \in A} \calf_{\alpha}
    \end{equation*}.

    Since $Q$ is totally ordered, let $A$ reflect its ordering and be a totally ordered index set, i.e., $\calf_\alpha < \calf_\beta \Leftrightarrow \alpha < \beta$.

    WTS $\calf$ is a filter in $P$. \begin{itemize}
    \item Clearly, $\calf_0 \subset \calf, \emptyset \not \in \calf, X \in \calf$. 
    \item  Take $B_1, B_2 \in \calf$. Then $B_1 \in \calf_{\alpha_1}, B_2 \in \calf_{\alpha_2}$. WLOG, $\alpha_1 < \alpha_2$ so $\calf_{\alpha_1} \subset \calf_{\alpha_2} \implies B_1 \in \calf_{\alpha_2}$. It follows that $B_1 \cap B_2 \in \calf_{\alpha_2} \subset \calf$.
    \item Take $B \in \calf$ and $B \subset A$. Then $ B \in \calf_{\alpha} \implies A \in \calf_\alpha \subset \calf$.
    \end{itemize}
    
    So $\calf$ is indeed a filter in $P$. And $\calf_\alpha < \calf \forall \alpha \in A$ so $\calf$ is indeed an upper bound for the chain.

    Using Zorn's Lemma, then there exists a maximal element in $P$. The partial order that we imposed on $P$ is precisely the requirement for a maximal filter, so we've found our maximal filter.

    \textbf{3.} WTS a filter is maximal if and only if it is an ultrafilter.
    
    \pffwd Let $\calf$ is a maximal filter, that is, for all filter $\calf'$ we have $\calf' \subset \calf$.

    Suppose that $\calf$ is not an ultrafilter, i.e., there exists $A \subset X$ such that either $A, A^C \not \in \calf$ or $A, A^C \in \calf$.

    Case 1: If $A, A^C \not \in \calf$ then let us investigate $\calg = \calf \cup \{A\}$. Since $A^C \not \in \calf$, for every $B \in \calf, B \not \subset A^C \implies B \cap A \neq \emptyset$. Since $\calf$ already satisfies the fip, so does $\calf'$ too. $\calg$ is then a collection of subsets of $X$ satisfying the fip, and so as shown previously, can be extended to a filter $\calf'$. But $\calf' \not \subset \calf$ so $\calf$ is not a maximal filter, \contra

    Case 2: If $A, A^C \in \calf \implies \emptyset \in \calf, \contra$.

    Therefore $\calf$ is indeed an ultrafilter.

    \pfbwd Let $\calf$ be an ultrafilter. Suppose $\calf$ is not maximal, i.e., there exists $\calf'$ such that it strictly extends $\calf$. Then take $A = \calf' - \calf$. $A \not \in \calf \implies A^C \in \calf \subset \calf' \implies A^C \in \calf'$. But then $A, A^C \in \calf' \implies \emptyset \in \calf', \contra$
\end{solution}

\begin{problem} [Problem 2]
    Recall from class the following construction: Let $X$ be a set (which can be topologized as a discrete space). Let $UX$ be the set of ultrafilters on $X$. For every subset $A$ of $X$ let $[A]$ denote the set of ultrafilters on $X$ that contain $A$ (thus $[A]$ is a subset of $UX$).
    \begin{enumerate}
    \item Show that the sets of the form $[A]$ form a basis for a topology on $UX$.
    \item Show that the map $X \to UX$ that takes each point $x$ in $X$ to the principal ultrafilter $F_x$ in $UX$ is injective, and the image of $X$ under this map is discrete in the subspace topology on $UX$.
    \item Show that $UX$ is compact and Hausdorff.
    \item (Bonus) Show that UX is the Stone-Cech compactification of $X$ (Hint: if $f:X \to K$ is a continuous map to a compact Hausdorff space, and $F$ is an ultrafilter on $X$, then $f_*F$ is an ultrafilter on $K$, which necessarily converges to some unique point $y$. Show that the map $g: UX \to K$ sending each $F$ to $y$ in this way is a continuous extension of $f$ to $UX$.)
    \end{enumerate}
\end{problem}
\begin{solution}
    \textbf{(a)} $UX$ is the set of ultrafilters on $X$. WTS $\{[A] : A \subset X\}$ forms a basis for a topology on $UX$.

    \begin{itemize}
        \item For each $\calf \in UX$, take any $U \subset X$, then either $U \in \calf$ or $U^C \in \calf$. WLOG $U \in \calf \implies \calf \in [U]$.
        \item Take $\calf \in [A_1] \cap [A_2]$ then it's clear that $\calf \in [A_1 \cap A_2]$
    \end{itemize}
    So $\{[A] : A \subset X\}$ satisfies the conditions of a basis.

    \textbf{(b)} Take $x \neq y \in X$. Let the map be $F: x \mapsto \calf_x$. Then the principal filters $\calf_x, \calf_y$ are ultra filters, since for any $A \subset X$, either $x \in A$ or $x \in A^C$; similar with $y$. And $\calf_x \neq \calf_y$ since $\{x\} \in \calf_x, \{x\} \not \in \calf_y$. The map is therefore injective.

    Then consider $\calf_x \in F(X)$. Want to show that $\{\calf_x\}$ is open in $F(X)$ in the subspace topology on $UX$.

    Denote $V = [\{x\}]$ is a basis element of the topology of $UX$ and is therefore open in $UX$. Then clearly $\calf_x \in V$ since it is an ultrafilter and it is a super set of $\{x\}$.

    Furthermore, there doesn't exist any other $\calf_y$ with $y \neq x$ that is in $V$, since $\{x \} \not \subset \calf_y$. It follows that \begin{equation*}
    \{\calf_x\} = V \cap f(X)
    \end{equation*}
    and is therefore open in $f(X)$ in the subspace topology of $UX$.

    \textbf{(c) 1.} WTS $UX$ is compact. We want to show that every family of closed subsets having the finite intersection property has non-empty intersection. Every closed set of $UX$ has the form:
    \begin{align*}
        UX - \bigcup_i [A_i] &= [\bigcap_i A_i^C] = [B]
    \end{align*}

    Therefore, let our family of closed subsets be $\{[B_i]\}_{i \in I}$ with the finite intersection property. Then we have for all $i_1, \ldots, i_n$, \begin{equation*}
    \emptyset \neq \bigcap_{k \in [n]} [B_{i_k}] = [\bigcap_{k \in [n]} B_{i_k}]
    \end{equation*}
    so $\bigcap_{k \in [n]} B_{i_k} \neq \emptyset$. It follows that $\{B_i\}$ has the finite intersection property, and can therefore be extended to a filter, and then to an ultrafilter $\calu$, so that $B_i \in \calu \forall i \in I$. It then follows that \begin{equation*}
    \calu \in [B_i] \forall i \in I \implies \calu \in \bigcap_{i \in I} [B_i]
    \end{equation*}
    so $\bigcap_{i \in I} [B_i] \neq \emptyset$ as required.

    \textbf{2.} WTS $UX$ is Hausdorff. Take $\calf_1 \neq \calf_2 \in UX$. Since $\calf_1 \neq \calf_2$ and they are both ultrafilters so neither one can be contained in the other, there exists some $A \subset X$ such that $A \in \calf_1 - \calf_2$. $\calf_2$ is an ultrafilter, so $A^c \in \calf_2$. Then $[A]$ and $[A^C]$ are open neighborhoods of $\calf_1$ and $\calf_2$, and they are clearly disjoint. It follows that $UX$ is Hausdorff.
\end{solution}
\begin{problem} [Problem 3]
    For each positive integer $n$ give an example of a topological space that has at least $n$ different compactifications. (Bonus: Give an example of a topological space that has infinitely many different compactifications.)
\end{problem}
\begin{solution}
    Fix $n$. Then take $X = \bigcup_{k = 1}^{n} (2k-2, 2k-1)$ in the usual topology in $\bbr$. For example, for $n = 2, X = (0, 1) \cup (2, 3)$, i.e., in $X$, we have $n$ translated copies of the open unit interval $(0, 1)$.

    Then we can compactify $X$ into the box $[-4n, 4n]^2 \subset \bbr^2$ (overly big box), which is compact Hausdorff, by specify the embedding for each of the translated copies. For each translated copy, we can either use the 2-point compactification to map $(2k-2, 2k-1)$ to $[(2k-2, 0), (2k-1, 0)]$, or to use the 1-point compactification to map $(2k-2, 2k-1)$ to the circle $B((2k-2, 0), 1)$.

    We therefore have $2^n$ different compactifications for $X$.
\end{solution}

\begin{problem} [38.2]
    Show that the bounded continuous function $g: (0, 1) \to \bbr$ defined by $g(x) = \cos(\frac{1}{x})$ cannot be extended to the compactification of Example 3. Define an embedding $h: (0, 1) \to [0, 1]^3$ such that the functions $x, \sin(\frac{1}{x}), \cos(\frac{1}{x})$ are all extendable to the compactification induced by $h$.
\end{problem}
\begin{solution}
    \textbf{1.} The compactification of Example 3 is the compactification induced by \begin{align*}
    h: (0, 1) &\to [-1, 1]^2 \\
    x &\mapsto x \times \sin(1/x)
    \end{align*}

    $X = (0, 1), Z = [-1, 1]^2$.

    Let $Y$ be the compactification of $X$, and $H$ be the extension of $h$.

    Suppose for sake of contradiction, that $g$ can be extended to a continuous map $G$ on $Y$.
    
    Consider the sequence of points $\{h(\frac{1}{n\pi})\}_{n \in \bbn} \subset Z$.

    Since $G$ is continuous and $H$ is an embedding, on the one hand,
     \begin{equation*}
    GH^{-1}(h(\frac{1}{n \pi})) = G(\frac{1}{n\pi}) \cvgn G(0)
    \end{equation*}
    but on the other hand,
    \begin{equation}
    GH^{-1}(h(\frac{1}{n \pi})) = g(\frac{1}{n\pi}) = (-1)^n
    \end{equation}
    which does not converge.

    \textbf{2.} Define \begin{equation*}
    h: (0, 1) \to [0, 1]^3, x \mapsto x \times \sin(1/x) \times \cos(1/x)
    \end{equation*} 
\end{solution}  

\begin{problem} [38.4]
    Let $Y$ be an arbitrary compactification of $X$, let $\beta(X)$ be the Stone-Cech compactification. Show there is a continuous surjective closed map $g: \beta(X) \to Y$ that equals the identity on $X$.

    (This exercise makes prices what we mean by saying that $\beta(X)$ is the ``maximal'' compactification of $X$. It shows that every compactification of $X$ is equivalent to a quotient space of $\beta(X)$.)
\end{problem}
\begin{solution}
    $Y$ is a compactification of $X$, so view $X$ as a subspace of $Y$ with $\cl{X} = Y$ and we can inspect the continuous inclusion map \begin{equation*}
        \iota: X \to Y
    \end{equation*}

    The universal property of the Stone Cech compactification then implies that there exists a unique extension of $\iota$ to continuous $g: \beta(X) \to Y$.

    Note that $\beta(X)$ is compact, a closed subset of $\beta(X)$ is compact, so its image through $g$ is compact, and is therefore closed in Hausdorff $Y$. It follows that $g$ is a closed map.

    It is a surjective map, since $\cl{i(X)} = Y \implies \cl{g(\beta X)} = Y$, but $\beta(X)$ is compact so $g(\beta X)$ is compact in Hausdorff $Y$, so $g(\beta X)$ is closed in $Y$, hence $Y = \cl{(g(\beta X))} = g(\beta X)$.

    It is clear that it is exactly the identity on $X$, since $g$ is an extension of $\iota$.
\end{solution}

\begin{problem} [38.6]
    Let $X$ be completely regular. Show that $X$ is connected if and only if $\beta(X)$ is connected. [Hint: If $X = A \cup B$ is a separation of $X$, let $f(x) = 0$ for $x \in A$ and $f(x) = 1$ for $x \in B$]
\end{problem}
\begin{solution}
    \pffwd By hypothesis, $X$ is connected. Suppose for sake of contradiction that $\beta(X)$ is not connected, i.e., there exists separation $\beta(X) = A \sqcup B$ where $A, B$ are disjoint and clopen in $\beta(X)$. 

    Then consider $X \cap A, X \cap B$.

    Since $A, B$ are disjoint, they are also disjoint.

    Also, $X \cap A$ is also open in $X$, since the conditions of the compactification is that the subspace topology of $Y$ on $X$ aligns with its topology. Same with $X \cap B$.

    They are also nonempty, since $\cl{X} = \beta(X)$.
    
    It then follows that $X = (X \cap A) \sqcup (X \cap B)$ is a separation of connected $X, \contra$

    Therefore $\beta(X)$ is connected.

    \pfbwd By hypothesis, $\beta X$ is connected. Suppose for sake of contradiction that $X$ is not connected, i.e., there exists separation $X = A \sqcup B$ where $A, B$ are disjoint and clopen in $X$. 
    Then define $f: X \to \{0, 1\}, f(x) = \begin{cases}
    1 & \:\text{for}\: x \in A \\
    0 & \:\text{for}\: x \in B
    \end{cases}$
    then $\inv{f}(\{1\}) = A, \inv{f}(\{0\}) = B$ are both open, so $f$ is continuous.

    It follows that there is a continuous extension of $f$, called $g: \beta(X) \to \{0, 1\}$. But then we can write $\beta(X) = \inv{g}(\{1\}) \sqcup \inv{g}(\{0\})$ which are disjoint (clearly), and open, since $g$ is continuous. This is then a separation for connected $\beta(X), \contra$. It follows that $X$ is connected.
\end{solution}
\end{document}