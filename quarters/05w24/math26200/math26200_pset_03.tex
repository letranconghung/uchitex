\documentclass[a4paper, 12pt]{article}
%%%%%%%%%%%%%%%%%%%%%%%%%%%%%%%%%%%%%%%%%%%%%%%%%%%%%%%%%%%%%%%%%%%%%%%%%%%%%%%
%                                Basic Packages                               %
%%%%%%%%%%%%%%%%%%%%%%%%%%%%%%%%%%%%%%%%%%%%%%%%%%%%%%%%%%%%%%%%%%%%%%%%%%%%%%%

% Gives us multiple colors.
\usepackage[usenames,dvipsnames,pdftex]{xcolor}
% Lets us style link colors.
\usepackage{hyperref}
% Lets us import images and graphics.
\usepackage{graphicx}
% Lets us use figures in floating environments.
\usepackage{float}
% Lets us create multiple columns.
\usepackage{multicol}
% Gives us better math syntax.
\usepackage{amsmath,amsfonts,mathtools,amsthm,amssymb}
% Lets us strikethrough text.
\usepackage{cancel}
% Lets us edit the caption of a figure.
\usepackage{caption}
% Lets us import pdf directly in our tex code.
\usepackage{pdfpages}
% Lets us do algorithm stuff.
\usepackage[ruled,vlined,linesnumbered]{algorithm2e}
% Use a smiley face for our qed symbol.
\usepackage{tikzsymbols}
% \usepackage{fullpage} %%smaller margins
\usepackage[shortlabels]{enumitem}

\setlist[enumerate]{font={\bfseries}} % global settings, for all lists

\usepackage{setspace}
\usepackage[margin=1in, headsep=12pt]{geometry}
\usepackage{wrapfig}
\usepackage{listings}
\usepackage{parskip}

\definecolor{codegreen}{rgb}{0,0.6,0}
\definecolor{codegray}{rgb}{0.5,0.5,0.5}
\definecolor{codepurple}{rgb}{0.58,0,0.82}
\definecolor{backcolour}{rgb}{0.95,0.95,0.95}

\lstdefinestyle{mystyle}{
    backgroundcolor=\color{backcolour},   
    commentstyle=\color{codegreen},
    keywordstyle=\color{magenta},
    numberstyle=\tiny\color{codegray},
    stringstyle=\color{codepurple},
    basicstyle=\ttfamily\footnotesize,
    breakatwhitespace=false,         
    breaklines=true,                 
    captionpos=b,                    
    keepspaces=true,                 
    numbers=left,                    
    numbersep=5pt,                  
    showspaces=false,                
    showstringspaces=false,
    showtabs=false,                  
    tabsize=2,
    numbers=none
}

\lstset{style=mystyle}
\def\class{article}


%%%%%%%%%%%%%%%%%%%%%%%%%%%%%%%%%%%%%%%%%%%%%%%%%%%%%%%%%%%%%%%%%%%%%%%%%%%%%%%
%                                Basic Settings                               %
%%%%%%%%%%%%%%%%%%%%%%%%%%%%%%%%%%%%%%%%%%%%%%%%%%%%%%%%%%%%%%%%%%%%%%%%%%%%%%%

%%%%%%%%%%%%%
%  Symbols  %
%%%%%%%%%%%%%

\let\implies\Rightarrow
\let\impliedby\Leftarrow
\let\iff\Leftrightarrow
\let\epsilon\varepsilon
%%%%%%%%%%%%
%  Tables  %
%%%%%%%%%%%%

\setlength{\tabcolsep}{5pt}
\renewcommand\arraystretch{1.5}

%%%%%%%%%%%%%%
%  SI Unitx  %
%%%%%%%%%%%%%%

\usepackage{siunitx}
\sisetup{locale = FR}

%%%%%%%%%%
%  TikZ  %
%%%%%%%%%%

\usepackage[framemethod=TikZ]{mdframed}
\usepackage{tikz}
\usepackage{tikz-cd}
\usepackage{tikzsymbols}

\usetikzlibrary{intersections, angles, quotes, calc, positioning}
\usetikzlibrary{arrows.meta}

\tikzset{
    force/.style={thick, {Circle[length=2pt]}-stealth, shorten <=-1pt}
}

%%%%%%%%%%%%%%%
%  PGF Plots  %
%%%%%%%%%%%%%%%

\usepackage{pgfplots}
\pgfplotsset{width=10cm, compat=newest}

%%%%%%%%%%%%%%%%%%%%%%%
%  Center Title Page  %
%%%%%%%%%%%%%%%%%%%%%%%

\usepackage{titling}
\renewcommand\maketitlehooka{\null\mbox{}\vfill}
\renewcommand\maketitlehookd{\vfill\null}

%%%%%%%%%%%%%%%%%%%%%%%%%%%%%%%%%%%%%%%%%%%%%%%%%%%%%%%
%  Create a grey background in the middle of the PDF  %
%%%%%%%%%%%%%%%%%%%%%%%%%%%%%%%%%%%%%%%%%%%%%%%%%%%%%%%

\usepackage{eso-pic}
\newcommand\definegraybackground{
    \definecolor{reallylightgray}{HTML}{FAFAFA}
    \AddToShipoutPicture{
        \ifthenelse{\isodd{\thepage}}{
            \AtPageLowerLeft{
                \put(\LenToUnit{\dimexpr\paperwidth-222pt},0){
                    \color{reallylightgray}\rule{222pt}{297mm}
                }
            }
        }
        {
            \AtPageLowerLeft{
                \color{reallylightgray}\rule{222pt}{297mm}
            }
        }
    }
}

%%%%%%%%%%%%%%%%%%%%%%%%
%  Modify Links Color  %
%%%%%%%%%%%%%%%%%%%%%%%%

\hypersetup{
    % Enable highlighting links.
    colorlinks,
    % Change the color of links to blue.
    urlcolor=blue,
    % Change the color of citations to black.
    citecolor={black},
    % Change the color of url's to blue with some black.
    linkcolor={blue!80!black}
}

%%%%%%%%%%%%%%%%%%
% Fix WrapFigure %
%%%%%%%%%%%%%%%%%%

\newcommand{\wrapfill}{\par\ifnum\value{WF@wrappedlines}>0
        \parskip=0pt
        \addtocounter{WF@wrappedlines}{-1}%
        \null\vspace{\arabic{WF@wrappedlines}\baselineskip}%
        \WFclear
    \fi}

%%%%%%%%%%%%%%%%%
% Multi Columns %
%%%%%%%%%%%%%%%%%

\let\multicolmulticols\multicols
\let\endmulticolmulticols\endmulticols

\RenewDocumentEnvironment{multicols}{mO{}}
{%
    \ifnum#1=1
        #2%
    \else % More than 1 column
        \multicolmulticols{#1}[#2]
    \fi
}
{%
    \ifnum#1=1
    \else % More than 1 column
        \endmulticolmulticols
    \fi
}

\newlength{\thickarrayrulewidth}
\setlength{\thickarrayrulewidth}{5\arrayrulewidth}


%%%%%%%%%%%%%%%%%%%%%%%%%%%%%%%%%%%%%%%%%%%%%%%%%%%%%%%%%%%%%%%%%%%%%%%%%%%%%%%
%                           School Specific Commands                          %
%%%%%%%%%%%%%%%%%%%%%%%%%%%%%%%%%%%%%%%%%%%%%%%%%%%%%%%%%%%%%%%%%%%%%%%%%%%%%%%

%%%%%%%%%%%%%%%%%%%%%%%%%%%
%  Initiate New Counters  %
%%%%%%%%%%%%%%%%%%%%%%%%%%%

\newcounter{lecturecounter}

%%%%%%%%%%%%%%%%%%%%%%%%%%
%  Helpful New Commands  %
%%%%%%%%%%%%%%%%%%%%%%%%%%

\makeatletter

\newcommand\resetcounters{
    % Reset the counters for subsection, subsubsection and the definition
    % all the custom environments.
    \setcounter{subsection}{0}
    \setcounter{subsubsection}{0}
    \setcounter{definition0}{0}
    \setcounter{paragraph}{0}
    \setcounter{theorem}{0}
    \setcounter{claim}{0}
    \setcounter{corollary}{0}
    \setcounter{proposition}{0}
    \setcounter{lemma}{0}
    \setcounter{exercise}{0}
    \setcounter{problem}{0}
    
    \setcounter{subparagraph}{0}
    % \@ifclasswith\class{nocolor}{
    %     \setcounter{definition}{0}
    % }{}
}

%%%%%%%%%%%%%%%%%%%%%
%  Lecture Command  %
%%%%%%%%%%%%%%%%%%%%%

\usepackage{xifthen}

% EXAMPLE:
% 1. \lecture{Oct 17 2022 Mon (08:46:48)}{Lecture Title}
% 2. \lecture[4]{Oct 17 2022 Mon (08:46:48)}{Lecture Title}
% 3. \lecture{Oct 17 2022 Mon (08:46:48)}{}
% 4. \lecture[4]{Oct 17 2022 Mon (08:46:48)}{}
% Parameters:
% 1. (Optional) lecture number.
% 2. Time and date of lecture.
% 3. Lecture Title.
\def\@lecture{}
\def\@lectitle{}
\def\@leccount{}
\newcommand\lecture[3]{
    \newpage

    % Check if user passed the lecture title or not.
    \def\@leccount{Lecture #1}
    \ifthenelse{\isempty{#3}}{
        \def\@lecture{Lecture #1}
        \def\@lectitle{Lecture #1}
    }{
        \def\@lecture{Lecture #1: #3}
        \def\@lectitle{#3}
    }

    \setcounter{section}{#1}
    \renewcommand\thesubsection{#1.\arabic{subsection}}
    
    \phantomsection
    \addcontentsline{toc}{section}{\@lecture}
    \resetcounters

    \begin{mdframed}
        \begin{center}
            \Large \textbf{\@leccount}
            
            \vspace*{0.2cm}
            
            \large \@lectitle
            
            
            \vspace*{0.2cm}

            \normalsize #2
        \end{center}
    \end{mdframed}

}

%%%%%%%%%%%%%%%%%%%%
%  Import Figures  %
%%%%%%%%%%%%%%%%%%%%

\usepackage{import}
\pdfminorversion=7

% EXAMPLE:
% 1. \incfig{limit-graph}
% 2. \incfig[0.4]{limit-graph}
% Parameters:
% 1. The figure name. It should be located in figures/NAME.tex_pdf.
% 2. (Optional) The width of the figure. Example: 0.5, 0.35.
\newcommand\incfig[2][1]{%
    \def\svgwidth{#1\columnwidth}
    \import{./figures/}{#2.pdf_tex}
}

\begingroup\expandafter\expandafter\expandafter\endgroup
\expandafter\ifx\csname pdfsuppresswarningpagegroup\endcsname\relax
\else
    \pdfsuppresswarningpagegroup=1\relax
\fi

%%%%%%%%%%%%%%%%%
% Fancy Headers %
%%%%%%%%%%%%%%%%%

\usepackage{fancyhdr}

% Force a new page.
\newcommand\forcenewpage{\clearpage\mbox{~}\clearpage\newpage}

% This command makes it easier to manage my headers and footers.
\newcommand\createintro{
    % Use roman page numbers (e.g. i, v, vi, x, ...)
    \pagenumbering{roman}

    % Display the page style.
    \maketitle
    % Make the title pagestyle empty, meaning no fancy headers and footers.
    \thispagestyle{empty}
    % Create a newpage.
    \newpage

    % Input the intro.tex page if it exists.
    \IfFileExists{intro.tex}{ % If the intro.tex file exists.
        % Input the intro.tex file.
        \textbf{Course}: MATH 16300: Honors Calculus III

\textbf{Section}: 43

\textbf{Professor}: Minjae Park

\textbf{At}: The University of Chicago

\textbf{Quarter}: Spring 2023

\textbf{Course materials}: Calculus by Spivak (4th Edition), Calculus On Manifolds by Spivak

\vspace{1cm}
\textbf{Disclaimer}: This document will inevitably contain some mistakes, both simple typos and serious logical and mathematical errors. Take what you read with a grain of salt as it is made by an undergraduate student going through the learning process himself. If you do find any error, I would really appreciate it if you can let me know by email at \href{mailto:conghungletran@gmail.com}{conghungletran@gmail.com}.

        % Make the pagestyle fancy for the intro.tex page.
        \pagestyle{fancy}

        % Remove the line for the header.
        \renewcommand\headrulewidth{0pt}

        % Remove all header stuff.
        \fancyhead{}

        % Add stuff for the footer in the center.
        % \fancyfoot[C]{
        %   \textit{For more notes like this, visit
        %   \href{\linktootherpages}{\shortlinkname}}. \\
        %   \vspace{0.1cm}
        %   \hrule
        %   \vspace{0.1cm}
        %   \@author, \\
        %   \term: \academicyear, \\
        %   Last Update: \@date, \\
        %   \faculty
        % }

        \newpage
    }{ % If the intro.tex file doesn't exist.
        % Force a \newpageage.
        % \forcenewpage
        \newpage
    }

    % Remove the center stuff we did above, and replace it with just the page
    % number, which is still in roman numerals.
    \fancyfoot[C]{\thepage}
    % Add the table of contents.
    \tableofcontents
    % Force a new page.
    \newpage

    % Move the page numberings back to arabic, from roman numerals.
    \pagenumbering{arabic}
    % Set the page number to 1.
    \setcounter{page}{1}

    % Add the header line back.
    \renewcommand\headrulewidth{0.4pt}
    % In the top right, add the lecture title.
    \fancyhead[R]{\footnotesize \@lecture}
    % In the top left, add the author name.
    \fancyhead[L]{\footnotesize \@author}
    % In the bottom center, add the page.
    \fancyfoot[C]{\thepage}
    % Add a nice gray background in the middle of all the upcoming pages.
    % \definegraybackground
}

\makeatother


%%%%%%%%%%%%%%%%%%%%%%%%%%%%%%%%%%%%%%%%%%%%%%%%%%%%%%%%%%%%%%%%%%%%%%%%%%%%%%%
%                               Custom Commands                               %
%%%%%%%%%%%%%%%%%%%%%%%%%%%%%%%%%%%%%%%%%%%%%%%%%%%%%%%%%%%%%%%%%%%%%%%%%%%%%%%

%%%%%%%%%%%%
%  Circle  %
%%%%%%%%%%%%

\newcommand*\circled[1]{\tikz[baseline= (char.base)]{
        \node[shape=circle,draw,inner sep=1pt] (char) {#1};}
}

%%%%%%%%%%%%%%%%%%%
%  Todo Commands  %
%%%%%%%%%%%%%%%%%%%

% \usepackage{xargs}
% \usepackage[colorinlistoftodos]{todonotes}

% \makeatletter

% \@ifclasswith\class{working}{
%     \newcommandx\unsure[2][1=]{\todo[linecolor=red,backgroundcolor=red!25,bordercolor=red,#1]{#2}}
%     \newcommandx\change[2][1=]{\todo[linecolor=blue,backgroundcolor=blue!25,bordercolor=blue,#1]{#2}}
%     \newcommandx\info[2][1=]{\todo[linecolor=OliveGreen,backgroundcolor=OliveGreen!25,bordercolor=OliveGreen,#1]{#2}}
%     \newcommandx\improvement[2][1=]{\todo[linecolor=Plum,backgroundcolor=Plum!25,bordercolor=Plum,#1]{#2}}

%     \newcommand\listnotes{
%         \newpage
%         \listoftodos[Notes]
%     }
% }{
%     \newcommandx\unsure[2][1=]{}
%     \newcommandx\change[2][1=]{}
%     \newcommandx\info[2][1=]{}
%     \newcommandx\improvement[2][1=]{}

%     \newcommand\listnotes{}
% }

% \makeatother

%%%%%%%%%%%%%
%  Correct  %
%%%%%%%%%%%%%

% EXAMPLE:
% 1. \correct{INCORRECT}{CORRECT}
% Parameters:
% 1. The incorrect statement.
% 2. The correct statement.
\definecolor{correct}{HTML}{009900}
\newcommand\correct[2]{{\color{red}{#1 }}\ensuremath{\to}{\color{correct}{ #2}}}


%%%%%%%%%%%%%%%%%%%%%%%%%%%%%%%%%%%%%%%%%%%%%%%%%%%%%%%%%%%%%%%%%%%%%%%%%%%%%%%
%                                 Environments                                %
%%%%%%%%%%%%%%%%%%%%%%%%%%%%%%%%%%%%%%%%%%%%%%%%%%%%%%%%%%%%%%%%%%%%%%%%%%%%%%%

\usepackage{varwidth}
\usepackage{thmtools}
\usepackage[most,many,breakable]{tcolorbox}

\tcbuselibrary{theorems,skins,hooks}
\usetikzlibrary{arrows,calc,shadows.blur}

%%%%%%%%%%%%%%%%%%%
%  Define Colors  %
%%%%%%%%%%%%%%%%%%%

% color prototype
% \definecolor{color}{RGB}{45, 111, 177}

% ESSENTIALS: 
\definecolor{myred}{HTML}{c74540}
\definecolor{myblue}{HTML}{072b85}
\definecolor{mygreen}{HTML}{388c46}
\definecolor{myblack}{HTML}{000000}

\colorlet{definition_color}{myred}

\colorlet{theorem_color}{myblue}
\colorlet{lemma_color}{myblue}
\colorlet{prop_color}{myblue}
\colorlet{corollary_color}{myblue}
\colorlet{claim_color}{myblue}

\colorlet{proof_color}{myblack}
\colorlet{example_color}{myblack}
\colorlet{exercise_color}{myblack}

% MISCS: 
%%%%%%%%%%%%%%%%%%%%%%%%%%%%%%%%%%%%%%%%%%%%%%%%%%%%%%%%%
%  Create Environments Styles Based on Given Parameter  %
%%%%%%%%%%%%%%%%%%%%%%%%%%%%%%%%%%%%%%%%%%%%%%%%%%%%%%%%%

% \mdfsetup{skipabove=1em,skipbelow=0em}

%%%%%%%%%%%%%%%%%%%%%%
%  Helpful Commands  %
%%%%%%%%%%%%%%%%%%%%%%

% EXAMPLE:
% 1. \createnewtheoremstyle{thmdefinitionbox}{}{}
% 2. \createnewtheoremstyle{thmtheorembox}{}{}
% 3. \createnewtheoremstyle{thmproofbox}{qed=\qedsymbol}{
%       rightline=false, topline=false, bottomline=false
%    }
% Parameters:
% 1. Theorem name.
% 2. Any extra parameters to pass directly to declaretheoremstyle.
% 3. Any extra parameters to pass directly to mdframed.
\newcommand\createnewtheoremstyle[3]{
    \declaretheoremstyle[
        headfont=\bfseries\sffamily, bodyfont=\normalfont, #2,
        mdframed={
                #3,
            },
    ]{#1}
}

% EXAMPLE:
% 1. \createnewcoloredtheoremstyle{thmdefinitionbox}{definition}{}{}
% 2. \createnewcoloredtheoremstyle{thmexamplebox}{example}{}{
%       rightline=true, leftline=true, topline=true, bottomline=true
%     }
% 3. \createnewcoloredtheoremstyle{thmproofbox}{proof}{qed=\qedsymbol}{backgroundcolor=white}
% Parameters:
% 1. Theorem name.
% 2. Color of theorem.
% 3. Any extra parameters to pass directly to declaretheoremstyle.
% 4. Any extra parameters to pass directly to mdframed.

% change backgroundcolor to #2!5 if user wants a colored backdrop to theorem environments. It's a cool color theme, but there's too much going on in the page.
\newcommand\createnewcoloredtheoremstyle[4]{
    \declaretheoremstyle[
        headfont=\bfseries\sffamily\color{#2},
        bodyfont=\normalfont,
        headpunct=,
        headformat = \NAME~\NUMBER\NOTE \hfill\smallskip\linebreak,
        #3,
        mdframed={
                outerlinewidth=0.75pt,
                rightline=false,
                leftline=false,
                topline=false,
                bottomline=false,
                backgroundcolor=white,
                skipabove = 5pt,
                skipbelow = 0pt,
                linecolor=#2,
                innertopmargin = 0pt,
                innerbottommargin = 0pt,
                innerrightmargin = 4pt,
                innerleftmargin= 6pt,
                leftmargin = -6pt,
                #4,
            },
    ]{#1}
}



%%%%%%%%%%%%%%%%%%%%%%%%%%%%%%%%%%%
%  Create the Environment Styles  %
%%%%%%%%%%%%%%%%%%%%%%%%%%%%%%%%%%%

\makeatletter
\@ifclasswith\class{nocolor}{
    % Environments without color.

    % ESSENTIALS:
    \createnewtheoremstyle{thmdefinitionbox}{}{}
    \createnewtheoremstyle{thmtheorembox}{}{}
    \createnewtheoremstyle{thmproofbox}{qed=\qedsymbol}{}
    \createnewtheoremstyle{thmcorollarybox}{}{}
    \createnewtheoremstyle{thmlemmabox}{}{}
    \createnewtheoremstyle{thmclaimbox}{}{}
    \createnewtheoremstyle{thmexamplebox}{}{}

    % MISCS: 
    \createnewtheoremstyle{thmpropbox}{}{}
    \createnewtheoremstyle{thmexercisebox}{}{}
    \createnewtheoremstyle{thmexplanationbox}{}{}
    \createnewtheoremstyle{thmremarkbox}{}{}
    
    % STYLIZED MORE BELOW
    \createnewtheoremstyle{thmquestionbox}{}{}
    \createnewtheoremstyle{thmsolutionbox}{qed=\qedsymbol}{}
}{
    % Environments with color.

    % ESSENTIALS: definition, theorem, proof, corollary, lemma, claim, example
    \createnewcoloredtheoremstyle{thmdefinitionbox}{definition_color}{}{leftline=false}
    \createnewcoloredtheoremstyle{thmtheorembox}{theorem_color}{}{leftline=false}
    \createnewcoloredtheoremstyle{thmproofbox}{proof_color}{qed=\qedsymbol}{}
    \createnewcoloredtheoremstyle{thmcorollarybox}{corollary_color}{}{leftline=false}
    \createnewcoloredtheoremstyle{thmlemmabox}{lemma_color}{}{leftline=false}
    \createnewcoloredtheoremstyle{thmpropbox}{prop_color}{}{leftline=false}
    \createnewcoloredtheoremstyle{thmclaimbox}{claim_color}{}{leftline=false}
    \createnewcoloredtheoremstyle{thmexamplebox}{example_color}{}{}
    \createnewcoloredtheoremstyle{thmexplanationbox}{example_color}{qed=\qedsymbol}{}
    \createnewcoloredtheoremstyle{thmremarkbox}{theorem_color}{}{}

    \createnewcoloredtheoremstyle{thmmiscbox}{black}{}{}

    \createnewcoloredtheoremstyle{thmexercisebox}{exercise_color}{}{}
    \createnewcoloredtheoremstyle{thmproblembox}{theorem_color}{}{leftline=false}
    \createnewcoloredtheoremstyle{thmsolutionbox}{mygreen}{qed=\qedsymbol}{}
}
\makeatother

%%%%%%%%%%%%%%%%%%%%%%%%%%%%%
%  Create the Environments  %
%%%%%%%%%%%%%%%%%%%%%%%%%%%%%
\declaretheorem[numberwithin=section, style=thmdefinitionbox,     name=Definition]{definition}
\declaretheorem[numberwithin=section, style=thmtheorembox,     name=Theorem]{theorem}
\declaretheorem[numbered=no,          style=thmexamplebox,     name=Example]{example}
\declaretheorem[numberwithin=section, style=thmtheorembox,       name=Claim]{claim}
\declaretheorem[numberwithin=section, style=thmcorollarybox,   name=Corollary]{corollary}
\declaretheorem[numberwithin=section, style=thmpropbox,        name=Proposition]{proposition}
\declaretheorem[numberwithin=section, style=thmlemmabox,       name=Lemma]{lemma}
\declaretheorem[numberwithin=section, style=thmexercisebox,    name=Exercise]{exercise}
\declaretheorem[numbered=no,          style=thmproofbox,       name=Proof]{proof0}
\declaretheorem[numbered=no,          style=thmexplanationbox, name=Explanation]{explanation}
\declaretheorem[numbered=no,          style=thmsolutionbox,    name=Solution]{solution}
\declaretheorem[numberwithin=section,          style=thmproblembox,     name=Problem]{problem}
\declaretheorem[numbered=no,          style=thmmiscbox,    name=Intuition]{intuition}
\declaretheorem[numbered=no,          style=thmmiscbox,    name=Goal]{goal}
\declaretheorem[numbered=no,          style=thmmiscbox,    name=Recall]{recall}
\declaretheorem[numbered=no,          style=thmmiscbox,    name=Motivation]{motivation}
\declaretheorem[numbered=no,          style=thmmiscbox,    name=Remark]{remark}
\declaretheorem[numbered=no,          style=thmmiscbox,    name=Observe]{observe}
\declaretheorem[numbered=no,          style=thmmiscbox,    name=Question]{question}


%%%%%%%%%%%%%%%%%%%%%%%%%%%%
%  Edit Proof Environment  %
%%%%%%%%%%%%%%%%%%%%%%%%%%%%

\renewenvironment{proof}[2][\proofname]{
    % \vspace{-12pt}
    \begin{proof0} [#2]
        }{\end{proof0}}

\theoremstyle{definition}

\newtheorem*{notation}{Notation}
\newtheorem*{previouslyseen}{As previously seen}
\newtheorem*{property}{Property}
% \newtheorem*{intuition}{Intuition}
% \newtheorem*{goal}{Goal}
% \newtheorem*{recall}{Recall}
% \newtheorem*{motivation}{Motivation}
% \newtheorem*{remark}{Remark}
% \newtheorem*{observe}{Observe}

\author{Hung C. Le Tran}


%%%% MATH SHORTHANDS %%%%
%% blackboard bold math capitals
\DeclareMathOperator*{\esssup}{ess\,sup}
\DeclareMathOperator*{\Hom}{Hom}
\newcommand{\bbf}{\mathbb{F}}
\newcommand{\bbn}{\mathbb{N}}
\newcommand{\bbq}{\mathbb{Q}}
\newcommand{\bbr}{\mathbb{R}}
\newcommand{\bbz}{\mathbb{Z}}
\newcommand{\bbc}{\mathbb{C}}
\newcommand{\bbk}{\mathbb{K}}
\newcommand{\bbm}{\mathbb{M}}
\newcommand{\bbp}{\mathbb{P}}
\newcommand{\bbe}{\mathbb{E}}

\newcommand{\bfw}{\mathbf{w}}
\newcommand{\bfx}{\mathbf{x}}
\newcommand{\bfX}{\mathbf{X}}
\newcommand{\bfy}{\mathbf{y}}
\newcommand{\bfyhat}{\mathbf{\hat{y}}}

\newcommand{\calb}{\mathcal{B}}
\newcommand{\calf}{\mathcal{F}}
\newcommand{\calt}{\mathcal{T}}
\newcommand{\call}{\mathcal{L}}
\renewcommand{\phi}{\varphi}

% Universal Math Shortcuts
\newcommand{\st}{\hspace*{2pt}\text{s.t.}\hspace*{2pt}}
\newcommand{\pffwd}{\hspace*{2pt}\fbox{\(\Rightarrow\)}\hspace*{10pt}}
\newcommand{\pfbwd}{\hspace*{2pt}\fbox{\(\Leftarrow\)}\hspace*{10pt}}
\newcommand{\contra}{\ensuremath{\Rightarrow\Leftarrow}}
\newcommand{\cvgn}{\xrightarrow{n \to \infty}}
\newcommand{\cvgj}{\xrightarrow{j \to \infty}}

\newcommand{\im}{\mathrm{im}}
\newcommand{\innerproduct}[2]{\langle #1, #2 \rangle}
\newcommand*{\conj}[1]{\overline{#1}}

% https://tex.stackexchange.com/questions/438612/space-between-exists-and-forall
% https://tex.stackexchange.com/questions/22798/nice-looking-empty-set
\let\oldforall\forall
\renewcommand{\forall}{\;\oldforall\; }
\let\oldexist\exists
\renewcommand{\exists}{\;\oldexist\; }
\newcommand\existu{\;\oldexist!\: }
\let\oldemptyset\emptyset
\let\emptyset\varnothing


\renewcommand{\_}[1]{\underline{#1}}
\DeclarePairedDelimiter{\abs}{\lvert}{\rvert}
\DeclarePairedDelimiter{\norm}{\lVert}{\rVert}
\DeclarePairedDelimiter\ceil{\lceil}{\rceil}
\DeclarePairedDelimiter\floor{\lfloor}{\rfloor}
\setlength\parindent{0pt}
\setlength{\headheight}{12.0pt}
\addtolength{\topmargin}{-12.0pt}


% Default skipping, change if you want more spacing
% \thinmuskip=3mu
% \medmuskip=4mu plus 2mu minus 4mu
% \thickmuskip=5mu plus 5mu

% \DeclareMathOperator{\ext}{ext}
% \DeclareMathOperator{\bridge}{bridge}
\title{MATH 26200: Point-Set Topology \\ \large Problem Set 3}
\date{20 Jan 2024}
\author{Hung Le Tran}
\begin{document}
\maketitle
\setcounter{section}{3}
\textbf{Textbook:} Munkres, \textit{Topology}
\begin{problem} [20.1 \redtext{done}]
    \begin{enumerate}
    \item []
    \item In $\bbr^n$, define \begin{equation*}
    d'(x, y) = \abs{x_1 - y_1} + \cdots + \abs{x_n - y_n}
    \end{equation*}
    Show that $d'$ is a metric that induces the usual topology on $\bbr^n$. Sketch the basis elements under $d'$ when $n = 2$.
    \item More generally, given $p \geq 1$, define \begin{equation*}
    d'(x, y) = \left[\sum_{i=1}^{n} \abs{x_i - y_i}^p\right]^{1/p}
    \end{equation*}
    for $x, y \in \bbr^n$. Assume that $d'$ is a metric. Show that it induces the usual topology on $\bbr$.
    \end{enumerate}
\end{problem}
\begin{solution}
    \textbf{(a)}
    Recall that the usual topology on $\bbr^n$ is the topology induced by the metric \begin{equation*}
    d(x, y) = \left[\sum_{i=1}^{n} (x_i - y_i)^2\right]^{1/2}
    \end{equation*}
    
    We prove that $d'$ and $d$ are comparable, i.e., that \begin{align*}
    d'(x, y)^2 &= \left(\sum_{i=1}^{n} \abs{x_i - y_i}\right)^2 \\
                &\geq \sum_{i=1}^{n} (x_i - y_i)^2 = d(x, y)^2 \\
    \implies d'(x, y) & \geq d(x, y) \\
\end{align*}
and 
\begin{align*}
    d'(x, y)^2 &= \left(\sum_{i=1}^{n} \abs{x_i - y_i}\right)^2 \\
        &\leq \left(\sum_{i=1}^{n} d(x, y)\right)^2 \\
        &\leq n^2d(x, y)^2 \\
    \implies d'(x, y) &\leq n d(x, y)
\end{align*}
Therefore we can exhibit basis elements for the $d'$ metric topology and the usual topology:
\begin{equation*}
B_{d, \epsilon}(x) \subset B_{d', \epsilon}(x) \subset B_{d, \frac{\epsilon}{n}}(x)
\end{equation*}
thus $d'$ induces the same topology.

Basis elements under $d'$ when $n=2$ are squares rotated by $\pi/4$.

\textbf{(b)} In the general case, we employ the same method. Call the $d'$ in part (a) $d_1$. Then
\begin{align*}
d_1(x, y)^p &=  \left(\sum_{i=1}^{n} \abs{x_i - y_i}\right)^p \\
&\geq d(x, y)^p \\
\implies d_1(x, y) &\geq d(x, y) 
\end{align*}
and 
\begin{align*}
    d_1(x, y)^p  &= \left(\sum_{i=1}^{n} \abs{x_i - y_i}\right)^p  \\
    &\leq \left(\sum_{i=1}^{n} d(x,y) \right)^p \\
    \implies d_1(x, y) &\leq n d(x, y)
\end{align*}
then by the same argument, $d'$ induces the same metric topology as $d_1$, which is the usual topology on $\bbr^n$.
\end{solution}

\begin{problem} [20.4 \redtext{done}]
    Consider the product, uniform and box topologies on $\bbr^\omega$.
    \begin{enumerate}
    \item In which topologies are the following functions from $\bbr$ to $\bbr^\omega$ continuous?
        \begin{align*}
            f(t) &= (t, 2t, 3t, \ldots) \\
            g(t) &= (t, t, t, \ldots) \\
            h(t) &= (t, \frac{1}{2} t, \frac{1}{3}t, \ldots)
        \end{align*}
    \item In which topologies do the following sequences converge?
        \begin{align*}
            w_1 &= (1, 1, 1, 1,\ldots) &x_1 &= (1, 1, 1, 1,\ldots)\\
            w_2 &= (0, 2, 2, 2,\ldots) &x_2 &= (0, \frac{1}{2}, \frac{1}{2}, \frac{1}{2}, \ldots) \\
            w_3 &= (0, 0, 3, 3,\ldots)  &x_3 &= (0, 0, \frac{1}{3}, \frac{1}{3}, \ldots)\\
            &\ldots \\
            y_1 &= (1, 0, 0, 0, \ldots) & z_1 &= (1, 1, 0, 0, \ldots)\\
            y_2 &= (\frac{1}{2}, \frac{1}{2}, 0, 0, \ldots) &  z_2 &= (\frac{1}{2}, \frac{1}{2}, 0, 0, \ldots) \\
            y_3 &= (\frac{1}{3}, \frac{1}{3}, \frac{1}{3}, 0, \ldots) & z_3 &= (\frac{1}{3}, \frac{1}{3}, 0, 0, \ldots)\\
        \end{align*}
    \end{enumerate}
\end{problem}
\begin{solution}
    \textbf{(a) 1.} Product topology. We have, by Theorem 20.5, that the metric \begin{equation*}
    D(\bfx, \bfy) = \sup \left\{\frac{\bar{d}(x_i, y_i)}{i}\right\}
    \end{equation*}
    where $\bar{d}(x_i, y_i) = \min\{\abs{x_i - y_i}, 1\} $ induces the product topology on $\bbr^\omega$. 
    
    \begin{itemize}
        \item $f$ is continuous: For all $\epsilon$, we can choose $\delta = \epsilon$, then $\abs{u - v} < \delta$ implies: 
        \begin{equation*}
            D(f(u), f(v)) \leq \sup\left\{\frac{\abs{iu - iv}}{i}\right\} = \sup\{\abs{u - v}\} < \delta = \epsilon
        \end{equation*}

        \item $g$ is continuous. For all $\epsilon$, we can choose $\delta = \epsilon$, then $\abs{u - v} < \delta$ implies: 
        \begin{equation*}
            D(g(u), g(v)) \leq \sup\left\{\frac{\abs{u - v}}{i}\right\} = \abs{u - v} < \delta = \epsilon
        \end{equation*}
        
        \item $h$ is continuous. For all $\epsilon$, we can choose $\delta = \epsilon$, then $\abs{u - v} < \delta$ implies: 
        \begin{equation*}
            D(g(u), g(v)) \leq \sup\left\{\frac{\abs{u - v}}{i^2}\right\} = \abs{u - v} < \delta = \epsilon
        \end{equation*}
    \end{itemize}    

\textbf{2.} Uniform topology. The uniform topology is the one induced by the metric \begin{equation*}
\bar{\rho}(\bfx, \bfy) = \sup\{\bar{d}(x_i, y_i)\}
\end{equation*}
\begin{itemize}
\item $f$ is not continuous at 1. $f(1) = (1, 2, \ldots)$. Let $\epsilon = 0.5$. Then we want to show that for all $\delta > 0$, there exists some $u \in (1 - \delta, 1 + \delta)$ such that $\bar{\rho}(f(u), f(1)) > \epsilon = 0.5$.

Indeed, for any $\delta$, pick $ u = 1 + \frac{\delta}{2}$. Then \begin{equation*}
f(u) = (1 + \frac{\delta}{2}, 2 + \delta, \ldots)
\end{equation*}
and there exists $N$ such that $N\delta/2 > 1$. Then \begin{equation*}
\bar{\rho}(f(u), f(1)) = \sup\{\bar{d}(f(u)_i, f(1)_i)\} \geq \bar{d}(f(u)_N, f(1)_N) = \bar{d}(N + N\delta/2, N) = 1 > 0.5
\end{equation*}
as required.

\item $g$ is continuous. For all $\epsilon$, we can choose $\delta = \epsilon$, then $\abs{u - v} < \delta$ implies \begin{equation*}
\bar{\rho}(g(u), g(v)) = \sup\{\bar{d} (g(u)_i, g(v)_i)\}  = \sup\{\bar{d} (u, v)\} \leq \abs{u - v} < \delta = \epsilon 
\end{equation*}

 \item $h$ is continuous. For all $\epsilon$, we can choose $\delta = \epsilon$, then $\abs{u - v} < \delta$ implies \begin{equation*}
    \bar{\rho}(h(u), h(v)) = \sup\{\bar{d} (h(u)_i, h(v)_i)\}  = \bar{d} (u, v) \leq \abs{u - v} < \delta = \epsilon 
    \end{equation*}
\end{itemize}

\textbf{3.} Box topology.
\begin{itemize}
\item The uniform topology is coarser than the box topology. $f$ is not continuous in the uniform topology, so it is also not continuous in the box topology.
\item $g$ is not continuous in the box topology. Because \begin{equation*}
g^{Pre}\left((-1, 1) \times \left(-\frac{1}{2}, \frac{1}{2}\right) \times \left(-\frac{1}{3}, \frac{1}{3}\right) \times \ldots \right) = \{0\}
\end{equation*}
is not open in $\bbr$.
\item $h$ is not continuous in the box topology. Because \begin{equation*}
h^{Pre}\left((-1, 1) \times \left(-\frac{1}{2^2}, \frac{1}{2^2}\right) \times \left(-\frac{1}{3^2}, \frac{1}{3^2}\right) \times \ldots \right) = \{0\}
\end{equation*}
is not open in $\bbr$.
\end{itemize}
\textbf{(b)}
\textbf{1.} Product topology.
\begin{itemize}
\item $w_n \cvgn (0, 0, \ldots)$, since \begin{equation*}
D(w_n, (0, 0, \ldots)) = \sup\left\{\frac{\bar{d}(n, 0)}{i} : i \geq n\right\} = \frac{1}{n} \cvgn 0
\end{equation*}
\item $x_n \cvgn (0, 0, \ldots)$, since \begin{equation*}
    D(x_n, (0, 0, \ldots)) = \sup\left\{\frac{\bar{d}(\frac{1}{n}, 0)}{i} : i \geq n\right\} = \frac{1}{n^2} \cvgn 0
    \end{equation*}

\item $y_n \cvgn (0, 0, \ldots)$, since \begin{equation*}
    D(y_n, (0, 0, \ldots)) = \sup\left\{\frac{\bar{d}(\frac{1}{n}, 0)}{i} : i \leq n\right\} = \frac{1}{n} \cvgn 0
    \end{equation*}
\item $z_n \cvgn (0, 0, \ldots)$,  since \begin{equation*}
    D(z_n, (0, 0, \ldots)) = \sup\left\{\frac{\bar{d}(\frac{1}{n}, 0)}{i} : i \leq 2\right\} = \frac{1}{n} \cvgn 0
\end{equation*}
\end{itemize}

\textbf{2.} Uniform topology.
\begin{itemize}
\item $w_n$ doesn't converge. Suppose that the sequence does: $w_n \cvgn a = (a_1, a_2, \ldots)$. Then for all $N$, we trivially have that \begin{equation*}
\abs{a_{N+2} - N} \geq 1 \:\text{or}\: \abs{a_{N+2} - (N+2)} \geq 1
\end{equation*}
which implies \begin{equation*}
\bar{\rho}(a, w_N) \geq 1 \:\text{or}\: \bar{\rho}(a, w_{N+2}) \geq 1
\end{equation*}
so $\bar{\rho}(a, w_n)$ does not go to 0 as $n \to \infty$.

\item $x_n \cvgn (0, 0, \ldots)$, since \begin{equation*}
\bar{\rho}(x_n, (0, 0, \ldots)) = \sup \left\{\bar{d}(\frac{1}{n}, 0): i\geq n\right\} = \frac{1}{n} \cvgn 0
\end{equation*}

\item $y_n \cvgn (0, 0, \ldots)$, since \begin{equation*}
\bar{\rho}(y_n, (0, 0, \ldots)) = \sup \left\{\bar{d}(\frac{1}{n}, 0): i \leq n\right\} = \frac{1}{n} \cvgn 0
\end{equation*}

\item $y_n \cvgn (0, 0, \ldots)$, since \begin{equation*}
    \bar{\rho}(z_n, (0, 0, \ldots)) = \sup \left\{\bar{d}(\frac{1}{n}, 0): i \leq 2\right\} = \frac{1}{n} \cvgn 0
    \end{equation*}
\end{itemize}

\textbf{3.} Box topology.
\begin{itemize}
\item $w_n$ doesn't converge in the uniform topology, and the uniform topology is coarser than the box topology, so it doesn't converge in the box topology either.
\item $x_n$ doesn't converge. Suppose it does to $a$. Then $a_1 = 0$, because otherwise, WLOG, $a_1 > 0$, the neighborhood $(a_1/2, 3a_1/2) \times \bbr \times \ldots$ only has 1 term, $x_1$. Similarly, $a_2 = a_3 = \ldots = 0$.

But there exists neighborhood of $a$:
\begin{equation*}
\left(-\frac{1}{2}, \frac{1}{2}\right) \times\left(-\frac{1}{4}, \frac{1}{4}\right) \times \ldots \times \left(-\frac{1}{2n}, \frac{1}{2n}\right) \times \ldots
\end{equation*}
that does not contain any $x_n$.

\item $y_n$ doesn't converge. Using the same reasoning as above, if $y_n$ does converge, then it must converge to $a = (0, 0, \ldots)$. However, the same neighborhood as above also does not contain any $y_n$.

\item $z_n \cvgn (0, 0, \ldots)$. Convergence in coordinates after the 3rd is clear, while $1/n \cvgn 0$ takes care of the first 2.
\end{itemize}
\end{solution}

\begin{problem} [21.3 \redtext{done}]
    Let $X_n$ be a metric space with metric $d_n$ for $n \in \bbz_+$. 
    \begin{enumerate}
    \item Show that \begin{equation*}
    \rho(x, y) = \max\{d_1(x_1, y_1), \ldots, d_n(x_n, y_n)\}
    \end{equation*}
    is a metric for the product space $X_1 \times \ldots \times X_n$.

    \item Let $\bar{d_i} = \min\{d_i, 1\}$. Show that \begin{equation*}
    D(x, y) = \sup\left\{\frac{\bar{d_i}(x_i, y_i)}{i}\right\}
    \end{equation*}
    is a metric for the product space $\prod X_i$
    \end{enumerate}
\end{problem}
\begin{solution}
    \textbf{(a)} We check the conditions for it to be a metric:
    \begin{enumerate} [(1)]
        \item $\rho(x, y) = \max\{d_1(x_1, y_1), \ldots, d_n(x_n, y_n)\} \geq d_1(x_1, y_1) \geq 0$.
        
        And $0 = \rho(x, y) = \max\{d_1(x_1, y_1), \ldots, d_n(x_n, y_n)\} \implies 0 = d_1(x_1, y_1) = d_2(x_2, y_2) = \ldots = d_n(x_n, y_n) \implies x = y$.

        \item $d_1, d_2, \ldots$ are symmetric so $\rho$ is symmetric.
        \item $\rho(x, y) + \rho(y, z) \geq d_i(x_i, y_i) + d_i(y_i, z_i) \geq d_i (x_i, z_i) \forall i \implies \rho(x, y) + \rho(y, z) \geq \max\{d_i(x_i, y_i)\} = \rho(x, z)$
    \end{enumerate}

    \textbf{(b)} To show that $D$ induces the same topology as the product topology on $\prod X_i$, we show that it is both finer and coarser.

    \textbf{1.} WTS the $D$-metric topology is finer than the product topology.

    Take a typical basis element $B$ of the product topology: the Cartesian product of $X_i$ at all index $i$ except for $U_{i_1}, U_{i_2}, \ldots, U_{i_N}$ at indices $i_1, i_2, \ldots, i_N$; this basis element contains the point $(x_1, x_2, \ldots)$. It is therefore necessary for $x_{i_j} \in U_{i_j} \forall j \in [N]$.

    Then we can take ball $B_{d_{i_j}}(x_{i_j}, \epsilon_{i_j}), \epsilon_{i_j} < 1 \forall j \in [N]$.

    Select $\epsilon = \min\left\{\frac{\epsilon_{i_j} }{i_j} \in [N]\right\}$ then \begin{equation*}
    y \in B_{D}(x, \epsilon) \implies \forall j \in [N], \frac{\bar{d}_{i_j}(x_{i_j}, y_{i_j})}{i_j} < \epsilon < \frac{\epsilon_{i_j}}{i_j} \implies \forall j \in [N], \bar{d}_{i_j}(x_{i_j}, y_{i_j}) < \epsilon_{i_j} < 1
    \end{equation*}
    which implies \begin{equation*}
    d_{i_j}(x_{i_j}, y_{i_j}) < \epsilon_{i_j} \forall j \in [N] \implies y \in B
    \end{equation*}

    Therefore we can find $B_D(x, \epsilon) \subset B$.

    \textbf{2.} WTS the product topology is finer than the $D$-metric topology.

    Take any point $x$, and typical basis element $B_{D}(x, 2\epsilon)$ (we can always do in reverse order). Then there exists $N$ such that $(N+1)\epsilon > 1$.

    Then we construct a basis element of the product topology that contains $x$, namely,\begin{equation*}
    U = B_{d_1}(x_1, \epsilon) \times B_{d_2}(x_2, 2\epsilon) \times \ldots \times B_{d_N}(x_N, N\epsilon) \times X_{N+1} \times X_{N+2} \times \ldots
    \end{equation*}
    and WTS $U \subset B_{D}(x, \epsilon)$.

    For any $y \in U$, we have that $\forall i \in [N], d_i(x_i, y_i) < i\epsilon \implies \frac{\bar{d}_i (x_i, y_i) }{i} < i\epsilon/i = \epsilon $.

    Meanwhile, $\forall i \geq N, \frac{\bar{d}_i (x_i, y_i) }{i} < \frac{1}{i} \leq \frac{1}{N+1} < \epsilon$

    It follows that \begin{equation*}
    D(x, y)  = \sup \left\{\frac{\bar{d}_i(x_i, y_i)}{i}\right\} \leq \epsilon < 2\epsilon
    \end{equation*}
    and therefore $U \subset B_D(x, 2\epsilon)$.

    \textbf{3.} From \textbf{1., 2.}, it follows that $D$ is a metric for the product space $\prod X_i$.
\end{solution}

\begin{problem} [26.8 \redtext{done}]
    Let $f: X \to Y$, $Y$ is compact Hausdorff. Then $f$ is continuous if and only if the graph of f, \begin{equation*}
    G_f = \{x \times f(x) \mid x \in X\}
    \end{equation*}
    is closed in $X \times Y$.

    Hint: If $G_f$ is closed and $V$ is a neighborhood of $f(x_0)$, then the intersection of $G_f$ and $X \times (Y - V)$ is closed. Apply Exercise 7.
\end{problem}
\begin{solution}
    \pffwd Suppose that $f$ is continuous. WTS $X \times Y - G_f$ is open.

    Take $x \times y \in (X \times Y - G_f)$, i.e., $y \in Y$ such that $y \neq f(x)$. Since $Y$ is Hausdorff, there exists disjoint open $V_{fx} \ni f(x), U_y \ni y$. Since $f$ is continuous, $f^{Pre}(V_{fx}) = V_x$ is  open in $X$.

    Then, $V_x \times U_y$ contains $x \times y$ and is a basis element of the topology on $X \times Y$. And $f(V_x) \cap U_y = V_{fx} \cap U_y = \emptyset$ so $V_x \times U_y \cap G_f = \emptyset$, i.e., $V_x \times U_y \subset (X \times Y - G_f)$. It follows that $G_f$ is closed.

    \pfbwd  We use the following lemma
    \begin{lemma} [Exercise 7]
        If $Y$ is compact, then the projection $\pi_1: X \times Y \to X$ is a closed map.
    \end{lemma}
    \begin{proof}
        Take $K \subset X \times Y$ closed. WTS $\pi_1(K)$ is closed in $X$, i.e., $X - \pi_1(K)$ is open in $X$.

        Take any $x \in (X - \pi_1(K))$, i.e., there doesn't exist $y$ such that $x \times y \in K$. This means that $\{x\} \times Y \cap K = \emptyset$. Since $K$ is closed in $(X \times Y)$, it follows that $(X \times Y - K)$ is an open set containing the slice $\{x\} \times Y$. By the tube lemma, there exists open $W \ni x$ such that $W \times Y \subset (X \times Y - K)$, which means $W \cap \pi_1(K) = \emptyset \implies W \subset (X - \pi_1(K)$. It follows that $(X - \pi_1(K))$ is open as required.
    \end{proof}

    Now the main proof. Suppose that $G_f$ is closed in $X \times Y$. To show that $f$ is continuous, we want to show that $f^{Pre}(V)$ is open in $X$ for all $V$ open. Take $f(x_0) \in V$ open. It follows that $X \times V$ is open in $X \times Y$, and therefore $X \times (Y - V)$ is closed. Then $G_f \cap X \times (Y - V)$ is also closed. Using Lemma (since $Y$ is compact), then $\pi_1(G_f \cap X \times (Y - V))$ is also closed. Then $U \coloneqq X - \pi_1(G_f \cap X \times (Y - V)) $ is open.

    But then $U = \pi_1(G_f) - \pi_1(G_f \cap X \times (Y - V)) = \pi_1(G_f \cap X \times V) = f^{Pre}(V)$. It is open.
\end{solution}

\begin{problem} [26.10 \redtext{done}]
    \begin{enumerate}
    \item []
    \item Prove the following partial converse to the uniform limit theorem:
    
    Let $f_n: X \to \bbr$ be a sequence of continuous functions, with $f_n(x) \cvgn f(x)$ for each $x \in X$. If $f$ is continuous, and if the sequence $f_n$ is monotone increasing, and if $X$ is compact, then the convergence is uniform.

    ($f_n$ is monotone increasing if $f_n(x) \leq f_{n+1}(x)$ for all $n$ and $x$.)
    \item Give examples to show that this theorem fails if you delete the requirement that $X$ be compact, or if you delete the requirement that the sequence be monotone.
    
    Hint: Exercises of Chapter 21.
    \end{enumerate}
\end{problem}
\begin{solution}
    \textbf{(a)} To show that $f_n \cvgn f$ uniformly, WTS for all $\epsilon > 0$, there exists $N$ such that \begin{equation*}
    \abs{f(x) - f_n(x)} < \epsilon
    \end{equation*}

    Fix $\epsilon > 0$. Then let $U_n \coloneqq \{x \in X : f(x) - f_n(x) < \epsilon\} = (f-f_n)^{Pre}(-\infty, \epsilon)$. Since $f, f_n$ are continuous, so is $(f-f_n)$ and $U_n$ is therefore open.

    Also, for all $x \in X$, since $f_n(x) \cvgn f(x)$, there exists $N$ \begin{equation*}
    \abs{f(x) - f_N(x)} < \epsilon \implies x \in U_N
    \end{equation*}
    It follows that $X = \bigcup_{n \in \bbn} U_n$. So $\{U_n\}_{n \in \bbn}$ is an open cover of $X$. $X$ is compact so there exists a finite subcover $\{U_{n_1}, \ldots, U_{n_m}\}$ for some $m$ finite.

    Take $N = \max\{n_1, \ldots, n_m\}$. Since $f_n$ is monotone increasing, $f(x) = \sup \{f_n(x)\}_{n \in \bbn}$ for every $x$. This means $\abs{f(x) - f_n(x)} = f(x) - f_n(x)$, and for any $x, n_k, n \geq N$, since $N \geq n_k$, we have that $f(x) - f_n(x) \leq f(x) - f_N(x) \leq f(x) - f{n_k}(x)$.
    
    Then for any $x$, since $x \in U_{n_k}$ for some $k \in [m]$, we have that for all $n \geq N$,
    \begin{equation*}
    f(x) - f_n(x) \leq f(x) - f_N(x) \leq f(x) - f_{n_k}(x) < \epsilon
    \end{equation*}

    \textbf{(b)} If $X$ is not compact: Define on $X = (0, 1)$ not compact, \begin{equation*}
        f_n(x) = x^n
    \end{equation*}

    then for all $x \in (0, 1)$, $f_n(x) \cvgn 0$ so $f \equiv 0$ on $(0, 1)$.

    However, for any $n$, $f_n((1/2)^{1/n}) = 1/2 \not < \epsilon$, so convergence is not uniform.

    If $f_n$ is not monotone: Define on $X = [0, 1]$ compact,
    \begin{equation*} 
    f_n(x) = \begin{cases}
        2nx & \:\text{on}\: [0, \frac{1}{2n}] \\
        2 - 2nx & \:\text{on}\: [\frac{1}{2n}, 1/n] \\
        0 & \:\text{otherwise}\: 
    \end{cases}
    \end{equation*}
    i.e. $f_n$ is a linear ``spike'', connecting $(0, 0), \left(\frac{1}{2n}, 1\right)$ and $\left(\frac{1}{n}, 0\right)$, then flat 0 on the rest.

    Evidently, $f_2(\frac{1}{4}) = 1 > f_1(\frac{1}{4}) = \frac{1}{2}$  so $f_n$ is not monotone increasing.

    But $f_n(x) \cvgn 0$ for all $x \in [0, 1]$, so $f \equiv 0$ on $[0, 1]$.

    But convergence is not uniform, since for all $n$, $f_n(\frac{1}{2n}) - f(\frac{1}{2n}) = 1 - 0 = 1 \not < \epsilon$, so convergence is not uniform.
\end{solution}

\begin{problem} [27.2 \redtext{done}]
    Let $X$ be a metric space with metric $d$; let $A \subset X$ be nonempty.
\begin{enumerate}
\item Show that $d(x, A) = 0$ if and only if $x \in \cl{A}$.
\item Show that if $A$ is compact, $d(x, A) = d(x, a)$ for some $a \in A$.
\item Define the $\epsilon$-neighborhood of $A$ in $X$ to be the set \begin{equation*}
U(A, \epsilon) = \{x \mid d(x, A) < \epsilon\}
\end{equation*}
Show that $U(A, \epsilon)$ equals the union of the open balls $B_d(a, \epsilon)$ for $a \in A$.
\item Assume that $A$ is compact; let $U$ be an open set containing $A$. Show that some $\epsilon$-neighborhood of $A$ is contained in $U$.
\item Show the result in (d) need not hold if $A$ is closed but not compact.
\end{enumerate}
\end{problem}
\begin{solution}
    \textbf{(a)} \pffwd Assume $d(x, A) = 0$. Take any basis $B_d(x, \epsilon) \ni x$. Since $0 = d(x, A) = \inf\{d(x, a) : a \in A\}$, there exists $a \in A$ such that $d(x, a) < \epsilon \implies a \in B_d(x, \epsilon)$. Since any basis element containing $x$ intersects $A$, any open neighborhood of $x$ also intersects $A$. So $x \in \cl{A}$.

    \pfbwd Assume $x \in \cl{A}$. Take basis elements $B_d(x, \frac{1}{n})$ for all $n \in \bbn$. Then there exists $a_n \in A \cap B_d(x, \frac{1}{n})$, i.e., $d(x, a_n) < \frac{1}{n}$. It then follows that $0 \leq d(x, A) = \inf\{d(x, a) : a \in A\} \leq \inf\{d(x, a_n) : n \in \bbn\} = 0 \implies d(x, A) = 0 $. \qed

    \textbf{(b)} For any $x$, we can define $d_x: A \to \bbr, d_x(a) = d(x, a) = d(a, x)$. Then $d_x$ is a continuous function. $A$ is compact, so $d_x$ achieves its minimum at some $a' \in A$. Therefore \begin{equation*}
    d(x, A) = \inf\{d(x, a): a \in A\} = d(x, a')
    \end{equation*}

    \textbf{(c)} WTS $U(A, \epsilon) = \bigcup_{a \in A} B_d(a, \epsilon)$.

    Take $x \in U(A, \epsilon)$. That means $d(x, A) < \epsilon$. Since $d(x, A) = \inf\{d(x, a): a \in A\}$, this means that there exists some $a' \in A$ such that $d(x, a') < \frac{d(x, A) + \epsilon}{2} < \epsilon \implies x \in B_d(a', \epsilon)$. It follows that $U(A, \epsilon) \subset \bigcup_{a \in A} B_d(a, \epsilon)$.

    Take $x \in \bigcup_{a \in A} B_d(a, \epsilon)$, which means $d(x, a') < \epsilon$ for some $a' \in A$. Then this means $d(x, A) = \inf\{d(x, a) : a \in A\} \leq d(x, a') < \epsilon \implies x \in U(A, \epsilon)$. It then follows that  $U(A, \epsilon) \supset \bigcup_{a \in A} B_d(a, \epsilon)$.

    Therefore $U(A, \epsilon) = \bigcup_{a \in A} B_d(a, \epsilon)$.

    \textbf{(d)} For every $a \in A \subset U$, since $U$ is open, we can draw $a \in B(a, \epsilon_a) \subset U$. Then $\{B(a, \epsilon_a)\}$ is an open cover of $A$ compact. Using the Lebesgue covering lemma, then there exists $\delta$ such that for any $a$, $B(a, \delta)$ is contained in some element of the covering, and therefore contained in $U$ in particular. It follows that $\bigcup_{a \in A} B(a, \delta) \subset U$. From (c), $\bigcup_{a \in A} B(a, \delta) = U(A, \delta)$.

    \textbf{(e)} Counter example: $X = [-1, 0) \cup (0, 1], A = U = (0, 1]$. $A$ and $U$ are clopen in the subspace topology. But any $\epsilon$-neighborhood of $A$, WLOG $\epsilon < 1$, would contain the point $-\epsilon/2$, for \begin{equation*}
    d(-\epsilon/2, A) = \inf\{d(-\epsilon/2, a) : a \in A\} = \epsilon/2 < \epsilon
    \end{equation*}
    Clearly $-\epsilon/2 \not \in U$.
\end{solution}

\begin{problem} [27.5 \redtext{done}]
    Let $X$ be a compact Hausdorff space, let $\{A_n\}$ be a countable collection of closed sets of $X$. Show that if each set $A_n$ has empty interior in $X$, then the union $\bigcup A_n$ has empty interior in in $X$.

    Hint: Imitate the proof of Theorem 27.7. This is a special case of the Baire category theorem.
\end{problem}
\begin{solution}
    \textit{(With Otto Reed)}

    Suppose, for sake of contradiction, that $U_0 = \interior (\bigcup A_n) \neq \emptyset$ is open.

    We claim that for any non-empty, open $U$ and any $A_n$, there exists non-empty $V$ such that $\cl{V} \subset U$ and $\cl{V} \cap A_n = \emptyset$. Take such $U$ and $A_n$. Since $\interior (A_n) = \emptyset$, $U$ is not a subset of $A_n$ (if $U \subset A_n$ then $U \subset \interior(A_n) \implies \interior(A_n) \neq \emptyset$), i.e., there exists some $a \in U$ such that $a \in U - A_n$. Then, take $K = A_n \cup (X - U)$; $K$ is closed since $A_n$ is closed and $U$ is open. $K$ is closed in compact $X$, so $K$ is compact. $a \in U - A_n \implies a \not \in K$. It follows from Lemma 26.4 ($K$ is compact, $x \not \in K$, $X$ is Hausdorff) that there exists disjoint open sets $V \ni a, W \supset K$. Then, since $V \cap W = \emptyset$, \begin{equation*}
    V \subset (X - W)
    \end{equation*}

    It follows that $\cl{V} \subset \cl{X - W}$. But $X - W$ is closed, so $\cl{X - W} = X - W$. It follows that $\cl{V} \subset X - W$.

    In turn, \begin{equation*}
    X - W \subset X - K = X - A_n \cup (X - U) = (X - A_n) \cap U = U - A_n \subset U
    \end{equation*}
    so \begin{equation*}
    \cl{V} \subset U
    \end{equation*}
    and recall that $a \in V$, so $V$ is non-empty. We have thus proven our claim.

    Apply the claim onto $U_0$ and $A_1$, it follows that there exists non-empty $U_1$ such that $\cl{U_1} \subset U_0$ and $\cl{U_1} \cap A_1 = \emptyset$. Iteratively, apply the claim onto $U_k$ and $A_{k+1}$, it follows that there exists non-empty $U_{k+1}$ such that $\cl{U_{k+1}} \subset U_k$ and $\cl{U_{k+1}} \cap A_{k+1} = \emptyset$.

    Each $\cl{U_k}$ is closed in compact $X$, so is compact. We therefore get a sequence of nested, non-empty compact $\cl{U_n}$:
    \begin{equation*}
    \cl{U_1} \supset \cl{U_2} \supset \ldots
    \end{equation*}

    It follows that there exists $p \in \bigcap_{k \in \bbn} \cl{U_k}$.

    However, $\cl{U_k} \cap A_k = \emptyset$ for all $k \geq 1$. On top of that, $\cl{U_{k+1}} \subset U_k \subset \cl{U_k}$ so $\cl{U_j} \cap A_k = \emptyset$ for all $j \geq k \geq 1$ too. It follows that \begin{equation*}
       \left( \bigcap_{k \in \bbn} \cl{U_k}\right) \cap \left(\bigcup_{k \in \bbn} A_k\right) = \emptyset
    \end{equation*}
    which implies $p \not \in \bigcup_{k \in \bbn} A_k$. 
    
    But $p \in \cl{V_1} \subset U = \interior \bigcup A_k \subset \bigcup A_k$, \contra.

    By contradiction, it follows that $\interior \bigcup A_n = \emptyset$.
\end{solution}
\end{document}