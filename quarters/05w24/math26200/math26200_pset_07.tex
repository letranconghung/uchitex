\documentclass[a4paper, 10pt]{article}
%%%%%%%%%%%%%%%%%%%%%%%%%%%%%%%%%%%%%%%%%%%%%%%%%%%%%%%%%%%%%%%%%%%%%%%%%%%%%%%
%                                Basic Packages                               %
%%%%%%%%%%%%%%%%%%%%%%%%%%%%%%%%%%%%%%%%%%%%%%%%%%%%%%%%%%%%%%%%%%%%%%%%%%%%%%%

% Gives us multiple colors.
\usepackage[usenames,dvipsnames,pdftex]{xcolor}
% Lets us style link colors.
\usepackage{hyperref}
% Lets us import images and graphics.
\usepackage{graphicx}
% Lets us use figures in floating environments.
\usepackage{float}
% Lets us create multiple columns.
\usepackage{multicol}
% Gives us better math syntax.
\usepackage{amsmath,amsfonts,mathtools,amsthm,amssymb}
% Lets us strikethrough text.
\usepackage{cancel}
% Lets us edit the caption of a figure.
\usepackage{caption}
% Lets us import pdf directly in our tex code.
\usepackage{pdfpages}
% Lets us do algorithm stuff.
\usepackage[ruled,vlined,linesnumbered]{algorithm2e}
% Use a smiley face for our qed symbol.
\usepackage{tikzsymbols}
% \usepackage{fullpage} %%smaller margins
\usepackage[shortlabels]{enumitem}

\setlist[enumerate]{font={\bfseries}} % global settings, for all lists

\usepackage{setspace}
\usepackage[margin=1in, headsep=12pt]{geometry}
\usepackage{wrapfig}
\usepackage{listings}
\usepackage{parskip}

\definecolor{codegreen}{rgb}{0,0.6,0}
\definecolor{codegray}{rgb}{0.5,0.5,0.5}
\definecolor{codepurple}{rgb}{0.58,0,0.82}
\definecolor{backcolour}{rgb}{0.95,0.95,0.95}

\lstdefinestyle{mystyle}{
    backgroundcolor=\color{backcolour},   
    commentstyle=\color{codegreen},
    keywordstyle=\color{magenta},
    numberstyle=\tiny\color{codegray},
    stringstyle=\color{codepurple},
    basicstyle=\ttfamily\footnotesize,
    breakatwhitespace=false,         
    breaklines=true,                 
    captionpos=b,                    
    keepspaces=true,                 
    numbers=left,                    
    numbersep=5pt,                  
    showspaces=false,                
    showstringspaces=false,
    showtabs=false,                  
    tabsize=2,
    numbers=none
}

\lstset{style=mystyle}
\def\class{article}


%%%%%%%%%%%%%%%%%%%%%%%%%%%%%%%%%%%%%%%%%%%%%%%%%%%%%%%%%%%%%%%%%%%%%%%%%%%%%%%
%                                Basic Settings                               %
%%%%%%%%%%%%%%%%%%%%%%%%%%%%%%%%%%%%%%%%%%%%%%%%%%%%%%%%%%%%%%%%%%%%%%%%%%%%%%%

%%%%%%%%%%%%%
%  Symbols  %
%%%%%%%%%%%%%

\let\implies\Rightarrow
\let\impliedby\Leftarrow
\let\iff\Leftrightarrow
\let\epsilon\varepsilon
%%%%%%%%%%%%
%  Tables  %
%%%%%%%%%%%%

\setlength{\tabcolsep}{5pt}
\renewcommand\arraystretch{1.5}

%%%%%%%%%%%%%%
%  SI Unitx  %
%%%%%%%%%%%%%%

\usepackage{siunitx}
\sisetup{locale = FR}

%%%%%%%%%%
%  TikZ  %
%%%%%%%%%%

\usepackage[framemethod=TikZ]{mdframed}
\usepackage{tikz}
\usepackage{tikz-cd}
\usepackage{tikzsymbols}

\usetikzlibrary{intersections, angles, quotes, calc, positioning}
\usetikzlibrary{arrows.meta}

\tikzset{
    force/.style={thick, {Circle[length=2pt]}-stealth, shorten <=-1pt}
}

%%%%%%%%%%%%%%%
%  PGF Plots  %
%%%%%%%%%%%%%%%

\usepackage{pgfplots}
\pgfplotsset{width=10cm, compat=newest}

%%%%%%%%%%%%%%%%%%%%%%%
%  Center Title Page  %
%%%%%%%%%%%%%%%%%%%%%%%

\usepackage{titling}
\renewcommand\maketitlehooka{\null\mbox{}\vfill}
\renewcommand\maketitlehookd{\vfill\null}

%%%%%%%%%%%%%%%%%%%%%%%%%%%%%%%%%%%%%%%%%%%%%%%%%%%%%%%
%  Create a grey background in the middle of the PDF  %
%%%%%%%%%%%%%%%%%%%%%%%%%%%%%%%%%%%%%%%%%%%%%%%%%%%%%%%

\usepackage{eso-pic}
\newcommand\definegraybackground{
    \definecolor{reallylightgray}{HTML}{FAFAFA}
    \AddToShipoutPicture{
        \ifthenelse{\isodd{\thepage}}{
            \AtPageLowerLeft{
                \put(\LenToUnit{\dimexpr\paperwidth-222pt},0){
                    \color{reallylightgray}\rule{222pt}{297mm}
                }
            }
        }
        {
            \AtPageLowerLeft{
                \color{reallylightgray}\rule{222pt}{297mm}
            }
        }
    }
}

%%%%%%%%%%%%%%%%%%%%%%%%
%  Modify Links Color  %
%%%%%%%%%%%%%%%%%%%%%%%%

\hypersetup{
    % Enable highlighting links.
    colorlinks,
    % Change the color of links to blue.
    urlcolor=blue,
    % Change the color of citations to black.
    citecolor={black},
    % Change the color of url's to blue with some black.
    linkcolor={blue!80!black}
}

%%%%%%%%%%%%%%%%%%
% Fix WrapFigure %
%%%%%%%%%%%%%%%%%%

\newcommand{\wrapfill}{\par\ifnum\value{WF@wrappedlines}>0
        \parskip=0pt
        \addtocounter{WF@wrappedlines}{-1}%
        \null\vspace{\arabic{WF@wrappedlines}\baselineskip}%
        \WFclear
    \fi}

%%%%%%%%%%%%%%%%%
% Multi Columns %
%%%%%%%%%%%%%%%%%

\let\multicolmulticols\multicols
\let\endmulticolmulticols\endmulticols

\RenewDocumentEnvironment{multicols}{mO{}}
{%
    \ifnum#1=1
        #2%
    \else % More than 1 column
        \multicolmulticols{#1}[#2]
    \fi
}
{%
    \ifnum#1=1
    \else % More than 1 column
        \endmulticolmulticols
    \fi
}

\newlength{\thickarrayrulewidth}
\setlength{\thickarrayrulewidth}{5\arrayrulewidth}


%%%%%%%%%%%%%%%%%%%%%%%%%%%%%%%%%%%%%%%%%%%%%%%%%%%%%%%%%%%%%%%%%%%%%%%%%%%%%%%
%                           School Specific Commands                          %
%%%%%%%%%%%%%%%%%%%%%%%%%%%%%%%%%%%%%%%%%%%%%%%%%%%%%%%%%%%%%%%%%%%%%%%%%%%%%%%

%%%%%%%%%%%%%%%%%%%%%%%%%%%
%  Initiate New Counters  %
%%%%%%%%%%%%%%%%%%%%%%%%%%%

\newcounter{lecturecounter}

%%%%%%%%%%%%%%%%%%%%%%%%%%
%  Helpful New Commands  %
%%%%%%%%%%%%%%%%%%%%%%%%%%

\makeatletter

\newcommand\resetcounters{
    % Reset the counters for subsection, subsubsection and the definition
    % all the custom environments.
    \setcounter{subsection}{0}
    \setcounter{subsubsection}{0}
    \setcounter{definition0}{0}
    \setcounter{paragraph}{0}
    \setcounter{theorem}{0}
    \setcounter{claim}{0}
    \setcounter{corollary}{0}
    \setcounter{proposition}{0}
    \setcounter{lemma}{0}
    \setcounter{exercise}{0}
    \setcounter{problem}{0}
    
    \setcounter{subparagraph}{0}
    % \@ifclasswith\class{nocolor}{
    %     \setcounter{definition}{0}
    % }{}
}

%%%%%%%%%%%%%%%%%%%%%
%  Lecture Command  %
%%%%%%%%%%%%%%%%%%%%%

\usepackage{xifthen}

% EXAMPLE:
% 1. \lecture{Oct 17 2022 Mon (08:46:48)}{Lecture Title}
% 2. \lecture[4]{Oct 17 2022 Mon (08:46:48)}{Lecture Title}
% 3. \lecture{Oct 17 2022 Mon (08:46:48)}{}
% 4. \lecture[4]{Oct 17 2022 Mon (08:46:48)}{}
% Parameters:
% 1. (Optional) lecture number.
% 2. Time and date of lecture.
% 3. Lecture Title.
\def\@lecture{}
\def\@lectitle{}
\def\@leccount{}
\newcommand\lecture[3]{
    \newpage

    % Check if user passed the lecture title or not.
    \def\@leccount{Lecture #1}
    \ifthenelse{\isempty{#3}}{
        \def\@lecture{Lecture #1}
        \def\@lectitle{Lecture #1}
    }{
        \def\@lecture{Lecture #1: #3}
        \def\@lectitle{#3}
    }

    \setcounter{section}{#1}
    \renewcommand\thesubsection{#1.\arabic{subsection}}
    
    \phantomsection
    \addcontentsline{toc}{section}{\@lecture}
    \resetcounters

    \begin{mdframed}
        \begin{center}
            \Large \textbf{\@leccount}
            
            \vspace*{0.2cm}
            
            \large \@lectitle
            
            
            \vspace*{0.2cm}

            \normalsize #2
        \end{center}
    \end{mdframed}

}

%%%%%%%%%%%%%%%%%%%%
%  Import Figures  %
%%%%%%%%%%%%%%%%%%%%

\usepackage{import}
\pdfminorversion=7

% EXAMPLE:
% 1. \incfig{limit-graph}
% 2. \incfig[0.4]{limit-graph}
% Parameters:
% 1. The figure name. It should be located in figures/NAME.tex_pdf.
% 2. (Optional) The width of the figure. Example: 0.5, 0.35.
\newcommand\incfig[2][1]{%
    \def\svgwidth{#1\columnwidth}
    \import{./figures/}{#2.pdf_tex}
}

\begingroup\expandafter\expandafter\expandafter\endgroup
\expandafter\ifx\csname pdfsuppresswarningpagegroup\endcsname\relax
\else
    \pdfsuppresswarningpagegroup=1\relax
\fi

%%%%%%%%%%%%%%%%%
% Fancy Headers %
%%%%%%%%%%%%%%%%%

\usepackage{fancyhdr}

% Force a new page.
\newcommand\forcenewpage{\clearpage\mbox{~}\clearpage\newpage}

% This command makes it easier to manage my headers and footers.
\newcommand\createintro{
    % Use roman page numbers (e.g. i, v, vi, x, ...)
    \pagenumbering{roman}

    % Display the page style.
    \maketitle
    % Make the title pagestyle empty, meaning no fancy headers and footers.
    \thispagestyle{empty}
    % Create a newpage.
    \newpage

    % Input the intro.tex page if it exists.
    \IfFileExists{intro.tex}{ % If the intro.tex file exists.
        % Input the intro.tex file.
        \textbf{Course}: MATH 16300: Honors Calculus III

\textbf{Section}: 43

\textbf{Professor}: Minjae Park

\textbf{At}: The University of Chicago

\textbf{Quarter}: Spring 2023

\textbf{Course materials}: Calculus by Spivak (4th Edition), Calculus On Manifolds by Spivak

\vspace{1cm}
\textbf{Disclaimer}: This document will inevitably contain some mistakes, both simple typos and serious logical and mathematical errors. Take what you read with a grain of salt as it is made by an undergraduate student going through the learning process himself. If you do find any error, I would really appreciate it if you can let me know by email at \href{mailto:conghungletran@gmail.com}{conghungletran@gmail.com}.

        % Make the pagestyle fancy for the intro.tex page.
        \pagestyle{fancy}

        % Remove the line for the header.
        \renewcommand\headrulewidth{0pt}

        % Remove all header stuff.
        \fancyhead{}

        % Add stuff for the footer in the center.
        % \fancyfoot[C]{
        %   \textit{For more notes like this, visit
        %   \href{\linktootherpages}{\shortlinkname}}. \\
        %   \vspace{0.1cm}
        %   \hrule
        %   \vspace{0.1cm}
        %   \@author, \\
        %   \term: \academicyear, \\
        %   Last Update: \@date, \\
        %   \faculty
        % }

        \newpage
    }{ % If the intro.tex file doesn't exist.
        % Force a \newpageage.
        % \forcenewpage
        \newpage
    }

    % Remove the center stuff we did above, and replace it with just the page
    % number, which is still in roman numerals.
    \fancyfoot[C]{\thepage}
    % Add the table of contents.
    \tableofcontents
    % Force a new page.
    \newpage

    % Move the page numberings back to arabic, from roman numerals.
    \pagenumbering{arabic}
    % Set the page number to 1.
    \setcounter{page}{1}

    % Add the header line back.
    \renewcommand\headrulewidth{0.4pt}
    % In the top right, add the lecture title.
    \fancyhead[R]{\footnotesize \@lecture}
    % In the top left, add the author name.
    \fancyhead[L]{\footnotesize \@author}
    % In the bottom center, add the page.
    \fancyfoot[C]{\thepage}
    % Add a nice gray background in the middle of all the upcoming pages.
    % \definegraybackground
}

\makeatother


%%%%%%%%%%%%%%%%%%%%%%%%%%%%%%%%%%%%%%%%%%%%%%%%%%%%%%%%%%%%%%%%%%%%%%%%%%%%%%%
%                               Custom Commands                               %
%%%%%%%%%%%%%%%%%%%%%%%%%%%%%%%%%%%%%%%%%%%%%%%%%%%%%%%%%%%%%%%%%%%%%%%%%%%%%%%

%%%%%%%%%%%%
%  Circle  %
%%%%%%%%%%%%

\newcommand*\circled[1]{\tikz[baseline= (char.base)]{
        \node[shape=circle,draw,inner sep=1pt] (char) {#1};}
}

%%%%%%%%%%%%%%%%%%%
%  Todo Commands  %
%%%%%%%%%%%%%%%%%%%

% \usepackage{xargs}
% \usepackage[colorinlistoftodos]{todonotes}

% \makeatletter

% \@ifclasswith\class{working}{
%     \newcommandx\unsure[2][1=]{\todo[linecolor=red,backgroundcolor=red!25,bordercolor=red,#1]{#2}}
%     \newcommandx\change[2][1=]{\todo[linecolor=blue,backgroundcolor=blue!25,bordercolor=blue,#1]{#2}}
%     \newcommandx\info[2][1=]{\todo[linecolor=OliveGreen,backgroundcolor=OliveGreen!25,bordercolor=OliveGreen,#1]{#2}}
%     \newcommandx\improvement[2][1=]{\todo[linecolor=Plum,backgroundcolor=Plum!25,bordercolor=Plum,#1]{#2}}

%     \newcommand\listnotes{
%         \newpage
%         \listoftodos[Notes]
%     }
% }{
%     \newcommandx\unsure[2][1=]{}
%     \newcommandx\change[2][1=]{}
%     \newcommandx\info[2][1=]{}
%     \newcommandx\improvement[2][1=]{}

%     \newcommand\listnotes{}
% }

% \makeatother

%%%%%%%%%%%%%
%  Correct  %
%%%%%%%%%%%%%

% EXAMPLE:
% 1. \correct{INCORRECT}{CORRECT}
% Parameters:
% 1. The incorrect statement.
% 2. The correct statement.
\definecolor{correct}{HTML}{009900}
\newcommand\correct[2]{{\color{red}{#1 }}\ensuremath{\to}{\color{correct}{ #2}}}


%%%%%%%%%%%%%%%%%%%%%%%%%%%%%%%%%%%%%%%%%%%%%%%%%%%%%%%%%%%%%%%%%%%%%%%%%%%%%%%
%                                 Environments                                %
%%%%%%%%%%%%%%%%%%%%%%%%%%%%%%%%%%%%%%%%%%%%%%%%%%%%%%%%%%%%%%%%%%%%%%%%%%%%%%%

\usepackage{varwidth}
\usepackage{thmtools}
\usepackage[most,many,breakable]{tcolorbox}

\tcbuselibrary{theorems,skins,hooks}
\usetikzlibrary{arrows,calc,shadows.blur}

%%%%%%%%%%%%%%%%%%%
%  Define Colors  %
%%%%%%%%%%%%%%%%%%%

% color prototype
% \definecolor{color}{RGB}{45, 111, 177}

% ESSENTIALS: 
\definecolor{myred}{HTML}{c74540}
\definecolor{myblue}{HTML}{072b85}
\definecolor{mygreen}{HTML}{388c46}
\definecolor{myblack}{HTML}{000000}

\colorlet{definition_color}{myred}

\colorlet{theorem_color}{myblue}
\colorlet{lemma_color}{myblue}
\colorlet{prop_color}{myblue}
\colorlet{corollary_color}{myblue}
\colorlet{claim_color}{myblue}

\colorlet{proof_color}{myblack}
\colorlet{example_color}{myblack}
\colorlet{exercise_color}{myblack}

% MISCS: 
%%%%%%%%%%%%%%%%%%%%%%%%%%%%%%%%%%%%%%%%%%%%%%%%%%%%%%%%%
%  Create Environments Styles Based on Given Parameter  %
%%%%%%%%%%%%%%%%%%%%%%%%%%%%%%%%%%%%%%%%%%%%%%%%%%%%%%%%%

% \mdfsetup{skipabove=1em,skipbelow=0em}

%%%%%%%%%%%%%%%%%%%%%%
%  Helpful Commands  %
%%%%%%%%%%%%%%%%%%%%%%

% EXAMPLE:
% 1. \createnewtheoremstyle{thmdefinitionbox}{}{}
% 2. \createnewtheoremstyle{thmtheorembox}{}{}
% 3. \createnewtheoremstyle{thmproofbox}{qed=\qedsymbol}{
%       rightline=false, topline=false, bottomline=false
%    }
% Parameters:
% 1. Theorem name.
% 2. Any extra parameters to pass directly to declaretheoremstyle.
% 3. Any extra parameters to pass directly to mdframed.
\newcommand\createnewtheoremstyle[3]{
    \declaretheoremstyle[
        headfont=\bfseries\sffamily, bodyfont=\normalfont, #2,
        mdframed={
                #3,
            },
    ]{#1}
}

% EXAMPLE:
% 1. \createnewcoloredtheoremstyle{thmdefinitionbox}{definition}{}{}
% 2. \createnewcoloredtheoremstyle{thmexamplebox}{example}{}{
%       rightline=true, leftline=true, topline=true, bottomline=true
%     }
% 3. \createnewcoloredtheoremstyle{thmproofbox}{proof}{qed=\qedsymbol}{backgroundcolor=white}
% Parameters:
% 1. Theorem name.
% 2. Color of theorem.
% 3. Any extra parameters to pass directly to declaretheoremstyle.
% 4. Any extra parameters to pass directly to mdframed.

% change backgroundcolor to #2!5 if user wants a colored backdrop to theorem environments. It's a cool color theme, but there's too much going on in the page.
\newcommand\createnewcoloredtheoremstyle[4]{
    \declaretheoremstyle[
        headfont=\bfseries\sffamily\color{#2},
        bodyfont=\normalfont,
        headpunct=,
        headformat = \NAME~\NUMBER\NOTE \hfill\smallskip\linebreak,
        #3,
        mdframed={
                outerlinewidth=0.75pt,
                rightline=false,
                leftline=false,
                topline=false,
                bottomline=false,
                backgroundcolor=white,
                skipabove = 5pt,
                skipbelow = 0pt,
                linecolor=#2,
                innertopmargin = 0pt,
                innerbottommargin = 0pt,
                innerrightmargin = 4pt,
                innerleftmargin= 6pt,
                leftmargin = -6pt,
                #4,
            },
    ]{#1}
}



%%%%%%%%%%%%%%%%%%%%%%%%%%%%%%%%%%%
%  Create the Environment Styles  %
%%%%%%%%%%%%%%%%%%%%%%%%%%%%%%%%%%%

\makeatletter
\@ifclasswith\class{nocolor}{
    % Environments without color.

    % ESSENTIALS:
    \createnewtheoremstyle{thmdefinitionbox}{}{}
    \createnewtheoremstyle{thmtheorembox}{}{}
    \createnewtheoremstyle{thmproofbox}{qed=\qedsymbol}{}
    \createnewtheoremstyle{thmcorollarybox}{}{}
    \createnewtheoremstyle{thmlemmabox}{}{}
    \createnewtheoremstyle{thmclaimbox}{}{}
    \createnewtheoremstyle{thmexamplebox}{}{}

    % MISCS: 
    \createnewtheoremstyle{thmpropbox}{}{}
    \createnewtheoremstyle{thmexercisebox}{}{}
    \createnewtheoremstyle{thmexplanationbox}{}{}
    \createnewtheoremstyle{thmremarkbox}{}{}
    
    % STYLIZED MORE BELOW
    \createnewtheoremstyle{thmquestionbox}{}{}
    \createnewtheoremstyle{thmsolutionbox}{qed=\qedsymbol}{}
}{
    % Environments with color.

    % ESSENTIALS: definition, theorem, proof, corollary, lemma, claim, example
    \createnewcoloredtheoremstyle{thmdefinitionbox}{definition_color}{}{leftline=false}
    \createnewcoloredtheoremstyle{thmtheorembox}{theorem_color}{}{leftline=false}
    \createnewcoloredtheoremstyle{thmproofbox}{proof_color}{qed=\qedsymbol}{}
    \createnewcoloredtheoremstyle{thmcorollarybox}{corollary_color}{}{leftline=false}
    \createnewcoloredtheoremstyle{thmlemmabox}{lemma_color}{}{leftline=false}
    \createnewcoloredtheoremstyle{thmpropbox}{prop_color}{}{leftline=false}
    \createnewcoloredtheoremstyle{thmclaimbox}{claim_color}{}{leftline=false}
    \createnewcoloredtheoremstyle{thmexamplebox}{example_color}{}{}
    \createnewcoloredtheoremstyle{thmexplanationbox}{example_color}{qed=\qedsymbol}{}
    \createnewcoloredtheoremstyle{thmremarkbox}{theorem_color}{}{}

    \createnewcoloredtheoremstyle{thmmiscbox}{black}{}{}

    \createnewcoloredtheoremstyle{thmexercisebox}{exercise_color}{}{}
    \createnewcoloredtheoremstyle{thmproblembox}{theorem_color}{}{leftline=false}
    \createnewcoloredtheoremstyle{thmsolutionbox}{mygreen}{qed=\qedsymbol}{}
}
\makeatother

%%%%%%%%%%%%%%%%%%%%%%%%%%%%%
%  Create the Environments  %
%%%%%%%%%%%%%%%%%%%%%%%%%%%%%
\declaretheorem[numberwithin=section, style=thmdefinitionbox,     name=Definition]{definition}
\declaretheorem[numberwithin=section, style=thmtheorembox,     name=Theorem]{theorem}
\declaretheorem[numbered=no,          style=thmexamplebox,     name=Example]{example}
\declaretheorem[numberwithin=section, style=thmtheorembox,       name=Claim]{claim}
\declaretheorem[numberwithin=section, style=thmcorollarybox,   name=Corollary]{corollary}
\declaretheorem[numberwithin=section, style=thmpropbox,        name=Proposition]{proposition}
\declaretheorem[numberwithin=section, style=thmlemmabox,       name=Lemma]{lemma}
\declaretheorem[numberwithin=section, style=thmexercisebox,    name=Exercise]{exercise}
\declaretheorem[numbered=no,          style=thmproofbox,       name=Proof]{proof0}
\declaretheorem[numbered=no,          style=thmexplanationbox, name=Explanation]{explanation}
\declaretheorem[numbered=no,          style=thmsolutionbox,    name=Solution]{solution}
\declaretheorem[numberwithin=section,          style=thmproblembox,     name=Problem]{problem}
\declaretheorem[numbered=no,          style=thmmiscbox,    name=Intuition]{intuition}
\declaretheorem[numbered=no,          style=thmmiscbox,    name=Goal]{goal}
\declaretheorem[numbered=no,          style=thmmiscbox,    name=Recall]{recall}
\declaretheorem[numbered=no,          style=thmmiscbox,    name=Motivation]{motivation}
\declaretheorem[numbered=no,          style=thmmiscbox,    name=Remark]{remark}
\declaretheorem[numbered=no,          style=thmmiscbox,    name=Observe]{observe}
\declaretheorem[numbered=no,          style=thmmiscbox,    name=Question]{question}


%%%%%%%%%%%%%%%%%%%%%%%%%%%%
%  Edit Proof Environment  %
%%%%%%%%%%%%%%%%%%%%%%%%%%%%

\renewenvironment{proof}[2][\proofname]{
    % \vspace{-12pt}
    \begin{proof0} [#2]
        }{\end{proof0}}

\theoremstyle{definition}

\newtheorem*{notation}{Notation}
\newtheorem*{previouslyseen}{As previously seen}
\newtheorem*{property}{Property}
% \newtheorem*{intuition}{Intuition}
% \newtheorem*{goal}{Goal}
% \newtheorem*{recall}{Recall}
% \newtheorem*{motivation}{Motivation}
% \newtheorem*{remark}{Remark}
% \newtheorem*{observe}{Observe}

\author{Hung C. Le Tran}


%%%% MATH SHORTHANDS %%%%
%% blackboard bold math capitals
\DeclareMathOperator*{\esssup}{ess\,sup}
\DeclareMathOperator*{\Hom}{Hom}
\newcommand{\bbf}{\mathbb{F}}
\newcommand{\bbn}{\mathbb{N}}
\newcommand{\bbq}{\mathbb{Q}}
\newcommand{\bbr}{\mathbb{R}}
\newcommand{\bbz}{\mathbb{Z}}
\newcommand{\bbc}{\mathbb{C}}
\newcommand{\bbk}{\mathbb{K}}
\newcommand{\bbm}{\mathbb{M}}
\newcommand{\bbp}{\mathbb{P}}
\newcommand{\bbe}{\mathbb{E}}

\newcommand{\bfw}{\mathbf{w}}
\newcommand{\bfx}{\mathbf{x}}
\newcommand{\bfX}{\mathbf{X}}
\newcommand{\bfy}{\mathbf{y}}
\newcommand{\bfyhat}{\mathbf{\hat{y}}}

\newcommand{\calb}{\mathcal{B}}
\newcommand{\calf}{\mathcal{F}}
\newcommand{\calt}{\mathcal{T}}
\newcommand{\call}{\mathcal{L}}
\renewcommand{\phi}{\varphi}

% Universal Math Shortcuts
\newcommand{\st}{\hspace*{2pt}\text{s.t.}\hspace*{2pt}}
\newcommand{\pffwd}{\hspace*{2pt}\fbox{\(\Rightarrow\)}\hspace*{10pt}}
\newcommand{\pfbwd}{\hspace*{2pt}\fbox{\(\Leftarrow\)}\hspace*{10pt}}
\newcommand{\contra}{\ensuremath{\Rightarrow\Leftarrow}}
\newcommand{\cvgn}{\xrightarrow{n \to \infty}}
\newcommand{\cvgj}{\xrightarrow{j \to \infty}}

\newcommand{\im}{\mathrm{im}}
\newcommand{\innerproduct}[2]{\langle #1, #2 \rangle}
\newcommand*{\conj}[1]{\overline{#1}}

% https://tex.stackexchange.com/questions/438612/space-between-exists-and-forall
% https://tex.stackexchange.com/questions/22798/nice-looking-empty-set
\let\oldforall\forall
\renewcommand{\forall}{\;\oldforall\; }
\let\oldexist\exists
\renewcommand{\exists}{\;\oldexist\; }
\newcommand\existu{\;\oldexist!\: }
\let\oldemptyset\emptyset
\let\emptyset\varnothing


\renewcommand{\_}[1]{\underline{#1}}
\DeclarePairedDelimiter{\abs}{\lvert}{\rvert}
\DeclarePairedDelimiter{\norm}{\lVert}{\rVert}
\DeclarePairedDelimiter\ceil{\lceil}{\rceil}
\DeclarePairedDelimiter\floor{\lfloor}{\rfloor}
\setlength\parindent{0pt}
\setlength{\headheight}{12.0pt}
\addtolength{\topmargin}{-12.0pt}


% Default skipping, change if you want more spacing
% \thinmuskip=3mu
% \medmuskip=4mu plus 2mu minus 4mu
% \thickmuskip=5mu plus 5mu

% \DeclareMathOperator{\ext}{ext}
% \DeclareMathOperator{\bridge}{bridge}
\title{MATH 26200: Point-Set Topology \\ \large Problem Set 7}
\date{29 Feb 2024}
\author{Hung Le Tran}
\begin{document}
\maketitle
\setcounter{section}{7}
\begin{problem} [\done]
    Construct an explicit homeomorphism from $\{0, 1\}^\bbn$ to $\{0, 1, 2\}^{\bbn}$. Here, $\{0, 1\}$ and $\{0, 1, 2\}$ denote the 2 and 3-element sets with the discrete topology.
\end{problem}
\begin{solution} [With Otto Reed's help]
    Construct the map $f: \{0, 1, 2\}^\bbn \to \{0, 1\}^\bbn$ as follows. Consider $g: \{0, 1, 2\} \to \{(0), (1, 0), (1, 1)\}$ such that
    \begin{equation*}
    g(0) = (0), g(1) = (1, 0), g(2) = (1, 1)
    \end{equation*}
    then map $f(x) = (\text{concat}_{n = 1}^{\infty} (g(x_n)))$.

    It is easy to check that $f$ is both injective and surjective, so it is bijective.

    To show continuity, each basic open set in $\{0, 1\}^\bbn$ is some $\prod U_i$ such that $U_i = \{0, 1\}$ for all but finitely many $\{i_1 < \ldots < i_K\}$. 

    Its preimage is then some 
    \begin{equation*}
        \bigcup_j (\prod_{k \in \bbn} U^{(j)}_k)
    \end{equation*}
    where for all $j$, $U^{(j)}_k = X_k$ for all $k \geq i_K$ (a very rough bound) and each $U^{(j)}$ is the ``preimage'' of the sequences truncated at $i_K$ (or $i_K + 1$ if $U_{i_K} = \{1\}$) that are contained in $\prod U_i$, is open, since $\{0, 1, 2\}$ has the discrete topology. Therefore $\prod_{k \in \bbn} U^{(j)}_k$ is a basic open set in $\{0, 1, 2\}^\bbn$, so the preimage of a basic open set in $\{0, 1\}^\bbn$ is open, so $f$ is continuous.

    The same proof applies to show that $\inv{f}$ is continuous, with the bound $k \geq 2 i_K$, since $g$ requires twice the number of indices in $\{0, 1\}^\bbn$ than in $\{0, 1, 2\}^\bbn$ to ``accommodate'' restrictions from the image set.

    Hence $f$ is a homeomorphism.
\end{solution}

\begin{problem} [\done]
    Suppose $X$ is a compact metric space, $Y$ is Hausdorff, and $f: X \to Y$ is continuous and surjective. Show that $Y$ is a compact metrizable space.
\end{problem}
\begin{solution}
    \textbf{1.} WTS $Y$ is compact. $f$ is surjective so $Y = f(X)$ is the continuous image of a compact set, so $Y$ is compact.

    \textbf{2.} WTS $X$ is 2nd countable. $X$ is a compact metric space. For each $x \in X$, construct and $n \in \bbn$, let $\calu_n = \{B(x, \frac{1}{n}) : x \in X\}$ then $\calu_n$ is an open cover of $X$, so it reduces to some finite subcover $\calv_n$.

    Then consider $\calv \coloneqq \bigcup_{n \in \bbn} \calv_n$ is a countable set. It is basis for $X$, since every element of $\calv$ is open, and if we take any $x \in U \subset X$ such that $U$ is open in $X$, since $x \in U$, there exists some (WLOG) $B(x, r) \subset U \subset X$. Then there exists some $N$ such that $\frac{1}{N} < \frac{r}{2}$. Consider the finite subcover $\calv_N$, then there has to exist some $B(x', \frac{1}{N}) \ni x$. But $\frac{1}{N} < \frac{r}{2}$ so in fact $B(x', \frac{1}{N}) \subset B(x, r) \subset U$, and $B(x', \frac{1}{N}) \in \calv_n \subset \calv$. So it follows that indeed $\calv$ is a basis for $X$. So $X$ is second countable.

    \textbf{3.} This $f$ is also a perfect map, since if $K \subset X$ is closed then it is compact, so $f(K)$ is compact in Hausdorff $Y$ so $f(K)$ is also closed. Also, $\inv{f}(\{y\})$ for any $y \in Y$ is the continuous inverse of a closed set so is closed, it is in compact $X$ so it is also compact.

    Hence $f$ is a perfect map, so second countability of $X$ implies second countability of $Y$ (per previous HW).
    
\end{solution}

\begin{problem} [\done]
    Fix a prime number $p$ and for each integer $n$, let $\bbz/p^n \bbz$ be the abelian group consisting of integers mod $p^n$.
    \begin{enumerate}
    \item Show that there is a directed system of surjective homomorphisms $\bbz / p^n \bbz \to \bbz / p^{n-1} \bbz$ given by ``reduction mod $p^{n-1}$''.
    \item Let $\bbz_p$ denote the inverse limit of this system, with the inverse limit topology (where each $\bbz/p^n \bbz$ has the discrete topology). Show that $\bbz_p$ is homeomorphic to a Cantor set.
    \item Show that $\bbz_p$ admits the natural structure of an abelian group (compatible with the group structures on all the $\bbz / p^n \bbz$), and with respect to this group structure, the operations of addition and inverse are continuous.
    \end{enumerate}
\end{problem}

\begin{solution}
    \textbf{(a)} Take any $n \in \bbn$. Then we can construct
    \begin{align*}
        f_n: \bbz/p^n \bbz &\to \bbz/p^{n-1} \bbz \\    
            a &\mapsto a \bmod p^{n-1}
    \end{align*}
    It is surjective, since for any $b \in \bbz /p^{n-1} \bbz$, we have that $f(b) = b$. It is also a homomorphism, since $f_n(0) = 0, f_n(a_1 + a_2) = (a_1 + a_2) \bmod p^{n-1} = a_1 \bmod p^{n-1} + a_2 \bmod p^{n-1} = f_n(a_1) + f_n(a_2)$. \qed

    \textbf{(b)} We have \begin{equation*}
    \bbz_p = \varprojlim \bbz/p^n \bbz
    \end{equation*}

    Each $\bbz/p^{n} \bbz$ is a finite, discrete space, so $\bbz_p = \varprojlim \bbz/p^n \bbz$ is a compact and totally disconnected space. So $\bbz_p$ is homeomorphic to a Cantor set. \qed

    \textbf{(c)} Take $x = (x_n), y = (y_n) \in \bbz_p$. Then $inv(x) = (-x_n) \in \bbz_p$ and $x + y \coloneqq (x_n + y_n) \in \bbz_p$, and this is well-defined since each $f_n$ is a group homomorphism. The coordinate wise group operation is exactly the group operation on all the $\bbz/p^n \bbz$.

    Consider the addition and inverse operations:
    \begin{equation*}
    +: \bbz_p \times \bbz_p \to \bbz_p, \quad inv: \bbz_p \to \bbz_p
    \end{equation*}

    To see this, we know that for the inverse operation, on each $\bbz/p^n \bbz$, $inv_n: \bbz/p^n \bbz \to \bbz/p^n \bbz$ is continuous (domain has discrete topology). View another directed system of $\bbz /p^n \bbz$ with inverse limit $\bbz_p$, with the same $f_n$, then from claim in class, there must uniquely exist some $\phi: \bbz_p \to \bbz_p$ such that for all $n$, \begin{equation*}
    \pi_n \phi = inv_n \pi_n
    \end{equation*}
    where $\pi_n: \bbz_p \to \bbz/p^n \bbz$ projects onto the $\bbz /p^n \bbz$ coordinate.
    It then follows that $\phi(x)_n = -x_n \implies \phi = inv$. $\phi$ is continuous so $inv$ is continuous.

    For the addition operation, consider $+(x, y) = x+y$. Then take a basic open set containing $x+y$, namely, $\prod U_n$ for some $U_n \subset \bbz/p^n \bbz$ open such that  $U_n = \bbz/p^n \bbz$ for all but finitely many $\{n_1, \ldots, n_J\}$. Then $x_{n_j} + y_{n_j} \in U_{n_j} \forall j \in [J]$. But then on $\bbz / p^{n_j} \bbz$, $+$ is continuous, so it follows that $\inv{+}(U_{n_j})$ is open in $\bbz / p^{n_j} \bbz \times \bbz / p^n \bbz$. It is a finite set, so it is the finite union of basic open sets \begin{equation*}
    \inv{+}(U_{n_j}) = \bigcup_{\alpha = 1}^{M_j} V_{\alpha, j} \times W_{\alpha, j}
    \end{equation*}
    So $\inv{+}(x + y) = \bigcup_{j \in J} \bigcup_{\alpha = 1}^{M_j} V_{\alpha, j} \times W_{\alpha, j}$ is a union of basic open sets in $\prod \bbz /p^n \bbz$, so is also open in the subspace topology, i.e., open in $\bbz_p$, that contains $x \times y$. It follows that $+$ is indeed continuous.

\end{solution}

\begin{problem}  [\done]
    A space $X$ is \textit{zero-dimensional} if for every point $x$ and any open neighborhood $U$ of $x$, there is a clopen set $V$ with $x$ in $V$ and $V$ in $U$.
    \begin{enumerate}
    \item Show that any zero dimensional Hausdorff space is totally disconnected.
    \item Suppose $X$ is Hausdorff, locally compact and totally disconnected. Show that it is zero dimensional.
    \end{enumerate}
\end{problem}
\begin{solution}
    \textbf{(a)} Let $X$ be a zero dimensional Hausdorff space. Suppose $X$ is not totally disconnected, i.e., there exists some $K \subset X$ connected with more than 1 point, say, $a \neq b \in K$. Since $X$ is Hausdorff, there exists some open $U_a \ni a, U_b \ni b$ such that $U_a \cap U_b = \emptyset$. Since $X$ is zero dimensional, there exists some clopen $V_a$ such that $a \in V_a \subset U_a$. Consequently, $b \not \in V_a$.
    
    Then $V_a \cap K$ is a proper clopen subset of $K$, so $K$ is not connected. \contra \qed

    \textbf{(b)} Let $X$ be Hausdorff, locally compact and totally disconnected. WTS it is zero-dimensional. 

    $X$ is Hausdorff and locally compact, so it is regular. So there exists some open $V$ such that $x \in \cl{V} \subset U$. $\cl{V}$ is closed in Hausdorff $X$, so it is compact. Let us inspect $\cl{V} \ni x$. Since $X$ is totally disconnected, $\cl{V}$ is also disconnected (with the subspace topology). $\cl{V}$ is therefore a compact and Hausdorff space. Following a Corollary from class, we have that $\{x\}$ is a component and $\{x\} \subset U$, so there exists some $W$ clopen such that $x \in W \subset U$.  We've thus found $W$.
\end{solution}


\begin{problem} [\done]
    Let $\calf$ be an equicontinuous family of functions from $[0, 1]$ to $[0, 1]$. Show that there is a continuous function $g: [0, 1] \times [0, 1] \to [0, 1]$ so that for every $f$ in $\calf$ there is some $t \in [0, 1]$ so that $g$ restricted to the horizontal interval $[0, 1] \times t$ agrees with $f$. (Hint: using Ascoli, show $\calf$ is contained in some compact subset $G$ of the space of continuous functions from $[0, 1]$ to $[0, 1]$ with some suitable topology. Show this compact space is metrizable. Deduce that there is a surjective map from the Cantor set to $G$. Use this surjective map to construct the function $g$.)
\end{problem}
\begin{solution}
    For any $a \in [0, 1]$, consider $\calf_a = \{f(a) : f \in \calf\}$, then its closure is closed in compact $[0, 1]$, so is compact. Thus, using Ascoli's Theorem, we have that $\calf$ is contained in a compact subspace $G$ of $\calc([0, 1], [0, 1])$ in the topology of compact convergence.

    Since $[0, 1]$ is compact and $[0, 1]$ is metric, the sets $\{B([0, 1], f, \epsilon)\} = \{f' \in Y^X \mid \sup_{x \in [0, 1]} \{d(fx, f'x) < \epsilon\}\}$ forms the basis for $\calc{([0, 1], [0, 1])}$. But this is exactly the uniform topology induced by the uniform metric. So it is metrizable!

    Therefore $G$ is a compact metric space. So there is a continuous surjective map $h: \calc \to G$, from the middle thirds Cantor set into $G$. Then for any $f \in \calf \subset G$, there exists some $\alpha \in \calc \subset [0, 1]$ such that $h(\alpha) = f$. Then define $g\mid_{[0, 1] \times \alpha} = h(\alpha) = f$.

    So we've define $g$ on $[0, 1] \times \calc \to [0, 1]$, it is clearly continuous since $f$ is continuous. Also, $\calc$ is compact, so $[0, 1] \times \calc$ is compact in Hausdorff $[0, 1] \times [0, 1]$ so it's closed. Using Tietze extension theorem, we can extend $g$ to $[0, 1] \times [0, 1] \to [0, 1]$. And we've found our $g$. 
\end{solution}

\end{document}