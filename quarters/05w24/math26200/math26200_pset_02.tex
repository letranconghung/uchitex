\documentclass[a4paper, 12pt]{article}
%%%%%%%%%%%%%%%%%%%%%%%%%%%%%%%%%%%%%%%%%%%%%%%%%%%%%%%%%%%%%%%%%%%%%%%%%%%%%%%
%                                Basic Packages                               %
%%%%%%%%%%%%%%%%%%%%%%%%%%%%%%%%%%%%%%%%%%%%%%%%%%%%%%%%%%%%%%%%%%%%%%%%%%%%%%%

% Gives us multiple colors.
\usepackage[usenames,dvipsnames,pdftex]{xcolor}
% Lets us style link colors.
\usepackage{hyperref}
% Lets us import images and graphics.
\usepackage{graphicx}
% Lets us use figures in floating environments.
\usepackage{float}
% Lets us create multiple columns.
\usepackage{multicol}
% Gives us better math syntax.
\usepackage{amsmath,amsfonts,mathtools,amsthm,amssymb}
% Lets us strikethrough text.
\usepackage{cancel}
% Lets us edit the caption of a figure.
\usepackage{caption}
% Lets us import pdf directly in our tex code.
\usepackage{pdfpages}
% Lets us do algorithm stuff.
\usepackage[ruled,vlined,linesnumbered]{algorithm2e}
% Use a smiley face for our qed symbol.
\usepackage{tikzsymbols}
% \usepackage{fullpage} %%smaller margins
\usepackage[shortlabels]{enumitem}

\setlist[enumerate]{font={\bfseries}} % global settings, for all lists

\usepackage{setspace}
\usepackage[margin=1in, headsep=12pt]{geometry}
\usepackage{wrapfig}
\usepackage{listings}
\usepackage{parskip}

\definecolor{codegreen}{rgb}{0,0.6,0}
\definecolor{codegray}{rgb}{0.5,0.5,0.5}
\definecolor{codepurple}{rgb}{0.58,0,0.82}
\definecolor{backcolour}{rgb}{0.95,0.95,0.95}

\lstdefinestyle{mystyle}{
    backgroundcolor=\color{backcolour},   
    commentstyle=\color{codegreen},
    keywordstyle=\color{magenta},
    numberstyle=\tiny\color{codegray},
    stringstyle=\color{codepurple},
    basicstyle=\ttfamily\footnotesize,
    breakatwhitespace=false,         
    breaklines=true,                 
    captionpos=b,                    
    keepspaces=true,                 
    numbers=left,                    
    numbersep=5pt,                  
    showspaces=false,                
    showstringspaces=false,
    showtabs=false,                  
    tabsize=2,
    numbers=none
}

\lstset{style=mystyle}
\def\class{article}


%%%%%%%%%%%%%%%%%%%%%%%%%%%%%%%%%%%%%%%%%%%%%%%%%%%%%%%%%%%%%%%%%%%%%%%%%%%%%%%
%                                Basic Settings                               %
%%%%%%%%%%%%%%%%%%%%%%%%%%%%%%%%%%%%%%%%%%%%%%%%%%%%%%%%%%%%%%%%%%%%%%%%%%%%%%%

%%%%%%%%%%%%%
%  Symbols  %
%%%%%%%%%%%%%

\let\implies\Rightarrow
\let\impliedby\Leftarrow
\let\iff\Leftrightarrow
\let\epsilon\varepsilon
%%%%%%%%%%%%
%  Tables  %
%%%%%%%%%%%%

\setlength{\tabcolsep}{5pt}
\renewcommand\arraystretch{1.5}

%%%%%%%%%%%%%%
%  SI Unitx  %
%%%%%%%%%%%%%%

\usepackage{siunitx}
\sisetup{locale = FR}

%%%%%%%%%%
%  TikZ  %
%%%%%%%%%%

\usepackage[framemethod=TikZ]{mdframed}
\usepackage{tikz}
\usepackage{tikz-cd}
\usepackage{tikzsymbols}

\usetikzlibrary{intersections, angles, quotes, calc, positioning}
\usetikzlibrary{arrows.meta}

\tikzset{
    force/.style={thick, {Circle[length=2pt]}-stealth, shorten <=-1pt}
}

%%%%%%%%%%%%%%%
%  PGF Plots  %
%%%%%%%%%%%%%%%

\usepackage{pgfplots}
\pgfplotsset{width=10cm, compat=newest}

%%%%%%%%%%%%%%%%%%%%%%%
%  Center Title Page  %
%%%%%%%%%%%%%%%%%%%%%%%

\usepackage{titling}
\renewcommand\maketitlehooka{\null\mbox{}\vfill}
\renewcommand\maketitlehookd{\vfill\null}

%%%%%%%%%%%%%%%%%%%%%%%%%%%%%%%%%%%%%%%%%%%%%%%%%%%%%%%
%  Create a grey background in the middle of the PDF  %
%%%%%%%%%%%%%%%%%%%%%%%%%%%%%%%%%%%%%%%%%%%%%%%%%%%%%%%

\usepackage{eso-pic}
\newcommand\definegraybackground{
    \definecolor{reallylightgray}{HTML}{FAFAFA}
    \AddToShipoutPicture{
        \ifthenelse{\isodd{\thepage}}{
            \AtPageLowerLeft{
                \put(\LenToUnit{\dimexpr\paperwidth-222pt},0){
                    \color{reallylightgray}\rule{222pt}{297mm}
                }
            }
        }
        {
            \AtPageLowerLeft{
                \color{reallylightgray}\rule{222pt}{297mm}
            }
        }
    }
}

%%%%%%%%%%%%%%%%%%%%%%%%
%  Modify Links Color  %
%%%%%%%%%%%%%%%%%%%%%%%%

\hypersetup{
    % Enable highlighting links.
    colorlinks,
    % Change the color of links to blue.
    urlcolor=blue,
    % Change the color of citations to black.
    citecolor={black},
    % Change the color of url's to blue with some black.
    linkcolor={blue!80!black}
}

%%%%%%%%%%%%%%%%%%
% Fix WrapFigure %
%%%%%%%%%%%%%%%%%%

\newcommand{\wrapfill}{\par\ifnum\value{WF@wrappedlines}>0
        \parskip=0pt
        \addtocounter{WF@wrappedlines}{-1}%
        \null\vspace{\arabic{WF@wrappedlines}\baselineskip}%
        \WFclear
    \fi}

%%%%%%%%%%%%%%%%%
% Multi Columns %
%%%%%%%%%%%%%%%%%

\let\multicolmulticols\multicols
\let\endmulticolmulticols\endmulticols

\RenewDocumentEnvironment{multicols}{mO{}}
{%
    \ifnum#1=1
        #2%
    \else % More than 1 column
        \multicolmulticols{#1}[#2]
    \fi
}
{%
    \ifnum#1=1
    \else % More than 1 column
        \endmulticolmulticols
    \fi
}

\newlength{\thickarrayrulewidth}
\setlength{\thickarrayrulewidth}{5\arrayrulewidth}


%%%%%%%%%%%%%%%%%%%%%%%%%%%%%%%%%%%%%%%%%%%%%%%%%%%%%%%%%%%%%%%%%%%%%%%%%%%%%%%
%                           School Specific Commands                          %
%%%%%%%%%%%%%%%%%%%%%%%%%%%%%%%%%%%%%%%%%%%%%%%%%%%%%%%%%%%%%%%%%%%%%%%%%%%%%%%

%%%%%%%%%%%%%%%%%%%%%%%%%%%
%  Initiate New Counters  %
%%%%%%%%%%%%%%%%%%%%%%%%%%%

\newcounter{lecturecounter}

%%%%%%%%%%%%%%%%%%%%%%%%%%
%  Helpful New Commands  %
%%%%%%%%%%%%%%%%%%%%%%%%%%

\makeatletter

\newcommand\resetcounters{
    % Reset the counters for subsection, subsubsection and the definition
    % all the custom environments.
    \setcounter{subsection}{0}
    \setcounter{subsubsection}{0}
    \setcounter{definition0}{0}
    \setcounter{paragraph}{0}
    \setcounter{theorem}{0}
    \setcounter{claim}{0}
    \setcounter{corollary}{0}
    \setcounter{proposition}{0}
    \setcounter{lemma}{0}
    \setcounter{exercise}{0}
    \setcounter{problem}{0}
    
    \setcounter{subparagraph}{0}
    % \@ifclasswith\class{nocolor}{
    %     \setcounter{definition}{0}
    % }{}
}

%%%%%%%%%%%%%%%%%%%%%
%  Lecture Command  %
%%%%%%%%%%%%%%%%%%%%%

\usepackage{xifthen}

% EXAMPLE:
% 1. \lecture{Oct 17 2022 Mon (08:46:48)}{Lecture Title}
% 2. \lecture[4]{Oct 17 2022 Mon (08:46:48)}{Lecture Title}
% 3. \lecture{Oct 17 2022 Mon (08:46:48)}{}
% 4. \lecture[4]{Oct 17 2022 Mon (08:46:48)}{}
% Parameters:
% 1. (Optional) lecture number.
% 2. Time and date of lecture.
% 3. Lecture Title.
\def\@lecture{}
\def\@lectitle{}
\def\@leccount{}
\newcommand\lecture[3]{
    \newpage

    % Check if user passed the lecture title or not.
    \def\@leccount{Lecture #1}
    \ifthenelse{\isempty{#3}}{
        \def\@lecture{Lecture #1}
        \def\@lectitle{Lecture #1}
    }{
        \def\@lecture{Lecture #1: #3}
        \def\@lectitle{#3}
    }

    \setcounter{section}{#1}
    \renewcommand\thesubsection{#1.\arabic{subsection}}
    
    \phantomsection
    \addcontentsline{toc}{section}{\@lecture}
    \resetcounters

    \begin{mdframed}
        \begin{center}
            \Large \textbf{\@leccount}
            
            \vspace*{0.2cm}
            
            \large \@lectitle
            
            
            \vspace*{0.2cm}

            \normalsize #2
        \end{center}
    \end{mdframed}

}

%%%%%%%%%%%%%%%%%%%%
%  Import Figures  %
%%%%%%%%%%%%%%%%%%%%

\usepackage{import}
\pdfminorversion=7

% EXAMPLE:
% 1. \incfig{limit-graph}
% 2. \incfig[0.4]{limit-graph}
% Parameters:
% 1. The figure name. It should be located in figures/NAME.tex_pdf.
% 2. (Optional) The width of the figure. Example: 0.5, 0.35.
\newcommand\incfig[2][1]{%
    \def\svgwidth{#1\columnwidth}
    \import{./figures/}{#2.pdf_tex}
}

\begingroup\expandafter\expandafter\expandafter\endgroup
\expandafter\ifx\csname pdfsuppresswarningpagegroup\endcsname\relax
\else
    \pdfsuppresswarningpagegroup=1\relax
\fi

%%%%%%%%%%%%%%%%%
% Fancy Headers %
%%%%%%%%%%%%%%%%%

\usepackage{fancyhdr}

% Force a new page.
\newcommand\forcenewpage{\clearpage\mbox{~}\clearpage\newpage}

% This command makes it easier to manage my headers and footers.
\newcommand\createintro{
    % Use roman page numbers (e.g. i, v, vi, x, ...)
    \pagenumbering{roman}

    % Display the page style.
    \maketitle
    % Make the title pagestyle empty, meaning no fancy headers and footers.
    \thispagestyle{empty}
    % Create a newpage.
    \newpage

    % Input the intro.tex page if it exists.
    \IfFileExists{intro.tex}{ % If the intro.tex file exists.
        % Input the intro.tex file.
        \textbf{Course}: MATH 16300: Honors Calculus III

\textbf{Section}: 43

\textbf{Professor}: Minjae Park

\textbf{At}: The University of Chicago

\textbf{Quarter}: Spring 2023

\textbf{Course materials}: Calculus by Spivak (4th Edition), Calculus On Manifolds by Spivak

\vspace{1cm}
\textbf{Disclaimer}: This document will inevitably contain some mistakes, both simple typos and serious logical and mathematical errors. Take what you read with a grain of salt as it is made by an undergraduate student going through the learning process himself. If you do find any error, I would really appreciate it if you can let me know by email at \href{mailto:conghungletran@gmail.com}{conghungletran@gmail.com}.

        % Make the pagestyle fancy for the intro.tex page.
        \pagestyle{fancy}

        % Remove the line for the header.
        \renewcommand\headrulewidth{0pt}

        % Remove all header stuff.
        \fancyhead{}

        % Add stuff for the footer in the center.
        % \fancyfoot[C]{
        %   \textit{For more notes like this, visit
        %   \href{\linktootherpages}{\shortlinkname}}. \\
        %   \vspace{0.1cm}
        %   \hrule
        %   \vspace{0.1cm}
        %   \@author, \\
        %   \term: \academicyear, \\
        %   Last Update: \@date, \\
        %   \faculty
        % }

        \newpage
    }{ % If the intro.tex file doesn't exist.
        % Force a \newpageage.
        % \forcenewpage
        \newpage
    }

    % Remove the center stuff we did above, and replace it with just the page
    % number, which is still in roman numerals.
    \fancyfoot[C]{\thepage}
    % Add the table of contents.
    \tableofcontents
    % Force a new page.
    \newpage

    % Move the page numberings back to arabic, from roman numerals.
    \pagenumbering{arabic}
    % Set the page number to 1.
    \setcounter{page}{1}

    % Add the header line back.
    \renewcommand\headrulewidth{0.4pt}
    % In the top right, add the lecture title.
    \fancyhead[R]{\footnotesize \@lecture}
    % In the top left, add the author name.
    \fancyhead[L]{\footnotesize \@author}
    % In the bottom center, add the page.
    \fancyfoot[C]{\thepage}
    % Add a nice gray background in the middle of all the upcoming pages.
    % \definegraybackground
}

\makeatother


%%%%%%%%%%%%%%%%%%%%%%%%%%%%%%%%%%%%%%%%%%%%%%%%%%%%%%%%%%%%%%%%%%%%%%%%%%%%%%%
%                               Custom Commands                               %
%%%%%%%%%%%%%%%%%%%%%%%%%%%%%%%%%%%%%%%%%%%%%%%%%%%%%%%%%%%%%%%%%%%%%%%%%%%%%%%

%%%%%%%%%%%%
%  Circle  %
%%%%%%%%%%%%

\newcommand*\circled[1]{\tikz[baseline= (char.base)]{
        \node[shape=circle,draw,inner sep=1pt] (char) {#1};}
}

%%%%%%%%%%%%%%%%%%%
%  Todo Commands  %
%%%%%%%%%%%%%%%%%%%

% \usepackage{xargs}
% \usepackage[colorinlistoftodos]{todonotes}

% \makeatletter

% \@ifclasswith\class{working}{
%     \newcommandx\unsure[2][1=]{\todo[linecolor=red,backgroundcolor=red!25,bordercolor=red,#1]{#2}}
%     \newcommandx\change[2][1=]{\todo[linecolor=blue,backgroundcolor=blue!25,bordercolor=blue,#1]{#2}}
%     \newcommandx\info[2][1=]{\todo[linecolor=OliveGreen,backgroundcolor=OliveGreen!25,bordercolor=OliveGreen,#1]{#2}}
%     \newcommandx\improvement[2][1=]{\todo[linecolor=Plum,backgroundcolor=Plum!25,bordercolor=Plum,#1]{#2}}

%     \newcommand\listnotes{
%         \newpage
%         \listoftodos[Notes]
%     }
% }{
%     \newcommandx\unsure[2][1=]{}
%     \newcommandx\change[2][1=]{}
%     \newcommandx\info[2][1=]{}
%     \newcommandx\improvement[2][1=]{}

%     \newcommand\listnotes{}
% }

% \makeatother

%%%%%%%%%%%%%
%  Correct  %
%%%%%%%%%%%%%

% EXAMPLE:
% 1. \correct{INCORRECT}{CORRECT}
% Parameters:
% 1. The incorrect statement.
% 2. The correct statement.
\definecolor{correct}{HTML}{009900}
\newcommand\correct[2]{{\color{red}{#1 }}\ensuremath{\to}{\color{correct}{ #2}}}


%%%%%%%%%%%%%%%%%%%%%%%%%%%%%%%%%%%%%%%%%%%%%%%%%%%%%%%%%%%%%%%%%%%%%%%%%%%%%%%
%                                 Environments                                %
%%%%%%%%%%%%%%%%%%%%%%%%%%%%%%%%%%%%%%%%%%%%%%%%%%%%%%%%%%%%%%%%%%%%%%%%%%%%%%%

\usepackage{varwidth}
\usepackage{thmtools}
\usepackage[most,many,breakable]{tcolorbox}

\tcbuselibrary{theorems,skins,hooks}
\usetikzlibrary{arrows,calc,shadows.blur}

%%%%%%%%%%%%%%%%%%%
%  Define Colors  %
%%%%%%%%%%%%%%%%%%%

% color prototype
% \definecolor{color}{RGB}{45, 111, 177}

% ESSENTIALS: 
\definecolor{myred}{HTML}{c74540}
\definecolor{myblue}{HTML}{072b85}
\definecolor{mygreen}{HTML}{388c46}
\definecolor{myblack}{HTML}{000000}

\colorlet{definition_color}{myred}

\colorlet{theorem_color}{myblue}
\colorlet{lemma_color}{myblue}
\colorlet{prop_color}{myblue}
\colorlet{corollary_color}{myblue}
\colorlet{claim_color}{myblue}

\colorlet{proof_color}{myblack}
\colorlet{example_color}{myblack}
\colorlet{exercise_color}{myblack}

% MISCS: 
%%%%%%%%%%%%%%%%%%%%%%%%%%%%%%%%%%%%%%%%%%%%%%%%%%%%%%%%%
%  Create Environments Styles Based on Given Parameter  %
%%%%%%%%%%%%%%%%%%%%%%%%%%%%%%%%%%%%%%%%%%%%%%%%%%%%%%%%%

% \mdfsetup{skipabove=1em,skipbelow=0em}

%%%%%%%%%%%%%%%%%%%%%%
%  Helpful Commands  %
%%%%%%%%%%%%%%%%%%%%%%

% EXAMPLE:
% 1. \createnewtheoremstyle{thmdefinitionbox}{}{}
% 2. \createnewtheoremstyle{thmtheorembox}{}{}
% 3. \createnewtheoremstyle{thmproofbox}{qed=\qedsymbol}{
%       rightline=false, topline=false, bottomline=false
%    }
% Parameters:
% 1. Theorem name.
% 2. Any extra parameters to pass directly to declaretheoremstyle.
% 3. Any extra parameters to pass directly to mdframed.
\newcommand\createnewtheoremstyle[3]{
    \declaretheoremstyle[
        headfont=\bfseries\sffamily, bodyfont=\normalfont, #2,
        mdframed={
                #3,
            },
    ]{#1}
}

% EXAMPLE:
% 1. \createnewcoloredtheoremstyle{thmdefinitionbox}{definition}{}{}
% 2. \createnewcoloredtheoremstyle{thmexamplebox}{example}{}{
%       rightline=true, leftline=true, topline=true, bottomline=true
%     }
% 3. \createnewcoloredtheoremstyle{thmproofbox}{proof}{qed=\qedsymbol}{backgroundcolor=white}
% Parameters:
% 1. Theorem name.
% 2. Color of theorem.
% 3. Any extra parameters to pass directly to declaretheoremstyle.
% 4. Any extra parameters to pass directly to mdframed.

% change backgroundcolor to #2!5 if user wants a colored backdrop to theorem environments. It's a cool color theme, but there's too much going on in the page.
\newcommand\createnewcoloredtheoremstyle[4]{
    \declaretheoremstyle[
        headfont=\bfseries\sffamily\color{#2},
        bodyfont=\normalfont,
        headpunct=,
        headformat = \NAME~\NUMBER\NOTE \hfill\smallskip\linebreak,
        #3,
        mdframed={
                outerlinewidth=0.75pt,
                rightline=false,
                leftline=false,
                topline=false,
                bottomline=false,
                backgroundcolor=white,
                skipabove = 5pt,
                skipbelow = 0pt,
                linecolor=#2,
                innertopmargin = 0pt,
                innerbottommargin = 0pt,
                innerrightmargin = 4pt,
                innerleftmargin= 6pt,
                leftmargin = -6pt,
                #4,
            },
    ]{#1}
}



%%%%%%%%%%%%%%%%%%%%%%%%%%%%%%%%%%%
%  Create the Environment Styles  %
%%%%%%%%%%%%%%%%%%%%%%%%%%%%%%%%%%%

\makeatletter
\@ifclasswith\class{nocolor}{
    % Environments without color.

    % ESSENTIALS:
    \createnewtheoremstyle{thmdefinitionbox}{}{}
    \createnewtheoremstyle{thmtheorembox}{}{}
    \createnewtheoremstyle{thmproofbox}{qed=\qedsymbol}{}
    \createnewtheoremstyle{thmcorollarybox}{}{}
    \createnewtheoremstyle{thmlemmabox}{}{}
    \createnewtheoremstyle{thmclaimbox}{}{}
    \createnewtheoremstyle{thmexamplebox}{}{}

    % MISCS: 
    \createnewtheoremstyle{thmpropbox}{}{}
    \createnewtheoremstyle{thmexercisebox}{}{}
    \createnewtheoremstyle{thmexplanationbox}{}{}
    \createnewtheoremstyle{thmremarkbox}{}{}
    
    % STYLIZED MORE BELOW
    \createnewtheoremstyle{thmquestionbox}{}{}
    \createnewtheoremstyle{thmsolutionbox}{qed=\qedsymbol}{}
}{
    % Environments with color.

    % ESSENTIALS: definition, theorem, proof, corollary, lemma, claim, example
    \createnewcoloredtheoremstyle{thmdefinitionbox}{definition_color}{}{leftline=false}
    \createnewcoloredtheoremstyle{thmtheorembox}{theorem_color}{}{leftline=false}
    \createnewcoloredtheoremstyle{thmproofbox}{proof_color}{qed=\qedsymbol}{}
    \createnewcoloredtheoremstyle{thmcorollarybox}{corollary_color}{}{leftline=false}
    \createnewcoloredtheoremstyle{thmlemmabox}{lemma_color}{}{leftline=false}
    \createnewcoloredtheoremstyle{thmpropbox}{prop_color}{}{leftline=false}
    \createnewcoloredtheoremstyle{thmclaimbox}{claim_color}{}{leftline=false}
    \createnewcoloredtheoremstyle{thmexamplebox}{example_color}{}{}
    \createnewcoloredtheoremstyle{thmexplanationbox}{example_color}{qed=\qedsymbol}{}
    \createnewcoloredtheoremstyle{thmremarkbox}{theorem_color}{}{}

    \createnewcoloredtheoremstyle{thmmiscbox}{black}{}{}

    \createnewcoloredtheoremstyle{thmexercisebox}{exercise_color}{}{}
    \createnewcoloredtheoremstyle{thmproblembox}{theorem_color}{}{leftline=false}
    \createnewcoloredtheoremstyle{thmsolutionbox}{mygreen}{qed=\qedsymbol}{}
}
\makeatother

%%%%%%%%%%%%%%%%%%%%%%%%%%%%%
%  Create the Environments  %
%%%%%%%%%%%%%%%%%%%%%%%%%%%%%
\declaretheorem[numberwithin=section, style=thmdefinitionbox,     name=Definition]{definition}
\declaretheorem[numberwithin=section, style=thmtheorembox,     name=Theorem]{theorem}
\declaretheorem[numbered=no,          style=thmexamplebox,     name=Example]{example}
\declaretheorem[numberwithin=section, style=thmtheorembox,       name=Claim]{claim}
\declaretheorem[numberwithin=section, style=thmcorollarybox,   name=Corollary]{corollary}
\declaretheorem[numberwithin=section, style=thmpropbox,        name=Proposition]{proposition}
\declaretheorem[numberwithin=section, style=thmlemmabox,       name=Lemma]{lemma}
\declaretheorem[numberwithin=section, style=thmexercisebox,    name=Exercise]{exercise}
\declaretheorem[numbered=no,          style=thmproofbox,       name=Proof]{proof0}
\declaretheorem[numbered=no,          style=thmexplanationbox, name=Explanation]{explanation}
\declaretheorem[numbered=no,          style=thmsolutionbox,    name=Solution]{solution}
\declaretheorem[numberwithin=section,          style=thmproblembox,     name=Problem]{problem}
\declaretheorem[numbered=no,          style=thmmiscbox,    name=Intuition]{intuition}
\declaretheorem[numbered=no,          style=thmmiscbox,    name=Goal]{goal}
\declaretheorem[numbered=no,          style=thmmiscbox,    name=Recall]{recall}
\declaretheorem[numbered=no,          style=thmmiscbox,    name=Motivation]{motivation}
\declaretheorem[numbered=no,          style=thmmiscbox,    name=Remark]{remark}
\declaretheorem[numbered=no,          style=thmmiscbox,    name=Observe]{observe}
\declaretheorem[numbered=no,          style=thmmiscbox,    name=Question]{question}


%%%%%%%%%%%%%%%%%%%%%%%%%%%%
%  Edit Proof Environment  %
%%%%%%%%%%%%%%%%%%%%%%%%%%%%

\renewenvironment{proof}[2][\proofname]{
    % \vspace{-12pt}
    \begin{proof0} [#2]
        }{\end{proof0}}

\theoremstyle{definition}

\newtheorem*{notation}{Notation}
\newtheorem*{previouslyseen}{As previously seen}
\newtheorem*{property}{Property}
% \newtheorem*{intuition}{Intuition}
% \newtheorem*{goal}{Goal}
% \newtheorem*{recall}{Recall}
% \newtheorem*{motivation}{Motivation}
% \newtheorem*{remark}{Remark}
% \newtheorem*{observe}{Observe}

\author{Hung C. Le Tran}


%%%% MATH SHORTHANDS %%%%
%% blackboard bold math capitals
\DeclareMathOperator*{\esssup}{ess\,sup}
\DeclareMathOperator*{\Hom}{Hom}
\newcommand{\bbf}{\mathbb{F}}
\newcommand{\bbn}{\mathbb{N}}
\newcommand{\bbq}{\mathbb{Q}}
\newcommand{\bbr}{\mathbb{R}}
\newcommand{\bbz}{\mathbb{Z}}
\newcommand{\bbc}{\mathbb{C}}
\newcommand{\bbk}{\mathbb{K}}
\newcommand{\bbm}{\mathbb{M}}
\newcommand{\bbp}{\mathbb{P}}
\newcommand{\bbe}{\mathbb{E}}

\newcommand{\bfw}{\mathbf{w}}
\newcommand{\bfx}{\mathbf{x}}
\newcommand{\bfX}{\mathbf{X}}
\newcommand{\bfy}{\mathbf{y}}
\newcommand{\bfyhat}{\mathbf{\hat{y}}}

\newcommand{\calb}{\mathcal{B}}
\newcommand{\calf}{\mathcal{F}}
\newcommand{\calt}{\mathcal{T}}
\newcommand{\call}{\mathcal{L}}
\renewcommand{\phi}{\varphi}

% Universal Math Shortcuts
\newcommand{\st}{\hspace*{2pt}\text{s.t.}\hspace*{2pt}}
\newcommand{\pffwd}{\hspace*{2pt}\fbox{\(\Rightarrow\)}\hspace*{10pt}}
\newcommand{\pfbwd}{\hspace*{2pt}\fbox{\(\Leftarrow\)}\hspace*{10pt}}
\newcommand{\contra}{\ensuremath{\Rightarrow\Leftarrow}}
\newcommand{\cvgn}{\xrightarrow{n \to \infty}}
\newcommand{\cvgj}{\xrightarrow{j \to \infty}}

\newcommand{\im}{\mathrm{im}}
\newcommand{\innerproduct}[2]{\langle #1, #2 \rangle}
\newcommand*{\conj}[1]{\overline{#1}}

% https://tex.stackexchange.com/questions/438612/space-between-exists-and-forall
% https://tex.stackexchange.com/questions/22798/nice-looking-empty-set
\let\oldforall\forall
\renewcommand{\forall}{\;\oldforall\; }
\let\oldexist\exists
\renewcommand{\exists}{\;\oldexist\; }
\newcommand\existu{\;\oldexist!\: }
\let\oldemptyset\emptyset
\let\emptyset\varnothing


\renewcommand{\_}[1]{\underline{#1}}
\DeclarePairedDelimiter{\abs}{\lvert}{\rvert}
\DeclarePairedDelimiter{\norm}{\lVert}{\rVert}
\DeclarePairedDelimiter\ceil{\lceil}{\rceil}
\DeclarePairedDelimiter\floor{\lfloor}{\rfloor}
\setlength\parindent{0pt}
\setlength{\headheight}{12.0pt}
\addtolength{\topmargin}{-12.0pt}


% Default skipping, change if you want more spacing
% \thinmuskip=3mu
% \medmuskip=4mu plus 2mu minus 4mu
% \thickmuskip=5mu plus 5mu

% \DeclareMathOperator{\ext}{ext}
% \DeclareMathOperator{\bridge}{bridge}
\title{MATH 26200: Point-Set Topology \\ \large Problem Set 2}
\date{17 Jan 2024}
\author{Hung Le Tran}
\begin{document}
\maketitle
\setcounter{section}{2}
\textbf{Textbook:} Munkres, \textit{Topology}
\begin{problem} [17.5 \redtext{done}]
    Let $X$ be an ordered set in the order topology. Show that $\overline{(a, b)} \subset [a, b]$. Under what conditions does equality hold?
\end{problem}
\begin{solution}
    Let $x \in \overline{(a, b)}$. WTS $x \in [a, b] \Leftrightarrow x \geq a, x \leq b$.

    We first prove that $x \geq a$. Suppose not, that is, $x < a$. Then there exists a neighborhood of $x$ that does not intersect $(a, b)$, namely, the open ray $(-\infty, a)$, which contains $x$, and: \begin{equation*}
    (-\infty, a) \cap (a, b) = \emptyset
    \end{equation*}
    This is a contradiction, since $x \in \overline{(a, b)}$. Therefore $x \geq a$.

    Similarly, $x \leq b$. Therefore $x \in [a, b]$, which implies $\overline{(a, b)} \subset [a, b]$.

    Equality holds when both $a$ and $b$ are limit points of $(a, b)$.
\end{solution}

\begin{problem} [17.6 \redtext{done}]
Let $A, B$ and $A_\alpha$ denote subsets of a space $X$. Prove the following.
\begin{enumerate}
\item If $A \subset B$, then $\overline{A} \subset \overline{B}$
\item $\overline{A \cup B} = \overline{A} \cup \overline{B}$
\item $\overline{\bigcup A_\alpha} \supset \bigcup \overline{A_\alpha}$, and give an example where equality fails.
\end{enumerate}
\end{problem}
\begin{solution}
    \textbf{(a)} Let $x \in \overline{A}$, WTS $x \in \overline{B}$. Take any open neighborhood $U$ of $x$.

    Since $x \in \overline{A}$, $U \cap A \neq \emptyset$. But $A \subset B \implies U \cap B \neq \emptyset$ too.

    This means that every open neighborhood $U$ of $x$ has non-empty intersection with $B$.

    Therefore $x \in \overline{B}$. \qed

    \textbf{(b) 1.} WTS $\overline{A \cup B} \subset \overline{A} \cup \overline{B}$.

    $\overline{A}, \overline{B}$ are closed in $X$. It follows that $\overline{A} \cup \overline{B}$ is also closed in $X$.

    Also, $A \subset \overline{A}, B \subset \overline{B} \implies A \cup B \subset \overline{A} \cup \overline{B}$.

    $\overline{A \cup B}$ is the smallest subset that contains $A \cup B$, so $\overline{A \cup B} \subset \overline{A} \cup \overline{B}$.

    \textbf{2.} WTS $\overline{A} \cup \overline{B} \subset \overline{A \cup B}$.

    Take $x \in \overline{A} \cup \overline{B}$. WLOG, $x \in \overline{A}$. This implies every open neighborhood $U$ of $x$ satisfies:\begin{equation*}
    U \cap A  \neq \emptyset \implies U \cap (A \cup B) \neq \emptyset
    \end{equation*}
    It follows that $x \in \overline{A \cup B}$. Therefore $\overline{A} \cup \overline{B} \subset \overline{A \cup B}$ as required.

    \textbf{3.} From \textbf{1., 2.}, it follows that $\overline{A \cup B} = \overline{A} \cup \overline{B}$. \qed

    \textbf{(c)} Take $x \in \bigcup \overline{A_\alpha}$. WTS $x \in \overline{\bigcup A_\alpha}$.

    Since $x \in \bigcup \overline{A_\alpha}$, $x \in \overline{A_\beta}$ for some $\beta$.

    It follows that every open neighborhood $U$ of $x$ satisfies: \begin{equation*}
    U \cap A_\beta \neq \emptyset \implies U \cap \bigcup A_\alpha \neq \emptyset
    \end{equation*}
    and therefore $x \in \overline{\bigcup A_\alpha}$. It follows that $\overline{\bigcup A_\alpha} \supset \bigcup \overline{A_\alpha}$ as required.

    An example of when equality fails:
    \begin{equation*}
    A_\alpha \coloneqq \{\alpha\} \forall \alpha \in \bbq
    \end{equation*}
    Then $\overline{A_\alpha} =\{ \alpha \}\implies \bigcup \overline{A_\alpha} = \bbq$, while \begin{equation*}
    \overline{\bigcup A_\alpha} = \overline{\bbq} = \bbr \neq \bbq
    \end{equation*}
\end{solution}

\begin{problem} [17.13 \redtext{done}]
    Show that $X$ is Hausdorff iff the \textit{diagonal} $\Delta = \{x \times x \mid x \in X\}$ is closed in $X \times X$.
\end{problem}
\begin{solution}
    \pffwd By hypothesis, $X$ is Hausdorff. WTS $\Delta = \{x \times x \mid x \in X\}$ is closed in $X \times X$. Equivalently, WTS $\cl{\Delta} = \Delta$.

    Suppose not, i.e., that there exists $y \times z \in \cl{\Delta} - \Delta$. $y \times z \not \in \Delta$ so $y \neq z$. Since $X$ is Hausdorff, there exists $U \ni y, V \ni z$ open such that $U \cap V = \emptyset$.

    Then, since $y \times z$ is a limit point of $\Delta$, and $U \times V$ is open $X \times X$, $y \times z \in U \times V$, it follows that \begin{equation*}
    U \times V \cap \Delta \neq \emptyset
    \end{equation*}
    Say, $x_0 \times x_o \in (U \times V) \cap \Delta$. Then $x_0 \in U, x_0 \in V \implies U \cap V \neq \emptyset, \contra$.

    Therefore $\Delta$ is closed in $X \times X$ as required. \qed

    \pfbwd By hypothesis, $\Delta$ is closed in $X \times X$. This implies $X \times X - \Delta = \{y \times z : y, z \in X;y \neq z\}$ is open.

    Take $y, z \in X;y \neq z $. Then $y \times z \in X \times X - \Delta$, so there exists $U \times V$ with $U, V$ open $X$ such that $y \times z \in U \times V \subset X \times X - \Delta$, i.e., $U \times V \cap \Delta = \emptyset$.

    WTS $U \cap V = \emptyset$. Suppose not, then there exists $w \in U \cap V$, then $w \times w \in U \times V$. But $w \times w \in \Delta$ too, so $U \times V \cap \Delta \neq \emptyset, \contra$. Therefore $U \cap V = \emptyset$.

    Therefore we've demonstrated $U, V$ open with $U \ni y, V \ni z, U \cap V = \emptyset$ for all $y, z \in X; y \neq z$. $X$ is therefore Hausdorff.
\end{solution}

\begin{problem} [18.12 \redtext{done}]
Let $F: \bbr \times \bbr \to \bbr$ be defined by the equation \begin{equation*}
F(x \times y) = \begin{cases}
\dfrac{xy}{x^2 + y^2} & \:\text{if}\: x \times y \neq 0 \times 0 \\
0 & \:\text{if}\: x \times y = 0 \times 0
\end{cases}
\end{equation*}
\begin{enumerate}
\item Show that $F$ is continuous in each variable separately.
\item Compute the function $g: \bbr \to \bbr$ defined by $g(x) = F(x \times x)$.
\item Show that $F$ is not continuous.
\end{enumerate}
\end{problem}
\begin{solution}
    \textbf{(a)} We write $F$ as a function in $x$:
    \begin{equation*}
    G_y(x) = G(x; y) = F(x \times y) = \begin{cases}
        \dfrac{xy}{x^2 + y^2} & \:\text{if}\: x \neq 0 \\
        0 & \:\text{if}\: x = 0
    \end{cases}
    \end{equation*}

    Then for $x \neq 0$, $G_y$ is clearly continuous at $x$, since it is the quotient of continuous functions in $x$, and the denominator $x^2 + y^2 > 0$.

    It remains to show that $G_y$ is continuous at $x = 0$. Fix $\epsilon > 0$. Then for all $x$ such that $\abs{x} = \abs{x - 0} < \delta := \epsilon \abs{y}$:
    \begin{equation*}
    \abs*{\dfrac{xy}{x^2 + y^2} - 0} \leq \abs*{\dfrac{xy}{y^2}} < \dfrac{\epsilon \abs{y}^2}{y^2} = \epsilon
    \end{equation*}
    $G_y$ is therefore continuous at $x = 0$ too. So it is continuous.

    $F$ is symmetric in $x$ and $y$, so $F$ is also continuous in $y$.

    \textbf{(b)}
    \begin{equation*}
    g(x) = F(x \times x) = \begin{cases}
    \dfrac{1}{2} & \:\text{if}\:  x \neq 0 \\
    0 & \:\text{if}\: x = 0
    \end{cases}
    \end{equation*}

    \textbf{(c)} For sake of contradiction, suppose that $F$ is indeed continuous. $\bbr$ is Hausdorff, so $\bbr \times \bbr$ is also Hausdorff. Then the diagonal $\Delta = \{x \times x \mid x \in \bbr\}$ is closed in $\bbr \times \bbr$, per the problem above.

    $F$ is continuous, so $F\mid_\Delta: \Delta \to \bbr$ is also continuous.
    
    $\left\{\dfrac{1}{2}\right\}$ is closed in $\bbr$ so $F|_\Delta^{-1}\left(\left\{\dfrac{1}{2}\right\}\right)$ must also be closed in $\Delta$. We can explicitly state, from $\textbf{(b)}$, that:
    \begin{equation*}
        F_\Delta^{-1}\left(\left\{\dfrac{1}{2}\right\}\right) = (\bbr - \{0\})^2
    \end{equation*}
    $(\bbr - \{0\})^2$ is closed in $\Delta$, and $\Delta $ is closed in $\bbr^2$, so $(\bbr - \{0\})^2$ is closed in $\bbr^2$. But it's clear that $\bbr-\{0\}$ is open in $\bbr$, so $(\bbr - \{0\})^2$ is open in $\bbr^2$, \contra.

    It follows that $F$ is not continuous.
\end{solution}

\begin{problem} [18.13 \redtext{done}]
Let $A \subset X$; let $f: A \to Y$ be continuous; let $Y$ be Hausdorff. Show that if $f$ maybe extended to a continuous function $g: \overline{A} \to Y$, then $g$ is uniquely determined by $f$.
\end{problem}
\begin{solution}
    Suppose not, i.e., that there exists $g, h: \cl{A} \to Y$ that are both continuous extension of $f$ from $A$ to $\cl{A}$, and $g \not \equiv h$, i.e., there exists $x \in \cl{A}$ where $g(x) \neq h(x)$. $x$ can't be in $A$ since $g(x) = f(x) = h(x)$, so $x$ must be a limit point of $A$.

    Then since $g(x) \neq h(x); g(x), h(x) \in Y$, and $Y$ is Hausdorff, it follows that there exists $U, V$ open in $Y$ such that $g(x) \in U, h(x) \in V, U \cap V = \emptyset$.

    Since $g, h$ are continuous, it follows that $g^{-1}(U)$ and $h^{-1}(V)$ are also open in $X$.

    Then $W = g^{-1}(U) \cap h^{-1}(V)$ is open in $X$ and contains $x$.  $x$ is a limit point of $A$, so there exists $y \neq x, y \in W \subset A$.

    But since $y \in A$, $g(y) = h(y)$. And $y \in W \implies g(y) \in U, h(y) \in V \implies U \cap V \neq \emptyset, \contra$.

    We therefore have that the continuous extension of $f$ from $A$ to $\cl{A}$, if possible, must be uniquely determined.
\end{solution}

\begin{problem} [19.4 \redtext{done}]
Show that $(X_1 \times \cdots \times X_{n-1}) \times X_n$ is homeomorphic with $X_1 \times \cdots \times X_n$.
\end{problem}
\begin{solution}
    Consider \begin{align*}
        f: (X_1 \times \cdots \times X_{n-1}) \times X_n &\to X_1 \times \cdots \times X_n \\
        (x_1 \times \cdots \times x_{n-1}) \times x_n & \mapsto x_1 \times \cdots \times x_n
    \end{align*}

    \textbf{1.} Clearly, $f$ is bijective.

    \textbf{2.} WTS $f$ is continuous. Take $U_1 \times \cdots \times U_n \in X_1 \cdots \times X_n$, a typical basis element of $X_1 \times \cdots \times X_n$, i.e., $U_i$ is open in $X_i$ for all $i \in [n]$.

    Then its preimage $f^{-1}(U_1 \times \cdots \times U_n) = (U_1 \times \cdots \times U_{n-1}) \times U_n$.

    Since $U_i$ is open in $X_i$ for all $i \in [n]$, particularly $i \in [n-1]$, $(U_1 \times \cdots \times U_{n-1})$ is a typical basis element in $(X_1 \times \cdots \times X_{n-1})$, so it is open. $U_n$ is also open in $X_n$, so $(U_1 \times \cdots \times U_{n-1}) \times U_n$ is a typical basis element in $(X_1 \times \cdots \times X_{n-1}) \times X_n$, so it is open. $f$ is therefore continuous.

    \textbf{3.} WTS $f^{-1}$ is continuous. Take $(U_1 \times \cdots \times U_{n-1}) \times U_n \in (X_1 \times \cdots \times X_{n-1}) \times X_n$, a typical basis element of $(X_1 \times \cdots \times X_{n-1}) \times X_n$, which means that $U_n$ is open in $X_n$, and $(U_1 \times \cdots \times U_{n-1})$ is open $(X_1 \times \cdots \times X_{n-1})$, which then means that $U_i$ is open in $X_i$ for $i \in [n-1]$. In summary, $U_i$ is open in $X_i$ for $i \in [n]$.

    Then its preimage through $f^{-1}$ is $f((U_1 \times \cdots U_{n-1}) \times U_n) = U_1 \times \cdots \times U_n$. It is clearly a typical basis element for $X_1 \times \cdots \times X_n$, because $U_i$ is open in $X_i$ for $i \in [n]$, and is therefore open. $f^{-1}$ is therefore also continuous.

    \textbf{4.} From \textbf{1., 2., 3.}, it follows that $f$ is a homeomorphism. We have demonstrated a homeomorphism between $(X_1 \times \cdots \times X_{n-1}) \times X_n$ and $X_1 \times \cdots \times X_n$, so they are homeomorphic.
\end{solution}

\begin{problem} [19.7 \redtext{done}]
Let $\bbr^\infty$ be the subset of $\bbr^\omega$ consisting of all sequences that are ``eventually zero,'' that is, all sequences $(x_1, x_2, \ldots)$ such that $x_i \neq 0$ for only finitely many values of $i$. What is the closure of $\bbr^\infty$ in $\bbr^\omega$ in the box and product topologies? Justify your answer.
\end{problem}
\begin{solution}
    Define $A \coloneqq \bbr^\omega - \bbr^\infty$, i.e., the set of sequences such that $x_i \neq 0$ for infinitely many values of $i$, i.e., $x_i = 0$ for finitely many values of $i$.

    \textbf{1.} In the box topology. WTS $\cl{\bbr^\infty} = {\bbr^\infty} \Leftrightarrow \bbr^\infty$ is closed $\Leftrightarrow A$ is open.
    

    Take $(y_i)_{i \in \bbn} = (y_1, \ldots) \in A$, i.e., $y_i = 0$ for finitely many values of $i$. WLOG, $y_1 = \ldots = y_n = 0; y_i > 0 \forall i \geq n+1$ for some $n \in \bbn$ (negative/mixed cases are handled similarly and trivially).

    Then there exists $U \coloneqq (-1, 1)^n \times (\frac{y_{n+1}}{2}, \frac{3y_{n+1}}{2}) \times (\frac{y_{n+2}}{2}, \frac{3y_{n+2}}{2}) \times \ldots$ is a basis element in the box topology, and is therefore open.

    Then it's clear that by construction, $(y_i) \in U$.

    Furthermore, we claim that $U \subset A$. Take any $(z_i) \in U$. Since, for all $i \geq n+1$, we have that $y_i > 0$ and  $z_i \in (\frac{y_{n+1}}{2}, \frac{3y_{n+1}}{2})$, it follows that $z_i \neq 0$ for all $i \geq n+1$, i.e., infinitely many values of $i$. It follows that $(z_i) \not \in \bbr^{\infty} \implies U \subset A  $.

    We have therefore demonstrated, for all $(y_i) \in A$, there exists open $U$ such that $(y_i) \in U \subset A$. $A$ is therefore open in the box topology.

    It follows that $\bbr^\omega - A = \bbr^\infty$ is closed, so $\cl{\bbr^\infty} = \bbr^\infty$.

    \textbf{2.} In the product topology. WTS $\cl{\bbr^\infty} = \bbr^\omega$, by showing that every point in $\bbr^\omega$ is a limit point of $\bbr^\infty$.

    Take $(y_i)_{i \in \bbn} \in \bbr^\omega$. Take a typical basis element in the product topology that contains $(y_i)$, that is:
    \begin{equation*}
    B = \bbr \times \ldots \times \bbr \times U_{i_1} \times \bbr \times \ldots \bbr \times U_{i_2} \times \bbr \times \ldots \times U_{i_n} \times \bbr \times \ldots
    \end{equation*}
    where only $U_{i_1}, U_{i_2}, \ldots, U_{i_n} \neq \bbr$ at the $i_1, i_2, \ldots, i_n$-th coordinate.

    For this basis element to contain $(y_i)$, it requires:
    \begin{equation*}
    y_{i_k} \in U_{i_k} \forall k \in [n],
    \end{equation*}
    with all other coordinates being trivially true (simply being in $\bbr$).

    We can then immediately show that $B \cap \bbr^\infty \neq \emptyset$, by demonstrating a point of intersection $(z_i)$ defined by \begin{equation*}
    z_{i_k} = y_{i_k} \forall k \in [n], z_i = 0 \:\text{otherwise}\: 
    \end{equation*}

    Then $(z_i) \in \bbr^\infty$ since $z_i \neq 0$ for only finitely many values of $i$. At the same time, since $z_{i_k} = y_{i_k} \in U_{i_k}$ and $0 \in \bbr$, it follows that $(z_i) \in B$ too. It follows that $B \cap \bbr^\infty \neq \emptyset$.

    Therefore, $\cl{\bbr^\infty} = \bbr^{\omega}$.
\end{solution}

\begin{problem} [17.21* (Bonus)]
(Kuratowski) Consider the collection of all subsets $A$ of the topological space $X$. The operations of closure $A \to \overline{A}$ and complementation $A \to X - A$ are functions from this collection to itself.
\begin{enumerate}
\item Show that starting with a given set $A$, one can form no more than 14 distinct sets by applying these two operations successively.
\item Find a subset of $A$ of $\bbr$ (in its usual topology) for which the maximum of 14 is obtained.
\end{enumerate}
\end{problem}
\begin{solution}
\end{solution}
\end{document}