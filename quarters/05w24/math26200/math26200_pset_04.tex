\documentclass[a4paper, 10pt]{article}
%%%%%%%%%%%%%%%%%%%%%%%%%%%%%%%%%%%%%%%%%%%%%%%%%%%%%%%%%%%%%%%%%%%%%%%%%%%%%%%
%                                Basic Packages                               %
%%%%%%%%%%%%%%%%%%%%%%%%%%%%%%%%%%%%%%%%%%%%%%%%%%%%%%%%%%%%%%%%%%%%%%%%%%%%%%%

% Gives us multiple colors.
\usepackage[usenames,dvipsnames,pdftex]{xcolor}
% Lets us style link colors.
\usepackage{hyperref}
% Lets us import images and graphics.
\usepackage{graphicx}
% Lets us use figures in floating environments.
\usepackage{float}
% Lets us create multiple columns.
\usepackage{multicol}
% Gives us better math syntax.
\usepackage{amsmath,amsfonts,mathtools,amsthm,amssymb}
% Lets us strikethrough text.
\usepackage{cancel}
% Lets us edit the caption of a figure.
\usepackage{caption}
% Lets us import pdf directly in our tex code.
\usepackage{pdfpages}
% Lets us do algorithm stuff.
\usepackage[ruled,vlined,linesnumbered]{algorithm2e}
% Use a smiley face for our qed symbol.
\usepackage{tikzsymbols}
% \usepackage{fullpage} %%smaller margins
\usepackage[shortlabels]{enumitem}

\setlist[enumerate]{font={\bfseries}} % global settings, for all lists

\usepackage{setspace}
\usepackage[margin=1in, headsep=12pt]{geometry}
\usepackage{wrapfig}
\usepackage{listings}
\usepackage{parskip}

\definecolor{codegreen}{rgb}{0,0.6,0}
\definecolor{codegray}{rgb}{0.5,0.5,0.5}
\definecolor{codepurple}{rgb}{0.58,0,0.82}
\definecolor{backcolour}{rgb}{0.95,0.95,0.95}

\lstdefinestyle{mystyle}{
    backgroundcolor=\color{backcolour},   
    commentstyle=\color{codegreen},
    keywordstyle=\color{magenta},
    numberstyle=\tiny\color{codegray},
    stringstyle=\color{codepurple},
    basicstyle=\ttfamily\footnotesize,
    breakatwhitespace=false,         
    breaklines=true,                 
    captionpos=b,                    
    keepspaces=true,                 
    numbers=left,                    
    numbersep=5pt,                  
    showspaces=false,                
    showstringspaces=false,
    showtabs=false,                  
    tabsize=2,
    numbers=none
}

\lstset{style=mystyle}
\def\class{article}


%%%%%%%%%%%%%%%%%%%%%%%%%%%%%%%%%%%%%%%%%%%%%%%%%%%%%%%%%%%%%%%%%%%%%%%%%%%%%%%
%                                Basic Settings                               %
%%%%%%%%%%%%%%%%%%%%%%%%%%%%%%%%%%%%%%%%%%%%%%%%%%%%%%%%%%%%%%%%%%%%%%%%%%%%%%%

%%%%%%%%%%%%%
%  Symbols  %
%%%%%%%%%%%%%

\let\implies\Rightarrow
\let\impliedby\Leftarrow
\let\iff\Leftrightarrow
\let\epsilon\varepsilon
%%%%%%%%%%%%
%  Tables  %
%%%%%%%%%%%%

\setlength{\tabcolsep}{5pt}
\renewcommand\arraystretch{1.5}

%%%%%%%%%%%%%%
%  SI Unitx  %
%%%%%%%%%%%%%%

\usepackage{siunitx}
\sisetup{locale = FR}

%%%%%%%%%%
%  TikZ  %
%%%%%%%%%%

\usepackage[framemethod=TikZ]{mdframed}
\usepackage{tikz}
\usepackage{tikz-cd}
\usepackage{tikzsymbols}

\usetikzlibrary{intersections, angles, quotes, calc, positioning}
\usetikzlibrary{arrows.meta}

\tikzset{
    force/.style={thick, {Circle[length=2pt]}-stealth, shorten <=-1pt}
}

%%%%%%%%%%%%%%%
%  PGF Plots  %
%%%%%%%%%%%%%%%

\usepackage{pgfplots}
\pgfplotsset{width=10cm, compat=newest}

%%%%%%%%%%%%%%%%%%%%%%%
%  Center Title Page  %
%%%%%%%%%%%%%%%%%%%%%%%

\usepackage{titling}
\renewcommand\maketitlehooka{\null\mbox{}\vfill}
\renewcommand\maketitlehookd{\vfill\null}

%%%%%%%%%%%%%%%%%%%%%%%%%%%%%%%%%%%%%%%%%%%%%%%%%%%%%%%
%  Create a grey background in the middle of the PDF  %
%%%%%%%%%%%%%%%%%%%%%%%%%%%%%%%%%%%%%%%%%%%%%%%%%%%%%%%

\usepackage{eso-pic}
\newcommand\definegraybackground{
    \definecolor{reallylightgray}{HTML}{FAFAFA}
    \AddToShipoutPicture{
        \ifthenelse{\isodd{\thepage}}{
            \AtPageLowerLeft{
                \put(\LenToUnit{\dimexpr\paperwidth-222pt},0){
                    \color{reallylightgray}\rule{222pt}{297mm}
                }
            }
        }
        {
            \AtPageLowerLeft{
                \color{reallylightgray}\rule{222pt}{297mm}
            }
        }
    }
}

%%%%%%%%%%%%%%%%%%%%%%%%
%  Modify Links Color  %
%%%%%%%%%%%%%%%%%%%%%%%%

\hypersetup{
    % Enable highlighting links.
    colorlinks,
    % Change the color of links to blue.
    urlcolor=blue,
    % Change the color of citations to black.
    citecolor={black},
    % Change the color of url's to blue with some black.
    linkcolor={blue!80!black}
}

%%%%%%%%%%%%%%%%%%
% Fix WrapFigure %
%%%%%%%%%%%%%%%%%%

\newcommand{\wrapfill}{\par\ifnum\value{WF@wrappedlines}>0
        \parskip=0pt
        \addtocounter{WF@wrappedlines}{-1}%
        \null\vspace{\arabic{WF@wrappedlines}\baselineskip}%
        \WFclear
    \fi}

%%%%%%%%%%%%%%%%%
% Multi Columns %
%%%%%%%%%%%%%%%%%

\let\multicolmulticols\multicols
\let\endmulticolmulticols\endmulticols

\RenewDocumentEnvironment{multicols}{mO{}}
{%
    \ifnum#1=1
        #2%
    \else % More than 1 column
        \multicolmulticols{#1}[#2]
    \fi
}
{%
    \ifnum#1=1
    \else % More than 1 column
        \endmulticolmulticols
    \fi
}

\newlength{\thickarrayrulewidth}
\setlength{\thickarrayrulewidth}{5\arrayrulewidth}


%%%%%%%%%%%%%%%%%%%%%%%%%%%%%%%%%%%%%%%%%%%%%%%%%%%%%%%%%%%%%%%%%%%%%%%%%%%%%%%
%                           School Specific Commands                          %
%%%%%%%%%%%%%%%%%%%%%%%%%%%%%%%%%%%%%%%%%%%%%%%%%%%%%%%%%%%%%%%%%%%%%%%%%%%%%%%

%%%%%%%%%%%%%%%%%%%%%%%%%%%
%  Initiate New Counters  %
%%%%%%%%%%%%%%%%%%%%%%%%%%%

\newcounter{lecturecounter}

%%%%%%%%%%%%%%%%%%%%%%%%%%
%  Helpful New Commands  %
%%%%%%%%%%%%%%%%%%%%%%%%%%

\makeatletter

\newcommand\resetcounters{
    % Reset the counters for subsection, subsubsection and the definition
    % all the custom environments.
    \setcounter{subsection}{0}
    \setcounter{subsubsection}{0}
    \setcounter{definition0}{0}
    \setcounter{paragraph}{0}
    \setcounter{theorem}{0}
    \setcounter{claim}{0}
    \setcounter{corollary}{0}
    \setcounter{proposition}{0}
    \setcounter{lemma}{0}
    \setcounter{exercise}{0}
    \setcounter{problem}{0}
    
    \setcounter{subparagraph}{0}
    % \@ifclasswith\class{nocolor}{
    %     \setcounter{definition}{0}
    % }{}
}

%%%%%%%%%%%%%%%%%%%%%
%  Lecture Command  %
%%%%%%%%%%%%%%%%%%%%%

\usepackage{xifthen}

% EXAMPLE:
% 1. \lecture{Oct 17 2022 Mon (08:46:48)}{Lecture Title}
% 2. \lecture[4]{Oct 17 2022 Mon (08:46:48)}{Lecture Title}
% 3. \lecture{Oct 17 2022 Mon (08:46:48)}{}
% 4. \lecture[4]{Oct 17 2022 Mon (08:46:48)}{}
% Parameters:
% 1. (Optional) lecture number.
% 2. Time and date of lecture.
% 3. Lecture Title.
\def\@lecture{}
\def\@lectitle{}
\def\@leccount{}
\newcommand\lecture[3]{
    \newpage

    % Check if user passed the lecture title or not.
    \def\@leccount{Lecture #1}
    \ifthenelse{\isempty{#3}}{
        \def\@lecture{Lecture #1}
        \def\@lectitle{Lecture #1}
    }{
        \def\@lecture{Lecture #1: #3}
        \def\@lectitle{#3}
    }

    \setcounter{section}{#1}
    \renewcommand\thesubsection{#1.\arabic{subsection}}
    
    \phantomsection
    \addcontentsline{toc}{section}{\@lecture}
    \resetcounters

    \begin{mdframed}
        \begin{center}
            \Large \textbf{\@leccount}
            
            \vspace*{0.2cm}
            
            \large \@lectitle
            
            
            \vspace*{0.2cm}

            \normalsize #2
        \end{center}
    \end{mdframed}

}

%%%%%%%%%%%%%%%%%%%%
%  Import Figures  %
%%%%%%%%%%%%%%%%%%%%

\usepackage{import}
\pdfminorversion=7

% EXAMPLE:
% 1. \incfig{limit-graph}
% 2. \incfig[0.4]{limit-graph}
% Parameters:
% 1. The figure name. It should be located in figures/NAME.tex_pdf.
% 2. (Optional) The width of the figure. Example: 0.5, 0.35.
\newcommand\incfig[2][1]{%
    \def\svgwidth{#1\columnwidth}
    \import{./figures/}{#2.pdf_tex}
}

\begingroup\expandafter\expandafter\expandafter\endgroup
\expandafter\ifx\csname pdfsuppresswarningpagegroup\endcsname\relax
\else
    \pdfsuppresswarningpagegroup=1\relax
\fi

%%%%%%%%%%%%%%%%%
% Fancy Headers %
%%%%%%%%%%%%%%%%%

\usepackage{fancyhdr}

% Force a new page.
\newcommand\forcenewpage{\clearpage\mbox{~}\clearpage\newpage}

% This command makes it easier to manage my headers and footers.
\newcommand\createintro{
    % Use roman page numbers (e.g. i, v, vi, x, ...)
    \pagenumbering{roman}

    % Display the page style.
    \maketitle
    % Make the title pagestyle empty, meaning no fancy headers and footers.
    \thispagestyle{empty}
    % Create a newpage.
    \newpage

    % Input the intro.tex page if it exists.
    \IfFileExists{intro.tex}{ % If the intro.tex file exists.
        % Input the intro.tex file.
        \textbf{Course}: MATH 16300: Honors Calculus III

\textbf{Section}: 43

\textbf{Professor}: Minjae Park

\textbf{At}: The University of Chicago

\textbf{Quarter}: Spring 2023

\textbf{Course materials}: Calculus by Spivak (4th Edition), Calculus On Manifolds by Spivak

\vspace{1cm}
\textbf{Disclaimer}: This document will inevitably contain some mistakes, both simple typos and serious logical and mathematical errors. Take what you read with a grain of salt as it is made by an undergraduate student going through the learning process himself. If you do find any error, I would really appreciate it if you can let me know by email at \href{mailto:conghungletran@gmail.com}{conghungletran@gmail.com}.

        % Make the pagestyle fancy for the intro.tex page.
        \pagestyle{fancy}

        % Remove the line for the header.
        \renewcommand\headrulewidth{0pt}

        % Remove all header stuff.
        \fancyhead{}

        % Add stuff for the footer in the center.
        % \fancyfoot[C]{
        %   \textit{For more notes like this, visit
        %   \href{\linktootherpages}{\shortlinkname}}. \\
        %   \vspace{0.1cm}
        %   \hrule
        %   \vspace{0.1cm}
        %   \@author, \\
        %   \term: \academicyear, \\
        %   Last Update: \@date, \\
        %   \faculty
        % }

        \newpage
    }{ % If the intro.tex file doesn't exist.
        % Force a \newpageage.
        % \forcenewpage
        \newpage
    }

    % Remove the center stuff we did above, and replace it with just the page
    % number, which is still in roman numerals.
    \fancyfoot[C]{\thepage}
    % Add the table of contents.
    \tableofcontents
    % Force a new page.
    \newpage

    % Move the page numberings back to arabic, from roman numerals.
    \pagenumbering{arabic}
    % Set the page number to 1.
    \setcounter{page}{1}

    % Add the header line back.
    \renewcommand\headrulewidth{0.4pt}
    % In the top right, add the lecture title.
    \fancyhead[R]{\footnotesize \@lecture}
    % In the top left, add the author name.
    \fancyhead[L]{\footnotesize \@author}
    % In the bottom center, add the page.
    \fancyfoot[C]{\thepage}
    % Add a nice gray background in the middle of all the upcoming pages.
    % \definegraybackground
}

\makeatother


%%%%%%%%%%%%%%%%%%%%%%%%%%%%%%%%%%%%%%%%%%%%%%%%%%%%%%%%%%%%%%%%%%%%%%%%%%%%%%%
%                               Custom Commands                               %
%%%%%%%%%%%%%%%%%%%%%%%%%%%%%%%%%%%%%%%%%%%%%%%%%%%%%%%%%%%%%%%%%%%%%%%%%%%%%%%

%%%%%%%%%%%%
%  Circle  %
%%%%%%%%%%%%

\newcommand*\circled[1]{\tikz[baseline= (char.base)]{
        \node[shape=circle,draw,inner sep=1pt] (char) {#1};}
}

%%%%%%%%%%%%%%%%%%%
%  Todo Commands  %
%%%%%%%%%%%%%%%%%%%

% \usepackage{xargs}
% \usepackage[colorinlistoftodos]{todonotes}

% \makeatletter

% \@ifclasswith\class{working}{
%     \newcommandx\unsure[2][1=]{\todo[linecolor=red,backgroundcolor=red!25,bordercolor=red,#1]{#2}}
%     \newcommandx\change[2][1=]{\todo[linecolor=blue,backgroundcolor=blue!25,bordercolor=blue,#1]{#2}}
%     \newcommandx\info[2][1=]{\todo[linecolor=OliveGreen,backgroundcolor=OliveGreen!25,bordercolor=OliveGreen,#1]{#2}}
%     \newcommandx\improvement[2][1=]{\todo[linecolor=Plum,backgroundcolor=Plum!25,bordercolor=Plum,#1]{#2}}

%     \newcommand\listnotes{
%         \newpage
%         \listoftodos[Notes]
%     }
% }{
%     \newcommandx\unsure[2][1=]{}
%     \newcommandx\change[2][1=]{}
%     \newcommandx\info[2][1=]{}
%     \newcommandx\improvement[2][1=]{}

%     \newcommand\listnotes{}
% }

% \makeatother

%%%%%%%%%%%%%
%  Correct  %
%%%%%%%%%%%%%

% EXAMPLE:
% 1. \correct{INCORRECT}{CORRECT}
% Parameters:
% 1. The incorrect statement.
% 2. The correct statement.
\definecolor{correct}{HTML}{009900}
\newcommand\correct[2]{{\color{red}{#1 }}\ensuremath{\to}{\color{correct}{ #2}}}


%%%%%%%%%%%%%%%%%%%%%%%%%%%%%%%%%%%%%%%%%%%%%%%%%%%%%%%%%%%%%%%%%%%%%%%%%%%%%%%
%                                 Environments                                %
%%%%%%%%%%%%%%%%%%%%%%%%%%%%%%%%%%%%%%%%%%%%%%%%%%%%%%%%%%%%%%%%%%%%%%%%%%%%%%%

\usepackage{varwidth}
\usepackage{thmtools}
\usepackage[most,many,breakable]{tcolorbox}

\tcbuselibrary{theorems,skins,hooks}
\usetikzlibrary{arrows,calc,shadows.blur}

%%%%%%%%%%%%%%%%%%%
%  Define Colors  %
%%%%%%%%%%%%%%%%%%%

% color prototype
% \definecolor{color}{RGB}{45, 111, 177}

% ESSENTIALS: 
\definecolor{myred}{HTML}{c74540}
\definecolor{myblue}{HTML}{072b85}
\definecolor{mygreen}{HTML}{388c46}
\definecolor{myblack}{HTML}{000000}

\colorlet{definition_color}{myred}

\colorlet{theorem_color}{myblue}
\colorlet{lemma_color}{myblue}
\colorlet{prop_color}{myblue}
\colorlet{corollary_color}{myblue}
\colorlet{claim_color}{myblue}

\colorlet{proof_color}{myblack}
\colorlet{example_color}{myblack}
\colorlet{exercise_color}{myblack}

% MISCS: 
%%%%%%%%%%%%%%%%%%%%%%%%%%%%%%%%%%%%%%%%%%%%%%%%%%%%%%%%%
%  Create Environments Styles Based on Given Parameter  %
%%%%%%%%%%%%%%%%%%%%%%%%%%%%%%%%%%%%%%%%%%%%%%%%%%%%%%%%%

% \mdfsetup{skipabove=1em,skipbelow=0em}

%%%%%%%%%%%%%%%%%%%%%%
%  Helpful Commands  %
%%%%%%%%%%%%%%%%%%%%%%

% EXAMPLE:
% 1. \createnewtheoremstyle{thmdefinitionbox}{}{}
% 2. \createnewtheoremstyle{thmtheorembox}{}{}
% 3. \createnewtheoremstyle{thmproofbox}{qed=\qedsymbol}{
%       rightline=false, topline=false, bottomline=false
%    }
% Parameters:
% 1. Theorem name.
% 2. Any extra parameters to pass directly to declaretheoremstyle.
% 3. Any extra parameters to pass directly to mdframed.
\newcommand\createnewtheoremstyle[3]{
    \declaretheoremstyle[
        headfont=\bfseries\sffamily, bodyfont=\normalfont, #2,
        mdframed={
                #3,
            },
    ]{#1}
}

% EXAMPLE:
% 1. \createnewcoloredtheoremstyle{thmdefinitionbox}{definition}{}{}
% 2. \createnewcoloredtheoremstyle{thmexamplebox}{example}{}{
%       rightline=true, leftline=true, topline=true, bottomline=true
%     }
% 3. \createnewcoloredtheoremstyle{thmproofbox}{proof}{qed=\qedsymbol}{backgroundcolor=white}
% Parameters:
% 1. Theorem name.
% 2. Color of theorem.
% 3. Any extra parameters to pass directly to declaretheoremstyle.
% 4. Any extra parameters to pass directly to mdframed.

% change backgroundcolor to #2!5 if user wants a colored backdrop to theorem environments. It's a cool color theme, but there's too much going on in the page.
\newcommand\createnewcoloredtheoremstyle[4]{
    \declaretheoremstyle[
        headfont=\bfseries\sffamily\color{#2},
        bodyfont=\normalfont,
        headpunct=,
        headformat = \NAME~\NUMBER\NOTE \hfill\smallskip\linebreak,
        #3,
        mdframed={
                outerlinewidth=0.75pt,
                rightline=false,
                leftline=false,
                topline=false,
                bottomline=false,
                backgroundcolor=white,
                skipabove = 5pt,
                skipbelow = 0pt,
                linecolor=#2,
                innertopmargin = 0pt,
                innerbottommargin = 0pt,
                innerrightmargin = 4pt,
                innerleftmargin= 6pt,
                leftmargin = -6pt,
                #4,
            },
    ]{#1}
}



%%%%%%%%%%%%%%%%%%%%%%%%%%%%%%%%%%%
%  Create the Environment Styles  %
%%%%%%%%%%%%%%%%%%%%%%%%%%%%%%%%%%%

\makeatletter
\@ifclasswith\class{nocolor}{
    % Environments without color.

    % ESSENTIALS:
    \createnewtheoremstyle{thmdefinitionbox}{}{}
    \createnewtheoremstyle{thmtheorembox}{}{}
    \createnewtheoremstyle{thmproofbox}{qed=\qedsymbol}{}
    \createnewtheoremstyle{thmcorollarybox}{}{}
    \createnewtheoremstyle{thmlemmabox}{}{}
    \createnewtheoremstyle{thmclaimbox}{}{}
    \createnewtheoremstyle{thmexamplebox}{}{}

    % MISCS: 
    \createnewtheoremstyle{thmpropbox}{}{}
    \createnewtheoremstyle{thmexercisebox}{}{}
    \createnewtheoremstyle{thmexplanationbox}{}{}
    \createnewtheoremstyle{thmremarkbox}{}{}
    
    % STYLIZED MORE BELOW
    \createnewtheoremstyle{thmquestionbox}{}{}
    \createnewtheoremstyle{thmsolutionbox}{qed=\qedsymbol}{}
}{
    % Environments with color.

    % ESSENTIALS: definition, theorem, proof, corollary, lemma, claim, example
    \createnewcoloredtheoremstyle{thmdefinitionbox}{definition_color}{}{leftline=false}
    \createnewcoloredtheoremstyle{thmtheorembox}{theorem_color}{}{leftline=false}
    \createnewcoloredtheoremstyle{thmproofbox}{proof_color}{qed=\qedsymbol}{}
    \createnewcoloredtheoremstyle{thmcorollarybox}{corollary_color}{}{leftline=false}
    \createnewcoloredtheoremstyle{thmlemmabox}{lemma_color}{}{leftline=false}
    \createnewcoloredtheoremstyle{thmpropbox}{prop_color}{}{leftline=false}
    \createnewcoloredtheoremstyle{thmclaimbox}{claim_color}{}{leftline=false}
    \createnewcoloredtheoremstyle{thmexamplebox}{example_color}{}{}
    \createnewcoloredtheoremstyle{thmexplanationbox}{example_color}{qed=\qedsymbol}{}
    \createnewcoloredtheoremstyle{thmremarkbox}{theorem_color}{}{}

    \createnewcoloredtheoremstyle{thmmiscbox}{black}{}{}

    \createnewcoloredtheoremstyle{thmexercisebox}{exercise_color}{}{}
    \createnewcoloredtheoremstyle{thmproblembox}{theorem_color}{}{leftline=false}
    \createnewcoloredtheoremstyle{thmsolutionbox}{mygreen}{qed=\qedsymbol}{}
}
\makeatother

%%%%%%%%%%%%%%%%%%%%%%%%%%%%%
%  Create the Environments  %
%%%%%%%%%%%%%%%%%%%%%%%%%%%%%
\declaretheorem[numberwithin=section, style=thmdefinitionbox,     name=Definition]{definition}
\declaretheorem[numberwithin=section, style=thmtheorembox,     name=Theorem]{theorem}
\declaretheorem[numbered=no,          style=thmexamplebox,     name=Example]{example}
\declaretheorem[numberwithin=section, style=thmtheorembox,       name=Claim]{claim}
\declaretheorem[numberwithin=section, style=thmcorollarybox,   name=Corollary]{corollary}
\declaretheorem[numberwithin=section, style=thmpropbox,        name=Proposition]{proposition}
\declaretheorem[numberwithin=section, style=thmlemmabox,       name=Lemma]{lemma}
\declaretheorem[numberwithin=section, style=thmexercisebox,    name=Exercise]{exercise}
\declaretheorem[numbered=no,          style=thmproofbox,       name=Proof]{proof0}
\declaretheorem[numbered=no,          style=thmexplanationbox, name=Explanation]{explanation}
\declaretheorem[numbered=no,          style=thmsolutionbox,    name=Solution]{solution}
\declaretheorem[numberwithin=section,          style=thmproblembox,     name=Problem]{problem}
\declaretheorem[numbered=no,          style=thmmiscbox,    name=Intuition]{intuition}
\declaretheorem[numbered=no,          style=thmmiscbox,    name=Goal]{goal}
\declaretheorem[numbered=no,          style=thmmiscbox,    name=Recall]{recall}
\declaretheorem[numbered=no,          style=thmmiscbox,    name=Motivation]{motivation}
\declaretheorem[numbered=no,          style=thmmiscbox,    name=Remark]{remark}
\declaretheorem[numbered=no,          style=thmmiscbox,    name=Observe]{observe}
\declaretheorem[numbered=no,          style=thmmiscbox,    name=Question]{question}


%%%%%%%%%%%%%%%%%%%%%%%%%%%%
%  Edit Proof Environment  %
%%%%%%%%%%%%%%%%%%%%%%%%%%%%

\renewenvironment{proof}[2][\proofname]{
    % \vspace{-12pt}
    \begin{proof0} [#2]
        }{\end{proof0}}

\theoremstyle{definition}

\newtheorem*{notation}{Notation}
\newtheorem*{previouslyseen}{As previously seen}
\newtheorem*{property}{Property}
% \newtheorem*{intuition}{Intuition}
% \newtheorem*{goal}{Goal}
% \newtheorem*{recall}{Recall}
% \newtheorem*{motivation}{Motivation}
% \newtheorem*{remark}{Remark}
% \newtheorem*{observe}{Observe}

\author{Hung C. Le Tran}


%%%% MATH SHORTHANDS %%%%
%% blackboard bold math capitals
\DeclareMathOperator*{\esssup}{ess\,sup}
\DeclareMathOperator*{\Hom}{Hom}
\newcommand{\bbf}{\mathbb{F}}
\newcommand{\bbn}{\mathbb{N}}
\newcommand{\bbq}{\mathbb{Q}}
\newcommand{\bbr}{\mathbb{R}}
\newcommand{\bbz}{\mathbb{Z}}
\newcommand{\bbc}{\mathbb{C}}
\newcommand{\bbk}{\mathbb{K}}
\newcommand{\bbm}{\mathbb{M}}
\newcommand{\bbp}{\mathbb{P}}
\newcommand{\bbe}{\mathbb{E}}

\newcommand{\bfw}{\mathbf{w}}
\newcommand{\bfx}{\mathbf{x}}
\newcommand{\bfX}{\mathbf{X}}
\newcommand{\bfy}{\mathbf{y}}
\newcommand{\bfyhat}{\mathbf{\hat{y}}}

\newcommand{\calb}{\mathcal{B}}
\newcommand{\calf}{\mathcal{F}}
\newcommand{\calt}{\mathcal{T}}
\newcommand{\call}{\mathcal{L}}
\renewcommand{\phi}{\varphi}

% Universal Math Shortcuts
\newcommand{\st}{\hspace*{2pt}\text{s.t.}\hspace*{2pt}}
\newcommand{\pffwd}{\hspace*{2pt}\fbox{\(\Rightarrow\)}\hspace*{10pt}}
\newcommand{\pfbwd}{\hspace*{2pt}\fbox{\(\Leftarrow\)}\hspace*{10pt}}
\newcommand{\contra}{\ensuremath{\Rightarrow\Leftarrow}}
\newcommand{\cvgn}{\xrightarrow{n \to \infty}}
\newcommand{\cvgj}{\xrightarrow{j \to \infty}}

\newcommand{\im}{\mathrm{im}}
\newcommand{\innerproduct}[2]{\langle #1, #2 \rangle}
\newcommand*{\conj}[1]{\overline{#1}}

% https://tex.stackexchange.com/questions/438612/space-between-exists-and-forall
% https://tex.stackexchange.com/questions/22798/nice-looking-empty-set
\let\oldforall\forall
\renewcommand{\forall}{\;\oldforall\; }
\let\oldexist\exists
\renewcommand{\exists}{\;\oldexist\; }
\newcommand\existu{\;\oldexist!\: }
\let\oldemptyset\emptyset
\let\emptyset\varnothing


\renewcommand{\_}[1]{\underline{#1}}
\DeclarePairedDelimiter{\abs}{\lvert}{\rvert}
\DeclarePairedDelimiter{\norm}{\lVert}{\rVert}
\DeclarePairedDelimiter\ceil{\lceil}{\rceil}
\DeclarePairedDelimiter\floor{\lfloor}{\rfloor}
\setlength\parindent{0pt}
\setlength{\headheight}{12.0pt}
\addtolength{\topmargin}{-12.0pt}


% Default skipping, change if you want more spacing
% \thinmuskip=3mu
% \medmuskip=4mu plus 2mu minus 4mu
% \thickmuskip=5mu plus 5mu

% \DeclareMathOperator{\ext}{ext}
% \DeclareMathOperator{\bridge}{bridge}
\title{MATH 26200: Point-Set Topology \\ \large Problem Set 4}
\date{08 Feb 2024}
\author{Hung Le Tran}
\begin{document}
\maketitle
\setcounter{section}{5}
\textbf{Textbook:} Munkres, \textit{Topology}.
\begin{problem} [30.2 \redtext{done}]
    Show that if $X$ has countable basis $\{B_n\}$, then every basis $\calc$ for $X$ contains a countable basis for $X$. (Hint: For every pair of indices $n, m$ for which it is possible, choose $C_{n, m}$ such that $B_n \subset C_{n,m} \subset B_m$.)
\end{problem}
\begin{solution}
    Let $D = \{C \in \calc : B_n \subset C \subset B_m, (n, m) \in \bbn^2\}$.

    $D$ is countable, so it remains for us to show that indeed $D$ is a basis for $X$.

    Take any $x \in U$, $U$ open in $X$, then since $\{B_n\}$ is a basis for $X$, there exists some $B_{m'}$ such that $x \in B_{m'} \subset U$.

    Since $\calc$ is a basis for $X$ and $B_{m'}$ is open, there exists some $C_x$ such that \begin{equation*}
    x \in C_x \subset B_{m'} \subset U
    \end{equation*}

    Since $\{B_n\}$ is a basis for $X$ and $C_x$ is open, there exists some $B_{n'}$ such that \begin{equation*}
    x \in B_{n'}\subset C_x \subset B_{m'}
    \end{equation*}
    
    Then $C_x \in D$, since $B_{n'} \subset C_x \subset B_{m'}$.

    It follows that for all $U$ open, for all $x \in U$, there exists $C_x \in D$ such that $x \in C_x \subset U$. $D$ is therefore a basis for $X$.
\end{solution}

\begin{problem} [31.7 \redtext{done}]
Let $p: X \to Y$ be a closed continuous surjective map such that $\inv{p}(\{y\})$ is compact for each $y \in Y$. (Such a map is called a \textit{perfect map}.)

\begin{enumerate}
\item Show that if $X$ is Hausdorff, then so is $Y$.
\item Show that if $X$ is regular, then so is $Y$.
\item Show that if $X$ is locally compact, then so is $Y$.
\item Show that if $X$ is second-countable, then so is $Y$. (Hint: Let $\calb$ be a countable basis for $X$. For each finite subset $J$ of $\calb$, let $U_J$ be the union of all sets of the form $\inv{p}(W)$, for $W$ open in $Y$, that are contained in the union of the elements of $J$.)
\end{enumerate}
\end{problem}
\begin{solution}
    We use the following lemmas:
    \begin{psetlemma} [Lemma 1]
    If $p: X \to Y$ is a perfect map, $B \subset Y$ and $U$ is an open set containing $\inv{p}(B)$ then there exists some open $B \subset W \subset Y$ such that $\inv{p}(W) \subset U$.
    \end{psetlemma}

    \begin{proof} [Lemma 1]
            $\inv{p}(B) \subset U$. $p$ is closed, so $p(X - U)$ is closed in $Y$. Then consider $W = Y - p(X - U)$ is open. Since $\inv{p}(B) \subset U$, $y \in W$. By construction, we also have that $\inv{p}(W) \subset U$.
    \end{proof}

    \begin{psetlemma} [Lemma 2]
        If $C$ and $K$ are compact subsets of Hausdorff $X$ such that $C \cap K = \emptyset$, then there exists open, disjoint $U \supset C, V \supset K$.
    \end{psetlemma}

    \begin{proof} [Lemma 2]
        Take $c \in C$, then $c \not \in K$. From Lemma 26.4, we know that there exists open disjoint $U_c$ and $V_c$ such that $U_c \ni c, V_c \supset K$.

        Since $C$ is compact, the open cover $\{U_c\}_{c \in C}$ reduces to a subcover $\{U_{c_1}, \ldots, U_{c_n}\}$. Then $U = \bigcup_{k = 1}^{n} U_{c_k}$ and $V = \bigcap_{k=1}^{n} V_{c_k}$ satisfies our requirements.
    \end{proof}

    \textbf{(a)} Suppose that $X$ is Hausdorff. Then $\inv{p}(\{y_1\})$ and $ \inv{p}(\{y_2\})$ are disjoint, compact subsets of Hausdorff $X$. Using Lemma 2, It follows that there exists open, disjoint $U_1, U_2 \subset X$ such that $U_1 \supset \inv{p}(\{y_1\}), U_2 \supset \inv{p}(\{y_2\})$. Using Lemma 1, it follows that there exists open $V_1, V_2 \subset Y$ such that $y_1 \in V_1, y_2 \in V_2$ and $\inv{p}(V_1) \subset U_1, \inv{p}(V_2) \subset U_2$. Since $U_1 \cap U_2 = \emptyset \implies V_1 \cap V_2 = \emptyset$. $Y$ is therefore Hausdorff.

    \textbf{(b)} Perform the same proof, since $B$ in Lemma 1 is general and not restricted to $\{y_1\}$.

    \textbf{(c)} Suppose $X$ is locally compact. WTS $Y$ is also locally compact, meaning for any $y \in Y$, there exists open neighborhood $V$ and compact $K$ such that $y \in V \subset K$.

    For all $x \in p^{-1}(\{y\})$, since $X$ is locally compact, there exists open $U_x$ and compact $C_x$ such that $x \in U_x \subset C_x$. $\{U_x\}$ is then an open cover of compact $\inv{p}(y)$, hence reduces to finite subcover $\{U_{x_1}, \ldots, U_{x_n} \}$. Then let $U = \bigcup_{k =1}^n U_{x_k}, C =  \bigcup_{k =1}^n C_{x_k}$ then $\inv{p}(y) \subset U$ and $C$ is a finite union of compacts and is therefore compact.

    Using Lemma 1, then there exists $V$ open in $Y$ such that $y \in V$ and $\inv{p}(V) \subset U \subset C \implies V \subset p(C)$. $p$ is continuous and $C$ is compact so $p(C)$ is also compact.

    \textbf{(d)} Suppose $X$ is second-countable. Let $\calb = \{B_j\}_{j \in \bbn}$ be a countable basis for $X$. For each finite subset $J$ of $\calb$, let $U_J$ be the union of all sets $\inv{p}(W)$ for some $W$ open in $Y$ such that $\inv{p}(W) \subset \bigcup_{j \in J}B_j$. The number of finite subsets of a countable set is countable, so $\{U_J\}$ is countable, and also $\{p(U_J)\}$.

    WTS $\{p(U_J)\}$ is a basis for $Y$. Take $W \subset Y$ open. Then $\inv{p}(W) = \bigcup_{y \in W} \inv{p}(\{y\})$ is a union of compacts. $p$ is continuous and $W$ is open, so $\inv{p}(W)$ is also open, and is therefore a union of basis elemenets, i.e., $\{B_j\}_{j \in J_W}$. Each $\inv{p}(y)$ is compact, and can therefore be covered by finitely many $\{B_j\}_{j \in J_y \subset J_W}$. Using Lemma 1, it then follows that there exists open $V_y \subset Y$ such that $y \in V_y, \inv{p}(V_y)\subset \bigcup_{j \in J_y} B_j$, so $\inv{p}(V_y) \subset U_{J_y} \subset \bigcup_{j \in J_W} \subset \inv{p}(W)$, which implies that $W = \bigcup_{y \in W} U_{J_y}$ as required.
\end{solution}

\begin{problem} [32.5 \redtext{done}]
Is $\bbr^\omega$ normal in the product topology? In the uniform topology?
\end{problem}
\begin{solution}
    $\bbr^\omega$ is metrizable in both topologies, and is therefore normal in both.
\end{solution}

\begin{problem} [32.6 \redtext{done}]
    A space $X$ is said to be \textit{completely normal} if every subspace of $X$ is normal. Show that $X$ is completely normal iff for every pair $A, B$ of separated sets in $X$ (that is, sets such that $\cl{A} \cap B = \emptyset = A \cap \cl{B}$), there exist disjoint open sets containing them. (Hint: If $X$ is completely normal, consider $X - (\cl{A} \cap \cl{B})$).
\end{problem}
\begin{solution}
    \pffwd Suppose $X$ is completely normal. Let $A, B$ be a pair of separated sets in $X$, meaning $\cl{A} \cap B = \emptyset = A \cap \cl{B}$.

    Consider $M = X - (\cl{A} \cap \cl{B}) \subset X$, then $M$ is normal. $\cl{A} \cap \cl{B}$ is closed, so $M$ is open in $X$.

    $M = (X - \cl{A}) \cup (X - \cl{B})$. Since $A \cap \cl{B} = \emptyset \implies A \subset X - \cl{B} \implies A \subset M$. Similarly, $B \subset M$.

    The closure of $A$ in $M$ is then $cl_M(A) = \cl{A} \cap M = \cl{A} \cup (X - \cl{B}) = \cl{A} - \cl{B}$, and similarly closure of $B$ in $M$ is $cl_M(B) = \cl{B} \cap M = \cl{B} - \cl{A}$. They are disjoint closed sets in normal $M$ and therefore there are disjoint open sets $U, V \subset M$ such that $U \supset cl_{M}(A), V \supset cl_{M}(B)$. It then also follows that $U \supset A, V \supset B$. $U, V$ are open in $M$ open in $X$, so $U, V$ are open in $X$. \qed

    \pfbwd WTS $X$ is completely normal. Take subspace $M$ of $X$ and disjoint closed subsets $A, B$ of $M$. Then $cl_M(A) = A, cl_M(B) = B$. Then since $A, B \subset M$,
    \begin{equation*}
        \cl{A} \cap B = (\cl{A} \cap M) \cap B = cl_{M}(A) \cap B = A \cap B = \emptyset
    \end{equation*}
    Similarly, $\cl{B} \cap A = \emptyset$. Hence $A, B$ are separated sets in $X$, so there exists disjoint open sets containing them. These open sets, when intersecting with $M$, are open in the subspace topology and of course still disjoint. $M$ is therefore normal, making $X$ completely normal.
\end{solution}

\begin{problem} [33.4 \redtext{done}]
    Recall that $A$ is a $G_{\delta}$ set in $X$ if $A$ is the intersection of a countable collection of open sets in $X$.

    Let $X$ be normal. Then there exists a continuous function $f: X \to [0, 1]$ such that $f(x) = 0$ for $x \in A$, and $f(x) > 0$ for $x \not \in A$ \textbf{if and only if} $A$ is a closed $G_{\delta}$ set in $X$.
    
    A function satisfying the requirements of this theorem is said to vanish precisely on $A$.
\end{problem}
\begin{solution}
    \pffwd Suppose there exists continuous $f: X \to [0, 1]$ such that $f(x) = 0$ for $x \in A$, $f(x) > 0$ for $x \not \in A$.

    It follows that \begin{equation*}
    A = \inv{f}(\{0\}) = \bigcap_{n \in \bbn} \inv{f}\left([0, \frac{1}{n})\right)
    \end{equation*}
    Since $f$ is continuous and $[0, 1/n)$ is open in $[0, 1]$ for all $n \in \bbn$, $A$ is thus a $G_\delta$ set.

    \pfbwd Suppose $A$ is a closed $G_\delta$ set in $X$.

    \begin{equation*}
    A = \bigcap_{n \in \bbn} U_n
    \end{equation*}
    where each $U_n$ is open in $X$. Then, for each $n$, $A \subset U_n \implies A \cap (X - U_n) = \emptyset$. Thus $A$ and $X - U_n$ are disjoint closed subsets of $X$, so by Urysohn's Lemma, there exists continuous $f_n: X \to [0, 1]$ such that $f_n(A) = \{0\}$ and $f_n(X - U_n) = \{1\}$.

    Then define \begin{equation*}
    s_n(x) = \sum_{k = 1}^{n} \frac{1}{2^k} f_k(x)
    \end{equation*}
    then $s_n$ is a finite linear combination of continuous functions and is therefore continuous for all $n \in \bbn$.

    And define \begin{equation*}
        f(x) = \lim_{n \to \infty} s_n(x)
    \end{equation*}
    as the pointwise convergence of $s_n$. The pointwise convergence exists because for fixed $x$, $s_n(x)$ is monotonically increasing and bounded above by 1.
    
    We want to show that this convergence is uniform. Indeed, \begin{align*}
        \abs*{s_n(x) - f(x)} = \abs*{\sum_{k = n+1}^{\infty} \frac{1}{2^{k+1}} f_k(x)} \leq \abs*{\sum_{k= n+1}^{\infty} \frac{1}{2^{k+1}}} \cvgn 0 \:\text{uniformly}\: 
    \end{align*}

    $f$ is therefore the uniform limit of continuous functions, and is therefore continuous itself.

    For any $x \in A$, $s_n(x) = 0 \implies f(x) = 0$. Then for $x \not \in A \implies x \in X - U_m$ for some $m$, then $f(x) \geq s_m(x) = \frac{1}{2^m} > 0$.
\end{solution}

\begin{problem} [34.1 \redtext{done}]
    Give an example showing that a Hausdorff space with a countable basis need not be metrizable.
\end{problem}
\begin{solution}
   $\bbr$ with $K$-topology, i.e., the open sets are the open sets in the usual topology, and sets of the form $(a, b) - K$ where $K = \{\frac{1}{n}\}_{n \in \bbn}$.

   $\bbr_K$ is finer than $\bbr$ with its usual topology, so it is Hausdorff. It has a countable basis: $\calb = \{(a, b), (a, b) - K : a, b \in \bbq\}$.

    But $\bbr_K$ is not regular, hence it can't be metrizable since metrizable implies normal implies regular.
\end{solution}

\begin{problem} [35.9]
    Let $X_1 \subset X_2 \subset \cdots$ be a sequence of spaces, where $X_i$ is a closed subspace of $X_{i+1}$ for each $i$. Let $X$ be the union of the $X_i$; let us topologize $X$ by declaring a set $U$ to be open in $X$ if $U \cap X_i$ is open in $X$ for each $i$.
    \begin{enumerate}
    \item Show that this is a topology on $X$ and that each space $X_i$ is a subspace (in fact, a closed subspace) of $X$ in this topology. This topology is called the topology \textit{coherent} with the subspaces $X_i$.
    \item Show that $f: X \to Y$ is continuous if $f|_{X_i}$ is continuous for each $i$.
    \item Show that if each space $X_i$ is normal, then $X$ is normal. (Hint: Given disjoint closed sets $A$ and $B$ in $X$, set $f$ equal to 0 on $A$ and 1 on $B$, and extend $f$ successively to $A \cup B \cup X_i$ for $i = 1, 2, \ldots$)
    \end{enumerate}
\end{problem}
\begin{solution}
\end{solution}
\end{document}