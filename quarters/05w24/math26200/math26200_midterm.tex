\documentclass[a4paper, 12pt]{article}
%%%%%%%%%%%%%%%%%%%%%%%%%%%%%%%%%%%%%%%%%%%%%%%%%%%%%%%%%%%%%%%%%%%%%%%%%%%%%%%
%                                Basic Packages                               %
%%%%%%%%%%%%%%%%%%%%%%%%%%%%%%%%%%%%%%%%%%%%%%%%%%%%%%%%%%%%%%%%%%%%%%%%%%%%%%%

% Gives us multiple colors.
\usepackage[usenames,dvipsnames,pdftex]{xcolor}
% Lets us style link colors.
\usepackage{hyperref}
% Lets us import images and graphics.
\usepackage{graphicx}
% Lets us use figures in floating environments.
\usepackage{float}
% Lets us create multiple columns.
\usepackage{multicol}
% Gives us better math syntax.
\usepackage{amsmath,amsfonts,mathtools,amsthm,amssymb}
% Lets us strikethrough text.
\usepackage{cancel}
% Lets us edit the caption of a figure.
\usepackage{caption}
% Lets us import pdf directly in our tex code.
\usepackage{pdfpages}
% Lets us do algorithm stuff.
\usepackage[ruled,vlined,linesnumbered]{algorithm2e}
% Use a smiley face for our qed symbol.
\usepackage{tikzsymbols}
% \usepackage{fullpage} %%smaller margins
\usepackage[shortlabels]{enumitem}

\setlist[enumerate]{font={\bfseries}} % global settings, for all lists

\usepackage{setspace}
\usepackage[margin=1in, headsep=12pt]{geometry}
\usepackage{wrapfig}
\usepackage{listings}
\usepackage{parskip}

\definecolor{codegreen}{rgb}{0,0.6,0}
\definecolor{codegray}{rgb}{0.5,0.5,0.5}
\definecolor{codepurple}{rgb}{0.58,0,0.82}
\definecolor{backcolour}{rgb}{0.95,0.95,0.95}

\lstdefinestyle{mystyle}{
    backgroundcolor=\color{backcolour},   
    commentstyle=\color{codegreen},
    keywordstyle=\color{magenta},
    numberstyle=\tiny\color{codegray},
    stringstyle=\color{codepurple},
    basicstyle=\ttfamily\footnotesize,
    breakatwhitespace=false,         
    breaklines=true,                 
    captionpos=b,                    
    keepspaces=true,                 
    numbers=left,                    
    numbersep=5pt,                  
    showspaces=false,                
    showstringspaces=false,
    showtabs=false,                  
    tabsize=2,
    numbers=none
}

\lstset{style=mystyle}
\def\class{article}


%%%%%%%%%%%%%%%%%%%%%%%%%%%%%%%%%%%%%%%%%%%%%%%%%%%%%%%%%%%%%%%%%%%%%%%%%%%%%%%
%                                Basic Settings                               %
%%%%%%%%%%%%%%%%%%%%%%%%%%%%%%%%%%%%%%%%%%%%%%%%%%%%%%%%%%%%%%%%%%%%%%%%%%%%%%%

%%%%%%%%%%%%%
%  Symbols  %
%%%%%%%%%%%%%

\let\implies\Rightarrow
\let\impliedby\Leftarrow
\let\iff\Leftrightarrow
\let\epsilon\varepsilon
%%%%%%%%%%%%
%  Tables  %
%%%%%%%%%%%%

\setlength{\tabcolsep}{5pt}
\renewcommand\arraystretch{1.5}

%%%%%%%%%%%%%%
%  SI Unitx  %
%%%%%%%%%%%%%%

\usepackage{siunitx}
\sisetup{locale = FR}

%%%%%%%%%%
%  TikZ  %
%%%%%%%%%%

\usepackage[framemethod=TikZ]{mdframed}
\usepackage{tikz}
\usepackage{tikz-cd}
\usepackage{tikzsymbols}

\usetikzlibrary{intersections, angles, quotes, calc, positioning}
\usetikzlibrary{arrows.meta}

\tikzset{
    force/.style={thick, {Circle[length=2pt]}-stealth, shorten <=-1pt}
}

%%%%%%%%%%%%%%%
%  PGF Plots  %
%%%%%%%%%%%%%%%

\usepackage{pgfplots}
\pgfplotsset{width=10cm, compat=newest}

%%%%%%%%%%%%%%%%%%%%%%%
%  Center Title Page  %
%%%%%%%%%%%%%%%%%%%%%%%

\usepackage{titling}
\renewcommand\maketitlehooka{\null\mbox{}\vfill}
\renewcommand\maketitlehookd{\vfill\null}

%%%%%%%%%%%%%%%%%%%%%%%%%%%%%%%%%%%%%%%%%%%%%%%%%%%%%%%
%  Create a grey background in the middle of the PDF  %
%%%%%%%%%%%%%%%%%%%%%%%%%%%%%%%%%%%%%%%%%%%%%%%%%%%%%%%

\usepackage{eso-pic}
\newcommand\definegraybackground{
    \definecolor{reallylightgray}{HTML}{FAFAFA}
    \AddToShipoutPicture{
        \ifthenelse{\isodd{\thepage}}{
            \AtPageLowerLeft{
                \put(\LenToUnit{\dimexpr\paperwidth-222pt},0){
                    \color{reallylightgray}\rule{222pt}{297mm}
                }
            }
        }
        {
            \AtPageLowerLeft{
                \color{reallylightgray}\rule{222pt}{297mm}
            }
        }
    }
}

%%%%%%%%%%%%%%%%%%%%%%%%
%  Modify Links Color  %
%%%%%%%%%%%%%%%%%%%%%%%%

\hypersetup{
    % Enable highlighting links.
    colorlinks,
    % Change the color of links to blue.
    urlcolor=blue,
    % Change the color of citations to black.
    citecolor={black},
    % Change the color of url's to blue with some black.
    linkcolor={blue!80!black}
}

%%%%%%%%%%%%%%%%%%
% Fix WrapFigure %
%%%%%%%%%%%%%%%%%%

\newcommand{\wrapfill}{\par\ifnum\value{WF@wrappedlines}>0
        \parskip=0pt
        \addtocounter{WF@wrappedlines}{-1}%
        \null\vspace{\arabic{WF@wrappedlines}\baselineskip}%
        \WFclear
    \fi}

%%%%%%%%%%%%%%%%%
% Multi Columns %
%%%%%%%%%%%%%%%%%

\let\multicolmulticols\multicols
\let\endmulticolmulticols\endmulticols

\RenewDocumentEnvironment{multicols}{mO{}}
{%
    \ifnum#1=1
        #2%
    \else % More than 1 column
        \multicolmulticols{#1}[#2]
    \fi
}
{%
    \ifnum#1=1
    \else % More than 1 column
        \endmulticolmulticols
    \fi
}

\newlength{\thickarrayrulewidth}
\setlength{\thickarrayrulewidth}{5\arrayrulewidth}


%%%%%%%%%%%%%%%%%%%%%%%%%%%%%%%%%%%%%%%%%%%%%%%%%%%%%%%%%%%%%%%%%%%%%%%%%%%%%%%
%                           School Specific Commands                          %
%%%%%%%%%%%%%%%%%%%%%%%%%%%%%%%%%%%%%%%%%%%%%%%%%%%%%%%%%%%%%%%%%%%%%%%%%%%%%%%

%%%%%%%%%%%%%%%%%%%%%%%%%%%
%  Initiate New Counters  %
%%%%%%%%%%%%%%%%%%%%%%%%%%%

\newcounter{lecturecounter}

%%%%%%%%%%%%%%%%%%%%%%%%%%
%  Helpful New Commands  %
%%%%%%%%%%%%%%%%%%%%%%%%%%

\makeatletter

\newcommand\resetcounters{
    % Reset the counters for subsection, subsubsection and the definition
    % all the custom environments.
    \setcounter{subsection}{0}
    \setcounter{subsubsection}{0}
    \setcounter{definition0}{0}
    \setcounter{paragraph}{0}
    \setcounter{theorem}{0}
    \setcounter{claim}{0}
    \setcounter{corollary}{0}
    \setcounter{proposition}{0}
    \setcounter{lemma}{0}
    \setcounter{exercise}{0}
    \setcounter{problem}{0}
    
    \setcounter{subparagraph}{0}
    % \@ifclasswith\class{nocolor}{
    %     \setcounter{definition}{0}
    % }{}
}

%%%%%%%%%%%%%%%%%%%%%
%  Lecture Command  %
%%%%%%%%%%%%%%%%%%%%%

\usepackage{xifthen}

% EXAMPLE:
% 1. \lecture{Oct 17 2022 Mon (08:46:48)}{Lecture Title}
% 2. \lecture[4]{Oct 17 2022 Mon (08:46:48)}{Lecture Title}
% 3. \lecture{Oct 17 2022 Mon (08:46:48)}{}
% 4. \lecture[4]{Oct 17 2022 Mon (08:46:48)}{}
% Parameters:
% 1. (Optional) lecture number.
% 2. Time and date of lecture.
% 3. Lecture Title.
\def\@lecture{}
\def\@lectitle{}
\def\@leccount{}
\newcommand\lecture[3]{
    \newpage

    % Check if user passed the lecture title or not.
    \def\@leccount{Lecture #1}
    \ifthenelse{\isempty{#3}}{
        \def\@lecture{Lecture #1}
        \def\@lectitle{Lecture #1}
    }{
        \def\@lecture{Lecture #1: #3}
        \def\@lectitle{#3}
    }

    \setcounter{section}{#1}
    \renewcommand\thesubsection{#1.\arabic{subsection}}
    
    \phantomsection
    \addcontentsline{toc}{section}{\@lecture}
    \resetcounters

    \begin{mdframed}
        \begin{center}
            \Large \textbf{\@leccount}
            
            \vspace*{0.2cm}
            
            \large \@lectitle
            
            
            \vspace*{0.2cm}

            \normalsize #2
        \end{center}
    \end{mdframed}

}

%%%%%%%%%%%%%%%%%%%%
%  Import Figures  %
%%%%%%%%%%%%%%%%%%%%

\usepackage{import}
\pdfminorversion=7

% EXAMPLE:
% 1. \incfig{limit-graph}
% 2. \incfig[0.4]{limit-graph}
% Parameters:
% 1. The figure name. It should be located in figures/NAME.tex_pdf.
% 2. (Optional) The width of the figure. Example: 0.5, 0.35.
\newcommand\incfig[2][1]{%
    \def\svgwidth{#1\columnwidth}
    \import{./figures/}{#2.pdf_tex}
}

\begingroup\expandafter\expandafter\expandafter\endgroup
\expandafter\ifx\csname pdfsuppresswarningpagegroup\endcsname\relax
\else
    \pdfsuppresswarningpagegroup=1\relax
\fi

%%%%%%%%%%%%%%%%%
% Fancy Headers %
%%%%%%%%%%%%%%%%%

\usepackage{fancyhdr}

% Force a new page.
\newcommand\forcenewpage{\clearpage\mbox{~}\clearpage\newpage}

% This command makes it easier to manage my headers and footers.
\newcommand\createintro{
    % Use roman page numbers (e.g. i, v, vi, x, ...)
    \pagenumbering{roman}

    % Display the page style.
    \maketitle
    % Make the title pagestyle empty, meaning no fancy headers and footers.
    \thispagestyle{empty}
    % Create a newpage.
    \newpage

    % Input the intro.tex page if it exists.
    \IfFileExists{intro.tex}{ % If the intro.tex file exists.
        % Input the intro.tex file.
        \textbf{Course}: MATH 16300: Honors Calculus III

\textbf{Section}: 43

\textbf{Professor}: Minjae Park

\textbf{At}: The University of Chicago

\textbf{Quarter}: Spring 2023

\textbf{Course materials}: Calculus by Spivak (4th Edition), Calculus On Manifolds by Spivak

\vspace{1cm}
\textbf{Disclaimer}: This document will inevitably contain some mistakes, both simple typos and serious logical and mathematical errors. Take what you read with a grain of salt as it is made by an undergraduate student going through the learning process himself. If you do find any error, I would really appreciate it if you can let me know by email at \href{mailto:conghungletran@gmail.com}{conghungletran@gmail.com}.

        % Make the pagestyle fancy for the intro.tex page.
        \pagestyle{fancy}

        % Remove the line for the header.
        \renewcommand\headrulewidth{0pt}

        % Remove all header stuff.
        \fancyhead{}

        % Add stuff for the footer in the center.
        % \fancyfoot[C]{
        %   \textit{For more notes like this, visit
        %   \href{\linktootherpages}{\shortlinkname}}. \\
        %   \vspace{0.1cm}
        %   \hrule
        %   \vspace{0.1cm}
        %   \@author, \\
        %   \term: \academicyear, \\
        %   Last Update: \@date, \\
        %   \faculty
        % }

        \newpage
    }{ % If the intro.tex file doesn't exist.
        % Force a \newpageage.
        % \forcenewpage
        \newpage
    }

    % Remove the center stuff we did above, and replace it with just the page
    % number, which is still in roman numerals.
    \fancyfoot[C]{\thepage}
    % Add the table of contents.
    \tableofcontents
    % Force a new page.
    \newpage

    % Move the page numberings back to arabic, from roman numerals.
    \pagenumbering{arabic}
    % Set the page number to 1.
    \setcounter{page}{1}

    % Add the header line back.
    \renewcommand\headrulewidth{0.4pt}
    % In the top right, add the lecture title.
    \fancyhead[R]{\footnotesize \@lecture}
    % In the top left, add the author name.
    \fancyhead[L]{\footnotesize \@author}
    % In the bottom center, add the page.
    \fancyfoot[C]{\thepage}
    % Add a nice gray background in the middle of all the upcoming pages.
    % \definegraybackground
}

\makeatother


%%%%%%%%%%%%%%%%%%%%%%%%%%%%%%%%%%%%%%%%%%%%%%%%%%%%%%%%%%%%%%%%%%%%%%%%%%%%%%%
%                               Custom Commands                               %
%%%%%%%%%%%%%%%%%%%%%%%%%%%%%%%%%%%%%%%%%%%%%%%%%%%%%%%%%%%%%%%%%%%%%%%%%%%%%%%

%%%%%%%%%%%%
%  Circle  %
%%%%%%%%%%%%

\newcommand*\circled[1]{\tikz[baseline= (char.base)]{
        \node[shape=circle,draw,inner sep=1pt] (char) {#1};}
}

%%%%%%%%%%%%%%%%%%%
%  Todo Commands  %
%%%%%%%%%%%%%%%%%%%

% \usepackage{xargs}
% \usepackage[colorinlistoftodos]{todonotes}

% \makeatletter

% \@ifclasswith\class{working}{
%     \newcommandx\unsure[2][1=]{\todo[linecolor=red,backgroundcolor=red!25,bordercolor=red,#1]{#2}}
%     \newcommandx\change[2][1=]{\todo[linecolor=blue,backgroundcolor=blue!25,bordercolor=blue,#1]{#2}}
%     \newcommandx\info[2][1=]{\todo[linecolor=OliveGreen,backgroundcolor=OliveGreen!25,bordercolor=OliveGreen,#1]{#2}}
%     \newcommandx\improvement[2][1=]{\todo[linecolor=Plum,backgroundcolor=Plum!25,bordercolor=Plum,#1]{#2}}

%     \newcommand\listnotes{
%         \newpage
%         \listoftodos[Notes]
%     }
% }{
%     \newcommandx\unsure[2][1=]{}
%     \newcommandx\change[2][1=]{}
%     \newcommandx\info[2][1=]{}
%     \newcommandx\improvement[2][1=]{}

%     \newcommand\listnotes{}
% }

% \makeatother

%%%%%%%%%%%%%
%  Correct  %
%%%%%%%%%%%%%

% EXAMPLE:
% 1. \correct{INCORRECT}{CORRECT}
% Parameters:
% 1. The incorrect statement.
% 2. The correct statement.
\definecolor{correct}{HTML}{009900}
\newcommand\correct[2]{{\color{red}{#1 }}\ensuremath{\to}{\color{correct}{ #2}}}


%%%%%%%%%%%%%%%%%%%%%%%%%%%%%%%%%%%%%%%%%%%%%%%%%%%%%%%%%%%%%%%%%%%%%%%%%%%%%%%
%                                 Environments                                %
%%%%%%%%%%%%%%%%%%%%%%%%%%%%%%%%%%%%%%%%%%%%%%%%%%%%%%%%%%%%%%%%%%%%%%%%%%%%%%%

\usepackage{varwidth}
\usepackage{thmtools}
\usepackage[most,many,breakable]{tcolorbox}

\tcbuselibrary{theorems,skins,hooks}
\usetikzlibrary{arrows,calc,shadows.blur}

%%%%%%%%%%%%%%%%%%%
%  Define Colors  %
%%%%%%%%%%%%%%%%%%%

% color prototype
% \definecolor{color}{RGB}{45, 111, 177}

% ESSENTIALS: 
\definecolor{myred}{HTML}{c74540}
\definecolor{myblue}{HTML}{072b85}
\definecolor{mygreen}{HTML}{388c46}
\definecolor{myblack}{HTML}{000000}

\colorlet{definition_color}{myred}

\colorlet{theorem_color}{myblue}
\colorlet{lemma_color}{myblue}
\colorlet{prop_color}{myblue}
\colorlet{corollary_color}{myblue}
\colorlet{claim_color}{myblue}

\colorlet{proof_color}{myblack}
\colorlet{example_color}{myblack}
\colorlet{exercise_color}{myblack}

% MISCS: 
%%%%%%%%%%%%%%%%%%%%%%%%%%%%%%%%%%%%%%%%%%%%%%%%%%%%%%%%%
%  Create Environments Styles Based on Given Parameter  %
%%%%%%%%%%%%%%%%%%%%%%%%%%%%%%%%%%%%%%%%%%%%%%%%%%%%%%%%%

% \mdfsetup{skipabove=1em,skipbelow=0em}

%%%%%%%%%%%%%%%%%%%%%%
%  Helpful Commands  %
%%%%%%%%%%%%%%%%%%%%%%

% EXAMPLE:
% 1. \createnewtheoremstyle{thmdefinitionbox}{}{}
% 2. \createnewtheoremstyle{thmtheorembox}{}{}
% 3. \createnewtheoremstyle{thmproofbox}{qed=\qedsymbol}{
%       rightline=false, topline=false, bottomline=false
%    }
% Parameters:
% 1. Theorem name.
% 2. Any extra parameters to pass directly to declaretheoremstyle.
% 3. Any extra parameters to pass directly to mdframed.
\newcommand\createnewtheoremstyle[3]{
    \declaretheoremstyle[
        headfont=\bfseries\sffamily, bodyfont=\normalfont, #2,
        mdframed={
                #3,
            },
    ]{#1}
}

% EXAMPLE:
% 1. \createnewcoloredtheoremstyle{thmdefinitionbox}{definition}{}{}
% 2. \createnewcoloredtheoremstyle{thmexamplebox}{example}{}{
%       rightline=true, leftline=true, topline=true, bottomline=true
%     }
% 3. \createnewcoloredtheoremstyle{thmproofbox}{proof}{qed=\qedsymbol}{backgroundcolor=white}
% Parameters:
% 1. Theorem name.
% 2. Color of theorem.
% 3. Any extra parameters to pass directly to declaretheoremstyle.
% 4. Any extra parameters to pass directly to mdframed.

% change backgroundcolor to #2!5 if user wants a colored backdrop to theorem environments. It's a cool color theme, but there's too much going on in the page.
\newcommand\createnewcoloredtheoremstyle[4]{
    \declaretheoremstyle[
        headfont=\bfseries\sffamily\color{#2},
        bodyfont=\normalfont,
        headpunct=,
        headformat = \NAME~\NUMBER\NOTE \hfill\smallskip\linebreak,
        #3,
        mdframed={
                outerlinewidth=0.75pt,
                rightline=false,
                leftline=false,
                topline=false,
                bottomline=false,
                backgroundcolor=white,
                skipabove = 5pt,
                skipbelow = 0pt,
                linecolor=#2,
                innertopmargin = 0pt,
                innerbottommargin = 0pt,
                innerrightmargin = 4pt,
                innerleftmargin= 6pt,
                leftmargin = -6pt,
                #4,
            },
    ]{#1}
}



%%%%%%%%%%%%%%%%%%%%%%%%%%%%%%%%%%%
%  Create the Environment Styles  %
%%%%%%%%%%%%%%%%%%%%%%%%%%%%%%%%%%%

\makeatletter
\@ifclasswith\class{nocolor}{
    % Environments without color.

    % ESSENTIALS:
    \createnewtheoremstyle{thmdefinitionbox}{}{}
    \createnewtheoremstyle{thmtheorembox}{}{}
    \createnewtheoremstyle{thmproofbox}{qed=\qedsymbol}{}
    \createnewtheoremstyle{thmcorollarybox}{}{}
    \createnewtheoremstyle{thmlemmabox}{}{}
    \createnewtheoremstyle{thmclaimbox}{}{}
    \createnewtheoremstyle{thmexamplebox}{}{}

    % MISCS: 
    \createnewtheoremstyle{thmpropbox}{}{}
    \createnewtheoremstyle{thmexercisebox}{}{}
    \createnewtheoremstyle{thmexplanationbox}{}{}
    \createnewtheoremstyle{thmremarkbox}{}{}
    
    % STYLIZED MORE BELOW
    \createnewtheoremstyle{thmquestionbox}{}{}
    \createnewtheoremstyle{thmsolutionbox}{qed=\qedsymbol}{}
}{
    % Environments with color.

    % ESSENTIALS: definition, theorem, proof, corollary, lemma, claim, example
    \createnewcoloredtheoremstyle{thmdefinitionbox}{definition_color}{}{leftline=false}
    \createnewcoloredtheoremstyle{thmtheorembox}{theorem_color}{}{leftline=false}
    \createnewcoloredtheoremstyle{thmproofbox}{proof_color}{qed=\qedsymbol}{}
    \createnewcoloredtheoremstyle{thmcorollarybox}{corollary_color}{}{leftline=false}
    \createnewcoloredtheoremstyle{thmlemmabox}{lemma_color}{}{leftline=false}
    \createnewcoloredtheoremstyle{thmpropbox}{prop_color}{}{leftline=false}
    \createnewcoloredtheoremstyle{thmclaimbox}{claim_color}{}{leftline=false}
    \createnewcoloredtheoremstyle{thmexamplebox}{example_color}{}{}
    \createnewcoloredtheoremstyle{thmexplanationbox}{example_color}{qed=\qedsymbol}{}
    \createnewcoloredtheoremstyle{thmremarkbox}{theorem_color}{}{}

    \createnewcoloredtheoremstyle{thmmiscbox}{black}{}{}

    \createnewcoloredtheoremstyle{thmexercisebox}{exercise_color}{}{}
    \createnewcoloredtheoremstyle{thmproblembox}{theorem_color}{}{leftline=false}
    \createnewcoloredtheoremstyle{thmsolutionbox}{mygreen}{qed=\qedsymbol}{}
}
\makeatother

%%%%%%%%%%%%%%%%%%%%%%%%%%%%%
%  Create the Environments  %
%%%%%%%%%%%%%%%%%%%%%%%%%%%%%
\declaretheorem[numberwithin=section, style=thmdefinitionbox,     name=Definition]{definition}
\declaretheorem[numberwithin=section, style=thmtheorembox,     name=Theorem]{theorem}
\declaretheorem[numbered=no,          style=thmexamplebox,     name=Example]{example}
\declaretheorem[numberwithin=section, style=thmtheorembox,       name=Claim]{claim}
\declaretheorem[numberwithin=section, style=thmcorollarybox,   name=Corollary]{corollary}
\declaretheorem[numberwithin=section, style=thmpropbox,        name=Proposition]{proposition}
\declaretheorem[numberwithin=section, style=thmlemmabox,       name=Lemma]{lemma}
\declaretheorem[numberwithin=section, style=thmexercisebox,    name=Exercise]{exercise}
\declaretheorem[numbered=no,          style=thmproofbox,       name=Proof]{proof0}
\declaretheorem[numbered=no,          style=thmexplanationbox, name=Explanation]{explanation}
\declaretheorem[numbered=no,          style=thmsolutionbox,    name=Solution]{solution}
\declaretheorem[numberwithin=section,          style=thmproblembox,     name=Problem]{problem}
\declaretheorem[numbered=no,          style=thmmiscbox,    name=Intuition]{intuition}
\declaretheorem[numbered=no,          style=thmmiscbox,    name=Goal]{goal}
\declaretheorem[numbered=no,          style=thmmiscbox,    name=Recall]{recall}
\declaretheorem[numbered=no,          style=thmmiscbox,    name=Motivation]{motivation}
\declaretheorem[numbered=no,          style=thmmiscbox,    name=Remark]{remark}
\declaretheorem[numbered=no,          style=thmmiscbox,    name=Observe]{observe}
\declaretheorem[numbered=no,          style=thmmiscbox,    name=Question]{question}


%%%%%%%%%%%%%%%%%%%%%%%%%%%%
%  Edit Proof Environment  %
%%%%%%%%%%%%%%%%%%%%%%%%%%%%

\renewenvironment{proof}[2][\proofname]{
    % \vspace{-12pt}
    \begin{proof0} [#2]
        }{\end{proof0}}

\theoremstyle{definition}

\newtheorem*{notation}{Notation}
\newtheorem*{previouslyseen}{As previously seen}
\newtheorem*{property}{Property}
% \newtheorem*{intuition}{Intuition}
% \newtheorem*{goal}{Goal}
% \newtheorem*{recall}{Recall}
% \newtheorem*{motivation}{Motivation}
% \newtheorem*{remark}{Remark}
% \newtheorem*{observe}{Observe}

\author{Hung C. Le Tran}


%%%% MATH SHORTHANDS %%%%
%% blackboard bold math capitals
\DeclareMathOperator*{\esssup}{ess\,sup}
\DeclareMathOperator*{\Hom}{Hom}
\newcommand{\bbf}{\mathbb{F}}
\newcommand{\bbn}{\mathbb{N}}
\newcommand{\bbq}{\mathbb{Q}}
\newcommand{\bbr}{\mathbb{R}}
\newcommand{\bbz}{\mathbb{Z}}
\newcommand{\bbc}{\mathbb{C}}
\newcommand{\bbk}{\mathbb{K}}
\newcommand{\bbm}{\mathbb{M}}
\newcommand{\bbp}{\mathbb{P}}
\newcommand{\bbe}{\mathbb{E}}

\newcommand{\bfw}{\mathbf{w}}
\newcommand{\bfx}{\mathbf{x}}
\newcommand{\bfX}{\mathbf{X}}
\newcommand{\bfy}{\mathbf{y}}
\newcommand{\bfyhat}{\mathbf{\hat{y}}}

\newcommand{\calb}{\mathcal{B}}
\newcommand{\calf}{\mathcal{F}}
\newcommand{\calt}{\mathcal{T}}
\newcommand{\call}{\mathcal{L}}
\renewcommand{\phi}{\varphi}

% Universal Math Shortcuts
\newcommand{\st}{\hspace*{2pt}\text{s.t.}\hspace*{2pt}}
\newcommand{\pffwd}{\hspace*{2pt}\fbox{\(\Rightarrow\)}\hspace*{10pt}}
\newcommand{\pfbwd}{\hspace*{2pt}\fbox{\(\Leftarrow\)}\hspace*{10pt}}
\newcommand{\contra}{\ensuremath{\Rightarrow\Leftarrow}}
\newcommand{\cvgn}{\xrightarrow{n \to \infty}}
\newcommand{\cvgj}{\xrightarrow{j \to \infty}}

\newcommand{\im}{\mathrm{im}}
\newcommand{\innerproduct}[2]{\langle #1, #2 \rangle}
\newcommand*{\conj}[1]{\overline{#1}}

% https://tex.stackexchange.com/questions/438612/space-between-exists-and-forall
% https://tex.stackexchange.com/questions/22798/nice-looking-empty-set
\let\oldforall\forall
\renewcommand{\forall}{\;\oldforall\; }
\let\oldexist\exists
\renewcommand{\exists}{\;\oldexist\; }
\newcommand\existu{\;\oldexist!\: }
\let\oldemptyset\emptyset
\let\emptyset\varnothing


\renewcommand{\_}[1]{\underline{#1}}
\DeclarePairedDelimiter{\abs}{\lvert}{\rvert}
\DeclarePairedDelimiter{\norm}{\lVert}{\rVert}
\DeclarePairedDelimiter\ceil{\lceil}{\rceil}
\DeclarePairedDelimiter\floor{\lfloor}{\rfloor}
\setlength\parindent{0pt}
\setlength{\headheight}{12.0pt}
\addtolength{\topmargin}{-12.0pt}


% Default skipping, change if you want more spacing
% \thinmuskip=3mu
% \medmuskip=4mu plus 2mu minus 4mu
% \thickmuskip=5mu plus 5mu

% \DeclareMathOperator{\ext}{ext}
% \DeclareMathOperator{\bridge}{bridge}
\title{MATH 26200: Point-Set Topology \\ \large Take-home Midterm Exam}
\date{31 Jan 2024}
\author{Hung Le Tran}
\begin{document}
\maketitle
\setcounter{section}{4}
\textbf{Textbook:} Munkres, \textit{Topology}  
\begin{problem} [\redtext{done}]
    Let $\bbr$ denote the real numbers. Give $\bbr$ the topology in which the closed sets (other than all of $\bbr$) are the finite subsets. Verify that this is a topology, and prove that if $p(x)$ is a polynomial (with real coefficients), the function $x \mapsto p(x)$ is continuous in this topology.
\end{problem}
\begin{solution}
    We verify the properties of a topology through its closed sets:
    \begin{enumerate} [1.]
        \item $\bbr$ is closed by hypothesis. $\emptyset$ has $0 < \infty$ elements, so is closed.
        \item Arbitrary intersections of closed $\{K_\alpha\}_{\alpha \in A}$, $\bigcap K_\alpha$, has at most $\abs{K_\alpha}$ elements for some $\alpha \in A$, but $\abs{K_\alpha} < \infty$ so $\bigcap K_\alpha$ is also closed.
        \item Finite unions of clsoed $\{K_i\}_{i \in [N]}$ has at most $\sum_{i=1}^{N} \abs{K_i}  < \infty$, so is also closed.
    \end{enumerate}

    It follows that this is indeed a topology.

    Now, let $p(x)$ be any polynomial in $x$. Let $N = \deg (p(x)) \in \bbz_{\geq 0}$. To show that $p$ is continuous, want to show preimages of closed sets are closed. Let $K = \{y_1, \ldots, y_n\}$ be any closed set. Then:
    \begin{align*}
    p^{-1}(K) &= \bigcup_{i \in [n]} p^{-1}(y_i) \\
    &= \bigcup_{i \in [n]} \{x : p(x) - y_i = 0\}
    \end{align*}
    And we have $\deg (p(x)) = N \implies \deg(p(x) - y_i) = N $ so it has at most $N$ roots. Therefore: 
    \begin{equation*}
    \abs{p^{-1}(K)} \leq Nn < \infty    
    \end{equation*}
    so $p^{-1}(K)$ is closed, as required.
\end{solution}

\begin{problem} [18.1 \redtext{done}]
    Prove that for functions $f: \bbr \to \bbr$, the $\epsilon - \delta$ definition of continuity implies the open set definition.
\end{problem}
\begin{solution}
    WTS if a function $f: \bbr \to \bbr$ satisfies the $\epsilon - \delta$ condition, then it is continuous in the open set definition.

    Take any $W \subset \bbr$ open, want to prove that $f^{-1}(W)$ is open. Take $x \in f^{-1}(W)$, i.e., $f(x) \in W$. $f(x) \in W$ open, so there exists a basis $f(x) \in B(f(x'), \epsilon') \subset W$. Define $\epsilon \coloneqq \frac{1}{2} \min\{\epsilon' - \abs{f(x) - f(x')}, \abs{f(x) - f(x')}\} > 0$, then $f(x) \in B(f(x), \epsilon) \subset B(f(x'), \epsilon') \subset W$.
    
    From the $\epsilon - \delta$ condition, it follows that there exists some $\delta > 0$ such that $f(B(x, \delta)) \subset B(f(x), \epsilon) \subset W$. It therefore follows that $B(x, \delta) \subset f^{-1}(W)$. We can do this for all $x \in f^{-1}(W)$, so it is open. $f$ is therefore continuous in the open set definition.
\end{solution}

\begin{problem} [19.8 \redtext{done}]
    Given sequences $(a_1, a_2, \ldots)$ and $(b_1, b_2, \ldots)$ of real numbers with $a_i > 0$ for all $i$, define $h: \bbr^\omega \to \bbr^\omega$ by the equaiton:\begin{equation*}
    h((x_1, x_2, \ldots)) = (a_1 x_1 + b_1, a_2x_2 + b_2, \ldots)
    \end{equation*}

Show that if $\bbr^\omega$ is given the product topology, $h$ is a homeomorphism of $\bbr^\omega$ with itself. What happens if $\bbr^\omega$ is given the box topology?
\end{problem}
\begin{solution}
    To show that $h$ is homeo, we want to show that it is bijective, continuous and that $h^{-1}$ is also continuous.

    \begin{enumerate} [1.]
    \item We write \begin{equation*}
        g((y_1, y_2, \ldots)) = \left(\frac{y_1 - b_1}{a_1}, \frac{y_2 - b_2}{a_2}, \ldots\right)  
    \end{equation*}
    This map is well-defined since $a_i > 0 \forall i \in \bbn$. It's trivial that $h \circ g = g \circ h = id$. It follows that $h$ is a bijection, with inverse $h^{-1} = g$.
    \item WTS each of $h$'s coordinate functions is continuous. Define, for each $i$, $f_i: \bbr \to \bbr, f_i(x) = a_i x + b_i$ then $f_i$ is trivially continuous. Then $h_i(x) = f_i \circ \pi_i$ is a composition of 2 continuous functions, and is therefore also continuous. Since each of $h$'s coordinate functions is continuous, $h$ is also continuous.
    \item $g = h^{-1}$ is continuous because of the same reason, each $g_i$ is similarly continuous.
    \end{enumerate}
    Therefore indeed $h$ is homeo.

    In the box topology, claim that $h$ is also a homeo. First, trivially, it is still a bijection. 
    
    Take the typical basis element $\prod_{i \in \bbn} U_i$, then we have 
    \begin{equation*}
        h^{-1}(\prod_{i \in \bbn} U_i) = \prod_{i \in \bbn} h_i^{-1}\left(U_i\right) 
    \end{equation*}
    Each $h_i$, as mentioned above, is continuous so $h_i^{-1}(U_i)$ is open in $\bbr$. Therefore $\prod_{i \in \bbn} h_i^{-1}\left(U_i\right)$ is open in the box topology, which implies $h$ is continuous.

    The proof for $g$ is continuous is similar, since each $g_i$ is similarly continuous. It follows that $h$ is a homeo in the box topology as well.
\end{solution}

\begin{problem} [23.5 \redtext{done}]
    A space is \textbf{totally disconnected} if its only connected subspaces are one-point sets. Show that if $X$ has the discrete topology, then $X$ is totally disconnected. Does the converse hold?
\end{problem}
\begin{solution}
    By hypothesis, $X$ has the discrete topology. One-point sets are clearly connected, since there can't be a separation with 2 non-empty disjoint clopen subsets, which would make the number of elements in the set $\geq 2$. Suppose $K$ with $\abs{K} \geq 2$ is also connected. Then there exists $a \in K$, and $K = \{a\} \sqcup (K - \{a\})$ is a separation of $K$ into 2 non-empty closed sets, so $K$ is not connected, \contra.

    It follows that the only connected subspaces are one-point sets (technically $\emptyset$ is also connected, but I figure the connected subspaces should be non-trivial), so $X$ is totally disconnected as required.

    % WTS the converse holds. Note that one-point sets are always connected, since its number of elements does not allow it to have a separation with 2 non-empty disjoint subsets.

    % If $X$ has 1 point then it holds trivially that the only (non-trivial) connected subspaces are one-point sets — there are no other sets.

    % If $X$ has more than 2 points, then for all $x \in X$, choose $y \in x, y \in X$, the set $\{x , y\}$ is not connected, meaning there exists some separation with 2 non-empty, disjoint, clopen subsets. The subsets can therefore only be $\{x\}$ and $\{y\}$, and they are clopen. Therefore $\{x\}$ is open for all $x \in X$, so $X$ has the discrete topology.

    The converse does not hold. Note that one-point sets are always connected, since its number of elements does not allow it to have a separation with 2 non-empty disjoint subsets. 
    
    Take $\bbq$ in the subspace topology, $\bbq \subset \bbr$. This is not the discrete topology, for each basis element $\bbq \cap B(x, r)$ contains infinitely many points. However, we claim that $\bbq$ is indeed totally disconnected. Suppose for sake of contradiction that there exists $U \subset \bbq$ connected such that $\abs{U} \geq 2$. Then we can find $q_1, q_2 \in U; q_1 < q_2$. We can then exhibit a separation of $U$:
    \begin{equation*}
    U = (U \cap (-\infty, p) \sqcup (U \cap (p, +\infty))
    \end{equation*}
    where $p$ is some irrational such that $q_1 < p < q_2$. They are clearly disjoint and open, and non-empty since $q_1$ and $q_2$ are respectively in them. So $U$ is not connected. $\bbq$ is therefore totally disconnected, but doesn't have the discrete topology.
\end{solution}

\begin{problem} [24.1 \redtext{done}]
    \begin{enumerate}
    \item[] 
    \item Show that no two of the spaces $(0, 1), (0, 1]$ and $[0, 1]$ are homeomorphic. [Hint: What happens if you remove a point from each of these spaces?]
    \item Suppose that there exist embeddings $f: X \to Y$ and $g: Y \to X$. Show by means of an example that $X$ and $Y$ need not be homeomorphic.
    \item Show $\bbr^n$ and $\bbr$ are not homeomorphic if $n > 1$.
\end{enumerate}
\end{problem}
\begin{solution}
\textbf{(a)} Removing any point $\alpha$ from $(0, 1)$ makes it a disconnected space: $(0, 1) = (0, \alpha) \sqcup (\alpha, 1)$, while removing $1$ from $(0, 1]$ keeps the space connected ($(0, 1)$ is connected) and removing $1$ from $[0, 1]$ also keeps the space connected ($[0, 1)$ is connected), so $(0, 1)$ is not homeomorphic to $(0, 1]$ nor is it homeomorphic to $[0, 1]$.

Removing any 2 distinct points from $(0, 1]$ makes it a disconnected space (removing $1$ and another point $\alpha \in (0, 1)$ makes it disconnected: $(0, \alpha) \sqcup (\alpha, 1)$; and removing $\alpha, \beta \in (0, 1)$ yields $(0, \alpha) \sqcup (\alpha, 1]$), while removing $0$ and $1$ from $[0, 1]$ keeps the space connected ($(0, 1)$ is connected), so $(0, 1]$ is not homeomorphic to $[0, 1]$.

Why this ``removing'' reasoning work is that suppose there exists a homeomorphism $f: (0, 1] \to (0, 1)$. Then $f((0, 1)) = (0, 1) - \{f(1)\}$; the LHS is connected (since $f$ is continuous and $(0, 1)$ is connected) while the RHS is not, \contra.

Similar for the removing-2-point case.

\textbf{(b)} Take $X = (0, 1), Y = [0, 1]$, then $f: X \to Y, f(x) = \frac{x}{2}$ and $g: Y \to X, g(y) = \frac{y+1}{4}$ are embeddings, but $(0, 1)$ and $[0, 1]$ are not homeomorphic as abovementioned.

\textbf{(c)} $n > 1$. 

Removing any point $\alpha$ from $\bbr$ makes it a disconnected space: $(- \infty, \alpha) \sqcup (\alpha, \infty)$.

However, removing $0 \in \bbr^n$ keeps it connected. So $\bbr^n - \{0\}$ is not homeomorphic to $\bbr$.
\end{solution}

\begin{problem} [24.10 \redtext{done}]
    Show that if $U$ is an open connected subspace of $\R^2$ then $U$ is path connected. [Hint: Show that given $x_0 \in U$, the set of points that can be joined to $x_0$ by a path in $U$ is both open and closed in $U$.]
\end{problem}
\begin{solution}
    Take $x_0 \in U$. Let $A = \{x \in U: \:\text{there exists a path}\: x_0 \to x\}$. Then $x_0 \in A$, so $A$ is non-empty. WTS $A$ is clopen in $U$.

    Take any $x \in A$, then since $U$ is open, there exists some $B(x, \epsilon)$ (can always center the ball) such that $x \in B(x, \epsilon) \subset U$. Take any $y \in B(x, \epsilon)$, then since the ball is convex, the segment $x \to y$ is contained in $B(x, \epsilon)$, and therefore in $U$. Therefore $f(t) = x + t(y-x)$ is a path from $x \to y$. $x \in A$, so there exists a path from $x_0 \to x$. Concatenate these 2 paths and we get a path from $x_0 \to y$, so $y \in A$. This works for any $y \in B(x, \epsilon)$, so $B(x, \epsilon) \subset A$. This works for any $x \in A$, so $A$ is open.

    $A$ is also closed, take $X - A = \{x \in U: \:\text{there is no path}\: x_0 \to x\}$. Take any $x \in X - A$, then since $U$ is open, there exists some $B(x, \epsilon)$ (can always center the ball) such that $x \in B(x, \epsilon) \subset U$. Take any $y \in B(x, \epsilon)$, then similarly, $f(t) = x + t(y-x)$ is a path from $x \to y$. We can conclude that $y \in X -A$, because if $y \in A$, meaning there's a path from $x_0 \to y$, then one can concatenate that path with the path from $x \to y$ in reverse to get a path from $x_0 \to x$, but $x \not \in A$, so it would be a contradiction. Therefore for any $y \in B(x, \epsilon)$, $y \in X - A \implies B(x, \epsilon) \subset X - A$. This works for any $x \in X - A$, so $X - A$ is open, so $A$ is closed.

    $A$ is clopen and non-empty, $A \subset U$ connected, so $A = U$. There is a path from $x_0$ to all $x \in U$. Then for all $x, y \in U$, concatenate path $x_0\to x$ in reverse to path $x_0\to y$, and we get path $x \to y$. So $U$ is path-connected.
\end{solution}

\begin{problem} [26.11 \redtext{done}]
Let $X$ be a compact Hausdorff space. Let $\cala$ be a collection of closed connected subsets of $X$ that is simply ordered by proper inclusion. Then \begin{equation*}
Y = \bigcap_{A \in \cala} A
\end{equation*}
is connected. [Hint: if $C \cup D$ is a separation of $Y$, choose disjoint open sets $U$ and $V$ of $X$ containing $C$ and $D$, respectively, and show that $\bigcap_{A \in \cala}(A - (U \cup V))$ is not empty.]
\end{problem}
\begin{solution}
    Each $A$ is closed so $Y = \bigcap_{A \in \cala} A$ is also closed in $X$.

    Suppose, for the sake of contradiction, that $Y$ is not connected, i.e., that there exists separation $Y = C \sqcup D$ of $C, D$ non-empty, disjoint, clopen sets in $Y$. $C, D$ are closed in $Y$, $Y$ closed in $X$ so they are closed in $X$.

    We use a fact, shown in class, that compact Hausdorff spaces are normal. So $X$ is normal. Therefore, since $C, D$ are closed and disjoint in $X$, there exists open $U, V \subset X$  such that $C \subset U, D \subset V$ and $U \cap V = \emptyset$.

    Then, for each $A \in \cala$, claim that $(A \cap U) \cup (A \cap V) \neq A$. Suppose not, that $(A \cap U) \cup (A \cap V) = A$, then $U \cap V = \emptyset$ implies that $(A \cap U) \cap (A \cap V) = \emptyset$, with both $(A \cap U), (A \cap V)$ open in $A$, and $A \cap U \supset C$, $A \cap V \supset D$ so they are both non-empty. Hence we get a separation of $A$, but $A$ is connected, \contra.

    It follows that $(A \cap U) \cup (A \cap V) \neq A \implies A - (A\cap U) \cup (A \cap V) = A - U \cup V \neq \emptyset$ for all $A \in \cala$.

    $U \cup V$ is open in $X$, so $A - U \cup V$ is closed in $X$. $X$ is compact, so $A - U \cup V$ is compact. $\cala$ is a collection of $A$ ordered by proper inclusion, so $\{A - U\cup V\}_{A \in \cala} $ is a collection of non-empty, compact, subsets ordered by proper inclusion.

    It then follows that \begin{equation*}
    \bigcap_{A \in \cala} (A - U\cup V) \neq \emptyset
    \end{equation*}

    However, \begin{equation*}
        \bigcap_{A \in \cala} (A - U\cup V) = \left(\bigcap_{A \in \cala} A\right) - U \cup V = Y - U \cup V = C \cup D - U \cup V = \emptyset, \contra
    \end{equation*}

    It follows that $Y$ is connected.
\end{solution}

\end{document}