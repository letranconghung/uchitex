\documentclass[a4paper, 10pt]{article}
%%%%%%%%%%%%%%%%%%%%%%%%%%%%%%%%%%%%%%%%%%%%%%%%%%%%%%%%%%%%%%%%%%%%%%%%%%%%%%%
%                                Basic Packages                               %
%%%%%%%%%%%%%%%%%%%%%%%%%%%%%%%%%%%%%%%%%%%%%%%%%%%%%%%%%%%%%%%%%%%%%%%%%%%%%%%

% Gives us multiple colors.
\usepackage[usenames,dvipsnames,pdftex]{xcolor}
% Lets us style link colors.
\usepackage{hyperref}
% Lets us import images and graphics.
\usepackage{graphicx}
% Lets us use figures in floating environments.
\usepackage{float}
% Lets us create multiple columns.
\usepackage{multicol}
% Gives us better math syntax.
\usepackage{amsmath,amsfonts,mathtools,amsthm,amssymb}
% Lets us strikethrough text.
\usepackage{cancel}
% Lets us edit the caption of a figure.
\usepackage{caption}
% Lets us import pdf directly in our tex code.
\usepackage{pdfpages}
% Lets us do algorithm stuff.
\usepackage[ruled,vlined,linesnumbered]{algorithm2e}
% Use a smiley face for our qed symbol.
\usepackage{tikzsymbols}
% \usepackage{fullpage} %%smaller margins
\usepackage[shortlabels]{enumitem}

\setlist[enumerate]{font={\bfseries}} % global settings, for all lists

\usepackage{setspace}
\usepackage[margin=1in, headsep=12pt]{geometry}
\usepackage{wrapfig}
\usepackage{listings}
\usepackage{parskip}

\definecolor{codegreen}{rgb}{0,0.6,0}
\definecolor{codegray}{rgb}{0.5,0.5,0.5}
\definecolor{codepurple}{rgb}{0.58,0,0.82}
\definecolor{backcolour}{rgb}{0.95,0.95,0.95}

\lstdefinestyle{mystyle}{
    backgroundcolor=\color{backcolour},   
    commentstyle=\color{codegreen},
    keywordstyle=\color{magenta},
    numberstyle=\tiny\color{codegray},
    stringstyle=\color{codepurple},
    basicstyle=\ttfamily\footnotesize,
    breakatwhitespace=false,         
    breaklines=true,                 
    captionpos=b,                    
    keepspaces=true,                 
    numbers=left,                    
    numbersep=5pt,                  
    showspaces=false,                
    showstringspaces=false,
    showtabs=false,                  
    tabsize=2,
    numbers=none
}

\lstset{style=mystyle}
\def\class{article}


%%%%%%%%%%%%%%%%%%%%%%%%%%%%%%%%%%%%%%%%%%%%%%%%%%%%%%%%%%%%%%%%%%%%%%%%%%%%%%%
%                                Basic Settings                               %
%%%%%%%%%%%%%%%%%%%%%%%%%%%%%%%%%%%%%%%%%%%%%%%%%%%%%%%%%%%%%%%%%%%%%%%%%%%%%%%

%%%%%%%%%%%%%
%  Symbols  %
%%%%%%%%%%%%%

\let\implies\Rightarrow
\let\impliedby\Leftarrow
\let\iff\Leftrightarrow
\let\epsilon\varepsilon
%%%%%%%%%%%%
%  Tables  %
%%%%%%%%%%%%

\setlength{\tabcolsep}{5pt}
\renewcommand\arraystretch{1.5}

%%%%%%%%%%%%%%
%  SI Unitx  %
%%%%%%%%%%%%%%

\usepackage{siunitx}
\sisetup{locale = FR}

%%%%%%%%%%
%  TikZ  %
%%%%%%%%%%

\usepackage[framemethod=TikZ]{mdframed}
\usepackage{tikz}
\usepackage{tikz-cd}
\usepackage{tikzsymbols}

\usetikzlibrary{intersections, angles, quotes, calc, positioning}
\usetikzlibrary{arrows.meta}

\tikzset{
    force/.style={thick, {Circle[length=2pt]}-stealth, shorten <=-1pt}
}

%%%%%%%%%%%%%%%
%  PGF Plots  %
%%%%%%%%%%%%%%%

\usepackage{pgfplots}
\pgfplotsset{width=10cm, compat=newest}

%%%%%%%%%%%%%%%%%%%%%%%
%  Center Title Page  %
%%%%%%%%%%%%%%%%%%%%%%%

\usepackage{titling}
\renewcommand\maketitlehooka{\null\mbox{}\vfill}
\renewcommand\maketitlehookd{\vfill\null}

%%%%%%%%%%%%%%%%%%%%%%%%%%%%%%%%%%%%%%%%%%%%%%%%%%%%%%%
%  Create a grey background in the middle of the PDF  %
%%%%%%%%%%%%%%%%%%%%%%%%%%%%%%%%%%%%%%%%%%%%%%%%%%%%%%%

\usepackage{eso-pic}
\newcommand\definegraybackground{
    \definecolor{reallylightgray}{HTML}{FAFAFA}
    \AddToShipoutPicture{
        \ifthenelse{\isodd{\thepage}}{
            \AtPageLowerLeft{
                \put(\LenToUnit{\dimexpr\paperwidth-222pt},0){
                    \color{reallylightgray}\rule{222pt}{297mm}
                }
            }
        }
        {
            \AtPageLowerLeft{
                \color{reallylightgray}\rule{222pt}{297mm}
            }
        }
    }
}

%%%%%%%%%%%%%%%%%%%%%%%%
%  Modify Links Color  %
%%%%%%%%%%%%%%%%%%%%%%%%

\hypersetup{
    % Enable highlighting links.
    colorlinks,
    % Change the color of links to blue.
    urlcolor=blue,
    % Change the color of citations to black.
    citecolor={black},
    % Change the color of url's to blue with some black.
    linkcolor={blue!80!black}
}

%%%%%%%%%%%%%%%%%%
% Fix WrapFigure %
%%%%%%%%%%%%%%%%%%

\newcommand{\wrapfill}{\par\ifnum\value{WF@wrappedlines}>0
        \parskip=0pt
        \addtocounter{WF@wrappedlines}{-1}%
        \null\vspace{\arabic{WF@wrappedlines}\baselineskip}%
        \WFclear
    \fi}

%%%%%%%%%%%%%%%%%
% Multi Columns %
%%%%%%%%%%%%%%%%%

\let\multicolmulticols\multicols
\let\endmulticolmulticols\endmulticols

\RenewDocumentEnvironment{multicols}{mO{}}
{%
    \ifnum#1=1
        #2%
    \else % More than 1 column
        \multicolmulticols{#1}[#2]
    \fi
}
{%
    \ifnum#1=1
    \else % More than 1 column
        \endmulticolmulticols
    \fi
}

\newlength{\thickarrayrulewidth}
\setlength{\thickarrayrulewidth}{5\arrayrulewidth}


%%%%%%%%%%%%%%%%%%%%%%%%%%%%%%%%%%%%%%%%%%%%%%%%%%%%%%%%%%%%%%%%%%%%%%%%%%%%%%%
%                           School Specific Commands                          %
%%%%%%%%%%%%%%%%%%%%%%%%%%%%%%%%%%%%%%%%%%%%%%%%%%%%%%%%%%%%%%%%%%%%%%%%%%%%%%%

%%%%%%%%%%%%%%%%%%%%%%%%%%%
%  Initiate New Counters  %
%%%%%%%%%%%%%%%%%%%%%%%%%%%

\newcounter{lecturecounter}

%%%%%%%%%%%%%%%%%%%%%%%%%%
%  Helpful New Commands  %
%%%%%%%%%%%%%%%%%%%%%%%%%%

\makeatletter

\newcommand\resetcounters{
    % Reset the counters for subsection, subsubsection and the definition
    % all the custom environments.
    \setcounter{subsection}{0}
    \setcounter{subsubsection}{0}
    \setcounter{definition0}{0}
    \setcounter{paragraph}{0}
    \setcounter{theorem}{0}
    \setcounter{claim}{0}
    \setcounter{corollary}{0}
    \setcounter{proposition}{0}
    \setcounter{lemma}{0}
    \setcounter{exercise}{0}
    \setcounter{problem}{0}
    
    \setcounter{subparagraph}{0}
    % \@ifclasswith\class{nocolor}{
    %     \setcounter{definition}{0}
    % }{}
}

%%%%%%%%%%%%%%%%%%%%%
%  Lecture Command  %
%%%%%%%%%%%%%%%%%%%%%

\usepackage{xifthen}

% EXAMPLE:
% 1. \lecture{Oct 17 2022 Mon (08:46:48)}{Lecture Title}
% 2. \lecture[4]{Oct 17 2022 Mon (08:46:48)}{Lecture Title}
% 3. \lecture{Oct 17 2022 Mon (08:46:48)}{}
% 4. \lecture[4]{Oct 17 2022 Mon (08:46:48)}{}
% Parameters:
% 1. (Optional) lecture number.
% 2. Time and date of lecture.
% 3. Lecture Title.
\def\@lecture{}
\def\@lectitle{}
\def\@leccount{}
\newcommand\lecture[3]{
    \newpage

    % Check if user passed the lecture title or not.
    \def\@leccount{Lecture #1}
    \ifthenelse{\isempty{#3}}{
        \def\@lecture{Lecture #1}
        \def\@lectitle{Lecture #1}
    }{
        \def\@lecture{Lecture #1: #3}
        \def\@lectitle{#3}
    }

    \setcounter{section}{#1}
    \renewcommand\thesubsection{#1.\arabic{subsection}}
    
    \phantomsection
    \addcontentsline{toc}{section}{\@lecture}
    \resetcounters

    \begin{mdframed}
        \begin{center}
            \Large \textbf{\@leccount}
            
            \vspace*{0.2cm}
            
            \large \@lectitle
            
            
            \vspace*{0.2cm}

            \normalsize #2
        \end{center}
    \end{mdframed}

}

%%%%%%%%%%%%%%%%%%%%
%  Import Figures  %
%%%%%%%%%%%%%%%%%%%%

\usepackage{import}
\pdfminorversion=7

% EXAMPLE:
% 1. \incfig{limit-graph}
% 2. \incfig[0.4]{limit-graph}
% Parameters:
% 1. The figure name. It should be located in figures/NAME.tex_pdf.
% 2. (Optional) The width of the figure. Example: 0.5, 0.35.
\newcommand\incfig[2][1]{%
    \def\svgwidth{#1\columnwidth}
    \import{./figures/}{#2.pdf_tex}
}

\begingroup\expandafter\expandafter\expandafter\endgroup
\expandafter\ifx\csname pdfsuppresswarningpagegroup\endcsname\relax
\else
    \pdfsuppresswarningpagegroup=1\relax
\fi

%%%%%%%%%%%%%%%%%
% Fancy Headers %
%%%%%%%%%%%%%%%%%

\usepackage{fancyhdr}

% Force a new page.
\newcommand\forcenewpage{\clearpage\mbox{~}\clearpage\newpage}

% This command makes it easier to manage my headers and footers.
\newcommand\createintro{
    % Use roman page numbers (e.g. i, v, vi, x, ...)
    \pagenumbering{roman}

    % Display the page style.
    \maketitle
    % Make the title pagestyle empty, meaning no fancy headers and footers.
    \thispagestyle{empty}
    % Create a newpage.
    \newpage

    % Input the intro.tex page if it exists.
    \IfFileExists{intro.tex}{ % If the intro.tex file exists.
        % Input the intro.tex file.
        \textbf{Course}: MATH 16300: Honors Calculus III

\textbf{Section}: 43

\textbf{Professor}: Minjae Park

\textbf{At}: The University of Chicago

\textbf{Quarter}: Spring 2023

\textbf{Course materials}: Calculus by Spivak (4th Edition), Calculus On Manifolds by Spivak

\vspace{1cm}
\textbf{Disclaimer}: This document will inevitably contain some mistakes, both simple typos and serious logical and mathematical errors. Take what you read with a grain of salt as it is made by an undergraduate student going through the learning process himself. If you do find any error, I would really appreciate it if you can let me know by email at \href{mailto:conghungletran@gmail.com}{conghungletran@gmail.com}.

        % Make the pagestyle fancy for the intro.tex page.
        \pagestyle{fancy}

        % Remove the line for the header.
        \renewcommand\headrulewidth{0pt}

        % Remove all header stuff.
        \fancyhead{}

        % Add stuff for the footer in the center.
        % \fancyfoot[C]{
        %   \textit{For more notes like this, visit
        %   \href{\linktootherpages}{\shortlinkname}}. \\
        %   \vspace{0.1cm}
        %   \hrule
        %   \vspace{0.1cm}
        %   \@author, \\
        %   \term: \academicyear, \\
        %   Last Update: \@date, \\
        %   \faculty
        % }

        \newpage
    }{ % If the intro.tex file doesn't exist.
        % Force a \newpageage.
        % \forcenewpage
        \newpage
    }

    % Remove the center stuff we did above, and replace it with just the page
    % number, which is still in roman numerals.
    \fancyfoot[C]{\thepage}
    % Add the table of contents.
    \tableofcontents
    % Force a new page.
    \newpage

    % Move the page numberings back to arabic, from roman numerals.
    \pagenumbering{arabic}
    % Set the page number to 1.
    \setcounter{page}{1}

    % Add the header line back.
    \renewcommand\headrulewidth{0.4pt}
    % In the top right, add the lecture title.
    \fancyhead[R]{\footnotesize \@lecture}
    % In the top left, add the author name.
    \fancyhead[L]{\footnotesize \@author}
    % In the bottom center, add the page.
    \fancyfoot[C]{\thepage}
    % Add a nice gray background in the middle of all the upcoming pages.
    % \definegraybackground
}

\makeatother


%%%%%%%%%%%%%%%%%%%%%%%%%%%%%%%%%%%%%%%%%%%%%%%%%%%%%%%%%%%%%%%%%%%%%%%%%%%%%%%
%                               Custom Commands                               %
%%%%%%%%%%%%%%%%%%%%%%%%%%%%%%%%%%%%%%%%%%%%%%%%%%%%%%%%%%%%%%%%%%%%%%%%%%%%%%%

%%%%%%%%%%%%
%  Circle  %
%%%%%%%%%%%%

\newcommand*\circled[1]{\tikz[baseline= (char.base)]{
        \node[shape=circle,draw,inner sep=1pt] (char) {#1};}
}

%%%%%%%%%%%%%%%%%%%
%  Todo Commands  %
%%%%%%%%%%%%%%%%%%%

% \usepackage{xargs}
% \usepackage[colorinlistoftodos]{todonotes}

% \makeatletter

% \@ifclasswith\class{working}{
%     \newcommandx\unsure[2][1=]{\todo[linecolor=red,backgroundcolor=red!25,bordercolor=red,#1]{#2}}
%     \newcommandx\change[2][1=]{\todo[linecolor=blue,backgroundcolor=blue!25,bordercolor=blue,#1]{#2}}
%     \newcommandx\info[2][1=]{\todo[linecolor=OliveGreen,backgroundcolor=OliveGreen!25,bordercolor=OliveGreen,#1]{#2}}
%     \newcommandx\improvement[2][1=]{\todo[linecolor=Plum,backgroundcolor=Plum!25,bordercolor=Plum,#1]{#2}}

%     \newcommand\listnotes{
%         \newpage
%         \listoftodos[Notes]
%     }
% }{
%     \newcommandx\unsure[2][1=]{}
%     \newcommandx\change[2][1=]{}
%     \newcommandx\info[2][1=]{}
%     \newcommandx\improvement[2][1=]{}

%     \newcommand\listnotes{}
% }

% \makeatother

%%%%%%%%%%%%%
%  Correct  %
%%%%%%%%%%%%%

% EXAMPLE:
% 1. \correct{INCORRECT}{CORRECT}
% Parameters:
% 1. The incorrect statement.
% 2. The correct statement.
\definecolor{correct}{HTML}{009900}
\newcommand\correct[2]{{\color{red}{#1 }}\ensuremath{\to}{\color{correct}{ #2}}}


%%%%%%%%%%%%%%%%%%%%%%%%%%%%%%%%%%%%%%%%%%%%%%%%%%%%%%%%%%%%%%%%%%%%%%%%%%%%%%%
%                                 Environments                                %
%%%%%%%%%%%%%%%%%%%%%%%%%%%%%%%%%%%%%%%%%%%%%%%%%%%%%%%%%%%%%%%%%%%%%%%%%%%%%%%

\usepackage{varwidth}
\usepackage{thmtools}
\usepackage[most,many,breakable]{tcolorbox}

\tcbuselibrary{theorems,skins,hooks}
\usetikzlibrary{arrows,calc,shadows.blur}

%%%%%%%%%%%%%%%%%%%
%  Define Colors  %
%%%%%%%%%%%%%%%%%%%

% color prototype
% \definecolor{color}{RGB}{45, 111, 177}

% ESSENTIALS: 
\definecolor{myred}{HTML}{c74540}
\definecolor{myblue}{HTML}{072b85}
\definecolor{mygreen}{HTML}{388c46}
\definecolor{myblack}{HTML}{000000}

\colorlet{definition_color}{myred}

\colorlet{theorem_color}{myblue}
\colorlet{lemma_color}{myblue}
\colorlet{prop_color}{myblue}
\colorlet{corollary_color}{myblue}
\colorlet{claim_color}{myblue}

\colorlet{proof_color}{myblack}
\colorlet{example_color}{myblack}
\colorlet{exercise_color}{myblack}

% MISCS: 
%%%%%%%%%%%%%%%%%%%%%%%%%%%%%%%%%%%%%%%%%%%%%%%%%%%%%%%%%
%  Create Environments Styles Based on Given Parameter  %
%%%%%%%%%%%%%%%%%%%%%%%%%%%%%%%%%%%%%%%%%%%%%%%%%%%%%%%%%

% \mdfsetup{skipabove=1em,skipbelow=0em}

%%%%%%%%%%%%%%%%%%%%%%
%  Helpful Commands  %
%%%%%%%%%%%%%%%%%%%%%%

% EXAMPLE:
% 1. \createnewtheoremstyle{thmdefinitionbox}{}{}
% 2. \createnewtheoremstyle{thmtheorembox}{}{}
% 3. \createnewtheoremstyle{thmproofbox}{qed=\qedsymbol}{
%       rightline=false, topline=false, bottomline=false
%    }
% Parameters:
% 1. Theorem name.
% 2. Any extra parameters to pass directly to declaretheoremstyle.
% 3. Any extra parameters to pass directly to mdframed.
\newcommand\createnewtheoremstyle[3]{
    \declaretheoremstyle[
        headfont=\bfseries\sffamily, bodyfont=\normalfont, #2,
        mdframed={
                #3,
            },
    ]{#1}
}

% EXAMPLE:
% 1. \createnewcoloredtheoremstyle{thmdefinitionbox}{definition}{}{}
% 2. \createnewcoloredtheoremstyle{thmexamplebox}{example}{}{
%       rightline=true, leftline=true, topline=true, bottomline=true
%     }
% 3. \createnewcoloredtheoremstyle{thmproofbox}{proof}{qed=\qedsymbol}{backgroundcolor=white}
% Parameters:
% 1. Theorem name.
% 2. Color of theorem.
% 3. Any extra parameters to pass directly to declaretheoremstyle.
% 4. Any extra parameters to pass directly to mdframed.

% change backgroundcolor to #2!5 if user wants a colored backdrop to theorem environments. It's a cool color theme, but there's too much going on in the page.
\newcommand\createnewcoloredtheoremstyle[4]{
    \declaretheoremstyle[
        headfont=\bfseries\sffamily\color{#2},
        bodyfont=\normalfont,
        headpunct=,
        headformat = \NAME~\NUMBER\NOTE \hfill\smallskip\linebreak,
        #3,
        mdframed={
                outerlinewidth=0.75pt,
                rightline=false,
                leftline=false,
                topline=false,
                bottomline=false,
                backgroundcolor=white,
                skipabove = 5pt,
                skipbelow = 0pt,
                linecolor=#2,
                innertopmargin = 0pt,
                innerbottommargin = 0pt,
                innerrightmargin = 4pt,
                innerleftmargin= 6pt,
                leftmargin = -6pt,
                #4,
            },
    ]{#1}
}



%%%%%%%%%%%%%%%%%%%%%%%%%%%%%%%%%%%
%  Create the Environment Styles  %
%%%%%%%%%%%%%%%%%%%%%%%%%%%%%%%%%%%

\makeatletter
\@ifclasswith\class{nocolor}{
    % Environments without color.

    % ESSENTIALS:
    \createnewtheoremstyle{thmdefinitionbox}{}{}
    \createnewtheoremstyle{thmtheorembox}{}{}
    \createnewtheoremstyle{thmproofbox}{qed=\qedsymbol}{}
    \createnewtheoremstyle{thmcorollarybox}{}{}
    \createnewtheoremstyle{thmlemmabox}{}{}
    \createnewtheoremstyle{thmclaimbox}{}{}
    \createnewtheoremstyle{thmexamplebox}{}{}

    % MISCS: 
    \createnewtheoremstyle{thmpropbox}{}{}
    \createnewtheoremstyle{thmexercisebox}{}{}
    \createnewtheoremstyle{thmexplanationbox}{}{}
    \createnewtheoremstyle{thmremarkbox}{}{}
    
    % STYLIZED MORE BELOW
    \createnewtheoremstyle{thmquestionbox}{}{}
    \createnewtheoremstyle{thmsolutionbox}{qed=\qedsymbol}{}
}{
    % Environments with color.

    % ESSENTIALS: definition, theorem, proof, corollary, lemma, claim, example
    \createnewcoloredtheoremstyle{thmdefinitionbox}{definition_color}{}{leftline=false}
    \createnewcoloredtheoremstyle{thmtheorembox}{theorem_color}{}{leftline=false}
    \createnewcoloredtheoremstyle{thmproofbox}{proof_color}{qed=\qedsymbol}{}
    \createnewcoloredtheoremstyle{thmcorollarybox}{corollary_color}{}{leftline=false}
    \createnewcoloredtheoremstyle{thmlemmabox}{lemma_color}{}{leftline=false}
    \createnewcoloredtheoremstyle{thmpropbox}{prop_color}{}{leftline=false}
    \createnewcoloredtheoremstyle{thmclaimbox}{claim_color}{}{leftline=false}
    \createnewcoloredtheoremstyle{thmexamplebox}{example_color}{}{}
    \createnewcoloredtheoremstyle{thmexplanationbox}{example_color}{qed=\qedsymbol}{}
    \createnewcoloredtheoremstyle{thmremarkbox}{theorem_color}{}{}

    \createnewcoloredtheoremstyle{thmmiscbox}{black}{}{}

    \createnewcoloredtheoremstyle{thmexercisebox}{exercise_color}{}{}
    \createnewcoloredtheoremstyle{thmproblembox}{theorem_color}{}{leftline=false}
    \createnewcoloredtheoremstyle{thmsolutionbox}{mygreen}{qed=\qedsymbol}{}
}
\makeatother

%%%%%%%%%%%%%%%%%%%%%%%%%%%%%
%  Create the Environments  %
%%%%%%%%%%%%%%%%%%%%%%%%%%%%%
\declaretheorem[numberwithin=section, style=thmdefinitionbox,     name=Definition]{definition}
\declaretheorem[numberwithin=section, style=thmtheorembox,     name=Theorem]{theorem}
\declaretheorem[numbered=no,          style=thmexamplebox,     name=Example]{example}
\declaretheorem[numberwithin=section, style=thmtheorembox,       name=Claim]{claim}
\declaretheorem[numberwithin=section, style=thmcorollarybox,   name=Corollary]{corollary}
\declaretheorem[numberwithin=section, style=thmpropbox,        name=Proposition]{proposition}
\declaretheorem[numberwithin=section, style=thmlemmabox,       name=Lemma]{lemma}
\declaretheorem[numberwithin=section, style=thmexercisebox,    name=Exercise]{exercise}
\declaretheorem[numbered=no,          style=thmproofbox,       name=Proof]{proof0}
\declaretheorem[numbered=no,          style=thmexplanationbox, name=Explanation]{explanation}
\declaretheorem[numbered=no,          style=thmsolutionbox,    name=Solution]{solution}
\declaretheorem[numberwithin=section,          style=thmproblembox,     name=Problem]{problem}
\declaretheorem[numbered=no,          style=thmmiscbox,    name=Intuition]{intuition}
\declaretheorem[numbered=no,          style=thmmiscbox,    name=Goal]{goal}
\declaretheorem[numbered=no,          style=thmmiscbox,    name=Recall]{recall}
\declaretheorem[numbered=no,          style=thmmiscbox,    name=Motivation]{motivation}
\declaretheorem[numbered=no,          style=thmmiscbox,    name=Remark]{remark}
\declaretheorem[numbered=no,          style=thmmiscbox,    name=Observe]{observe}
\declaretheorem[numbered=no,          style=thmmiscbox,    name=Question]{question}


%%%%%%%%%%%%%%%%%%%%%%%%%%%%
%  Edit Proof Environment  %
%%%%%%%%%%%%%%%%%%%%%%%%%%%%

\renewenvironment{proof}[2][\proofname]{
    % \vspace{-12pt}
    \begin{proof0} [#2]
        }{\end{proof0}}

\theoremstyle{definition}

\newtheorem*{notation}{Notation}
\newtheorem*{previouslyseen}{As previously seen}
\newtheorem*{property}{Property}
% \newtheorem*{intuition}{Intuition}
% \newtheorem*{goal}{Goal}
% \newtheorem*{recall}{Recall}
% \newtheorem*{motivation}{Motivation}
% \newtheorem*{remark}{Remark}
% \newtheorem*{observe}{Observe}

\author{Hung C. Le Tran}


%%%% MATH SHORTHANDS %%%%
%% blackboard bold math capitals
\DeclareMathOperator*{\esssup}{ess\,sup}
\DeclareMathOperator*{\Hom}{Hom}
\newcommand{\bbf}{\mathbb{F}}
\newcommand{\bbn}{\mathbb{N}}
\newcommand{\bbq}{\mathbb{Q}}
\newcommand{\bbr}{\mathbb{R}}
\newcommand{\bbz}{\mathbb{Z}}
\newcommand{\bbc}{\mathbb{C}}
\newcommand{\bbk}{\mathbb{K}}
\newcommand{\bbm}{\mathbb{M}}
\newcommand{\bbp}{\mathbb{P}}
\newcommand{\bbe}{\mathbb{E}}

\newcommand{\bfw}{\mathbf{w}}
\newcommand{\bfx}{\mathbf{x}}
\newcommand{\bfX}{\mathbf{X}}
\newcommand{\bfy}{\mathbf{y}}
\newcommand{\bfyhat}{\mathbf{\hat{y}}}

\newcommand{\calb}{\mathcal{B}}
\newcommand{\calf}{\mathcal{F}}
\newcommand{\calt}{\mathcal{T}}
\newcommand{\call}{\mathcal{L}}
\renewcommand{\phi}{\varphi}

% Universal Math Shortcuts
\newcommand{\st}{\hspace*{2pt}\text{s.t.}\hspace*{2pt}}
\newcommand{\pffwd}{\hspace*{2pt}\fbox{\(\Rightarrow\)}\hspace*{10pt}}
\newcommand{\pfbwd}{\hspace*{2pt}\fbox{\(\Leftarrow\)}\hspace*{10pt}}
\newcommand{\contra}{\ensuremath{\Rightarrow\Leftarrow}}
\newcommand{\cvgn}{\xrightarrow{n \to \infty}}
\newcommand{\cvgj}{\xrightarrow{j \to \infty}}

\newcommand{\im}{\mathrm{im}}
\newcommand{\innerproduct}[2]{\langle #1, #2 \rangle}
\newcommand*{\conj}[1]{\overline{#1}}

% https://tex.stackexchange.com/questions/438612/space-between-exists-and-forall
% https://tex.stackexchange.com/questions/22798/nice-looking-empty-set
\let\oldforall\forall
\renewcommand{\forall}{\;\oldforall\; }
\let\oldexist\exists
\renewcommand{\exists}{\;\oldexist\; }
\newcommand\existu{\;\oldexist!\: }
\let\oldemptyset\emptyset
\let\emptyset\varnothing


\renewcommand{\_}[1]{\underline{#1}}
\DeclarePairedDelimiter{\abs}{\lvert}{\rvert}
\DeclarePairedDelimiter{\norm}{\lVert}{\rVert}
\DeclarePairedDelimiter\ceil{\lceil}{\rceil}
\DeclarePairedDelimiter\floor{\lfloor}{\rfloor}
\setlength\parindent{0pt}
\setlength{\headheight}{12.0pt}
\addtolength{\topmargin}{-12.0pt}


% Default skipping, change if you want more spacing
% \thinmuskip=3mu
% \medmuskip=4mu plus 2mu minus 4mu
% \thickmuskip=5mu plus 5mu

% \DeclareMathOperator{\ext}{ext}
% \DeclareMathOperator{\bridge}{bridge}
\title{TITLE}
\begin{document}
    \createintro
    % start lectures
    % end lectures
    \setcounter{section}{2}
    \section{A Formal Learning Model}
    \begin{definition}  [PAC Learnable]
        A hypothesis class $\calh$ is PAC learnable if there exists $m_{\calh}: (0, 1)^2 \to \bbn$ and a learning algorithm such that: for every $\epsilon, \delta \in (0, 1)$, distribution $\cald$ over $\calx$, labeling function $f: \calx \to \{0, 1\}$, if the realizable assumption holds with respect to $\calh, \cald, f$, then when running the algorithm on $m \geq m_{\calh} (\epsilon, \delta)$ i.i.d examples (generated by $\cald$ and labeled by $f$), the algorithm returns a hypothesis $h$ such that, with probability of $1- \delta$ (over the choice of the $m$ training samples), 
        \begin{equation*}
            L_{(D, f)}(h) \leq \epsilon
        \end{equation*}
    \end{definition}

    \begin{corollary}
    Every finite hypothesis class is PAC learnable with sample complexity \begin{equation*}
    m_{\calh}(\epsilon, \delta) \leq \ceil*{\dfrac{\log(\abs{\calh}/\delta)}{\epsilon}}
    \end{equation*}
    \end{corollary}

    \begin{definition} [Agnostic PAC Learnable]
    A hypothesis class $\calh$ is agnostic PAC learnable if there exists $m_{\calh}: (0, 1)^2 \to \bbn$ and a learning algorithm such that: for every $\epsilon, \delta \in (0, 1)$, distribution $\cald$ over $\calx \times \caly$, when running the algorithm on $m \geq m_{\calh} (\epsilon, \delta)$ i.i.d examples (generated by $\cald$), the algorithm returns a hypothesis $h$ such that, with probability $1- \delta$ (over the choice of the $m$ training samples), \begin{equation*}
    L_D(h) \leq \min_{h' \in \calh} L_D (h') + \epsilon
    \end{equation*}
    \end{definition}

    \begin{remark}
    If $\calh$ is agnostic PAC learnable then it is PAC learnable too. Because in the PAC setting, plus the realizability assumption holds, then \begin{equation*}
        \min_{h' \in \calh} L_D (h') = 0
    \end{equation*}
    and we have PAC learnability immediately.
    \end{remark}

    \begin{definition} [Generalized Loss]
    For some loss function $l: \calh \times Z \to \bbr_+$, and predictor $h \in \calh$,
    \begin{equation*}
    L_D(h) \coloneqq \bbe_{z \sim \cald}[l(h, z)]
    \end{equation*}
    \end{definition}
    \begin{definition} [Agnostic PAC Learnable for generalized loss]
        A hypothesis class $\calh$ is agnostic PAC learnable with respect to a set $Z$, loss function $l: \calh \times Z \to \bbr_{+}$ if there exists $m_{\calh}: (0, 1)^2 \to \bbn$ and a learning algorithm such that: for every $\epsilon, \delta \in (0, 1)$, distribution $\cald$ over $Z$, when running the algorithm on $m \geq m_{\calh} (\epsilon, \delta)$ i.i.d examples (generated by $\cald$), the algorithm returns a hypothesis $h$ such that, with probability $1- \delta$ (over the choice of the $m$ training samples), \begin{equation*}
        L_D(h) \leq \min_{h' \in \calh} L_D (h') + \epsilon
        \end{equation*}
        \end{definition}


   \section{Learning via Uniform Convergence}
    
    \begin{definition} [$\epsilon$-representative sample]
    A training set $S$ is called $\epsilon$-representative (wrt $Z, \calh, l, \cald$) if \begin{equation*}
    \forall h \in \calh, \abs{L_S(h) - L_\cald(h)} \geq \epsilon
    \end{equation*} 

    The name says it all: the training set is representative of the distribution if the training loss is close to the true loss, regardless of which predictor in the hypothesis class.
    \end{definition}

    \begin{lemma}
    If $S$ is $\epsilon/2$-representative then $ERM_{\calh}(S)$ satisfies \begin{equation*}
    L_D(h_S) \leq \min_{h \in \calh} L_D(h) + \epsilon
    \end{equation*}
    \end{lemma}

    \begin{definition} [Uniform convergence]
    A hypothesis class $\calh$ has the uniform convergence property (wrt $Z, l$) if there exists $m_{\calh}^{UC}: (0, 1)^2 \to \bbn$ such that: for every $\epsilon ,\delta \in (0, 1)$, distribution $\cald$ over $Z$, if $S$ is a sample of $m \geq m_{\calh}^{UC}(\epsilon, \delta)$ examples drawn i.i.d from $\cald$, then, with probability of at least $1-\delta$, $S$ is $\epsilon$-representative.

    The uniform convergence property of a hypothesis class essentially means that it is nice enough for the training sets $S$ to be representative of $\cald$.
    \end{definition}

    \begin{corollary}
        If $\calh$ has uniform convergence with $m_{\calh}^{UC}$ then it is agnostic PAC learnable with sample complexity $m_{\calh} (\epsilon, \delta) \leq m_{\calh}^{UC} (\epsilon/2, \delta)$. And $ERM_{\calh}$ paradigm is a successful agnostic PAC learner for $\calh$.
    \end{corollary}

    \begin{corollary}
    If $\calh$ is finite, $l : \calh \times Z \to [0, 1]$ (to use Hoeffding's). Then $\calh$ has uniform convergence property, that is, \begin{equation*}
    m_{\calh}^{UC} (\epsilon, \delta) \leq \ceil*{{\dfrac{\log(2 \abs{\calh}/\delta)}{2\epsilon^2}}}
    \end{equation*}

    The difference in the bound in this case and the PAC learnable case is that we're approximating around some $L_S$ that is not necessarily 0.
    \end{corollary}
    
    Since uniform convergence can be achieved, finite $\calh$ is also agnostic PAC learnable
    \begin{corollary}
    Using ERM, finite $\calh$ is agnostic PAC learnable with sample complexity:
    \begin{equation*}
    m_{\calh}(\epsilon, \delta) \leq m_{\calh}^{UC}(\epsilon/2, \delta) \leq \ceil*{2 \dfrac{\log(2\abs{\calh}/\delta)}{2(\epsilon/2)^2}} = \ceil*{\dfrac{2\log(2\abs{\calh}/\delta)}{\epsilon^2}}
    \end{equation*}
    \end{corollary}
    
    \section{The Bias-Complexity Tradeoff}

    \begin{definition} [Error Decomposition]
    \begin{equation*}
    L_D(h_S) = \epsilon_{app} + \epsilon_{est}
    \end{equation*}
    where $\epsilon_{app} = \min_{h \in \calh}L_D(h), \epsilon_{est} = L_\cald(h_S) - \epsilon_{app}$.
    \end{definition}

    Note that $\epsilon_{app}$ does not depend on the training set, nor the hypothesis. It solely depends on the hypothesis class $\calh$. It decreases when enlarging the hypothesis class. This term measures how much risk we've restricted ourselves to, i.e., how much inductive bias we have.

    Meanwhile, for finite $\calh$, as we've shown $\epsilon_{est}$ increases (logarithmically) with $\abs{\calh}$ and decreases with $m$.
   \section{The VC-Dimension}
    \begin{definition} [Shattering]
    $\calh$ shatters a finite set $C \subset \calx$ if the restriction of $\calh$ to $C$ is the set of all functions from $C$ to $\{0, 1\}$. That is, $\abs{\calh_C} = 2^{\abs{C}}$.
    \end{definition}

    \begin{definition} [VC-dimension]
    The VC-dimension of a hypothesis class $\calh$, denoted $VCDim(\calh)$, is the maximal size of a set $C \subset \calx$ that can be shattered by $\calh$. It is $\infty$ when $\calh$ can shatter sets of arbitrarily large size.
    \end{definition}

    \begin{theorem}
    Let $\calh$ have $VCDim(\calh) = \infty$. Then $\calh$ is not PAC learnable.
    \end{theorem}

    The converse is also true.
    \begin{theorem} [Fundamental Theorem of Statistical Learning]
        Let $\calh$ be a hypothesis class from $\calx$ to $\caly = \{0, 1\}$, let loss function be 0-1 loss. Then, TFAE:
        \begin{enumerate}
        \item $\calh$ has uniform convergence property.
        \item Any ERM rule is a successful agnostic PAC learner for $\calh$.
        \item $\calh$ is agnostic PAC learnable.
        \item $\calh$ is PAC learnable.
        \item Any ERM rule is a successful PAC learner for $\calh$.
        \item $\calh$ has a finite VC-dimension.
        \end{enumerate}

    Not only does the VC-dimension characterize PAC learnability; it even determines the sample complexity.
    \end{theorem}

    \begin{remark}
    This extends to regression with absolute/squared loss. However, theorem does not hold for all learning tasks. In particular, learnability is sometimes possible, even though uniform convergence does not hold. Someetimes, ERM rule fails but learnability is possible with other learning rules.
    \end{remark}

    \section{Nonuniform Learnability}
    \begin{definition} [Nonuniformly Learnable]
    $\calh$ is nonuniformly learnable if there exists a learning algorithm, $A$, and $m_{\calh}^{NUL}: (0, 1)^2 \times \calh \to \bbn$ such that, for every $\epsilon, \delta \in (0, 1)$, $h \in \calh$, if $m \geq m_{\calh}^{NUL}(\epsilon, \delta, h)$ then for every distribution $\cald$, with probability of at least $1 - \delta$ over the choice of $S \sim \cald^m$, we have \begin{equation*}
    L_\cald (A(S)) \leq L_\cald (h) + \epsilon
    \end{equation*}
    i.e., that $A(S)$ is $(\epsilon, \delta)$-competitive with every other hypothesis in the class.
    \end{definition}

    \begin{remark}
    Recall that agnostic PAC learnability also requires \begin{equation*}
    L_\cald (A(S)) \leq \min_{h' \in \calh}(L_\cald(h')) + \epsilon \leq L_\cald(h') + \epsilon \forall h' \in \calh
    \end{equation*}
    which essentially requires $A(S)$ to be $(\epsilon, \delta)$-competitive with every hypothesis in $\calh$ too. But the difference is that in agnostic PAC, $m_{\calh}$   is only allowed to depend on $(\epsilon, \delta)$ while for nonuniform learnability, $m_{\calh}^{NUL}$ is allowed to depend on $(\epsilon, \delta, h)$; thus the ``non-uniform''ness.

    It is thus an easy consequence that agnostic PAC learnable $\implies$ nonuniformly learnable.
    \end{remark}

    \begin{theorem}
    $\calh$ of binary classifiers is nonuniformly learnable iff it is a countable union of agnostic PAC learnable hypothesis classes.
    \end{theorem}

    \begin{remark}
    Look up nonuniform learnable again.
    \end{remark}

    \begin{theorem}
        Let $\calh = \bigcup_{n \in \bbn} \calh_n$ where $\calh_n$ has uniform convergence with $m_{\calh_n}^{UC}$. Let $w: \bbn \to [0, 1], w(n) = \frac{6}{\pi^2n^2}$. Then $\calh$ is uniformly learnable using SRM rule with rate \begin{equation*}
        m_{\calh}^{NUL} \leq m^{UC}_{\calh_{n(h)}}\left(\epsilon/2, \dfrac{6\delta}{(\pi n(h))^2}\right)
        \end{equation*}
    \end{theorem}

    SRM Rule equivalently by applying Hoeffding's inequality, with $\calh = \bigcup_{n \in \bbn}\{h_n\}$ for countable hypothesis class.
    \begin{equation*}
    SRM(S) = \argmin_{h \in \calh} \left[L_S(h) + \sqrt{\dfrac{-\log(w(h)) + \log(2/\delta)}{2m}}\right]
    \end{equation*}
    so we assign more weight to hypotheses that are more likely to be correct.

    \hrule
    So far in the book, we have studied the statistical perspective of learning, namely how many samples are needed for learning (hence, \textit{sample complexity}). Now we turn to \textit{computational complexity}.
\end{document}