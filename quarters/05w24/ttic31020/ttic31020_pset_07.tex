\documentclass[a4paper, 10pt]{article}
%%%%%%%%%%%%%%%%%%%%%%%%%%%%%%%%%%%%%%%%%%%%%%%%%%%%%%%%%%%%%%%%%%%%%%%%%%%%%%%
%                                Basic Packages                               %
%%%%%%%%%%%%%%%%%%%%%%%%%%%%%%%%%%%%%%%%%%%%%%%%%%%%%%%%%%%%%%%%%%%%%%%%%%%%%%%

% Gives us multiple colors.
\usepackage[usenames,dvipsnames,pdftex]{xcolor}
% Lets us style link colors.
\usepackage{hyperref}
% Lets us import images and graphics.
\usepackage{graphicx}
% Lets us use figures in floating environments.
\usepackage{float}
% Lets us create multiple columns.
\usepackage{multicol}
% Gives us better math syntax.
\usepackage{amsmath,amsfonts,mathtools,amsthm,amssymb}
% Lets us strikethrough text.
\usepackage{cancel}
% Lets us edit the caption of a figure.
\usepackage{caption}
% Lets us import pdf directly in our tex code.
\usepackage{pdfpages}
% Lets us do algorithm stuff.
\usepackage[ruled,vlined,linesnumbered]{algorithm2e}
% Use a smiley face for our qed symbol.
\usepackage{tikzsymbols}
% \usepackage{fullpage} %%smaller margins
\usepackage[shortlabels]{enumitem}

\usepackage{setspace}
\usepackage[margin=1in, headsep=12pt]{geometry}
\usepackage{wrapfig}
\def\class{article}


%%%%%%%%%%%%%%%%%%%%%%%%%%%%%%%%%%%%%%%%%%%%%%%%%%%%%%%%%%%%%%%%%%%%%%%%%%%%%%%
%                                Basic Settings                               %
%%%%%%%%%%%%%%%%%%%%%%%%%%%%%%%%%%%%%%%%%%%%%%%%%%%%%%%%%%%%%%%%%%%%%%%%%%%%%%%

%%%%%%%%%%%%%
%  Symbols  %
%%%%%%%%%%%%%

\let\implies\Rightarrow
\let\impliedby\Leftarrow
\let\iff\Leftrightarrow
\let\epsilon\varepsilon
%%%%%%%%%%%%
%  Tables  %
%%%%%%%%%%%%

\setlength{\tabcolsep}{5pt}
\renewcommand\arraystretch{1.5}

%%%%%%%%%%%%%%
%  SI Unitx  %
%%%%%%%%%%%%%%

\usepackage{siunitx}
\sisetup{locale = FR}

%%%%%%%%%%
%  TikZ  %
%%%%%%%%%%

\usepackage[framemethod=TikZ]{mdframed}
\usepackage{tikz}
\usepackage{tikz-cd}
\usepackage{tikzsymbols}

\usetikzlibrary{intersections, angles, quotes, calc, positioning}
\usetikzlibrary{arrows.meta}

\tikzset{
    force/.style={thick, {Circle[length=2pt]}-stealth, shorten <=-1pt}
}

%%%%%%%%%%%%%%%
%  PGF Plots  %
%%%%%%%%%%%%%%%

\usepackage{pgfplots}
\pgfplotsset{width=10cm, compat=newest}

%%%%%%%%%%%%%%%%%%%%%%%
%  Center Title Page  %
%%%%%%%%%%%%%%%%%%%%%%%

\usepackage{titling}
\renewcommand\maketitlehooka{\null\mbox{}\vfill}
\renewcommand\maketitlehookd{\vfill\null}

%%%%%%%%%%%%%%%%%%%%%%%%%%%%%%%%%%%%%%%%%%%%%%%%%%%%%%%
%  Create a grey background in the middle of the PDF  %
%%%%%%%%%%%%%%%%%%%%%%%%%%%%%%%%%%%%%%%%%%%%%%%%%%%%%%%

\usepackage{eso-pic}
\newcommand\definegraybackground{
    \definecolor{reallylightgray}{HTML}{FAFAFA}
    \AddToShipoutPicture{
        \ifthenelse{\isodd{\thepage}}{
            \AtPageLowerLeft{
                \put(\LenToUnit{\dimexpr\paperwidth-222pt},0){
                    \color{reallylightgray}\rule{222pt}{297mm}
                }
            }
        }
        {
            \AtPageLowerLeft{
                \color{reallylightgray}\rule{222pt}{297mm}
            }
        }
    }
}

%%%%%%%%%%%%%%%%%%%%%%%%
%  Modify Links Color  %
%%%%%%%%%%%%%%%%%%%%%%%%

\hypersetup{
    % Enable highlighting links.
    colorlinks,
    % Change the color of links to blue.
    urlcolor=blue,
    % Change the color of citations to black.
    citecolor={black},
    % Change the color of url's to blue with some black.
    linkcolor={blue!80!black}
}

%%%%%%%%%%%%%%%%%%
% Fix WrapFigure %
%%%%%%%%%%%%%%%%%%

\newcommand{\wrapfill}{\par\ifnum\value{WF@wrappedlines}>0
        \parskip=0pt
        \addtocounter{WF@wrappedlines}{-1}%
        \null\vspace{\arabic{WF@wrappedlines}\baselineskip}%
        \WFclear
    \fi}

%%%%%%%%%%%%%%%%%
% Multi Columns %
%%%%%%%%%%%%%%%%%

\let\multicolmulticols\multicols
\let\endmulticolmulticols\endmulticols

\RenewDocumentEnvironment{multicols}{mO{}}
{%
    \ifnum#1=1
        #2%
    \else % More than 1 column
        \multicolmulticols{#1}[#2]
    \fi
}
{%
    \ifnum#1=1
    \else % More than 1 column
        \endmulticolmulticols
    \fi
}

\newlength{\thickarrayrulewidth}
\setlength{\thickarrayrulewidth}{5\arrayrulewidth}


%%%%%%%%%%%%%%%%%%%%%%%%%%%%%%%%%%%%%%%%%%%%%%%%%%%%%%%%%%%%%%%%%%%%%%%%%%%%%%%
%                           School Specific Commands                          %
%%%%%%%%%%%%%%%%%%%%%%%%%%%%%%%%%%%%%%%%%%%%%%%%%%%%%%%%%%%%%%%%%%%%%%%%%%%%%%%

%%%%%%%%%%%%%%%%%%%%%%%%%%%
%  Initiate New Counters  %
%%%%%%%%%%%%%%%%%%%%%%%%%%%

\newcounter{lecturecounter}

%%%%%%%%%%%%%%%%%%%%%%%%%%
%  Helpful New Commands  %
%%%%%%%%%%%%%%%%%%%%%%%%%%

\makeatletter

\newcommand\resetcounters{
    % Reset the counters for subsection, subsubsection and the definition
    % all the custom environments.
    \setcounter{subsection}{0}
    \setcounter{subsubsection}{0}
    \setcounter{paragraph}{0}
    \setcounter{subparagraph}{0}
    \setcounter{theorem}{0}
    \setcounter{claim}{0}
    \setcounter{corollary}{0}
    \setcounter{lemma}{0}
    \setcounter{exercise}{0}

    \@ifclasswith\class{nocolor}{
        \setcounter{definition}{0}
    }{}
}

%%%%%%%%%%%%%%%%%%%%%
%  Lecture Command  %
%%%%%%%%%%%%%%%%%%%%%

\usepackage{xifthen}

% EXAMPLE:
% 1. \lecture{Oct 17 2022 Mon (08:46:48)}{Lecture Title}
% 2. \lecture[4]{Oct 17 2022 Mon (08:46:48)}{Lecture Title}
% 3. \lecture{Oct 17 2022 Mon (08:46:48)}{}
% 4. \lecture[4]{Oct 17 2022 Mon (08:46:48)}{}
% Parameters:
% 1. (Optional) lecture number.
% 2. Time and date of lecture.
% 3. Lecture Title.
\def\@lecture{}
\def\@lectitle{}
\def\@leccount{}
\newcommand\lecture[3]{
    %   % Add 1 to the lecture counter.
    %   \addtocounter{lecturecounter}{1}

    % Set the section number to the lecture counter.
    \setcounter{section}{#1}
    \renewcommand\thesubsection{#1.\arabic{subsection}}

    % Reset the counters.
    \resetcounters

    % Check if user passed the lecture title or not.
    \def\@leccount{Lecture #1}
    \ifthenelse{\isempty{#3}}{
        \def\@lecture{Lecture #1}
    }{
        \def\@lecture{Lecture #1: #3}
        \def\@lectitle{#3}
    }

    % Display the information like the following:
    %                                                  Oct 17 2022 Mon (08:49:10)
    % ---------------------------------------------------------------------------
    % Lecture 1: Lecture Title
    \newpage
    \begin{mdframed}
        % \section*{\@lecture}
        \begin{center}
            \Large \textbf{\@leccount}

            \vspace*{0.2cm}

            \large\@lectitle

            \vspace*{0.2cm}
            \normalsize #2
        \end{center}
    \end{mdframed}
    \addcontentsline{toc}{section}{\@lecture}
}

%%%%%%%%%%%%%%%%%%%%
%  Import Figures  %
%%%%%%%%%%%%%%%%%%%%

\usepackage{import}
\pdfminorversion=7

% EXAMPLE:
% 1. \incfig{limit-graph}
% 2. \incfig[0.4]{limit-graph}
% Parameters:
% 1. The figure name. It should be located in figures/NAME.tex_pdf.
% 2. (Optional) The width of the figure. Example: 0.5, 0.35.
\newcommand\incfig[2][1]{%
    \def\svgwidth{#1\columnwidth}
    \import{./figures/}{#2.pdf_tex}
}

\begingroup\expandafter\expandafter\expandafter\endgroup
\expandafter\ifx\csname pdfsuppresswarningpagegroup\endcsname\relax
\else
    \pdfsuppresswarningpagegroup=1\relax
\fi

%%%%%%%%%%%%%%%%%
% Fancy Headers %
%%%%%%%%%%%%%%%%%

\usepackage{fancyhdr}

% Force a new page.
\newcommand\forcenewpage{\clearpage\mbox{~}\clearpage\newpage}

% This command makes it easier to manage my headers and footers.
\newcommand\createintro{
    % Use roman page numbers (e.g. i, v, vi, x, ...)
    \pagenumbering{roman}

    % Display the page style.
    \maketitle
    % Make the title pagestyle empty, meaning no fancy headers and footers.
    \thispagestyle{empty}
    % Create a newpage.
    \newpage

    % Input the intro.tex page if it exists.
    \IfFileExists{intro.tex}{ % If the intro.tex file exists.
        % Input the intro.tex file.
        \textbf{Course}: COURSE

\textbf{Section}: SECTION

\textbf{Professor}: PROFESSOR

\textbf{At}: The University of Chicago

\textbf{Quarter}: QUARTER

\textbf{Course materials}: COURSE_MATERIALS

\vspace{1cm}
\textbf{Disclaimer}: This document will inevitably contain some mistakes, both simple typos and serious logical and mathematical errors. Take what you read with a grain of salt as it is made by an undergraduate student going through the learning process himself. If you do find any error, I would really appreciate it if you can let me know by email at \href{mailto:conghungletran@gmail.com}{conghungletran@gmail.com}.

        % Make the pagestyle fancy for the intro.tex page.
        \pagestyle{fancy}

        % Remove the line for the header.
        \renewcommand\headrulewidth{0pt}

        % Remove all header stuff.
        \fancyhead{}

        % Add stuff for the footer in the center.
        % \fancyfoot[C]{
        %   \textit{For more notes like this, visit
        %   \href{\linktootherpages}{\shortlinkname}}. \\
        %   \vspace{0.1cm}
        %   \hrule
        %   \vspace{0.1cm}
        %   \@author, \\
        %   \term: \academicyear, \\
        %   Last Update: \@date, \\
        %   \faculty
        % }
    }{ % If the intro.tex file doesn't exist.
        % Force a \newpageage.
        \forcenewpage
    }

    % Create a new page.
    \newpage

    % Remove the center stuff we did above, and replace it with just the page
    % number, which is still in roman numerals.
    \fancyfoot[C]{\thepage}
    % Add the table of contents.
    \tableofcontents
    % Force a new page.
    \forcenewpage

    % Move the page numberings back to arabic, from roman numerals.
    \pagenumbering{arabic}
    % Set the page number to 1.
    \setcounter{page}{1}

    % Add the header line back.
    \renewcommand\headrulewidth{0.4pt}
    % In the top right, add the lecture title.
    \fancyhead[R]{\footnotesize \@lecture}
    % In the top left, add the author name.
    \fancyhead[L]{\footnotesize \@author}
    % In the bottom center, add the page.
    \fancyfoot[C]{\thepage}
    % Add a nice gray background in the middle of all the upcoming pages.
    % \definegraybackground
}

\makeatother


%%%%%%%%%%%%%%%%%%%%%%%%%%%%%%%%%%%%%%%%%%%%%%%%%%%%%%%%%%%%%%%%%%%%%%%%%%%%%%%
%                               Custom Commands                               %
%%%%%%%%%%%%%%%%%%%%%%%%%%%%%%%%%%%%%%%%%%%%%%%%%%%%%%%%%%%%%%%%%%%%%%%%%%%%%%%

%%%%%%%%%%%%
%  Circle  %
%%%%%%%%%%%%

\newcommand*\circled[1]{\tikz[baseline=(char.base)]{
        \node[shape=circle,draw,inner sep=1pt] (char) {#1};}
}

%%%%%%%%%%%%%%%%%%%
%  Todo Commands  %
%%%%%%%%%%%%%%%%%%%

% \usepackage{xargs}
% \usepackage[colorinlistoftodos]{todonotes}

% \makeatletter

% \@ifclasswith\class{working}{
%     \newcommandx\unsure[2][1=]{\todo[linecolor=red,backgroundcolor=red!25,bordercolor=red,#1]{#2}}
%     \newcommandx\change[2][1=]{\todo[linecolor=blue,backgroundcolor=blue!25,bordercolor=blue,#1]{#2}}
%     \newcommandx\info[2][1=]{\todo[linecolor=OliveGreen,backgroundcolor=OliveGreen!25,bordercolor=OliveGreen,#1]{#2}}
%     \newcommandx\improvement[2][1=]{\todo[linecolor=Plum,backgroundcolor=Plum!25,bordercolor=Plum,#1]{#2}}

%     \newcommand\listnotes{
%         \newpage
%         \listoftodos[Notes]
%     }
% }{
%     \newcommandx\unsure[2][1=]{}
%     \newcommandx\change[2][1=]{}
%     \newcommandx\info[2][1=]{}
%     \newcommandx\improvement[2][1=]{}

%     \newcommand\listnotes{}
% }

% \makeatother

%%%%%%%%%%%%%
%  Correct  %
%%%%%%%%%%%%%

% EXAMPLE:
% 1. \correct{INCORRECT}{CORRECT}
% Parameters:
% 1. The incorrect statement.
% 2. The correct statement.
\definecolor{correct}{HTML}{009900}
\newcommand\correct[2]{{\color{red}{#1 }}\ensuremath{\to}{\color{correct}{ #2}}}


%%%%%%%%%%%%%%%%%%%%%%%%%%%%%%%%%%%%%%%%%%%%%%%%%%%%%%%%%%%%%%%%%%%%%%%%%%%%%%%
%                                 Environments                                %
%%%%%%%%%%%%%%%%%%%%%%%%%%%%%%%%%%%%%%%%%%%%%%%%%%%%%%%%%%%%%%%%%%%%%%%%%%%%%%%

\usepackage{varwidth}
\usepackage{thmtools}
\usepackage[most,many,breakable]{tcolorbox}

\tcbuselibrary{theorems,skins,hooks}
\usetikzlibrary{arrows,calc,shadows.blur}

%%%%%%%%%%%%%%%%%%%
%  Define Colors  %
%%%%%%%%%%%%%%%%%%%

\definecolor{myblue}{RGB}{45, 111, 177}
\definecolor{mygreen}{RGB}{56, 140, 70}
\definecolor{myred}{RGB}{199, 68, 64}
\definecolor{mypurple}{RGB}{197, 92, 212}

% ESSENTIALS: 
\definecolor{definition_color}{HTML}{c74540}

\definecolor{theorem_color}{HTML}{00007B}
\colorlet{proof_color}{theorem_color}
\colorlet{prop_color}{theorem_color}

\colorlet{corollary_color}{mypurple!85!black}

\definecolor{lemma_color}{HTML}{983b0f}

\definecolor{example_color}{HTML}{2A7F7F}
\colorlet{exercise_color}{example_color}

\colorlet{claim_color}{mygreen!85!black}

% MISCS: 
%%%%%%%%%%%%%%%%%%%%%%%%%%%%%%%%%%%%%%%%%%%%%%%%%%%%%%%%%
%  Create Environments Styles Based on Given Parameter  %
%%%%%%%%%%%%%%%%%%%%%%%%%%%%%%%%%%%%%%%%%%%%%%%%%%%%%%%%%

\mdfsetup{skipabove=1em,skipbelow=0em}

%%%%%%%%%%%%%%%%%%%%%%
%  Helpful Commands  %
%%%%%%%%%%%%%%%%%%%%%%

% EXAMPLE:
% 1. \createnewtheoremstyle{thmdefinitionbox}{}{}
% 2. \createnewtheoremstyle{thmtheorembox}{}{}
% 3. \createnewtheoremstyle{thmproofbox}{qed=\qedsymbol}{
%       rightline=false, topline=false, bottomline=false
%    }
% Parameters:
% 1. Theorem name.
% 2. Any extra parameters to pass directly to declaretheoremstyle.
% 3. Any extra parameters to pass directly to mdframed.
\newcommand\createnewtheoremstyle[3]{
    \declaretheoremstyle[
        headfont=\bfseries\sffamily, bodyfont=\normalfont, #2,
        mdframed={
                #3,
            },
    ]{#1}
}

% EXAMPLE:
% 1. \createnewcoloredtheoremstyle{thmdefinitionbox}{definition}{}{}
% 2. \createnewcoloredtheoremstyle{thmexamplebox}{example}{}{
%       rightline=true, leftline=true, topline=true, bottomline=true
%     }
% 3. \createnewcoloredtheoremstyle{thmproofbox}{proof}{qed=\qedsymbol}{backgroundcolor=white}
% Parameters:
% 1. Theorem name.
% 2. Color of theorem.
% 3. Any extra parameters to pass directly to declaretheoremstyle.
% 4. Any extra parameters to pass directly to mdframed.

% change backgroundcolor to #2!5 if user wants a colored backdrop to theorem environments. It's a cool color theme, but there's too much going on in the page.
\newcommand\createnewcoloredtheoremstyle[4]{
    \declaretheoremstyle[
        headfont=\bfseries\sffamily\color{#2}, bodyfont=\normalfont, #3,
        mdframed={
                linewidth=2pt,
                rightline=false, leftline=true, topline=false, bottomline=false,
                linecolor=#2, backgroundcolor=white, #4,
            },
    ]{#1}
}



%%%%%%%%%%%%%%%%%%%%%%%%%%%%%%%%%%%
%  Create the Environment Styles  %
%%%%%%%%%%%%%%%%%%%%%%%%%%%%%%%%%%%

\makeatletter
\@ifclasswith\class{nocolor}{
    % Environments without color.

    % ESSENTIALS:
    \createnewtheoremstyle{thmdefinitionbox}{}{}
    \createnewtheoremstyle{thmtheorembox}{}{}
    \createnewtheoremstyle{thmproofbox}{qed=\qedsymbol}{}
    \createnewtheoremstyle{thmcorollarybox}{}{}
    \createnewtheoremstyle{thmlemmabox}{}{}
    \createnewtheoremstyle{thmclaimbox}{}{}
    \createnewtheoremstyle{thmexamplebox}{}{}

    % MISCS: 
    \createnewtheoremstyle{thmpropbox}{}{}
    \createnewtheoremstyle{thmexercisebox}{}{}
    \createnewtheoremstyle{thmexplanationbox}{}{}
    \createnewtheoremstyle{thmremarkbox}{}{}

    % STYLIZED MORE BELOW
    \createnewtheoremstyle{thmquestionbox}{}{}
    \createnewtheoremstyle{thmsolutionbox}{qed=\qedsymbol}{}
}{
    % Environments with color.

    % ESSENTIALS: definition, theorem, proof, corollary, lemma, claim, example
    \createnewcoloredtheoremstyle{thmdefinitionbox}{definition_color}{}{
        leftline=true
    }
    \createnewcoloredtheoremstyle{thmtheorembox}{theorem_color}{}{
        leftline=true
    }
    \createnewcoloredtheoremstyle{thmproofbox}{proof_color}{qed=\qedsymbol}{backgroundcolor=white}
    \createnewcoloredtheoremstyle{thmcorollarybox}{corollary_color}{}{backgroundcolor=white}
    \createnewcoloredtheoremstyle{thmlemmabox}{lemma_color}{}{backgroundcolor=white}
    \createnewcoloredtheoremstyle{thmclaimbox}{claim_color}{}{}
    \createnewcoloredtheoremstyle{thmexamplebox}{example_color}{}{backgroundcolor=white}
    \createnewcoloredtheoremstyle{thmexplanationbox}{example_color}{qed=\qedsymbol}{backgroundcolor=white}
    \createnewcoloredtheoremstyle{thmremarkbox}{theorem_color}{}{backgroundcolor=white}

    \createnewcoloredtheoremstyle{thmmiscbox}{black}{}{leftline=false,backgroundcolor=white}


    \createnewcoloredtheoremstyle{thmpropbox}{prop_color}{}{backgroundcolor=white}
    \createnewcoloredtheoremstyle{thmexercisebox}{exercise_color}{}{backgroundcolor=white}

    \createnewcoloredtheoremstyle{thmproblembox}{myred}{}{backgroundcolor=white}
    \createnewcoloredtheoremstyle{thmsolutionbox}{mygreen}{qed=\qedsymbol}{backgroundcolor=white}
}
\makeatother

%%%%%%%%%%%%%%%%%%%%%%%%%%%%%
%  Create the Environments  %
%%%%%%%%%%%%%%%%%%%%%%%%%%%%%
\declaretheorem[numberwithin=section, style=thmdefinitionbox,     name=Definition]{definition0}
\declaretheorem[numberwithin=section, style=thmtheorembox,     name=Theorem]{theorem}
\declaretheorem[numbered=no,          style=thmexamplebox,     name=Example]{example}
\declaretheorem[numberwithin=section, style=thmclaimbox,       name=Claim]{claim}
\declaretheorem[numberwithin=section, style=thmcorollarybox,   name=Corollary]{corollary}
\declaretheorem[numberwithin=section, style=thmpropbox,        name=Proposition]{proposition}
\declaretheorem[numberwithin=section, style=thmlemmabox,       name=Lemma]{lemma}
\declaretheorem[numberwithin=section, style=thmexercisebox,    name=Exercise]{exercise}
\declaretheorem[numbered=no,          style=thmproofbox,       name=Proof]{replacementproof}
\declaretheorem[numbered=no,          style=thmexplanationbox, name=Explanation]{explanation}
\declaretheorem[numbered=no,          style=thmsolutionbox,    name=Solution]{solution}
\declaretheorem[numberwithin=section,          style=thmproblembox,     name=Problem]{problem}
\declaretheorem[numbered=no,          style=thmmiscbox,    name=Intuition]{intuition}
\declaretheorem[numbered=no,          style=thmmiscbox,    name=Goal]{goal}
\declaretheorem[numbered=no,          style=thmmiscbox,    name=Recall]{recall}
\declaretheorem[numbered=no,          style=thmmiscbox,    name=Motivation]{motivation}
\declaretheorem[numbered=no,          style=thmmiscbox,    name=Remark]{remark}
\declaretheorem[numbered=no,          style=thmmiscbox,    name=Observe]{observe}




%%%% FANCY GRAPHICS:

% \makeatletter
% \@ifclasswith\class{nocolor}{
%   % Environments without color.

%   \newtheorem*{note}{Note}

%   \declaretheorem[numberwithin=section, style=thmdefinitionbox, name=Definition]{definition}
%   \declaretheorem[numberwithin=section, style=thmquestionbox,   name=Question]{question}
%   \declaretheorem[numberwithin=section, style=thmsolutionbox,   name=Solution]{solution}
% }{
%   % Environments with color.

%   \newtcbtheorem[number within=section]{Definition}{Definition}{
%     enhanced,
%     before skip=2mm,
%     after skip=2mm,
%     colback=red!5,
%     colframe=red!80!black,
%     colbacktitle=red!75!black,
%     boxrule=0.5mm,
%     attach boxed title to top left={
%       xshift=1cm,
%       yshift*=1mm-\tcboxedtitleheight
%     },
%     varwidth boxed title*=-3cm,
%     boxed title style={
%       interior engine=empty,
%       frame code={
%         \path[fill=tcbcolback]
%         ([yshift=-1mm,xshift=-1mm]frame.north west)
%         arc[start angle=0,end angle=180,radius=1mm]
%         ([yshift=-1mm,xshift=1mm]frame.north east)
%         arc[start angle=180,end angle=0,radius=1mm];
%         \path[left color=tcbcolback!60!black,right color=tcbcolback!60!black,
%         middle color=tcbcolback!80!black]
%         ([xshift=-2mm]frame.north west) -- ([xshift=2mm]frame.north east)
%         [rounded corners=1mm]-- ([xshift=1mm,yshift=-1mm]frame.north east)
%         -- (frame.south east) -- (frame.south west)
%         -- ([xshift=-1mm,yshift=-1mm]frame.north west)
%         [sharp corners]-- cycle;
%       },
%     },
%     fonttitle=\bfseries,
%     title={#2},
%     #1
%   }{def}

%   \NewDocumentEnvironment{definition}{O{}O{}}
%     {\begin{Definition}{#1}{#2}}{\end{Definition}}

%   \newtcolorbox{note}[1][]{%
%     enhanced jigsaw,
%     colback=gray!20!white,%
%     colframe=gray!80!black,
%     size=small,
%     boxrule=1pt,
%     title=\textbf{Note:-},
%     halign title=flush center,
%     coltitle=black,
%     breakable,
%     drop shadow=black!50!white,
%     attach boxed title to top left={xshift=1cm,yshift=-\tcboxedtitleheight/2,yshifttext=-\tcboxedtitleheight/2},
%     minipage boxed title=1.5cm,
%     boxed title style={%
%       colback=white,
%       size=fbox,
%       boxrule=1pt,
%       boxsep=2pt,
%       underlay={%
%         \coordinate (dotA) at ($(interior.west) + (-0.5pt,0)$);
%         \coordinate (dotB) at ($(interior.east) + (0.5pt,0)$);
%         \begin{scope}
%           \clip (interior.north west) rectangle ([xshift=3ex]interior.east);
%           \filldraw [white, blur shadow={shadow opacity=60, shadow yshift=-.75ex}, rounded corners=2pt] (interior.north west) rectangle (interior.south east);
%         \end{scope}
%         \begin{scope}[gray!80!black]
%           \fill (dotA) circle (2pt);
%           \fill (dotB) circle (2pt);
%         \end{scope}
%       },
%     },
%     #1,
%   }

%   \newtcbtheorem{Question}{Question}{enhanced,
%     breakable,
%     colback=white,
%     colframe=myblue!80!black,
%     attach boxed title to top left={yshift*=-\tcboxedtitleheight},
%     fonttitle=\bfseries,
%     title=\textbf{Question:-},
%     boxed title size=title,
%     boxed title style={%
%       sharp corners,
%       rounded corners=northwest,
%       colback=tcbcolframe,
%       boxrule=0pt,
%     },
%     underlay boxed title={%
%       \path[fill=tcbcolframe] (title.south west)--(title.south east)
%       to[out=0, in=180] ([xshift=5mm]title.east)--
%       (title.center-|frame.east)
%       [rounded corners=\kvtcb@arc] |-
%       (frame.north) -| cycle;
%     },
%     #1
%   }{def}

%   \NewDocumentEnvironment{question}{O{}O{}}
%   {\begin{Question}{#1}{#2}}{\end{Question}}

%   \newtcolorbox{Solution}{enhanced,
%     breakable,
%     colback=white,
%     colframe=mygreen!80!black,
%     attach boxed title to top left={yshift*=-\tcboxedtitleheight},
%     title=\textbf{Solution:-},
%     boxed title size=title,
%     boxed title style={%
%       sharp corners,
%       rounded corners=northwest,
%       colback=tcbcolframe,
%       boxrule=0pt,
%     },
%     underlay boxed title={%
%       \path[fill=tcbcolframe] (title.south west)--(title.south east)
%       to[out=0, in=180] ([xshift=5mm]title.east)--
%       (title.center-|frame.east)
%       [rounded corners=\kvtcb@arc] |-
%       (frame.north) -| cycle;
%     },
%   }

%   \NewDocumentEnvironment{solution}{O{}O{}}
%   {\vspace{-10pt}\begin{Solution}{#1}{#2}}{\end{Solution}}
% }
% \makeatother


%%%%% END OF FANCY GRAPHICS %%%%%%%%%%



%%%%%%%%%%%%%%%%%%%%%%%%%%%%
%  Edit Proof Environment  %
%%%%%%%%%%%%%%%%%%%%%%%%%%%%

\renewenvironment{proof}[2][\proofname]{
    % \vspace{-12pt}
    \begin{replacementproof} [#2]
}{\end{replacementproof}}

\newenvironment{definition}[1]{
    \begin{definition0}[#1]

    \hfill
        
    \vspace{0.2cm}

}{

    \vspace{0.2cm}
    \end{definition0}
}


\theoremstyle{definition}

\newtheorem*{notation}{Notation}
\newtheorem*{previouslyseen}{As previously seen}
\newtheorem*{property}{Property}
% \newtheorem*{intuition}{Intuition}
% \newtheorem*{goal}{Goal}
% \newtheorem*{recall}{Recall}
% \newtheorem*{motivation}{Motivation}
% \newtheorem*{remark}{Remark}
% \newtheorem*{observe}{Observe}

\author{Cong Hung Le Tran}


%%%% MATH SHORTHANDS %%%%
%% blackboard bold math capitals
\newcommand{\bbf}{\mathbb{F}}
\newcommand{\bbn}{\mathbb{N}}
\newcommand{\bbq}{\mathbb{Q}}
\newcommand{\bbr}{\mathbb{R}}
\newcommand{\bbz}{\mathbb{Z}}
\newcommand{\bbc}{\mathbb{C}}
\newcommand{\bbk}{\mathbb{K}}
\newcommand{\bbm}{\mathbb{M}}
\renewcommand{\phi}{\varphi}
\newcommand{\st}{\;\text{such that}\;}


% MATH 20250 %
\newcommand{\Hom}{\mathrm{Hom}}
\newcommand{\im}{\mathrm{im}}

% https://tex.stackexchange.com/questions/438612/space-between-exists-and-forall
\let\oldforall\forall
\renewcommand{\forall}{\;\oldforall\; }
\let\oldexist\exists
\renewcommand{\exists}{\;\oldexist\; }
\newcommand\existu{\;\oldexist!\: }


\renewcommand{\_}[1]{\underline{ #1 }}
\DeclarePairedDelimiter{\abs}{\lvert}{\rvert}
\DeclarePairedDelimiter{\norm}{\lVert}{\rVert}
\setlength\parindent{0pt}
\setlength{\headheight}{12.0pt}
\addtolength{\topmargin}{-12.0pt}


% Default skipping, change if you want more spacing
% \thinmuskip=3mu
% \medmuskip=4mu plus 2mu minus 4mu
% \thickmuskip=5mu plus 5mu



% \DeclareMathOperator{\ext}{ext}
% \DeclareMathOperator{\bridge}{bridge}
\title{TTIC 31020: Introduction to Machine Learning \\ \large Problem Set 7}
\date{20 Feb 2024}
\author{Hung Le Tran}
\begin{document}
\maketitle
\setcounter{section}{7}
\begin{problem} [Problem 1]
    \textbf{(a)} \begin{equation*}
    y = \frac{3}{\sqrt{10}} x_{100} + \frac{1}{\sqrt{10}} x_1
    \end{equation*}
    where $x \sim N(0, \Sigma)$  with \begin{equation*}
    \Sigma = \begin{bmatrix}
    1 & 0 & 0 & \cdots & 0 \\
    0 & 1 & 0.9 & \cdots & 0 \\
    0 & 0.9 & 1 & \cdots & 0 \\
    &&\vdots &&\\
    0 & 0.9 & 0.9 & \cdots & 1
    \end{bmatrix} \in \bbr^{100 \times 100}
    \end{equation*}

    \textbf{1.} Optimal Feature Selection

    Recall that \begin{equation*}
    w_k = \argmin_{\norm{w}_0 \leq k} L_S(w)
    \end{equation*}

    Then we'll have $I_1 = \{100\}$, $I_2 = \{1, 100\}$. There's no particular ``order'' as to which values are inserted into $I_k$, since for each $k$ we search from scratch. $w_2$ already achieves \begin{equation*}
    L_S(w_2) = L_\cald(w_2) = 0 \leq 0.01   
    \end{equation*}

    \textbf{2.} Greedy Feature Selection
    \begin{equation*}
        I_{k + 1} = \argmin_{I} \min_{\supp(w) \subseteq I} L_S(w) \:\text{such that}\:  I_k \subset I, \abs{I} = k+1
    \end{equation*}

    Then $I_0 = \emptyset, I_1  = \{100\}, I_2 = \{100, 1\}$ with 100 getting added in the first iteration and 1 getting added in the second iteration. $w_2$ here also achieves 0 error.

    \textbf{3.} $\ell_1$-norm relaxation.

    Choose $B_1 = \frac{3}{\sqrt{10}}$ then $w_{B_1} = (0, 0, \ldots, 0, \frac{3}{\sqrt{10}})$ and $I_1 = \{100\}$.

    Choose $B_2 = 1$ then $w_{B_2} = (\frac{1}{\sqrt{10}}, 0, \ldots, 0, \frac{3}{\sqrt{10}})$ and $I_2 = \{1, 100\}$. Fit $w_2$, then $w_2 = w_{B_2}$ and achieves 0 error.

    \textbf{4.} Filter-by-correlation.

    We have that \begin{equation*}
    \bbe[y] = 0, Var(y) = \frac{9}{10} 1 + \frac{1}{10} 1 + 0 = 1
    \end{equation*}
    Therefore all the variables concerned ($x_1 \to x_{100}, y$) have mean 0 and variance 1. We can calculate the correlation coefficients:
    \begin{align*}
    \rho_{100} &= \frac{\cov(x_{100}, y)}{1} = \cov(x_{100}, \frac{3}{\sqrt{10}}x_{100} + \frac{1}{\sqrt{10}} x_1) = \frac{3}{\sqrt{10}} \\
    \rho_1 &= \frac{1}{\sqrt{10}}
    \end{align*}
    then for $i$ with $2 \leq i \leq 99$, then \begin{align*}
    \rho_{i} &= \cov(x_i, \frac{3}{\sqrt{10}}x_{100} + \frac{1}{\sqrt{10}} x_1) \\ 
    &= \frac{3}{\sqrt{10}} \cov(x_i, x_{100}) + \frac{1}{\sqrt{10}} \cov(x_i, x_1) \\
    &= \frac{3}{\sqrt{10}} 0.9 + 0 = \frac{2.7}{\sqrt{10}}
    \end{align*}

    Therefore $I_1 = \{100\}$, and $I_k$ for $k \in [2, 99]$ would include 100 and $(k-1)$ numbers in range $[2, 99]$ with arbitrary tie breaking. $I_{100} = [100]$ trivially.

    But then for $I_k$, as $k$ increases, the inclusion of new features does not add any information to predict $y$. WLOG, we limit the prediction to \begin{equation*}
    h_w(x) = w_{100} x_{100} + w_j x_j
    \end{equation*}
    for some $j \in [2, 99]$ for $k \leq 99$. Then \begin{align*}
    L_S(w) &= \bbe[w_{100} x_{100} + w_j x_j - \frac{3}{\sqrt{10}} x_{100} - \frac{1}{\sqrt{10}} x_1] \\
    &= \ldots \\
    &= \left(w_{100} - \frac{3}{\sqrt{10}}\right)^2 + w_j^2 + 0.9 \left(w_{100} - \frac{3}{\sqrt{10}}\right)w_j + \frac{1}{10}
    \end{align*}
    hence the optimal weight would then be when \begin{equation*}
    0 = \frac{\partial L}{\partial w_{100}} = \frac{\partial L}{\partial w_j}
    \end{equation*}
    which gives $w_{100} = \frac{3}{\sqrt{10}}, w_j = 0$, i.e., not using $x_j$ at all. This makes sense. Then the error would be \begin{equation*}
    L_S(w) = 1/10
    \end{equation*}
    so we can't get any lower using $k \leq 99$. Hence the smallest $k$ such that $L_{\cald}(w_k) \leq 0.01$ would be 100.

    \textbf{5.} Conclusion: method 2 and 3 work as well as the optimal method.

    \textbf{(b)} \begin{equation*}
    x_1 = z_1, y = z_2, x_2 = z_1 + 0.0001 z_2, x_i = z_i + 0.0001z_2
    \end{equation*}
    for $i \in \{3, 4, \ldots, 100\}$, where $z \sim N(0, I)$.

    \textbf{1.} Optimal Feature Selection.

    $I_1$ is the singleton of any number in $[2, 100]$ with arbitrary tie breaking. WLOG, $I_1 = \{2\}$. Then $w = (w_2)$, and 
    \begin{align*}
    L_S(w) &= \bbe[(w_2(z_1 + 0.0001z_2) - z_2)^2] \\
    &= w_2^2 + (0.0001w_2 - 1)^2
    \end{align*}
    which has minimum of $\approx 0.9999$ at $w_2 = \frac{0.0002}{2 + 2 \times 10^{-8}} \approx 0.0001$. Our loss $0.9999 > 0.01$. Continue:

    $I_2 = \{1, j\}$ for any $j \in [2, 100]$ with arbitrary tie breaking. WLOG, $j = 2$. Then $I_2 = \{1, 2\}$ with optimal weight $w = (-10^4, 10^4)$ achieving zero loss.

    \textbf{2.} Greedy Feature Selection

    Greedy selection selects $I_1 = \{j\}$ with arbitrary tie breaking for some $j \in [2, 100]$. This is because the best loss for $j \in [2, 100]$ would be $\approx 0.9999$, while if $I_1 = \{1\}$ then \begin{equation*}
    \exp[w_1 x_1 - y] = \exp[w_1 z_1 - z_2] = w_1^2 + 1 \geq 1 > 0.9999
    \end{equation*}

    Then, $I_2 = \{j, 1\}$, with 1 added as the next feature, and the optimal weight is the aforementioned optimal weight. This weight achieves 0 loss.

    \textbf{3.} $\ell_1$-norm relaxation.

    Choose $B_1 =  \frac{0.0002}{2 + 2 \times 10^{-8}} \approx 0.0001$, then $w_{B_1}$ is the 1-sparse tuple containing $\frac{0.0002}{2 + 2 \times 10^{-8}} \approx 0.0001$ at some index $j \in [2, 100]$. WLOG $j = 2$. Then $I_1 = \{2\}$. This weight, as aforementioned, achieves $\approx 0.9999$ loss.

    Choose $B_2 = 2 \times 10^4$, then $w_{B_2}$ is 2-sparse with $10^{-4}$ at its first index and $10^4$ at some index $j \in [2, 100]$. WLOG $j = 2$, then $I_2 = \{1, 2\}$. This weight achieves 0 loss.

    \textbf{4.} Filter-by-correlation.

    We have trivially that $\rho_1 = 0$, while for $j \in [2, 100]$, say, $j = 2$, we have \begin{equation*}
    \rho_2 = \frac{0.0001}{\sqrt{(1^2 + 0.0001^2) (1)}}  \approx 9.9 \times 10^{-5}
    \end{equation*}

    Therefore, filter by correlation, for $k \in [99]$, would select $k$ numbers from $[2, 100]$ with arbitrary tie breaking, since $\rho_j \approx 9.9 \times 10^{-5} > 0 = \rho_1 \forall j \in [99]$.

    However, $w_k$ would only be able to achieve $\approx 0.9999$ loss at best, therefore the smallest $k$ such that $L < 0.01$ would be 100, when $I_{100} = [100]$ trivially.

    \textbf{5.} Conclusion: method 2 and 3 work as well as optimal feature selection.
\end{problem}

\begin{problem}
    \textbf{(a)} WTS $L_{D^{(t+1)}}(h_t) = 0.5$.

    We state the update rule for $D$:\begin{equation*}
    D_i^{(t+1)} = \frac{D_i^{(t)} \exp(-\alpha_t y_i h_t(x_i))}{\sum_{j=1}^{m} D_j^{(t)}\exp(-\alpha_t y_j h_t(x_j))}
    \end{equation*}

    Let $E = \{i : h_t(x_i) \neq y_i\}$. Then for $i \in E$, we have
    \begin{align*}
        D_i^{(t)} \exp(-\alpha_t y_t h_t(x_i)) &= D_i^{(t)} \exp(\alpha_t) \\
        &= D_i^{(t)} \left(\frac{1}{\epsilon_t} - 1\right)^{1/2}
    \end{align*}
    and similarly if $i \not \in E$ then \begin{equation*}
        D_i^{(t)}  \exp(-\alpha_t y_t h_t(x_i)) = D_i^{(t)} \left(\frac{1}{\epsilon_t} - 1\right)^{-1/2}
    \end{equation*}
    It then follows that
    \begin{align*}
    L_{D^{(t+1)}}(h_t) &= \sum D_{i}^{(t+1)} \1 \{h_t(x_i) \neq y_i\} \\
    &= \sum_{i \in E} D_{i}^{(t+1)} \\
    &= \sum_{i \in E} \frac{D_i^{(t)} \left(\frac{1}{\epsilon_t} - 1\right)^{1/2}}{\sum_{j \in E} D_j^{(t)} \exp(-\alpha_t y_i h_t(x_i))} \\
    &= \frac{\sum_{j \in E} D_j^{(t)} \left(\frac{1}{\epsilon_t} - 1\right)^{1/2}}{\sum_{j \in E} D_j^{(t)} \left(\frac{1}{\epsilon_t} - 1\right)^{1/2} + \sum_{j \not \in E} D_j^{(t)} \left(\frac{1}{\epsilon_t} - 1\right)^{-1/2}}  \\
    &= \frac{\sum_{j \in E} D_j^{(t)} \left(\frac{1}{\epsilon_t} - 1\right)}{\sum_{j \in E} D_j^{(t)} \left(\frac{1}{\epsilon_t} - 1\right) + \sum_{j \not \in E} D_j^{(t)}}  \\
    \end{align*}
    Recall that \begin{equation*}
    \epsilon_t = \sum_{j \in E} D_j^{(t)}
    \end{equation*}
    and \begin{equation*}
    \sum_{j \not \in E} D_j^{(t)} = 1 - \sum_{j \in E} D_j^{(t)}
    \end{equation*}
    so \begin{align*}
        L_{D^{(t+1)}}(h_t) &= \frac{\sum_{j \in E} D_j^{(t)} (\sum_{j \in E} D_j^{(t)} - 1)}{\sum_{j \in E}D_j(t)(\sum_{j \in E} D_j^{(t)} - 1) - (1 - \sum_{j \in E} D_j^{(t)})(\sum_{j \in E} D_j^{(t)})} \\
        &= \frac{a(a-1)}{a(a-1) - (1-a)a} = \frac{1}{2}
    \end{align*}
    as required.

    \textbf{(b)} We rewrite what we want to prove, using $T$ instead of $t$ to avoid confusion. WTS \begin{equation*}
    \frac{\partial L_S^{\exp} (h_{w^{(T)}})}{\partial w[h]} \propto L^{01}_{D^{(T)}}(h) - \frac{1}{2}
    \end{equation*}

    We have that\begin{equation*}
        h_{w^{(T)}} (x) = \sum_{h} w^{(T)}[h] \phi(x)[h] = \sum_{h} w^{(T)}[h] h(x)
    \end{equation*}
    which implies \begin{align}
    L_S^{\exp}(h_{w^{(T)}}) &= \frac{1}{m} \sum_{i = 1}^{m} e^{-y_i h_{w^{(T)}}(x_i)} \nonumber\\
    &= \frac{1}{m} \sum_{i = 1}^{m} e^{-y_i \sum_{h} w^{(T)}[h]h(x_i)}\nonumber \\
    \implies \frac{\partial L_S^{\exp} (h_{w^{(T)}})}{\partial w[h]} &= \frac{-1}{m} \sum_{i=1}^{m} \left[y_i h(x_i) e^{-y_i \sum_{h} w^{(T)}[h] h(x_i)}\right] \nonumber\\
    &= \frac{-1}{m} \sum_{i=1}^{m} \left[y_i h(x_i) e^{-y_i h_{w^{(T)}}(x_i)}\right] \label{eqn1}
    \end{align}

    We now want to show that \begin{equation*}
    \forall i \in [m], D_i^{(T)} = C_T \exp(-y_i h_{w^{(T)}}(x_i))
    \end{equation*}
    for some constant $C_T$ that only depends on $T$.

    We will prove by induction on $T$.

    For $T = 0$, for all $i \in [m]$, we have
    \begin{equation*}   
        D_i^{(0)} = \frac{1}{m}, \quad \exp(y_i h_{w^{(0)}(x_i)}) = e^0 = 1
    \end{equation*}
    so we have $C_0 = \frac{1}{m}$.

    Suppose that the proposition holds for $T = T$, we now want to show that it is also true for $T = T+1$. Indeed, for all $i \in [m]$, we have
    \begin{align*}
        D^{(T+1)}_i &= \frac{D^{(T)}_i \exp(-\alpha_{T} y_i h_{T}(x_i))}{\sum_{j} D^{(T)}_j \exp(-\alpha_{T} y_j h_{T}(x_j))} \\
        &= \frac{C_T}{C_T} \frac{\exp(-y_i h_{w^{(T)}}(x_i) - \alpha_T y_i h_T(x_i))}{\sum_{j} \exp(-y_j h_{w^{(T)}}(x_j) - \alpha_T y_j h_T(x_j))} \\
        &= \frac{\exp(-y_i h_{w^{(T+1)}}(x_i))}{\sum_{j} \exp(-y_j h_{w^{(T+1)}}(x_j))} \\
        &= C_{T+1} \exp(-y_i h_{w^{(T+1)}}(x_i))
    \end{align*}
    where \begin{equation*}
    C_{T+1} \coloneqq \frac{1}{\sum_{j} \exp(-y_j h_{w^{(T+1)}}(x_j))}
    \end{equation*}
    is only dependent on $T$.

    By induction, we have that \begin{equation*}
    D_i^{(T)} = C_T \exp(-y_i h_{w^{(T)}}(x_i))
    \end{equation*}
    holds for all $T$.

    We now return to \eqref{eqn1}, recall that $E = \{i : h(x_i) \neq y_i\}$, and substitute $\exp(-y_i h_{w^{(T)}}(x_i)) = \frac{1}{C_T} D_i^{(T)}$, to have
    \begin{align*}
    \frac{\partial L_S^{\exp}(h_{w^{(T)}})}{\partial w[h]} &= \frac{-1}{mC_T} \sum_{i=1}^{m} \left[y_i h(x_i) D_i^{(T)}\right] \\
    &= \frac{1}{mC_T} \left[\sum_{i \in E} D_i^{(T)} - \sum_{i \not \in E} D_i^{(T)}\right] \\
    &= \frac{1}{C_T} \left[L^{01}_{D^{(T)}}(h) - (1 - L^{01}_{D^{(T)}}(h))\right] \\
    &= \frac{2}{C_T} \left[L^{01}_{D^{(T)}}(h) - \frac{1}{2}\right] \propto \left[L^{01}_{D^{(T)}}(h) - \frac{1}{2}\right] 
    \end{align*}
    as required.
\end{problem}

\end{document}