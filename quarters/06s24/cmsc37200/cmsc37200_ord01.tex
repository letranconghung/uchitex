\documentclass{amsart}
\usepackage{amssymb}
\usepackage{fancyhdr}
\usepackage{parskip}

\setlength{\parindent}{0pt}

\usepackage{amsmath,mathtools,amssymb}
\usepackage{xcolor}
% 5. Math shorthands
% 5.1. Fonts
\newcommand*{\bbf}{\mathbb{F}}
\newcommand*{\bbn}{\mathbb{N}}
\newcommand*{\bbq}{\mathbb{Q}}
\newcommand*{\bbr}{\mathbb{R}}
\newcommand*{\R}{\mathbb{R}}
\newcommand*{\Rn}{\mathbb{R}^n}
\newcommand*{\bbz}{\mathbb{Z}}
\newcommand*{\bbc}{\mathbb{C}}
\newcommand*{\bbk}{\mathbb{K}}
\newcommand*{\bbm}{\mathbb{M}}
\newcommand*{\bbp}{\mathbb{P}}
\newcommand*{\bbe}{\mathbb{E}}
\newcommand*{\1}{\mathds{1}}

\newcommand*{\bfw}{\mathbf{w}}
\newcommand*{\bfx}{\mathbf{x}}
\newcommand*{\bfX}{\mathbf{X}}
\newcommand*{\bfU}{\mathbf{U}}
\newcommand*{\bfP}{\mathbf{P}}   
\newcommand*{\bfy}{\mathbf{y}}
\newcommand*{\bfyhat}{\mathbf{\hat{y}}}
\newcommand*{\bfv}{\mathbf{v}}
\newcommand*{\bfV}{\mathbf{V}}
\newcommand*{\bfu}{\mathbf{u}}
\newcommand*{\bfSigma}{\mathbf{\Sigma}}


\newcommand*{\cala}{\mathcal{A}}
\newcommand*{\calb}{\mathcal{B}}
\newcommand*{\calc}{\mathcal{C}}
\newcommand*{\cald}{\mathcal{D}}
\newcommand*{\calf}{\mathcal{F}}
\newcommand*{\calg}{\mathcal{G}}
\newcommand*{\calh}{\mathcal{H}}
\newcommand*{\call}{\mathcal{L}}
\newcommand*{\calm}{\mathcal{M}}
\newcommand*{\calp}{\mathcal{P}}
\newcommand*{\calr}{\mathcal{R}}
\newcommand*{\cals}{\mathcal{S}}
\newcommand*{\calt}{\mathcal{T}}
\newcommand*{\calu}{\mathcal{U}}
\newcommand*{\calv}{\mathcal{V}}
\newcommand*{\calw}{\mathcal{W}}
\newcommand*{\calx}{\mathcal{X}}
\newcommand*{\caly}{\mathcal{Y}}

\newcommand*{\rme}{\mathrm{e}}
\newcommand*{\rmd}{\mathrm{d}}
\newcommand*{\rmosc}{\mathrm{osc}}
\newcommand*{\lhs}{\ensuremath{\mathrm{LHS}}}
\newcommand*{\rhs}{\ensuremath{\mathrm{RHS}}}

\newcommand*{\frake}{\mathfrak{e}}
% Proof Writing
\newcommand*{\st}{\hspace*{2pt}\text{s.t.}\hspace*{2pt}}
\newcommand*{\pffwd}{\hspace*{2pt}\fbox{\(\Rightarrow\)}\hspace*{10pt}}
\newcommand*{\pfbwd}{\hspace*{2pt}\fbox{\(\Leftarrow\)}\hspace*{10pt}}
\newcommand*{\contra}{\ensuremath{\Rightarrow\Leftarrow}}

% Shorthands
\newcommand*{\cvgi}{\xrightarrow{j \to \infty}}
\newcommand*{\cvgj}{\xrightarrow{j \to \infty}}
\newcommand*{\cvgk}{\xrightarrow{k \to \infty}}
\newcommand*{\cvgn}{\xrightarrow{n \to \infty}}
\newcommand*{\cvgx}{\xrightarrow{x\to \infty}}
\newcommand*{\cvgg}[1]{\xrightarrow{#1}}
\newcommand*{\eval}[3]{\left[#1\right]^{#2}_{#3}}
\newcommand*{\inv}[1]{{#1}^{-1}}

\newcommand*{\unicvg}{\rightrightarrows}
\newcommand*{\dx}{\mathrm{d} x}
\newcommand*{\ddx}{\frac{\mathrm{d}}{\mathrm{d} x}}
\newcommand*{\dy}{\mathrm{d} y}
\newcommand*{\dz}{\mathrm{d} z}
\newcommand*{\du}{\mathrm{d} u}
\newcommand*{\dv}{\mathrm{d} v}
\newcommand*{\dt}{\mathrm{d} t}
\newcommand*{\sigmoid}{\mathrm{sigmoid}}
\newcommand*{\softmax}{\mathrm{softmax}}
\newcommand*{\rk}{\mathrm{rank}}
\newcommand*{\iprod}[2]{\langle #1, #2 \rangle}
\newcommand*{\conj}[1]{\overline{#1}}
\newcommand*{\sign}[1]{\mathrm{sign}\left(#1\right)}
\newcommand*{\cl}[1]{\overline{#1}}

% 5.2. Symbols maneuvering
% https://tex.stackexchange.com/questions/438612/space-between-exists-and-forall
% https://tex.stackexchange.com/questions/22798/nice-looking-empty-set
\let\implies\Rightarrow
\let\impliedby\Leftarrow
\let\iff\Leftrightarrow
\let\epsilon\varepsilon
% \renewcommand{\phi}{\varphi}
\let\oldforall\forall
\renewcommand{\forall}{\;\oldforall\; }
\let\oldexist\exists
\renewcommand{\exists}{\;\oldexist\; }
\newcommand\existu{\;\oldexist!\: }
\let\oldemptyset\emptyset
\let\emptyset\varnothing

% 5.3. New math operators
\DeclareMathOperator*{\esssup}{ess\,sup}
\DeclareMathOperator*{\Hom}{Hom}
\DeclareMathOperator*{\im}{Im}
\DeclareMathOperator*{\spann}{span}
\DeclareMathOperator*{\argmax}{arg\,max}
\DeclareMathOperator*{\argmin}{arg\,min}
\DeclareMathOperator*{\interior}{int}
\DeclareMathOperator*{\supp}{supp}
\DeclareMathOperator*{\var}{Var}
\DeclareMathOperator*{\cov}{Cov}
\DeclareMathOperator*{\diam}{diam}
\DeclareMathOperator*{\range}{range}
\DeclareMathOperator*{\lcm}{lcm}
\DeclarePairedDelimiter{\abs}{\lvert}{\rvert}
\DeclarePairedDelimiter{\norm}{\lVert}{\rVert}
\DeclarePairedDelimiter{\ceil}{\lceil}{\rceil}
\DeclarePairedDelimiter{\floor}{\lfloor}{\rfloor}

% 5.4. Formatting
\newcommand*{\redtext}[1]{\textcolor{red}{#1}}
\newcommand*{\done}{\redtext{done}}
\newcommand*{\todo}[1]{\colorbox{red}{#1}}
\pagestyle{fancy}

\rhead{Hung Le Tran}   %% <-- your name here
\chead{}
\cfoot{\thepage}
\rfoot{\today}  


%% your macros -->

\begin{document}
\Large
\noindent
Assignment {\bf ORD 1}   %% <-- Gradescope assignment title here ***

\medskip\noindent
\hrule


\medskip\noindent
Solutions by {\bf Hung Le Tran} \qquad   %% <-- your name here ***
  {\tt conghunglt(at)u.e.}      %% <-- your uchicago email address here ***

\vspace{0.5cm}

\noindent
This document contains my solutions to
the Gradescope assignment named on the top of this page.
Specifically, my solutions to the following problems
are included:

\vspace{0.4cm}
\begin{itemize}  %% *** update this list
\item 01.31 (page 3)
\item 01.35 (page 4)
\item 01.45 (page 5)
\item 01.54 (page 6)
\item 01.62 (page 7)
\item 01.68 (page 8)
\item 01.71 (page 9)
\item 01.77 (page 10)
\item 01.87(a)(b) (pages 11-12)
\item 01.92 (page 13)
\item 01.101 (page 14)
\item 01.151 (page 15)
\item 01.164 (page 16)
\end{itemize}

\vspace{0.5cm}

\noindent
I did not forget 
\begin{itemize}
  \item to REFRESH my browser for the latest information
   about each problem
  \item \underline{to link problems to pages}.\\
  This page is linked to the problems I did not solve.
\item to update the items marked *** in the template
  (my name, email, the Gradescope title of the assignment,
  the list of problems solved, the \verb>\lhead> statements
  (left page headers: list of (sub)problems solved on each page)
\item to make sure no subproblem solution spills over to the next page
  (except when this is unavoidable, i.e., when the solution to a
  subproblem does not fit on a page)
\item if a problem takes more than one page, I linked
  each of those pages to the problem
\item I took care not to defeat the mechanisms provided by this template.
\end{itemize}
  With each problem, {\bf I stated my sources and collaborations}.\\
  By submitting this solution \emph{I certify} that\\
  \emph{my statement of sources and collaborations is accurate and complete}.\\
  I understand that without this certification, my solutions will
  not be accepted.
  
\vspace{1cm}

% ******** START OF CONTENT ******* %
\newpage
\Large
\lhead{Problem 01.31}

\noindent
(\done) 01.31 Question. \\
Give a simple sentence in plain English expressing the negation of the statement that ``the sequence $S$ is eventually nonzero.'' Do not use the word ``not'' and avoid mathematical usage such as ``there exist(s)''. You may use the word ``zero'' and variants of the word ``infinity''.

\medskip\noindent
\emph{Sources and collaborations.}\\
None

\medskip\noindent
\emph{Proof.}\\
Sequence $S$ has an infinite number of zeros. \qed

\newpage
\Large
\lhead{Problem 01.35}

\noindent
(\done) 01.35 Question.\\
Denote the Fibonacci numbers: $F_0 = 0, F_1 = 1, F_{n+2} = F_{n+1} + F_n$. Let $m \in \bbn$. Prove that the sequence $(F_n \bmod m)$ is periodic with period $\leq m^2 -1$.\\
Example: The $(F_n \bmod 3)$ sequence is periodic with period 8; the repeating part is $0, 1, 1, 2, 0, 2, 2, 1$.

\medskip\noindent
\emph{Sources and collaborations.}\\
None.

\medskip
\emph{Proof.} Fix $m \in \bbn$. Define a new sequence $G_n \coloneqq F_n \bmod m$. WTS $(G_n)$ is periodic with period $\leq m^2 - 1$. It trivially follows that $0 \leq G_n \leq m-1$.

Investigate the tuples $\{(G_i, G_{i+1}) : 0 \leq i \leq m^2 - 1\}$.

If there exists some $i$ such that $G_i = G_{i+1} = 0$ then we are done, since the $G_n$ sequence would be identically zero and thus have period 1. 

If there doesn't, then there remains $m^2 - 1$ possible 2-tuples of numbers from $0$ to $m-1$, but there are $m^2$ such tuples, so by the Pigeonhole Principle, it follows that there exists some $0 \leq i < j \leq m^2-1$ such that $G_i = G_j, G_{i+1} = G_{j+1}$.

Let $d = j - i \leq m^2 - 1$. WTS $G_n$ is periodic with period $d$, i.e., $\forall n \in \bbn_0, G_{n + d} = G_n$. We shall use the following Lemma to continue.

\noindent \textbf{Lemma 1.}\\
If $G_k = G_{k + d}, G_{k+1} = G_{k+1+d}$ then $G_{k-1} = G_{k-1+d}$ and $G_{k+2} = G_{k+2+d}$.

\noindent \emph{Proof of Lemma 1.}\\
We have $G_{k-1} = F_{k - 1} \bmod m = F_{k+1} \bmod m - F_{k} \bmod m = G_{k+1} - G_k = G_{k+1+d} - G_{k+d} = F_{k+1+d} \bmod m - F_{k + d} \bmod m = F_{k-1+d} \bmod m = G_{k-1+d}$
and $G_{k+2} = F_{k+2} \bmod m = F_{k} \bmod m + F_{k+1} \bmod m = G_k + G_{k+1} = G_{k + d} + G_{k+1+d} = G_{k+2+d}$ as required. \qed

\noindent\emph{Back to main proof.}\\
Use Lemma 1 to perform mathematical induction from $i$ downwards, so that $G_n = G_{n+d}$ for all $n \leq i$, then perform mathematical induction from $i$ upwards, so that $G_n = G_{n+d}$ for all $n \geq i$. It then follows that $G_n = G_{n+d}$ for all $n \in \bbn_0$. Hence $(G_n)$ has period $d \leq m^2-1$. \qed

\bigskip\hrule

\vspace{0.5cm}

\newpage
\Large
\lhead{Problem 01.45}

\noindent
(\done) 01.45 Question. \\
Find the quotients of all Fibonacci-type geometric progressions. \textit{Hint.} There are two such ratios; one of them is the golden ratio.

\medskip\noindent
\emph{Sources and collaborations.}\\
None.

\medskip\noindent
\emph{Proof.}\\
Let $(a_n = aq^n)_{n \in \bbn_0}$ be a geometric progression for some $a \in \bbr$. If it is Fibonacci-type then for all $n \in \bbn_0$ we have \begin{equation*}
a_{n + 2} = a_n + a_{n+1}
\end{equation*}
This implies \begin{equation*}
aq^{n+2} = aq^n + aq^{n+1}
\end{equation*}
The trivial sequence of all zeros would admit any number as a quotient. Let us, then, only consider non-trivial sequences, hence with $a, q \neq 0$. Then the above equation implies \begin{equation*}
q^2 - q - 1 = 0
\end{equation*}
which has roots $\frac{1 + \sqrt{5}}{2}$ (golden ratio) and $\frac{1 - \sqrt{5}}{2}$.

In conclusion, the valid quotients are $\frac{1 + \sqrt{5}}{2}$ and $\frac{1 - \sqrt{5}}{2}$. \qed

\bigskip\hrule

\vspace{0.5cm}

\newpage
\Large
\lhead{Problem 01.54}

\noindent
(\done) 01.54 Question. \\
Find two bounded sequences, $(a_n)$ and $(b_n)$ such that $\limsup (a_n + b_n) < \limsup a_n + \limsup b_n$.

\medskip\noindent
\emph{Sources and collaborations.}\\
None.

\medskip\noindent
\emph{Proof.}\\
Consider $(a_n) = (1, 0, 1, 0, \ldots)$ the $\{0, 1\}$-alternating sequence that starts with 1, and $(b_n) = (0, 1, 0, 1, \ldots)$ the $\{0, 1\}$-alternating sequence that starts with 0. Then $\limsup a_n = \limsup b_n = 1$ and $\limsup a_n + b_n = 1$.
\qed

\bigskip\hrule

\vspace{0.5cm}

\newpage
\Large
\lhead{Problm 01.62}

\noindent
(\done) 01.62 Question. \\
Prove that there exist real numbers $a, b$ such that $\binom{n}{5} \sim a \cdot n^b$. Find $a, b$. Make your proof elegant.

\medskip\noindent
\emph{Sources and collaborations.}\\
None.

\medskip\noindent
\emph{Proof.}\\
\begin{equation*}
    \binom{n}{5} = \frac{n(n-1)(n-2)(n-3)(n-4)}{5!}
\end{equation*}
is a degree 5 polynomial in $n$ with leading term $\frac{n^5}{5!} = \frac{n^5}{120}$ so it is asymptotically equal to $a \cdot n^b$ where $a = \frac{1}{120}$ and $b = 5$.
\qed

\bigskip\hrule

\vspace{0.5cm}


\newpage
\Large
\lhead{Problem 01.68}

\noindent
(\done) 01.68 Question. \\
Prove $\sqrt{n^2 + 1} -n \sim 1/(2n)$.

\medskip\noindent
\emph{Sources and collaborations.}\\
None.

\medskip\noindent
\emph{Proof.}\\
We have 
\begin{align*}
    \sqrt{n^2 + 1} - n &= \frac{n^2 + 1 - n^2}{\sqrt{n^2 + 1} + n} \\
    &= \frac{1}{\sqrt{n^2 + 1} + n}
\end{align*}
and
\begin{equation*}
\lim_{n \to \infty} \frac{1/(\sqrt{n^2 + 1} + n)}{1/(2n)} = \lim_{n \to \infty} \frac{2n}{\sqrt{n^2 + 1} + n} = 1
\end{equation*}
so $\sqrt{n^2 + 1} - n \sim 1/(2n)$ as required.
\qed

\bigskip\hrule

\vspace{0.5cm}

\newpage
\Large
\lhead{Problem 01.71}
\noindent
(\done) 01.71 Question. \\
If $f$ is a function, differentiable at zero, $f(0) = 0$, and $f'(0) \neq 0$, then $f(1/n) \sim f'(0)/n$.

\medskip\noindent
\emph{Sources and collaborations.}\\
None.

\medskip\noindent
\emph{Proof.}\\
$f$ is differentiable at 0 so we can expand:
\begin{equation*}
f(x) = f(0) + f'(0)x + o(\abs{x}) = f'(0)x + R(x)
\end{equation*}
where $\lim_{x \to 0} \frac{R(x)}{\abs{x}} = 0$.

It then follows that \begin{equation*}
\lim_{n \to \infty} \frac{f(1/n)}{f'(0)/n} = \lim_{n \to \infty} \frac{f'(0)/n + R(1/n)}{f'(0)/n} = 1 + \lim_{n \to \infty} \frac{R(1/n)}{f'(0)/n} = 1
\end{equation*}
\qed

\bigskip\hrule

\vspace{0.5cm}

\newpage
\Large
\lhead{Problem 01.77}

\noindent
(\done) 01.77 Question. \\
Prove that there exist numbers $a, b, c$ such that $\binom{2n}{n} \sim an^bc^n$. Find $a, b, c$.

\medskip\noindent
\emph{Sources and collaborations.}\\
None

\medskip\noindent
\emph{Proof.}\\
Using Stirling's Formula, we have
\begin{align*}
    \binom{2n}{n} &= \frac{(2n)!}{n!n!} \\
    &\sim \sqrt{2\pi(2n)} \left(\frac{2n}{e}\right)^{2n} \left(\sqrt{2\pi n}\left(\frac{n}{e}\right)^n\right)^{-2} \\
    &= \frac{1}{\sqrt{n}}n^{-1/2}4^{n}
\end{align*}
so $a = \frac{1}{\sqrt{\pi}}, b = -1/2, c = 4$.
\hfill $\Box$

\bigskip\hrule

\vspace{0.5cm}





\newpage
\Large
\lhead{Problem 01.87 (a)}

\noindent
(\done) 01.87(a) Question. \\
Assume $a_n, b_n > 1$. Consider the following statements:\\
(A) $a_n \sim b_n$\\
(B) $\ln a_n \sim \ln b_n$\\
Prove (A) does not imply (B).

\medskip\noindent
\emph{Sources and collaborations.}\\
None.

\medskip\noindent
\emph{Proof.}\\
Consider $(a_n = e^{2/n})$ and $(b_n = e^{1/n})$ then we have \begin{equation*}
\lim_{n \to \infty} \frac{a_n}{b_n} = \lim_{n \to \infty} \frac{e^{2/n}}{e^{1/n}} = \lim_{n \to \infty} e^{1/n} = 1
\end{equation*}
so $a_n \sim b_n$ but \begin{equation*}
\lim_{n \to \infty} \frac{\ln a_n}{\ln b_n} = \lim_{n \to \infty} \frac{2/n}{1/n} = 2
\end{equation*}
so $\ln a_n \not \sim \ln b_n$.

It then follows that (A) does not imply (B).
\qed

\bigskip\hrule

\vspace{0.5cm}

\newpage
\Large
\lhead{Problem 01.87(b)}



\noindent
(\done) 01.87(b) Question. \\
Prove (A) does imply (B) under the stronger assumption that $a_n \geq 1.01$.

\medskip\noindent
\emph{Sources and collaborations.}\\
None.

\medskip\noindent
\emph{Proof.}\\
We have $a_n \geq 1.01 \implies \ln a_n \geq \ln 1.01 \eqqcolon C > 0$.

Let us investigate if the following limit exists:
\begin{equation*}
\lim_{n \to \infty} \frac{\ln(b_n/a_n)}{\ln a_n}
\end{equation*}

Indeed,
\begin{equation*}
    \abs*{\frac{\ln(b_n / a_n)}{\ln(a_n)}} = \frac{\abs{\ln(b_n / a_n)}}{\ln a_n} \leq \frac{\abs{\ln(b_n/a_n)}}{C} \cvgn 0 
\end{equation*}
since $\lim_{n \to \infty} \frac{b_n}{a_n} = 1$.

It then follows by Squeeze Theorem that \begin{equation*}
\lim_{n \to \infty} \frac{\ln(b_n / a_n)}{\ln a_n} = 0
\end{equation*}
so
\begin{equation*}
    \lim_{n \to \infty} \frac{\ln b_n}{\ln a_n} = \lim_{n \to \infty} \frac{\ln a_n + \ln (b_n /a_n)}{\ln a_n} = 1 + \lim_{n \to \infty} \frac{\ln(b_n /a_n)}{\ln a_n} = 1
    \end{equation*}
hence $\ln b_n \sim \ln a_n$ as required.
\qed

\bigskip\hrule

\vspace{0.5cm}

\newpage
\Large
\lhead{Problem 01.92}

\noindent
(\done) 01.92 Question. \\
Prove that for all $n \geq 1$, we have $n! > \left(\frac{n}{e}\right)^n$. Use the power series expansion of $e^x$.

\medskip\noindent
\emph{Sources and collaborations.}\\
None.

\medskip\noindent
\emph{Proof.}\\
We have for $x \geq 1$, \begin{equation*}
e^x = \sum_{n = 0}^{\infty} \frac{1}{n!}x^n >\frac{x^n}{n!}
\end{equation*}
In particular, this holds for $x = n$, which implies \begin{equation*}
e^n > \frac{n^n}{n!} \implies n! > \left(\frac{n}{e}\right)^n
\end{equation*}
\qed

\bigskip\hrule

\vspace{0.5cm}


\newpage
\Large
\lhead{Problem 01.101}

\noindent
(\done) 01.101 Question. \\
Prove $\ln (n!) \sim n \ln n$ without using Stirling's formula, using ASY 6.8 instead.

\medskip\noindent
\emph{Sources and collaborations.}\\
None.

\medskip\noindent
\emph{Proof.}\\
Using ASY 6.8, we have
\begin{equation*}
n^n > n! > \left(\frac{n}{e}\right)^n
\end{equation*}
which implies
\begin{equation*}
n \ln n > \ln n! > n (\ln n - 1)
\end{equation*}
which implies
\begin{equation*}
1 + \frac{n}{\ln n!} > \frac{n \ln n}{\ln n!} > 1
\end{equation*}

For $n$ large we have $1 + \frac{n}{ \ln n!} = 1 + \frac{1}{\ln (n-1)!}$ so it converges to 1 as $n$ tends to $\infty$. By Squeeze Theorem, it follows that $\lim_{n \to \infty} \frac{n \ln n}{\ln n!} = 1$ so $n \ln n \sim \ln n!$ as required.
\qed

\bigskip\hrule

\vspace{0.5cm}

\newpage
\Large
\lhead{Problem 01.151}

\noindent
(\done) 01.151 Question. \\
Prove that the bound in the Eventown Theorem is tight, i.e., $m = 2^{\floor{n/2}}$ is achievable for all $n$.

\medskip\noindent
\emph{Sources and collaborations.}\\
None.

\medskip\noindent
\emph{Proof.}\\
Denote the members as $\Omega = \{0, \ldots, n-1\}$. Let us pair up members $(2i, 2i+1)$ for $0 \leq i \leq \floor{n/2} -1$ so we have $\floor{n/2}$ pairs.

We claim that the system that consist of all subsets of the set of pairs (which contains $2^{\floor{n/2}}$ clubs) is an Eventown system.

Indeed, the intersection between each pair of clubs is even since they intersect at some whole number of pairs, which pack an even number of members.

Hence $m = 2^{\floor{n/2}}$ is achievable for all $n$.
\qed

\bigskip\hrule

\vspace{0.5cm}

\newpage
\Large
\lhead{Problem 01.164}

\noindent
(\done) 01.164 Question. \\
For every $n \in \bbn$ determine the minimum number of clubs in a maximal Oddtown system.

\medskip\noindent
\emph{Sources and collaborations.}\\
None.

\medskip\noindent
\emph{Proof.}\\
If $n$ is odd, then the system that only has 1 club, namely the club that contains everyone, is a maximal system. Because if any other club is added, it would intersect with this club in its entirety, i.e., with an odd number of members, so not even.

If $n$ is even, then the system that has 2 clubs, club 1 consisting of everyone except for person $1$ and club 2 consisting only of person $1$, is a maximal system. Because if a new club wants to be added, then if it contains person 1, its intersection with club 2 would be odd, hence invalid. It then has to exclude person 1 and is therefore contained in club 1 and we run into the same problem as above.

So the answer is 1 for odd $n$ and 2 for even $n$.
\qed

\bigskip\hrule

\vspace{0.5cm}





\end{document}