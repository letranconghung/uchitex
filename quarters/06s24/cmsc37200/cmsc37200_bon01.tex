\documentclass{amsart}
\usepackage{amssymb}
\usepackage{fancyhdr}
\usepackage{parskip}

\setlength{\parindent}{0pt}

\usepackage{amsmath,mathtools,amssymb}
\usepackage{xcolor}
% 5. Math shorthands
% 5.1. Fonts
\newcommand*{\bbf}{\mathbb{F}}
\newcommand*{\bbn}{\mathbb{N}}
\newcommand*{\bbq}{\mathbb{Q}}
\newcommand*{\bbr}{\mathbb{R}}
\newcommand*{\R}{\mathbb{R}}
\newcommand*{\Rn}{\mathbb{R}^n}
\newcommand*{\bbz}{\mathbb{Z}}
\newcommand*{\bbc}{\mathbb{C}}
\newcommand*{\bbk}{\mathbb{K}}
\newcommand*{\bbm}{\mathbb{M}}
\newcommand*{\bbp}{\mathbb{P}}
\newcommand*{\bbe}{\mathbb{E}}
\newcommand*{\1}{\mathds{1}}

\newcommand*{\bfw}{\mathbf{w}}
\newcommand*{\bfx}{\mathbf{x}}
\newcommand*{\bfX}{\mathbf{X}}
\newcommand*{\bfU}{\mathbf{U}}
\newcommand*{\bfP}{\mathbf{P}}   
\newcommand*{\bfy}{\mathbf{y}}
\newcommand*{\bfyhat}{\mathbf{\hat{y}}}
\newcommand*{\bfv}{\mathbf{v}}
\newcommand*{\bfV}{\mathbf{V}}
\newcommand*{\bfu}{\mathbf{u}}
\newcommand*{\bfSigma}{\mathbf{\Sigma}}


\newcommand*{\cala}{\mathcal{A}}
\newcommand*{\calb}{\mathcal{B}}
\newcommand*{\calc}{\mathcal{C}}
\newcommand*{\cald}{\mathcal{D}}
\newcommand*{\calf}{\mathcal{F}}
\newcommand*{\calg}{\mathcal{G}}
\newcommand*{\calh}{\mathcal{H}}
\newcommand*{\call}{\mathcal{L}}
\newcommand*{\calm}{\mathcal{M}}
\newcommand*{\calp}{\mathcal{P}}
\newcommand*{\calr}{\mathcal{R}}
\newcommand*{\cals}{\mathcal{S}}
\newcommand*{\calt}{\mathcal{T}}
\newcommand*{\calu}{\mathcal{U}}
\newcommand*{\calv}{\mathcal{V}}
\newcommand*{\calw}{\mathcal{W}}
\newcommand*{\calx}{\mathcal{X}}
\newcommand*{\caly}{\mathcal{Y}}

\newcommand*{\rme}{\mathrm{e}}
\newcommand*{\rmd}{\mathrm{d}}
\newcommand*{\rmosc}{\mathrm{osc}}
\newcommand*{\lhs}{\ensuremath{\mathrm{LHS}}}
\newcommand*{\rhs}{\ensuremath{\mathrm{RHS}}}

\newcommand*{\frake}{\mathfrak{e}}
% Proof Writing
\newcommand*{\st}{\hspace*{2pt}\text{s.t.}\hspace*{2pt}}
\newcommand*{\pffwd}{\hspace*{2pt}\fbox{\(\Rightarrow\)}\hspace*{10pt}}
\newcommand*{\pfbwd}{\hspace*{2pt}\fbox{\(\Leftarrow\)}\hspace*{10pt}}
\newcommand*{\contra}{\ensuremath{\Rightarrow\Leftarrow}}

% Shorthands
\newcommand*{\cvgi}{\xrightarrow{j \to \infty}}
\newcommand*{\cvgj}{\xrightarrow{j \to \infty}}
\newcommand*{\cvgk}{\xrightarrow{k \to \infty}}
\newcommand*{\cvgn}{\xrightarrow{n \to \infty}}
\newcommand*{\cvgx}{\xrightarrow{x\to \infty}}
\newcommand*{\cvgg}[1]{\xrightarrow{#1}}
\newcommand*{\eval}[3]{\left[#1\right]^{#2}_{#3}}
\newcommand*{\inv}[1]{{#1}^{-1}}

\newcommand*{\unicvg}{\rightrightarrows}
\newcommand*{\dx}{\mathrm{d} x}
\newcommand*{\ddx}{\frac{\mathrm{d}}{\mathrm{d} x}}
\newcommand*{\dy}{\mathrm{d} y}
\newcommand*{\dz}{\mathrm{d} z}
\newcommand*{\du}{\mathrm{d} u}
\newcommand*{\dv}{\mathrm{d} v}
\newcommand*{\dt}{\mathrm{d} t}
\newcommand*{\sigmoid}{\mathrm{sigmoid}}
\newcommand*{\softmax}{\mathrm{softmax}}
\newcommand*{\rk}{\mathrm{rank}}
\newcommand*{\iprod}[2]{\langle #1, #2 \rangle}
\newcommand*{\conj}[1]{\overline{#1}}
\newcommand*{\sign}[1]{\mathrm{sign}\left(#1\right)}
\newcommand*{\cl}[1]{\overline{#1}}

% 5.2. Symbols maneuvering
% https://tex.stackexchange.com/questions/438612/space-between-exists-and-forall
% https://tex.stackexchange.com/questions/22798/nice-looking-empty-set
\let\implies\Rightarrow
\let\impliedby\Leftarrow
\let\iff\Leftrightarrow
\let\epsilon\varepsilon
% \renewcommand{\phi}{\varphi}
\let\oldforall\forall
\renewcommand{\forall}{\;\oldforall\; }
\let\oldexist\exists
\renewcommand{\exists}{\;\oldexist\; }
\newcommand\existu{\;\oldexist!\: }
\let\oldemptyset\emptyset
\let\emptyset\varnothing

% 5.3. New math operators
\DeclareMathOperator*{\esssup}{ess\,sup}
\DeclareMathOperator*{\Hom}{Hom}
\DeclareMathOperator*{\im}{Im}
\DeclareMathOperator*{\spann}{span}
\DeclareMathOperator*{\argmax}{arg\,max}
\DeclareMathOperator*{\argmin}{arg\,min}
\DeclareMathOperator*{\interior}{int}
\DeclareMathOperator*{\supp}{supp}
\DeclareMathOperator*{\var}{Var}
\DeclareMathOperator*{\cov}{Cov}
\DeclareMathOperator*{\diam}{diam}
\DeclareMathOperator*{\range}{range}
\DeclareMathOperator*{\lcm}{lcm}
\DeclarePairedDelimiter{\abs}{\lvert}{\rvert}
\DeclarePairedDelimiter{\norm}{\lVert}{\rVert}
\DeclarePairedDelimiter{\ceil}{\lceil}{\rceil}
\DeclarePairedDelimiter{\floor}{\lfloor}{\rfloor}

% 5.4. Formatting
\newcommand*{\redtext}[1]{\textcolor{red}{#1}}
\newcommand*{\done}{\redtext{done}}
\newcommand*{\todo}[1]{\colorbox{red}{#1}}
\pagestyle{fancy}

\rhead{Hung Le Tran}   %% <-- your name here
\chead{}
\cfoot{\thepage}
\rfoot{\today}  


%% your macros -->

\begin{document}
\Large
\noindent
Assignment {\bf BON 1}   %% <-- Gradescope assignment title here ***

\medskip\noindent
\hrule


\medskip\noindent
Solutions by {\bf Hung Le Tran} \qquad   %% <-- your name here ***
  {\tt conghunglt(at)u.e.}      %% <-- your uchicago email address here ***

\vspace{0.5cm}

\noindent
This document contains my solutions to
the Gradescope assignment named on the top of this page.
Specifically, my solutions to the following problems
are included:

\vspace{0.4cm}
\begin{itemize}  %% *** update this list
\item 01.104 (pages 2-3)
\item 01.159 (page 4)
\end{itemize}

\vspace{0.5cm}

\noindent
I did not forget 
\begin{itemize}
  \item to REFRESH my browser for the latest information
   about each problem
  \item \underline{to link problems to pages}.\\
  This page is linked to the problems I did not solve.
\item to update the items marked *** in the template
  (my name, email, the Gradescope title of the assignment,
  the list of problems solved, the \verb>\lhead> statements
  (left page headers: list of (sub)problems solved on each page)
\item to make sure no subproblem solution spills over to the next page
  (except when this is unavoidable, i.e., when the solution to a
  subproblem does not fit on a page)
\item if a problem takes more than one page, I linked
  each of those pages to the problem
\item I took care not to defeat the mechanisms provided by this template.
\end{itemize}
  With each problem, {\bf I stated my sources and collaborations}.\\
  By submitting this solution \emph{I certify} that\\
  \emph{my statement of sources and collaborations is accurate and complete}.\\
  I understand that without this certification, my solutions will
  not be accepted.
  
\vspace{1cm}

\newpage
\Large
\lhead{Problem 01.104}

\noindent
(\done) 01.104 Question. \\
Let $p_n$ be the $n$-th prime number. Consider the statement:
\begin{equation*}
p_n \sim n \ln n
\end{equation*}

Prove that this statement is equivalent to the Prime Number Theorem.

\medskip\noindent
\emph{Sources and collaborations.}\\
None

\medskip\noindent
\emph{Proof.}\\
Let us restate the Prime Number Theorem (PNT).

\textbf{Theorem.} Let $\pi(x)$ be the numbers of prime numbers $\leq x$. Then \begin{equation*}
\pi(x) \sim \frac{x}{\ln x}
\end{equation*}

\pffwd We first show that $p_n \sim n \ln n$ implies PNT.

Let $0 < \epsilon \ll 1$ be arbitrary. Take $\delta = \epsilon/2$.

Since $p_n \sim n \ln n$, there exists some $N = N_\delta = N_\epsilon \in \bbn$ such that $n \geq N \implies (1-\delta) n \ln n \leq p_n \leq (1+\delta) n \ln n$.

Take $X = p_N = X_\epsilon$. For all $x \geq X$, let $m = \pi(x) \geq \pi(X) = N$ then we have
\begin{align*}
    p_{m} \leq x \leq p_{m + 1}
\end{align*}
but $m \geq N$ so \begin{equation*}
(1-\delta)m \ln m \leq x \leq (1 + \delta) (m+1) \ln (m+1)
\end{equation*}
$\ln$ is monotonic, so:
\begin{equation*}
\ln(1-\delta) + \ln m  + \ln\ln m \leq \ln x \leq \ln(1 + \delta) + \ln (m+1) \ln\ln (m+1)
\end{equation*}
It then follows that for all $x \geq X$,
\begin{multline*}
    \frac{(1-\delta) m \ln m}{m[\ln(1+\delta) + \ln (m+1) + \ln \ln (m+1)]} \leq \frac{x}{\pi(x)\ln x} \\
    \leq \frac{(1+\delta) (m+1) \ln (m+1)}{m [\ln(1-\delta) + \ln m + \ln \ln m]}
\end{multline*}

Investigating the limit as $m \to \infty$, we have:
\begin{align*}
    \lim_{m \to \infty} \lhs &= \lim_{m \to \infty}(1 - \delta) \frac{\ln m}{\ln (1 + \delta) + \ln (m+1) + \ln \ln (m+1)} \\
    &= 1 - \delta,
\end{align*}
using the $\ln$ asymptotic result in 01.87(b) to get $\ln (m+1) \sim \ln m$, and that $\ln m$ grows exponentially faster than $\ln \ln (m+1)$.

Similarly, $\lim_{m \to \infty} \rhs = 1 + \delta$.

What the 2 limits above imply is that there exists some $M = M_\epsilon$ such that $\pi(x) \geq M_\epsilon \implies \abs{\lhs - (1 - \delta)}, \abs{\rhs - (1 + \delta)} < \frac{\epsilon}{2}$, implying $\abs{\lhs - 1}, \abs{\rhs - 1} < \epsilon$, so \begin{equation*}
\abs*{\frac{\pi(x) \ln x}{x} - 1} < \epsilon
\end{equation*}

Therefore, for all $x \geq p_{\ceil{M_\epsilon}}$, the above inequality is achieved.

$\epsilon$ was arbitrary, so $\lim_{x \to \infty} \frac{\pi(x) \ln x}{x} = 1$, so $\pi(x) \sim \frac{x}{\ln x}$ as required. \qed

\pfbwd WTS PNT implies $p_n \sim n \ln n$.

Let $0 < \epsilon \ll 1$ be arbitrary. Since $\pi(x) \sim \frac{x}{\ln x}$, there exists some $X_\epsilon$ such that \begin{equation} \label{eq1}
x \geq X_\epsilon \implies (1-\epsilon) x \leq \pi(x) \ln x \leq (1 + \epsilon)x
\end{equation}

Take $N = N_\epsilon = \pi(X_\epsilon) + 1$, then for all $n \geq N$, we have that $p_n \geq p_{N + 1} \geq X_\epsilon$ so \eqref{eq1} holds:
\begin{equation*}
(1-\epsilon) p_n \leq n \ln n \leq (1 + \epsilon) p_n
\end{equation*}
which implies
\begin{equation*}
1 - \epsilon \leq \frac{n \ln n}{p_n} \leq 1 + \epsilon
\end{equation*}

$\epsilon$ was arbitrary, so $\lim_{n \to \infty} \frac{n \ln n}{p_n} = 1 \implies p_n \sim n \ln n$ as required. \qed

\bigskip\hrule

\vspace{0.5cm}

\newpage
\Large
\lhead{Problem 01.159}

\noindent
(\done) 01.159 Question. \\
Let $\Omega$ be a set of $n$ elements. Let $C_1, \ldots, C_m$ be an Oddtown system in $\Omega$ and let $\bfv_1, \ldots, \bfv_m$ be the corresponding incidence vectors. Prove that the $\bfv_i$ are linearly independent over $\bbq$. (Note that the Oddtown Theorem follows from this)

\medskip\noindent
\emph{Sources and collaborations.}\\
None.

\medskip\noindent
\emph{Proof.}\\
WTS Oddtown incidence vectors $\bfv_1, \ldots, \bfv_m \in \bbq^n$ are linearly independent.

Suppose there exists $(p_i/q_i)_{i \in [m]} \in \bbq$ such that \begin{equation*}
\sum_{i=1}^{m}   \alpha_i \bfv_i = \mathbf{0}
\end{equation*}
Let $a_i = (p_i/q_i) \lcm(q_1, \ldots, q_m) \in \bbz$ then
\begin{equation} \label{eq2}
\sum_{i=1}^{m} a_i \bfv_i = \mathbf{0}
\end{equation}
WLOG, $\gcd(a_1, \ldots, a_m) = 1$ (If not, divide them all by $\gcd(a_1, \ldots, a_m)$).

Define the symmetric bilinear function $\iprod{\cdot}{\cdot}: \bbq^n \times \bbq^n \to \bbz, \bfu \times \bfv \mapsto \sum_{i=1}^{m} \bfu^{(i)} \bfv^{(i)}$ where $\bfu^{(i)}$ denotes the $i$-th index of $\bfu$. Then the Oddtown conditions mean that $\iprod{\bfv_i}{\bfv_j}$ is even if $i \neq j$ and odd if $i = j$.

For any $i \in [m]$, from \eqref{eq2}, we get $\iprod{\lhs}{\bfv_i} = \iprod{\rhs}{\bfv_i}$ which implies:
\begin{equation*}
a_i \cdot (odd) + \sum (even) = 0
\end{equation*}
where $(odd), (even)$ denote some odd and even integers respectively. It then follows that $a_i$ is even.

This holds for all $i \in [m]$, so $(a_i)_{i \in [m]}$ have 2 as a common divisor, \contra
\qed
\bigskip\hrule

\vspace{0.5cm}




\end{document}