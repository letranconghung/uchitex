\lecture{5}{06 Apr 2023}{Span, Linear Independence, Basis}

\begin{recall}
    Linear Combination: Let \(V = \:\text{\(\bbk\)-vector space}\: \) with \(v_1, v_2, \dots, v_r \in V\) then \[
        \bbk\langle v_1, v_2, \dots, v_r \rangle \coloneqq \{w \in W \mid = w = a_1v_1 + \dots + a_rv_r; a_i \in \bbk\} \subseteq V (\:\text{is a subspace of}\: V)
    \]
\end{recall}

\begin{definition} {Span}
    \(\{v_1, v_2, \dots, v_r\}\) span \(V\) if \[
        \bbk\langle v_1, v_2, \dots, v_r \rangle = V
    \]
    i.e. equality is achieved: every vector in \(V\) can be written as linear combinations of \(\{v_1, v_2, \dots, v_r\}\)
\end{definition}

Connecting to the previous lecture, let \(\psi: \bbk^r \to V\) then \(\psi \in \Hom_\bbk(\bbk^r, V) \xrightarrow{\sim} V^{\oplus r}\), i.e. \(\psi\) corresponds to \((v_1, v_2, \dots, v_r)\) in \(V\).

In particular, \((v_1, v_2, \dots, v_r) \in V^{\oplus r}\) determines the map:
\begin{align*}
    \psi: (1, 0, \dots, 0) \in \bbk^r                & \to v_1                                             \\
    (0, 1, \dots, 0) \in \bbk^r                      & \to v_2                                             \\
                                                     & \vdots                                              \\
    (0, 0, \dots, 1) \in \bbk^r                      & \to v_r                                             \\
    (\alpha_1, \alpha_2, \dots, \alpha_r) \in \bbk^r & \to \alpha_1v_1 + \alpha_2v_2 + \dots + \alpha_rv_r
\end{align*}

\begin{lemma}
    \begin{enumerate}
        \item Let \(\psi: \bbk^r \to V\) be a linear transformation determined by \(v_1, v_2, \dots, v_r \in V\), i.e. \(\psi(\alpha_1, \alpha_2, \dots, \alpha_r) \coloneqq \sum_{i=1}^r \alpha_iv_i\), then \[
                  \im(\psi) = \bbk\langle v_1, v_2, \dots, v_r \rangle
              \] is a subspace of \(V\)
        \item \(\{v_1, v_2, \dots, v_r\}\) span \(V \Leftrightarrow \psi \:\text{is surjective}\: \)

              i.e. a surjection \(\bbk^r \to V\) corresponds to \(r\) vectors \(v_1, v_2, \dots, v_r \in V\) that span \(V\)
    \end{enumerate}
\end{lemma}

\begin{remark}
    \(V\) is finite dimensional when \(\exists\) surjection \(\bbk^d \to V\)

    \(\Leftrightarrow \exists d\) vectors \(v_1, v_2, \dots, v_r\) that span \(V\).

    Recall: \(\dim V = \min\{r \in \bbz_{\geq 0} \st \exists \:\text{surjective}\: \bbk^r \to V\}\).

    Next, what does it mean for \(\psi\) to be injective?
\end{remark}

\begin{definition} {Linear Independence}
    \(v_1, v_2, \dots, v_r \in V\) are \textbf{linearly independent} if \[
        a_1v_1 + a_2v_2 + \cdots + a_rv_r = 0; a_i \in \bbk \implies a_1 = a_2 = \cdots = a_r = 0
    \]

    i.e. there doesn't exist non-trivial relations between the vectors.
\end{definition}

\begin{example}
    In \(\bbr^2\), (0, 1) and (0, 2) are not linearly independent because \[
        (-2) (0, 1) + (0, 2) = (0, 0)
    \]

    But (0, 1) and (1,0) are linearly independent.
\end{example}

Consequentially, they are \textbf{linearly dependent} otherwise, i.e. \[
    \exists a_i \:\text{not all }\: 0 \st \sum a_iv_i = 0
\]
\begin{lemma} \label{lemma:5.2}
    Given \(\psi: \bbk^r \to V\) corresponds to \(v_1, v_2, \dots, v_r\) then \(v_1, v_2, \dots, v_r\) are linearly independent if and only if \(\psi\) is injective
\end{lemma}


In order to prove the lemma above, we shall make use of a more convenient test for whether a map \(\phi: \bbk^r \to V\) is injective.

\begin{lemma} \label{lemma:5.3}
    Let \(\phi: V \to W\) be a linear transformation then \(\phi\) is injective if and only if \[
        \ker(\phi) = \{0\} \subseteq V
    \]
\end{lemma}

\begin{proof} {Lemma \ref{lemma:5.3}}
    \pffwd We assume that \(\phi\) is injective, want to show that \(\ker(\phi) = \{0\}\).

    We know that \(\phi(0) = 0 \implies 0 \in \ker(\phi) \) but since \(\phi\) is injective, \(\nexists v \neq 0 \in V \st \phi(v) = 0\).

    It follows that \(\ker(\phi) = {0}\)

    \pfbwd We want to show that \(x, y \in V \st \phi(x) = \phi(y) \implies x = y\)

    Since \(\phi(x-y)  = \phi(x + (-y)) = \phi(x) - \phi(y) = 0\), combined with \(\ker(\phi) = {0}\) \[
        \implies x - y = 0 \implies x =y
    \]
\end{proof}

\begin{proof} {Lemma \ref{lemma:5.2}}
    Applying Lemma \ref{lemma:5.3}, we want to show: \(\ker(\phi) = {0}\) iff \(v_1, v_2, \dots, v_r\) are linearly independent.

    \pffwd Suppose \(\ker(\phi) = \{0\}\) then want to show \[
        a_1v_1 + a_2v_2 + \dots + a_rv_r = 0 \implies a_i = 0 \forall i
    \]

    But \(LHS = \phi((a_1, a_2, \dots, a_r))  \implies (a_1, a_2, \dots, a_r) \in \ker(\phi) \implies (a_1, a_2, \dots, a_r)  = 0\).

    Therefore \(a_i = 0 \forall i\).

    \pfbwd Suppose that \(v_1, v_2, \dots, v_r\) are linearly independent.

    Then for \(v \in \ker(\phi) \implies \phi(v) = 0\), with \(v = (a_1, a_2, \dots, a_r)\)
    \begin{align*}
        \implies 0 & =\phi(v)                           \\
                   & = \phi((a_1, a_2, \dots, a_r))     \\
                   & = a_1v_1 + a_2v_2 + \dots + a_rv_r
    \end{align*}
    But since \(v_1, v_2, \dots, v_r\) are linearly independent
    \[
        \implies a_i = 0 \forall i \implies v = 0 \implies \ker(\phi) = {0}
    \]
\end{proof}

\begin{corollary}
    If \(V\) has dimension \(d\) over \(\bbk\) then there exists isomorphic \(\phi: \bbk^d \xrightarrow{\sim} V\)

    i.e. \(\phi\) is a bijective linear transformation
\end{corollary}

\begin{proof}{Corollary}
    Since \(d = \dim V\), by definition there exists surjective linear transformation \(\pi: \bbk^d \twoheadrightarrow V\)

    We then claim that \(\pi\) is also injective.

    Proving by contradiction, we suppose that \(\pi\) is not injective.
    
    let \(v_1, v_2, \dots, v_d\) be the \(d\) vectors that correspond to \(\pi\), i.e. \[
    \pi((a_1, a_2, \dots, a_d)) = a_1v_1 + \dots + a_dv_d
    \]

    By Lemma \ref{lemma:5.2}, \(\pi\) being not injective implies that \(v_1, v_2, \dots, v_d\) are linearly dependent.

    i.e. there exists \(b_1, b_2, \dots, b_d \in \bbk\) not identically 0 s.t. \[
    b_1v_1 + b_2v_2 + \dots + b_dv_d = 0
    \]

    WLOG, assume \(b_1 \neq 0\).
    \begin{align*}
        \implies b_1v_1 &= -(b_2v_2 \dots b_dv_d) \\
        \implies v_1 &= -b^{-1}(b_2v_2 \dots b_dv_d) (\exists b^{-1} \because b_1 \neq 0) \\
        &= c_2v_2 + c_3v_3 + \dots + c_dv_d
    \end{align*}

    We already know that since \(\pi\) is surjective, thus \(v_1, v_2, \dots, v_d\) span \(V\). However, the above equality implies that \(v_2, \dots, v_d\) already span \(V\)!
    
    It follows that there must exist a surjective linear transformation \(\pi': \bbk^{d-1} \twoheadrightarrow V\)

    \contra, since \(d = \min\{r \mid \exists \:\text{surjective}\: \pi^r: \bbk^r \twoheadrightarrow V\}\)

    Therefore \(\pi\) is injective. It is already surjective, and therefore bijective, making it an isomorphism.
\end{proof}

\begin{recall}
    \(\psi: \bbk^d \to V\) as determined by \(v_1, v_2, \dots, v_d\) is 
    \begin{enumerate}
        \item \textbf{injective} when \(v_1, v_2, \dots, v_d\) are linearly independent
        \item \textbf{surjective} when \(v_1, v_2, \dots, v_d\) span \(V\)
    \end{enumerate}

    This naturally leads to our next definition.
\end{recall}
\begin{definition} {Basis}
    \(\{v_1, v_2, \dots, v_r\}\) is called a \textbf{basis} of \(V\) if they span \(V\) and are linearly independent,
    
    i.e. \(\psi_{(v_1, v_2, \dots, v_r)}: \bbk^r \to V\) is an isomorphism.
\end{definition}

\begin{corollary}
    \(
    \dim_\bbk V = d \Leftrightarrow \exists \:\text{basis}\: \{v_1, v_2, \dots, v_d\} \:\text{for}\: V
    \)
\end{corollary}
\begin{corollary}
    If \(\{v_1, v_2, \dots, v_d\}\) and \(\{w_1, w_2, \dots, v_{d'}\}\) are basis for \(V\) then \(d = d'\).
\end{corollary}