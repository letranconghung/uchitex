\lecture{4}{04 Apr 2023}{Linear Transformation, Homomorphism, Kernel, Image}
\subsection{Vector Subspace}
\begin{definition} {Vector Subspace}
    Let \(V\) be a \(\bbk\)-vector space. A \textbf{subspace} (or \textbf{sub-vector space}) of \(V\) is a subset \(W \subseteq V\) \st \(W\) is itself a \(\bbk\)-vector space under addition and scaling induced from \(V\). A priori, we know that \[
        + : W \times W \to V, \cdot: W \times W \to V
    \]
    but this subspace requirement implies that
    \[
        \forall x, y \in W, x+y \in W
    \]
    \[
        \forall \alpha \in \bbk, x \in W, \alpha \cdot x \in W
    \]
    In other words, the subspace is closed under addition and scaling.
\end{definition}

\begin{example}
    Take \(\bbk = \bbr, V = \bbr^2\), with ordinary addition and scaling.

    Consider the subset represented by line \(y = 1\).
    \begin{figure} [H]
    \begin{center}
        \begin{tikzpicture}[scale=0.5]
            \begin{axis} [
                axis lines = middle,
                xmin = -10,
                xmax = 10,
                ymin = -10,
                ymax = 10,
                xticklabels = {,,},
                yticklabels = {,,},
            ]
                \addplot[domain = -10:10]{5};
            \end{axis}
        \end{tikzpicture}
    \end{center}
\end{figure}


    This is not a subspace because there exists no 0 element. This kinda implies that any subspace of \(\bbr^2\) must pass through the origin \((0, 0)\).

    Consider another instance, this time the following ray:
    \begin{figure} [H]
    \begin{center}
        \begin{tikzpicture}[scale=0.5]
            \begin{axis} [
                axis lines = middle,
                xmin = -10,
                xmax = 10,
                ymin = -10,
                ymax = 10,
                xticklabels = {,,},
                yticklabels = {,,},
            ]
                \addplot[domain = 0:10]{-x};
            \end{axis}
        \end{tikzpicture}
    \end{center}
\end{figure}
    This is also not a subspace, since there's no additive inverse. Therefore a subspace shall look something like this:
    \begin{figure} [H]
    \begin{center}
        \begin{tikzpicture}[scale=0.5]
            \begin{axis} [
                axis lines = middle,
                xmin = -10,
                xmax = 10,
                ymin = -10,
                ymax = 10,
                xticklabels = {,,},
                yticklabels = {,,},
            ]
                \addplot[domain = -10:10]{-x};
            \end{axis}
        \end{tikzpicture}
    \end{center}
\end{figure}
\end{example}
\subsection{Mapping}
\begin{motivation}
    A map from sets to sets can be anything. e.g. \(x:\bbz \mapsto x^2:\bbz\) doesn't preserve the ``group'' structure \((x+y) ^2 \neq x^2 + y^2\) most of the time.
\end{motivation}

\begin{definition} {Group Homomorphism}
    Let \(A, B\) be Abelian groups. Map \(\psi: A \to B\) is called a \textbf{group homomorphism} if: \[
        \psi (x + y) = \psi (x) + \psi (y)
    \]
\end{definition}

Then \(x:\bbz \mapsto x^2:\bbz\) is not a group homomorphism, but \(x: \bbz \mapsto nx: \bbz\) for fixed \(n\) is a group homomorphism.

Here, a natural question arises: If given 2 vector spaces, what maps are allowed between them? What structures do we have to preserve?

\begin{definition} {Linear Transformation}
    Let \(V, W\) be \(\bbk\)-vector spaces. Then a \textbf{vector space homomorphism} is also called a \textbf{linear transformation}, a map \(\psi: V \to W\) \st
    \begin{enumerate}
        \item \(\psi(v_1 + v_2) = \psi(v_1) + \psi(v_2) \forall v_1, v_2 \in V\)
        \item \(\psi(\alpha \cdot v) = \alpha \cdot \psi(v) \forall \alpha \in \bbk, v \in V\)
    \end{enumerate}

    Denote \(\mathbf{Hom_\bbk(V, W)}\) as the set of all linear transformations \(V \to W\).
\end{definition}

\begin{example}
    \(\bbk = \bbr, V = W = \bbr\)

    \(\Hom_\bbr(V, W) = \{\psi: \bbr \to \bbr \mid (1), (2) \:\text{are satisfied}\: \} \)

    We claim that \(\psi(1)\) uniquely determines the map \(\psi\), because \[
        \psi(\alpha) = \alpha \cdot \psi(1)
    \]

    Essentially, there exists a bijection between \(\Hom_\bbr(V, W)\) and \(\bbr\):
    \begin{align*}
        \Hom_\bbr(V, W)              & \to \bbr         \\
        \psi                                 & \to \psi(1)      \\
        (\psi_\beta: x \mapsto x \cdot\beta) & \leftarrow \beta
    \end{align*}
\end{example}

\begin{example}
    \(\bbk = \bbr, V = \bbr, W \:\text{is any vector space (over \(\bbr\))}\: \)

    We, similarly, claim that there is a bijection between \(\Hom_\bbr(V, W) \:\text{and}\: \bbr\). With the same reasoning, \(\psi\) is determined by \(\psi(1)\), though this time \(\psi(1) \in W\).

    \begin{align*}
        \Hom_\bbr(V, W)           & \to W             \\
        \psi                              & \to \psi(1) \in W \\
        (\psi_\beta: x \mapsto x \cdot w) & \leftarrow w
    \end{align*}
\end{example}

\begin{example}
    As a sub-example of the example above, consider \(W = \bbr^2\):
    \begin{figure} [H]
    \begin{center}
        \begin{tikzpicture}[scale=0.8]
            \begin{axis} [
                axis lines = middle,
                xmin = -10,
                xmax = 10,
                ymin = -10,
                ymax = 10,
                xticklabels = {,,},
                yticklabels = {,,},
            ]
                \draw[->] (0, 0) -- (4, 5);
                \draw[dotted] (-8, -10) -- (8, 10);
            \end{axis}
        \end{tikzpicture}
    \end{center}
\end{figure}
    Then if \(\psi(1) = (4, 5)\) as above (and \(\psi(0) = (0, 0)\) implicit), then \(\psi\) would map the rest of \(V = \bbr\) onto the dotted line above.

    An interesting point to note is that if \(\psi(1) = (0, 0)\), then the entire real line would get sent (and compressed) to \((0, 0)\). \(\psi_{(0, 0)}\) therefore contracts \(\bbr\) into one point (the origin (0, 0)) while others output a subspace of \(\bbr^2\).
\end{example}

\begin{example}
    \(\bbk = \bbr, V = \bbr^2, W = \:\text{any \(\bbr\)-vector space}\: \)

    We claim that there exists a bijection between \(\Hom_\bbr(\bbr^2, W)\) and \(W \oplus W\); as each \(\psi\) is determined by \(\psi((1, 0)) \:\text{and}\: \psi((0, 1))\).

    The notation \(\oplus\) is defined as: If \(V, W\) are \(\bbk\)-vector spaces then \[
        V \oplus W = \{(v, w) \mid v \in V, w \in W\}
    \]
    e.g. \(\bbr^2 = \bbr \oplus \bbr\)

    Then \(V \oplus W\) would also be a \(\bbk\)-vector space with operations \(+, \cdot\) defined intuitively:
    \begin{align*}
        (v_1, w_1) + (v_2, w_2) & = (v_1+v_2, w_1 + w_2)             \\
        \alpha \cdot (v, w)     & = (\alpha \cdot v, \alpha \cdot w)
    \end{align*}

    Back to the example, \(\forall v = (x, y) \in V, v = x (1, 0) + y(0, 1)\), therefore \[
        \psi(v) = \psi((x, y)) = x \cdot\psi((1, 0)) + y\cdot\psi((0, 1))
    \]
    \(\psi\) is therefore uniquely defined by \(\psi((1, 0)) \:\text{and}\: \psi((0, 1))\).
\end{example}

\begin{example}
    \(\bbk = \bbr, V = \bbr^m, W = \:\text{any \(\bbr\)-vector space}\: \)

    Think about \(W = \bbr^n\) with similar reasoning.

    \textbf{Hint: } We want to show there exists a bijection between \(\Hom_\bbr(\bbr^m, \bbr^n) \) and \(\bbr^{m\cdot n}\), but this is often rewritten as \(\bbm_{m\times n}(\bbr)\)
\end{example}

\subsection{Isomorphism, Kernel, Image}
Every linear transformation is just a map, and we can therefore question if it is injective, surjective or bijective. In all cases, these concepts simply deal with the sets (vector spaces) as simply sets.

\begin{definition} {Isomorphism}
    A \(\bbk\)-linear transformation \(\psi: V \to W\) is an \textbf{isomorphism} if it is bijective.
\end{definition}

\begin{definition} {Kernel, Image}
    Let \(\psi: V \to W\) be a linear transformation over \(\bbk\). Then: \begin{enumerate}
        \item \textbf{Kernel}: \(\ker(\psi) \coloneqq \{v \in V \mid \psi(v) = 0\} \subseteq V\)
        \item \textbf{Image}: \(\im(\psi) \coloneqq \{w \in W \mid \exists v \in V \st \psi(v) = w\}\)
    \end{enumerate}
\end{definition}

\begin{lemma}
    \hfill
    \begin{enumerate}
        \item \(\ker(\psi)\) is a \(\bbk\)-vector subspace of \(V\)
        \item \(\im(\psi)\) is a \(\bbk\)-vector subspace of \(W\)
    \end{enumerate}
\end{lemma}

\begin{proof} {Lemma}    
    We want to show that if \(x, y \in \ker(\psi)\) then \(x + y \in \ker(\psi)\).
    \[
        \begin{split}  
            \psi(x + y) &= \psi(x) + \psi(y) (\:\text{since \(\psi\) is a linear transformation}\: ) \\
            & = 0 + 0 \\
            &= 0
        \end{split}
    \]

    Therefore \(x + y \in \ker(\psi)\)

    Furthermore, \(\forall \alpha \in \bbk, x \in \ker(\psi)\) then \[
    \psi(\alpha, x) = \alpha \cdot \psi(x) = \alpha \cdot 0 = 0 \implies \alpha \cdot x \in \ker(\psi)
    \]

    Therefore \(\ker(\psi)\) is a subspace.

    Similarly, \(\im(\psi)\) is a subspace.
\end{proof}

\begin{definition} {Finite Dimensional, Dimension}
    \begin{enumerate}
        \item Let \(V\) be a \(\bbk\)-vector space. \(V\) is called \textbf{finite dimensional} if there exists a surjective linear transformation \(\bbk^r \to V\) where \(r \in \bbz_{\geq 0}\). As a consequence, \(\bbk^r\) is also finite dimensional, with an identity mapping.
        \item If $V$ is finite dimensional then \textbf{dimension} of $V$ is defined as \[\dim V \coloneqq \min\{k \in \bbz_{\geq 0} \mid \exists \:\text{linear transformation}\: \bbk^r \to V\}\]
    \end{enumerate}
\end{definition}