\documentclass[a4paper, 10pt]{article}
%%%%%%%%%%%%%%%%%%%%%%%%%%%%%%%%%%%%%%%%%%%%%%%%%%%%%%%%%%%%%%%%%%%%%%%%%%%%%%%
%                                Basic Packages                               %
%%%%%%%%%%%%%%%%%%%%%%%%%%%%%%%%%%%%%%%%%%%%%%%%%%%%%%%%%%%%%%%%%%%%%%%%%%%%%%%

% Gives us multiple colors.
\usepackage[usenames,dvipsnames,pdftex]{xcolor}
% Lets us style link colors.
\usepackage{hyperref}
% Lets us import images and graphics.
\usepackage{graphicx}
% Lets us use figures in floating environments.
\usepackage{float}
% Lets us create multiple columns.
\usepackage{multicol}
% Gives us better math syntax.
\usepackage{amsmath,amsfonts,mathtools,amsthm,amssymb}
% Lets us strikethrough text.
\usepackage{cancel}
% Lets us edit the caption of a figure.
\usepackage{caption}
% Lets us import pdf directly in our tex code.
\usepackage{pdfpages}
% Lets us do algorithm stuff.
\usepackage[ruled,vlined,linesnumbered]{algorithm2e}
% Use a smiley face for our qed symbol.
\usepackage{tikzsymbols}
% \usepackage{fullpage} %%smaller margins
\usepackage[shortlabels]{enumitem}

\setlist[enumerate]{font={\bfseries}} % global settings, for all lists

\usepackage{setspace}
\usepackage[margin=1in, headsep=12pt]{geometry}
\usepackage{wrapfig}
\usepackage{listings}
\usepackage{parskip}

\definecolor{codegreen}{rgb}{0,0.6,0}
\definecolor{codegray}{rgb}{0.5,0.5,0.5}
\definecolor{codepurple}{rgb}{0.58,0,0.82}
\definecolor{backcolour}{rgb}{0.95,0.95,0.95}

\lstdefinestyle{mystyle}{
    backgroundcolor=\color{backcolour},   
    commentstyle=\color{codegreen},
    keywordstyle=\color{magenta},
    numberstyle=\tiny\color{codegray},
    stringstyle=\color{codepurple},
    basicstyle=\ttfamily\footnotesize,
    breakatwhitespace=false,         
    breaklines=true,                 
    captionpos=b,                    
    keepspaces=true,                 
    numbers=left,                    
    numbersep=5pt,                  
    showspaces=false,                
    showstringspaces=false,
    showtabs=false,                  
    tabsize=2,
    numbers=none
}

\lstset{style=mystyle}
\def\class{article}


%%%%%%%%%%%%%%%%%%%%%%%%%%%%%%%%%%%%%%%%%%%%%%%%%%%%%%%%%%%%%%%%%%%%%%%%%%%%%%%
%                                Basic Settings                               %
%%%%%%%%%%%%%%%%%%%%%%%%%%%%%%%%%%%%%%%%%%%%%%%%%%%%%%%%%%%%%%%%%%%%%%%%%%%%%%%

%%%%%%%%%%%%%
%  Symbols  %
%%%%%%%%%%%%%

\let\implies\Rightarrow
\let\impliedby\Leftarrow
\let\iff\Leftrightarrow
\let\epsilon\varepsilon
%%%%%%%%%%%%
%  Tables  %
%%%%%%%%%%%%

\setlength{\tabcolsep}{5pt}
\renewcommand\arraystretch{1.5}

%%%%%%%%%%%%%%
%  SI Unitx  %
%%%%%%%%%%%%%%

\usepackage{siunitx}
\sisetup{locale = FR}

%%%%%%%%%%
%  TikZ  %
%%%%%%%%%%

\usepackage[framemethod=TikZ]{mdframed}
\usepackage{tikz}
\usepackage{tikz-cd}
\usepackage{tikzsymbols}

\usetikzlibrary{intersections, angles, quotes, calc, positioning}
\usetikzlibrary{arrows.meta}

\tikzset{
    force/.style={thick, {Circle[length=2pt]}-stealth, shorten <=-1pt}
}

%%%%%%%%%%%%%%%
%  PGF Plots  %
%%%%%%%%%%%%%%%

\usepackage{pgfplots}
\pgfplotsset{width=10cm, compat=newest}

%%%%%%%%%%%%%%%%%%%%%%%
%  Center Title Page  %
%%%%%%%%%%%%%%%%%%%%%%%

\usepackage{titling}
\renewcommand\maketitlehooka{\null\mbox{}\vfill}
\renewcommand\maketitlehookd{\vfill\null}

%%%%%%%%%%%%%%%%%%%%%%%%%%%%%%%%%%%%%%%%%%%%%%%%%%%%%%%
%  Create a grey background in the middle of the PDF  %
%%%%%%%%%%%%%%%%%%%%%%%%%%%%%%%%%%%%%%%%%%%%%%%%%%%%%%%

\usepackage{eso-pic}
\newcommand\definegraybackground{
    \definecolor{reallylightgray}{HTML}{FAFAFA}
    \AddToShipoutPicture{
        \ifthenelse{\isodd{\thepage}}{
            \AtPageLowerLeft{
                \put(\LenToUnit{\dimexpr\paperwidth-222pt},0){
                    \color{reallylightgray}\rule{222pt}{297mm}
                }
            }
        }
        {
            \AtPageLowerLeft{
                \color{reallylightgray}\rule{222pt}{297mm}
            }
        }
    }
}

%%%%%%%%%%%%%%%%%%%%%%%%
%  Modify Links Color  %
%%%%%%%%%%%%%%%%%%%%%%%%

\hypersetup{
    % Enable highlighting links.
    colorlinks,
    % Change the color of links to blue.
    urlcolor=blue,
    % Change the color of citations to black.
    citecolor={black},
    % Change the color of url's to blue with some black.
    linkcolor={blue!80!black}
}

%%%%%%%%%%%%%%%%%%
% Fix WrapFigure %
%%%%%%%%%%%%%%%%%%

\newcommand{\wrapfill}{\par\ifnum\value{WF@wrappedlines}>0
        \parskip=0pt
        \addtocounter{WF@wrappedlines}{-1}%
        \null\vspace{\arabic{WF@wrappedlines}\baselineskip}%
        \WFclear
    \fi}

%%%%%%%%%%%%%%%%%
% Multi Columns %
%%%%%%%%%%%%%%%%%

\let\multicolmulticols\multicols
\let\endmulticolmulticols\endmulticols

\RenewDocumentEnvironment{multicols}{mO{}}
{%
    \ifnum#1=1
        #2%
    \else % More than 1 column
        \multicolmulticols{#1}[#2]
    \fi
}
{%
    \ifnum#1=1
    \else % More than 1 column
        \endmulticolmulticols
    \fi
}

\newlength{\thickarrayrulewidth}
\setlength{\thickarrayrulewidth}{5\arrayrulewidth}


%%%%%%%%%%%%%%%%%%%%%%%%%%%%%%%%%%%%%%%%%%%%%%%%%%%%%%%%%%%%%%%%%%%%%%%%%%%%%%%
%                           School Specific Commands                          %
%%%%%%%%%%%%%%%%%%%%%%%%%%%%%%%%%%%%%%%%%%%%%%%%%%%%%%%%%%%%%%%%%%%%%%%%%%%%%%%

%%%%%%%%%%%%%%%%%%%%%%%%%%%
%  Initiate New Counters  %
%%%%%%%%%%%%%%%%%%%%%%%%%%%

\newcounter{lecturecounter}

%%%%%%%%%%%%%%%%%%%%%%%%%%
%  Helpful New Commands  %
%%%%%%%%%%%%%%%%%%%%%%%%%%

\makeatletter

\newcommand\resetcounters{
    % Reset the counters for subsection, subsubsection and the definition
    % all the custom environments.
    \setcounter{subsection}{0}
    \setcounter{subsubsection}{0}
    \setcounter{definition0}{0}
    \setcounter{paragraph}{0}
    \setcounter{theorem}{0}
    \setcounter{claim}{0}
    \setcounter{corollary}{0}
    \setcounter{proposition}{0}
    \setcounter{lemma}{0}
    \setcounter{exercise}{0}
    \setcounter{problem}{0}
    
    \setcounter{subparagraph}{0}
    % \@ifclasswith\class{nocolor}{
    %     \setcounter{definition}{0}
    % }{}
}

%%%%%%%%%%%%%%%%%%%%%
%  Lecture Command  %
%%%%%%%%%%%%%%%%%%%%%

\usepackage{xifthen}

% EXAMPLE:
% 1. \lecture{Oct 17 2022 Mon (08:46:48)}{Lecture Title}
% 2. \lecture[4]{Oct 17 2022 Mon (08:46:48)}{Lecture Title}
% 3. \lecture{Oct 17 2022 Mon (08:46:48)}{}
% 4. \lecture[4]{Oct 17 2022 Mon (08:46:48)}{}
% Parameters:
% 1. (Optional) lecture number.
% 2. Time and date of lecture.
% 3. Lecture Title.
\def\@lecture{}
\def\@lectitle{}
\def\@leccount{}
\newcommand\lecture[3]{
    \newpage

    % Check if user passed the lecture title or not.
    \def\@leccount{Lecture #1}
    \ifthenelse{\isempty{#3}}{
        \def\@lecture{Lecture #1}
        \def\@lectitle{Lecture #1}
    }{
        \def\@lecture{Lecture #1: #3}
        \def\@lectitle{#3}
    }

    \setcounter{section}{#1}
    \renewcommand\thesubsection{#1.\arabic{subsection}}
    
    \phantomsection
    \addcontentsline{toc}{section}{\@lecture}
    \resetcounters

    \begin{mdframed}
        \begin{center}
            \Large \textbf{\@leccount}
            
            \vspace*{0.2cm}
            
            \large \@lectitle
            
            
            \vspace*{0.2cm}

            \normalsize #2
        \end{center}
    \end{mdframed}

}

%%%%%%%%%%%%%%%%%%%%
%  Import Figures  %
%%%%%%%%%%%%%%%%%%%%

\usepackage{import}
\pdfminorversion=7

% EXAMPLE:
% 1. \incfig{limit-graph}
% 2. \incfig[0.4]{limit-graph}
% Parameters:
% 1. The figure name. It should be located in figures/NAME.tex_pdf.
% 2. (Optional) The width of the figure. Example: 0.5, 0.35.
\newcommand\incfig[2][1]{%
    \def\svgwidth{#1\columnwidth}
    \import{./figures/}{#2.pdf_tex}
}

\begingroup\expandafter\expandafter\expandafter\endgroup
\expandafter\ifx\csname pdfsuppresswarningpagegroup\endcsname\relax
\else
    \pdfsuppresswarningpagegroup=1\relax
\fi

%%%%%%%%%%%%%%%%%
% Fancy Headers %
%%%%%%%%%%%%%%%%%

\usepackage{fancyhdr}

% Force a new page.
\newcommand\forcenewpage{\clearpage\mbox{~}\clearpage\newpage}

% This command makes it easier to manage my headers and footers.
\newcommand\createintro{
    % Use roman page numbers (e.g. i, v, vi, x, ...)
    \pagenumbering{roman}

    % Display the page style.
    \maketitle
    % Make the title pagestyle empty, meaning no fancy headers and footers.
    \thispagestyle{empty}
    % Create a newpage.
    \newpage

    % Input the intro.tex page if it exists.
    \IfFileExists{intro.tex}{ % If the intro.tex file exists.
        % Input the intro.tex file.
        \textbf{Course}: MATH 16300: Honors Calculus III

\textbf{Section}: 43

\textbf{Professor}: Minjae Park

\textbf{At}: The University of Chicago

\textbf{Quarter}: Spring 2023

\textbf{Course materials}: Calculus by Spivak (4th Edition), Calculus On Manifolds by Spivak

\vspace{1cm}
\textbf{Disclaimer}: This document will inevitably contain some mistakes, both simple typos and serious logical and mathematical errors. Take what you read with a grain of salt as it is made by an undergraduate student going through the learning process himself. If you do find any error, I would really appreciate it if you can let me know by email at \href{mailto:conghungletran@gmail.com}{conghungletran@gmail.com}.

        % Make the pagestyle fancy for the intro.tex page.
        \pagestyle{fancy}

        % Remove the line for the header.
        \renewcommand\headrulewidth{0pt}

        % Remove all header stuff.
        \fancyhead{}

        % Add stuff for the footer in the center.
        % \fancyfoot[C]{
        %   \textit{For more notes like this, visit
        %   \href{\linktootherpages}{\shortlinkname}}. \\
        %   \vspace{0.1cm}
        %   \hrule
        %   \vspace{0.1cm}
        %   \@author, \\
        %   \term: \academicyear, \\
        %   Last Update: \@date, \\
        %   \faculty
        % }

        \newpage
    }{ % If the intro.tex file doesn't exist.
        % Force a \newpageage.
        % \forcenewpage
        \newpage
    }

    % Remove the center stuff we did above, and replace it with just the page
    % number, which is still in roman numerals.
    \fancyfoot[C]{\thepage}
    % Add the table of contents.
    \tableofcontents
    % Force a new page.
    \newpage

    % Move the page numberings back to arabic, from roman numerals.
    \pagenumbering{arabic}
    % Set the page number to 1.
    \setcounter{page}{1}

    % Add the header line back.
    \renewcommand\headrulewidth{0.4pt}
    % In the top right, add the lecture title.
    \fancyhead[R]{\footnotesize \@lecture}
    % In the top left, add the author name.
    \fancyhead[L]{\footnotesize \@author}
    % In the bottom center, add the page.
    \fancyfoot[C]{\thepage}
    % Add a nice gray background in the middle of all the upcoming pages.
    % \definegraybackground
}

\makeatother


%%%%%%%%%%%%%%%%%%%%%%%%%%%%%%%%%%%%%%%%%%%%%%%%%%%%%%%%%%%%%%%%%%%%%%%%%%%%%%%
%                               Custom Commands                               %
%%%%%%%%%%%%%%%%%%%%%%%%%%%%%%%%%%%%%%%%%%%%%%%%%%%%%%%%%%%%%%%%%%%%%%%%%%%%%%%

%%%%%%%%%%%%
%  Circle  %
%%%%%%%%%%%%

\newcommand*\circled[1]{\tikz[baseline= (char.base)]{
        \node[shape=circle,draw,inner sep=1pt] (char) {#1};}
}

%%%%%%%%%%%%%%%%%%%
%  Todo Commands  %
%%%%%%%%%%%%%%%%%%%

% \usepackage{xargs}
% \usepackage[colorinlistoftodos]{todonotes}

% \makeatletter

% \@ifclasswith\class{working}{
%     \newcommandx\unsure[2][1=]{\todo[linecolor=red,backgroundcolor=red!25,bordercolor=red,#1]{#2}}
%     \newcommandx\change[2][1=]{\todo[linecolor=blue,backgroundcolor=blue!25,bordercolor=blue,#1]{#2}}
%     \newcommandx\info[2][1=]{\todo[linecolor=OliveGreen,backgroundcolor=OliveGreen!25,bordercolor=OliveGreen,#1]{#2}}
%     \newcommandx\improvement[2][1=]{\todo[linecolor=Plum,backgroundcolor=Plum!25,bordercolor=Plum,#1]{#2}}

%     \newcommand\listnotes{
%         \newpage
%         \listoftodos[Notes]
%     }
% }{
%     \newcommandx\unsure[2][1=]{}
%     \newcommandx\change[2][1=]{}
%     \newcommandx\info[2][1=]{}
%     \newcommandx\improvement[2][1=]{}

%     \newcommand\listnotes{}
% }

% \makeatother

%%%%%%%%%%%%%
%  Correct  %
%%%%%%%%%%%%%

% EXAMPLE:
% 1. \correct{INCORRECT}{CORRECT}
% Parameters:
% 1. The incorrect statement.
% 2. The correct statement.
\definecolor{correct}{HTML}{009900}
\newcommand\correct[2]{{\color{red}{#1 }}\ensuremath{\to}{\color{correct}{ #2}}}


%%%%%%%%%%%%%%%%%%%%%%%%%%%%%%%%%%%%%%%%%%%%%%%%%%%%%%%%%%%%%%%%%%%%%%%%%%%%%%%
%                                 Environments                                %
%%%%%%%%%%%%%%%%%%%%%%%%%%%%%%%%%%%%%%%%%%%%%%%%%%%%%%%%%%%%%%%%%%%%%%%%%%%%%%%

\usepackage{varwidth}
\usepackage{thmtools}
\usepackage[most,many,breakable]{tcolorbox}

\tcbuselibrary{theorems,skins,hooks}
\usetikzlibrary{arrows,calc,shadows.blur}

%%%%%%%%%%%%%%%%%%%
%  Define Colors  %
%%%%%%%%%%%%%%%%%%%

% color prototype
% \definecolor{color}{RGB}{45, 111, 177}

% ESSENTIALS: 
\definecolor{myred}{HTML}{c74540}
\definecolor{myblue}{HTML}{072b85}
\definecolor{mygreen}{HTML}{388c46}
\definecolor{myblack}{HTML}{000000}

\colorlet{definition_color}{myred}

\colorlet{theorem_color}{myblue}
\colorlet{lemma_color}{myblue}
\colorlet{prop_color}{myblue}
\colorlet{corollary_color}{myblue}
\colorlet{claim_color}{myblue}

\colorlet{proof_color}{myblack}
\colorlet{example_color}{myblack}
\colorlet{exercise_color}{myblack}

% MISCS: 
%%%%%%%%%%%%%%%%%%%%%%%%%%%%%%%%%%%%%%%%%%%%%%%%%%%%%%%%%
%  Create Environments Styles Based on Given Parameter  %
%%%%%%%%%%%%%%%%%%%%%%%%%%%%%%%%%%%%%%%%%%%%%%%%%%%%%%%%%

% \mdfsetup{skipabove=1em,skipbelow=0em}

%%%%%%%%%%%%%%%%%%%%%%
%  Helpful Commands  %
%%%%%%%%%%%%%%%%%%%%%%

% EXAMPLE:
% 1. \createnewtheoremstyle{thmdefinitionbox}{}{}
% 2. \createnewtheoremstyle{thmtheorembox}{}{}
% 3. \createnewtheoremstyle{thmproofbox}{qed=\qedsymbol}{
%       rightline=false, topline=false, bottomline=false
%    }
% Parameters:
% 1. Theorem name.
% 2. Any extra parameters to pass directly to declaretheoremstyle.
% 3. Any extra parameters to pass directly to mdframed.
\newcommand\createnewtheoremstyle[3]{
    \declaretheoremstyle[
        headfont=\bfseries\sffamily, bodyfont=\normalfont, #2,
        mdframed={
                #3,
            },
    ]{#1}
}

% EXAMPLE:
% 1. \createnewcoloredtheoremstyle{thmdefinitionbox}{definition}{}{}
% 2. \createnewcoloredtheoremstyle{thmexamplebox}{example}{}{
%       rightline=true, leftline=true, topline=true, bottomline=true
%     }
% 3. \createnewcoloredtheoremstyle{thmproofbox}{proof}{qed=\qedsymbol}{backgroundcolor=white}
% Parameters:
% 1. Theorem name.
% 2. Color of theorem.
% 3. Any extra parameters to pass directly to declaretheoremstyle.
% 4. Any extra parameters to pass directly to mdframed.

% change backgroundcolor to #2!5 if user wants a colored backdrop to theorem environments. It's a cool color theme, but there's too much going on in the page.
\newcommand\createnewcoloredtheoremstyle[4]{
    \declaretheoremstyle[
        headfont=\bfseries\sffamily\color{#2},
        bodyfont=\normalfont,
        headpunct=,
        headformat = \NAME~\NUMBER\NOTE \hfill\smallskip\linebreak,
        #3,
        mdframed={
                outerlinewidth=0.75pt,
                rightline=false,
                leftline=false,
                topline=false,
                bottomline=false,
                backgroundcolor=white,
                skipabove = 5pt,
                skipbelow = 0pt,
                linecolor=#2,
                innertopmargin = 0pt,
                innerbottommargin = 0pt,
                innerrightmargin = 4pt,
                innerleftmargin= 6pt,
                leftmargin = -6pt,
                #4,
            },
    ]{#1}
}



%%%%%%%%%%%%%%%%%%%%%%%%%%%%%%%%%%%
%  Create the Environment Styles  %
%%%%%%%%%%%%%%%%%%%%%%%%%%%%%%%%%%%

\makeatletter
\@ifclasswith\class{nocolor}{
    % Environments without color.

    % ESSENTIALS:
    \createnewtheoremstyle{thmdefinitionbox}{}{}
    \createnewtheoremstyle{thmtheorembox}{}{}
    \createnewtheoremstyle{thmproofbox}{qed=\qedsymbol}{}
    \createnewtheoremstyle{thmcorollarybox}{}{}
    \createnewtheoremstyle{thmlemmabox}{}{}
    \createnewtheoremstyle{thmclaimbox}{}{}
    \createnewtheoremstyle{thmexamplebox}{}{}

    % MISCS: 
    \createnewtheoremstyle{thmpropbox}{}{}
    \createnewtheoremstyle{thmexercisebox}{}{}
    \createnewtheoremstyle{thmexplanationbox}{}{}
    \createnewtheoremstyle{thmremarkbox}{}{}
    
    % STYLIZED MORE BELOW
    \createnewtheoremstyle{thmquestionbox}{}{}
    \createnewtheoremstyle{thmsolutionbox}{qed=\qedsymbol}{}
}{
    % Environments with color.

    % ESSENTIALS: definition, theorem, proof, corollary, lemma, claim, example
    \createnewcoloredtheoremstyle{thmdefinitionbox}{definition_color}{}{leftline=false}
    \createnewcoloredtheoremstyle{thmtheorembox}{theorem_color}{}{leftline=false}
    \createnewcoloredtheoremstyle{thmproofbox}{proof_color}{qed=\qedsymbol}{}
    \createnewcoloredtheoremstyle{thmcorollarybox}{corollary_color}{}{leftline=false}
    \createnewcoloredtheoremstyle{thmlemmabox}{lemma_color}{}{leftline=false}
    \createnewcoloredtheoremstyle{thmpropbox}{prop_color}{}{leftline=false}
    \createnewcoloredtheoremstyle{thmclaimbox}{claim_color}{}{leftline=false}
    \createnewcoloredtheoremstyle{thmexamplebox}{example_color}{}{}
    \createnewcoloredtheoremstyle{thmexplanationbox}{example_color}{qed=\qedsymbol}{}
    \createnewcoloredtheoremstyle{thmremarkbox}{theorem_color}{}{}

    \createnewcoloredtheoremstyle{thmmiscbox}{black}{}{}

    \createnewcoloredtheoremstyle{thmexercisebox}{exercise_color}{}{}
    \createnewcoloredtheoremstyle{thmproblembox}{theorem_color}{}{leftline=false}
    \createnewcoloredtheoremstyle{thmsolutionbox}{mygreen}{qed=\qedsymbol}{}
}
\makeatother

%%%%%%%%%%%%%%%%%%%%%%%%%%%%%
%  Create the Environments  %
%%%%%%%%%%%%%%%%%%%%%%%%%%%%%
\declaretheorem[numberwithin=section, style=thmdefinitionbox,     name=Definition]{definition}
\declaretheorem[numberwithin=section, style=thmtheorembox,     name=Theorem]{theorem}
\declaretheorem[numbered=no,          style=thmexamplebox,     name=Example]{example}
\declaretheorem[numberwithin=section, style=thmtheorembox,       name=Claim]{claim}
\declaretheorem[numberwithin=section, style=thmcorollarybox,   name=Corollary]{corollary}
\declaretheorem[numberwithin=section, style=thmpropbox,        name=Proposition]{proposition}
\declaretheorem[numberwithin=section, style=thmlemmabox,       name=Lemma]{lemma}
\declaretheorem[numberwithin=section, style=thmexercisebox,    name=Exercise]{exercise}
\declaretheorem[numbered=no,          style=thmproofbox,       name=Proof]{proof0}
\declaretheorem[numbered=no,          style=thmexplanationbox, name=Explanation]{explanation}
\declaretheorem[numbered=no,          style=thmsolutionbox,    name=Solution]{solution}
\declaretheorem[numberwithin=section,          style=thmproblembox,     name=Problem]{problem}
\declaretheorem[numbered=no,          style=thmmiscbox,    name=Intuition]{intuition}
\declaretheorem[numbered=no,          style=thmmiscbox,    name=Goal]{goal}
\declaretheorem[numbered=no,          style=thmmiscbox,    name=Recall]{recall}
\declaretheorem[numbered=no,          style=thmmiscbox,    name=Motivation]{motivation}
\declaretheorem[numbered=no,          style=thmmiscbox,    name=Remark]{remark}
\declaretheorem[numbered=no,          style=thmmiscbox,    name=Observe]{observe}
\declaretheorem[numbered=no,          style=thmmiscbox,    name=Question]{question}


%%%%%%%%%%%%%%%%%%%%%%%%%%%%
%  Edit Proof Environment  %
%%%%%%%%%%%%%%%%%%%%%%%%%%%%

\renewenvironment{proof}[2][\proofname]{
    % \vspace{-12pt}
    \begin{proof0} [#2]
        }{\end{proof0}}

\theoremstyle{definition}

\newtheorem*{notation}{Notation}
\newtheorem*{previouslyseen}{As previously seen}
\newtheorem*{property}{Property}
% \newtheorem*{intuition}{Intuition}
% \newtheorem*{goal}{Goal}
% \newtheorem*{recall}{Recall}
% \newtheorem*{motivation}{Motivation}
% \newtheorem*{remark}{Remark}
% \newtheorem*{observe}{Observe}

\author{Hung C. Le Tran}


%%%% MATH SHORTHANDS %%%%
%% blackboard bold math capitals
\DeclareMathOperator*{\esssup}{ess\,sup}
\DeclareMathOperator*{\Hom}{Hom}
\newcommand{\bbf}{\mathbb{F}}
\newcommand{\bbn}{\mathbb{N}}
\newcommand{\bbq}{\mathbb{Q}}
\newcommand{\bbr}{\mathbb{R}}
\newcommand{\bbz}{\mathbb{Z}}
\newcommand{\bbc}{\mathbb{C}}
\newcommand{\bbk}{\mathbb{K}}
\newcommand{\bbm}{\mathbb{M}}
\newcommand{\bbp}{\mathbb{P}}
\newcommand{\bbe}{\mathbb{E}}

\newcommand{\bfw}{\mathbf{w}}
\newcommand{\bfx}{\mathbf{x}}
\newcommand{\bfX}{\mathbf{X}}
\newcommand{\bfy}{\mathbf{y}}
\newcommand{\bfyhat}{\mathbf{\hat{y}}}

\newcommand{\calb}{\mathcal{B}}
\newcommand{\calf}{\mathcal{F}}
\newcommand{\calt}{\mathcal{T}}
\newcommand{\call}{\mathcal{L}}
\renewcommand{\phi}{\varphi}

% Universal Math Shortcuts
\newcommand{\st}{\hspace*{2pt}\text{s.t.}\hspace*{2pt}}
\newcommand{\pffwd}{\hspace*{2pt}\fbox{\(\Rightarrow\)}\hspace*{10pt}}
\newcommand{\pfbwd}{\hspace*{2pt}\fbox{\(\Leftarrow\)}\hspace*{10pt}}
\newcommand{\contra}{\ensuremath{\Rightarrow\Leftarrow}}
\newcommand{\cvgn}{\xrightarrow{n \to \infty}}
\newcommand{\cvgj}{\xrightarrow{j \to \infty}}

\newcommand{\im}{\mathrm{im}}
\newcommand{\innerproduct}[2]{\langle #1, #2 \rangle}
\newcommand*{\conj}[1]{\overline{#1}}

% https://tex.stackexchange.com/questions/438612/space-between-exists-and-forall
% https://tex.stackexchange.com/questions/22798/nice-looking-empty-set
\let\oldforall\forall
\renewcommand{\forall}{\;\oldforall\; }
\let\oldexist\exists
\renewcommand{\exists}{\;\oldexist\; }
\newcommand\existu{\;\oldexist!\: }
\let\oldemptyset\emptyset
\let\emptyset\varnothing


\renewcommand{\_}[1]{\underline{#1}}
\DeclarePairedDelimiter{\abs}{\lvert}{\rvert}
\DeclarePairedDelimiter{\norm}{\lVert}{\rVert}
\DeclarePairedDelimiter\ceil{\lceil}{\rceil}
\DeclarePairedDelimiter\floor{\lfloor}{\rfloor}
\setlength\parindent{0pt}
\setlength{\headheight}{12.0pt}
\addtolength{\topmargin}{-12.0pt}


% Default skipping, change if you want more spacing
% \thinmuskip=3mu
% \medmuskip=4mu plus 2mu minus 4mu
% \thickmuskip=5mu plus 5mu

% \DeclareMathOperator{\ext}{ext}
% \DeclareMathOperator{\bridge}{bridge}
\title{Definitions}

\begin{document}

This document lists out definitions in Math that I couldn't possibly organize in my tiny brain. The definitions are expected to be built off of the previous ones.


\section{Binary Relations}
\begin{definition} {Binary Relation}
    A \textbf{binary relation} \(R\) over sets \(X, Y\) is a subset of \(X \times Y\). \((x, y) \in R \) is equivalent to \(xRy\). We say \textit{\(x\) is \(R-\)related to \(y\)}.

    In other words, \(R\) imposes some condition that \((x, y) \in R\) must satisfy to be included in the set.

    e.g. \(x \geq y\)
\end{definition}

\begin{definition} {Homogenous Relation/Endorelation}
    A \textbf{homogenous relation} is a binary relation from a set to itself, i.e. \(R: X \times X \to X\).

\end{definition}

\begin{definition} {Reflexive, Irreflexive, Symmetric, Antisymmetric, Asymmetric, Transitive, Connected, Strongly Connected}
    Let \(R\) be a homogenous relation over set \(X\). Then the following properties are defined as:

    \begin{enumerate}
        \item \(R\) is \textbf{reflexive} if \(\forall x \in X, xRx\) e.g. \(\geq\)
        \item \(R\) is \textbf{irreflexive} if \(\forall x \in X, \neg xRx\) e.g. \(>\)
        \item \(R\) is \textbf{symmetric} if \(\forall x, y \in X, xRy \Leftrightarrow yRx\) e.g. shares the same house
        \item \(R\) is \textbf{antisymmetric} if \(\forall x, y \in X, xRy \land yRx \implies x = y\) e.g. \(\geq\)
        \item \(R\) is \textbf{asymmetric} if \(\forall x, y \in X, xRy \implies \neg yRx\) e.g. \(>\)
        \item \(R\) is \textbf{transitive} if \(\forall x, y, z \in X, xRy \land yRz \implies xRz\) e.g. \(>, \geq\)
        \item \(R\) is \textbf{connected} if \(\forall x, y \in X, x \neq y \implies xRy \lor yRx\)
        \item \(R\) is \textbf{strongly connected} if \(\forall x, y \in X, xRy \lor yRx\)
    \end{enumerate}
\end{definition}
\begin{remark}
    In some sense, \textbf{asymmetry} is \textbf{antisymmetry + irreflexivity}; antisymmetry gets upgraded when there is irreflexivity.
\end{remark}

\begin{definition} {Partial Order}
    A \textbf{partial order} is a relation that is reflexive, antisymmetric and transitive.

    e.g. \(\geq\)
\end{definition}

\begin{definition} {Strict Partial Order}
    A \textbf{strict partial order} is a relation that is irreflexive, asymmetric and transitive.

    e.g. \(>\)
\end{definition}

\begin{definition} {Total Order}
    A \textbf{total order} is a relation that is reflexive, antisymmetric, transitive and connected.
\end{definition}

\begin{definition} {Strict Total Order}
    A \textbf{strict total order} is a relation that is irreflexive, asymmetric, transitive and connected.
\end{definition}

\begin{remark}
    In short, \textit{total} adds on the requirement that it must be \textit{connected}, while \textit{strict} changes \textit{reflexivity} into \textit{irreflexivity} (and since everyone has \textit{antisymmetry}, this changes into \textit{asymmetry} too)
\end{remark}
\begin{definition} {Equivalence Relation}
    An \textbf{equivalence relation} is a relation that is reflexive, symmetric and transitive.
\end{definition}

\section{Basic Algebraic Structures}
\begin{definition} {Group}
    A non-empty set \(G\) equipped with a binary operation \(\cdot: G \times G \to G\) is called a \textbf{group} if they satisfy the following \textit{group axioms}:

    \begin{enumerate}
        \item (Associativity) \(\forall a, b, c \in G, (a \cdot b) \cdot c = a \cdot (b \cdot c)\)
        \item (Identity) \(\exists e \in G \st \forall g \in G, e \cdot g = g \cdot e = g\)
        \item (Inverse) \(\forall g \in G, \exists g' \in G \st g'g = e\). Then \(g^{-1} \coloneqq g'\)
    \end{enumerate}
\end{definition}
\begin{remark}
    The above axioms of a group imply:

    \begin{enumerate}
        \item Identity is unique
        \item Inverse is unique
    \end{enumerate}

    Even when one restricts the axioms to just simply having left identity and left inverse, the above remark still holds
\end{remark}
\begin{definition} {Subgroup}
    \(H \subseteq G\) is a \textbf{subgroup} of \(G\) if

    \begin{enumerate}
        \item \(e \in H\)
        \item \(H\) is closed under \(\cdot\) and taking inverses
    \end{enumerate}
\end{definition}

\begin{definition} {Abelian Group}
    A \textbf{group} \((G, \cdot)\) is \textbf{Abelian} if the operation \(\cdot\) is also commutative.

    Recap: Associativity, Commutativity, Identity, Inverse
\end{definition}
\begin{definition} {Ring}
    A \textbf{ring} is a set \(R\) equipped with two binary operations (addition, multiplication) \(+, \cdot: R \times R \to R\) \st they satisfy the following \textit{ring axioms}:

    \begin{enumerate}
        \item \((R, +)\) is an Abelian group (Associativity, Commutativity, Identity, Inverse)
        \item (Multiplicative Associativity) \(\forall a, b, c \in R, (a \cdot b) \cdot c = a \cdot (b \cdot c)\)
        \item (Multiplicative Identity) \(\forall r \in R, \exists 1 \in R \st 1 \cdot r = r \cdot 1 = r\)
        \item (Left, Right Distributivity) This property governs how the 2 operations interact \[
                  a \cdot (b + c) = (a \cdot b) + (a \cdot c)
              \]
              \[
                  (b + c) \cdot a = (b \cdot a) + (c \cdot a)
              \]
    \end{enumerate}

    Note that multiplicative commutativity and multiplicative inverse are not here!
\end{definition}

\begin{remark}
    One can derive that when a ring has 0 = 1 (additive identity = multiplicative identity), then it must be a trivial \textit{zero ring} of \(\{0\}\)
\end{remark}

\begin{definition} {Rng}
    A \textbf{rng} is a ring without the requirement of a multiplicative identity.

    Recap: Abelian group, multiplicative associativity, distributivity
\end{definition}

\begin{definition} {Commutative Ring}
    A \textbf{commutative ring} is a ring, with the additional requirement of multiplication being commutative.

    Recap: Abelian group; multiplicative associativity, commutativity and identity; distributivity
\end{definition}

\begin{definition} {Field}
    A \textbf{field} is a set \(F\) equipped with 2 binary operations (addition, multiplication) \(+, \cdot: F \times F \to F\) \st

    \begin{enumerate}
        \item Addition and multiplication are associative
        \item Addition and multiplication are commutative
        \item There is an additive inverse 0
        \item \(\forall a \in F \setminus {0}, \exists a^{-1} \st a^{-1}a = 1\)
        \item Distributivity
    \end{enumerate}

    In short, a field is a commutative ring where non-zero elements have multiplicative inverses. The non-zero elements then form a group equipped with multiplication with 1 as their identity.

    Recap: Abelian group; multiplicative associativity, commutativity, identity and inverse (inverse only for non-zero); distributivity
\end{definition}

\begin{definition} {Ordered Field}
    [Munkres - Ch.1 p.32]
\end{definition}
\section{The -isms}
\begin{definition} {Homomorphism}
    \textit{homo}: Greek \textit{homos}, meaning ``same''

    \textit{morphism}: Greek \textit{morphism}, meaning ``shape, form''

    A \textbf{homomorphism} is a map \(f: A \to B\),  where \(A, B\) are (very generically) algebraic structures of the same type \(G\) and are therefore equipped with the same kind(s) of operation, WLOG, namely \(\cdot_A, \cdot_B: G^k \to G\) \; (e.g. groups, vector spaces). A homomorphism \(f\) then preserves the structures of these operations, i.e. \[
        f(x) \cdot_B f(y) = f(x \cdot_A y) \forall x, y \in A
    \]
    if \(\cdot\) is binary, and the same concept applies for the general \(k-\)ary case.
\end{definition}

\:\newline
\begin{example}
    A \textbf{group homomorphism} is a homomorphism in a group, where the homomorphism preserves the \(\cdot\) equipped by the group.
\end{example}

\begin{definition} {Isomorphism}
    \textit{iso}: Greek \textit{isos}, meaning ``equal''

    An \textbf{isomorphism} is a homomorphism that is bijective.
\end{definition}

\begin{remark}
    For me, it's so common to just intuitively think that a homomorphism must be bijective, but no!

    e.g. \(f: \bbz \to \{0\}\), both equipped with \(+\)
\end{remark}

\begin{definition} {Endomorphism}
    \textit{endo}: Greek \textit{endon}, meaning ``in, within''

    An \textbf{endomorphism} is a homomorphism that has the same domain and codomain, i.e. \[
        f: A \to A
    \]
\end{definition}

\begin{definition} {Automorphism}
    \textit{auto}: Greek \textit{autos}, meaning ``self''

    An \textbf{automorphism} is an endomorphism that is also an isomorphism, (or vice versa)
\end{definition}

\section{The -algebras}
This section stemmed from my need to study probability and felt the urgent need to categorize what I'm working with immediately.

\begin{definition} {Field of sets, Algebra}
    A \textbf{field of sets} is a pair \((X, \calf)\), where \(X\) is a set and \(\calf\) is a \textit{collection} of subsets of \(X\), that satisfies:
    \begin{enumerate}
        \item \(\O \in \calf\)
        \item (Closed under complementation) \[\calf \setminus F \in \calf \forall F \in \calf\]
        \item (Closed under finite unions) \[\bigcup_{k=1}^n F_k \in \calf \forall F_1, F_2, \dots, F_k \in \calf\]
    \end{enumerate}

    \(\calf\) is then called an \textbf{algebra over \(X\)}
\end{definition}
\begin{remark}
    The property that \(\calf\) is closed under finite unions also implies that it is closed under finite intersections, simply by applying De Morgan's Law.

    Furthermore, one can think of \(\calf\) as consisting of \textit{the admissible sets of \(X\)}, the \textit{complexes of \(X\)}, the nice ones that we can handle and get a hold of. In most contexts, this \(\calf\) would be where we define a lot of things on, as not all subsets of \(X\) are nice to work with.
\end{remark}

\begin{definition} {\(\sigma-\)field of sets, \(\sigma-\)algebra}
    A \textbf{\(\sigma-\)field of sets} is a field of sets \((X, \calf)\) that also satisfies:
    \begin{enumerate}
        \item[4.] (Closed under countable unions) \[
                \bigcup_{i=1}^{\infty} F_i \in \calf \forall F_1, F_2, \dots \in \calf
            \]
    \end{enumerate}
    \(\calf\) is then called a \textbf{\(\sigma-\)algebra over \(X\)}
\end{definition}
\begin{remark}
    The property that a \(\sigma-\)algebra is closed under countable unions also implies that it is closed under countable intersections, again, by an application of De Morgan's Law.
\end{remark}

\section{Linear Algebra}
\subsection{Vector Spaces}
\begin{definition} {Vector Spaces}
    A \textbf{vector space} over a field \(F\) is a non-empty set \(V\) equipped with 2 binary operations (vector addition, scalar multiplication): \(+: V \times V \to V, \cdot: F \times V \to V\) that satisfies:
    \begin{enumerate}
        \item (Abelian Group) \((V, +)\) forms an abelian group
        \item (Scalar and Field Multiplication)  \[
                  (a \cdot_F b) \cdot v = a \cdot (b \cdot V)
              \]
        \item (Field Multiplicative Identity) \[
                  1_F \cdot v = v \forall v \in V
              \]
        \item (Distributivity) \(\forall a \in F; u, v \in V\)
              \[
                  a \cdot (u + v) = (a \cdot u ) + (a \cdot v)
              \]\[
                  (a +_F b) \cdot v = a \cdot v + b \cdot v
              \]
    \end{enumerate}
\end{definition}
\begin{definition} {Linear Map}
    Let \(V, W\) be vector spaces over the same field \(F\). Then a \textbf{linear map} is a function \(f: V \to W\) that is operations-preserving, i.e. satisfying:
    \begin{enumerate}
        \item (Preserving Addition) \[
                  f(v_1 +_V v_2) = f(v_1) +_W f(v_2) \forall v_1, v_2 \in V
              \]
        \item (Preserving Scalar Multiplication) \[
                  f(c \cdot_V v) = c \cdot_W f(v) \forall c \in F, v \in V
              \]
    \end{enumerate}
    More generally, \[
        f(c_1 v_1 + c_2 v_2 + \cdots + c_n v_n) = c_1 f(v_1) + c_2 f(v_2) + \cdots + c_n f(v_n)
    \]

    In other words, a linear map is a vector space homomorphism.
\end{definition}

\begin{definition} {Linear Isomorphism}
    A \textbf{linear isomorphism} is a linear map that is also bijective.
\end{definition}

\begin{definition} {Linear Operator/Linear Endomorphism}
    A \textbf{linear operator} or a \textbf{linear endomorphism} is a linear map that has the same domain and codomain, i.e. a linear map \(f: V \to V\).
\end{definition}
\section{Analysis}
\subsection{Metric Spaces}
\begin{definition} {Metric Spaces}

\end{definition}
\begin{definition} {Isometry}
    \textit{metry}: Greek \textit{metria}, meaning ``measuring, measure''

    An \(\)
\end{definition}

\section{Topology}
\begin{remark}
    On indexing: \begin{enumerate}
        \item \(\{E_1, E_2, \dots, E_N\}\), sometimes I like to use \(\{{E_k}\}_{k\leq N}\), suggests a finite indexing
        \item \(\{E_1, E_2, \dots\}\) or \(\{E_n\}_{n \in \bbn}\) suggests a countable indexing
        \item \(\{E_\alpha\}\) suggests an uncountable indexing (which kinda encompasses all previous cases and alludes to the ``arbitrary'' nature)
    \end{enumerate}
\end{remark}
\begin{definition} {Topology, Topological Space, Open Set}
    A \textbf{topology} on set \(X\) is a collection \(\calt\) of subsets of \(X\) having the following properties:
    \begin{enumerate}
        \item \(\O, X \in \calt\)
        \item Arbitrary union of elements in \(\calt\) is in \(\calt\), i.e. \[
                  \bigcup_\alpha E_\alpha \in \calt \forall \{E_\alpha\} \subseteq \calt
              \]
        \item Finite intersection of elements in \(\calt\) is in \(\calt\), i.e. \[
                  \bigcap_{k=1}^N E_k \in \calt \forall \{E_k\}_{k \leq N} \subseteq \calt
              \]
    \end{enumerate}
    A set \(X\) with a specified topology \(\calt\) is a \textbf{topological space}. \(U \subseteq X\) is called an \textbf{open set} iff \(U \in \calt\), so think of \(\calt\) as a (huge) collection of open sets.
\end{definition}

\begin{definition} {Basis For A Topology}
    Oftentimes, one is unable to specify the entire topology \(\calt\), so we can instead specify a smaller collection oof subsets of \(X\) and then define the topology using that.

    Let \(X\) be a set, then a \textbf{basis} is a collection \(\calb\) of subsets of \(X\) (called \textbf{basis elements}) \st \begin{enumerate}
        \item \(\forall x \in X, \exists B \in \calb \st x \in B \) (wherever you are, I got you)
        \item If \(x \in B_1 \cap B_2; B_1, B_2 \in \calb\) then \(\exists B_3 \in \calb \st x \in B_3 \subseteq B_1 \cap B_2 \)
    \end{enumerate}
\end{definition}
\begin{definition} {Topology Generated By Basis}
    Call this entity that we want to generate \(\calt_\calb\) (my own notation).

    Then for subset \(U \subseteq X\), 
    \begin{center}
        \( U \in \calt_\calb\) if \(\forall x \in U, \exists B \in \calb \st x \in B \subseteq U\)
    \end{center}
\end{definition}

\subsection{Others}
\begin{definition} {Linear Continuum}

\end{definition}


\section{REU 2023}
\begin{definition} {Harmonic Function}
Let \(U\) be an open subset of \(\bbr^d\), then \(f\) is \textbf{harmonic} in \(U\) if and only if it is continuous and satisfies the mean value property: for every \(x \in U, \forall 0 < \epsilon < dist(x, \partial U)\),
\[
f(x) = MV(f; x, \epsilon) = \int_{\abs{y-x} = \epsilon} f(y) ds(y)
\]
\end{definition}
\end{document}