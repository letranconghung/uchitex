\documentclass[openany, amssymb, psamsfonts]{amsart}
\usepackage{mathrsfs,comment}
\usepackage[usenames,dvipsnames]{color}
\usepackage[normalem]{ulem}
\usepackage{url}
\usepackage[all,arc,2cell]{xy}
\UseAllTwocells
\usepackage{enumerate}
%%% hyperref stuff is taken from AGT style file
\usepackage{hyperref}  
\hypersetup{%
  bookmarksnumbered=true,%
  bookmarks=true,%
  colorlinks=true,%
  linkcolor=blue,%
  citecolor=blue,%
  filecolor=blue,%
  menucolor=blue,%
  pagecolor=blue,%
  urlcolor=blue,%
  pdfnewwindow=true,%
  pdfstartview=FitBH}   
  
\let\fullref\autoref
%
%  \autoref is very crude.  It uses counters to distinguish environments
%  so that if say {lemma} uses the {theorem} counter, then autrorefs
%  which should come out Lemma X.Y in fact come out Theorem X.Y.  To
%  correct this give each its own counter eg:
%                 \newtheorem{theorem}{Theorem}[section]
%                 \newtheorem{lemma}{Lemma}[section]
%  and then equate the counters by commands like:
%                 \makeatletter
%                   \let\c@lemma\c@theorem
%                  \makeatother
%
%  To work correctly the environment name must have a corrresponding 
%  \XXXautorefname defined.  The following command does the job:
%
\def\makeautorefname#1#2{\expandafter\def\csname#1autorefname\endcsname{#2}}
%
%  Some standard autorefnames.  If the environment name for an autoref 
%  you need is not listed below, add a similar line to your TeX file:
%  
%\makeautorefname{equation}{Equation}%
\def\equationautorefname~#1\null{(#1)\null}
\makeautorefname{footnote}{footnote}%
\makeautorefname{item}{item}%
\makeautorefname{figure}{Figure}%
\makeautorefname{table}{Table}%
\makeautorefname{part}{Part}%
\makeautorefname{appendix}{Appendix}%
\makeautorefname{chapter}{Chapter}%
\makeautorefname{section}{Section}%
\makeautorefname{subsection}{Section}%
\makeautorefname{subsubsection}{Section}%
\makeautorefname{theorem}{Theorem}%
\makeautorefname{thm}{Theorem}%
\makeautorefname{cor}{Corollary}%
\makeautorefname{lem}{Lemma}%
\makeautorefname{prop}{Proposition}%
\makeautorefname{pro}{Property}
\makeautorefname{conj}{Conjecture}%
\makeautorefname{defn}{Definition}%
\makeautorefname{notn}{Notation}
\makeautorefname{notns}{Notations}
\makeautorefname{rem}{Remark}%
\makeautorefname{quest}{Question}%
\makeautorefname{exmp}{Example}%
\makeautorefname{ax}{Axiom}%
\makeautorefname{claim}{Claim}%
\makeautorefname{ass}{Assumption}%
\makeautorefname{asss}{Assumptions}%
\makeautorefname{con}{Construction}%
\makeautorefname{prob}{Problem}%
\makeautorefname{warn}{Warning}%
\makeautorefname{obs}{Observation}%
\makeautorefname{conv}{Convention}%


%
%                  *** End of hyperref stuff ***

%theoremstyle{plain} --- default
\newtheorem{thm}{Theorem}[section]
\newtheorem{cor}{Corollary}[section]
\newtheorem{prop}{Proposition}[section]
\newtheorem{lem}{Lemma}[section]
\newtheorem{prob}{Problem}[section]
\newtheorem{conj}{Conjecture}[section]
%\newtheorem{ass}{Assumption}[section]
%\newtheorem{asses}{Assumptions}[section]

\theoremstyle{definition}
\newtheorem{defn}{Definition}[section]
\newtheorem{ass}{Assumption}[section]
\newtheorem{asss}{Assumptions}[section]
\newtheorem{ax}{Axiom}[section]
\newtheorem{con}{Construction}[section]
\newtheorem{exmp}{Example}[section]
\newtheorem{notn}{Notation}[section]
\newtheorem{notns}{Notations}[section]
\newtheorem{pro}{Property}[section]
\newtheorem{quest}{Question}[section]
\newtheorem{rem}{Remark}[section]
\newtheorem{warn}{Warning}[section]
\newtheorem{sch}{Scholium}[section]
\newtheorem{obs}{Observation}[section]
\newtheorem{conv}{Convention}[section]

%%%% hack to get fullref working correctly
\makeatletter
\let\c@obs=\c@thm
\let\c@cor=\c@thm
\let\c@prop=\c@thm
\let\c@lem=\c@thm
\let\c@prob=\c@thm
\let\c@con=\c@thm
\let\c@conj=\c@thm
\let\c@defn=\c@thm
\let\c@notn=\c@thm
\let\c@notns=\c@thm
\let\c@exmp=\c@thm
\let\c@ax=\c@thm
\let\c@pro=\c@thm
\let\c@ass=\c@thm
\let\c@warn=\c@thm
\let\c@rem=\c@thm
\let\c@sch=\c@thm
\let\c@equation\c@thm
\numberwithin{equation}{section}
\makeatother

%%%% MATH SHORTHANDS %%%%
%% blackboard bold math capitals
\newcommand{\bbf}{\mathbb{F}}
\newcommand{\bbn}{\mathbb{N}}
\newcommand{\bbq}{\mathbb{Q}}
\newcommand{\bbr}{\mathbb{R}}
\newcommand{\bbz}{\mathbb{Z}}
\newcommand{\bbc}{\mathbb{C}}
\newcommand{\bbk}{\mathbb{K}}
\newcommand{\bbm}{\mathbb{M}}
\newcommand{\bbp}{\mathbb{P}}
\newcommand{\bbe}{\mathbb{E}}


\newcommand{\calb}{\mathcal{B}}
\newcommand{\calf}{\mathcal{F}}
\newcommand{\calt}{\mathcal{T}}
\newcommand{\call}{\mathcal{L}}

\renewcommand{\phi}{\varphi}

% Universal Math Shortcuts
\newcommand{\st}{\hspace*{4pt}\text{s.t.}\hspace*{4pt}}
\newcommand{\pffwd}{\hspace*{2pt}\fbox{\(\Rightarrow\)}\hspace*{10pt}}
\newcommand{\pfbwd}{\hspace*{2pt}\fbox{\(\Leftarrow\)}\hspace*{10pt}}
\newcommand{\contra}{\ensuremath{\Rightarrow\Leftarrow}}

\let\oldforall\forall
\renewcommand{\forall}{\;\oldforall\; }
\let\oldexist\exists
\renewcommand{\exists}{\;\oldexist\; }
\newcommand\existu{\;\oldexist!\: }


\renewcommand{\_}[1]{\underline{#1}}
\bibliographystyle{plain}

%--------Meta Data: Fill in your info------
\title{Something Harmonic Function}

\author{Hung Le Tran}

\date{Summer 2023}

\begin{document}

\begin{abstract}

    This is the abstract

\end{abstract}

\maketitle

\tableofcontents

\section{Harmonic Function}

\begin{defn} [Harmonic Function]
    Recall that a function \(f\) on \(\bbz^d\) is harmonic at \(x\) if \(f(x)\) equals the average of \(f\) on its nearest neighbors. If \(U\) is an open subset of \(\bbr^d\), we will say that \(f\) is \textbf{harmonic} in \(U\) if and only if it is continuous and satisfies the following \textbf{mean value property}: for every \(x \in U\), and every \(0 < \epsilon < dist(x, \partial U)\),
    \begin{equation}
        f(x) = MV(f; x, \epsilon) = \int_{|y-x| = \epsilon}f(y) ds(y)
    \end{equation}
\end{defn}

\begin{rem}
    Value at \(x\) equals to average of ball radius \(\epsilon\) around \(x\) for all \(\epsilon\)
\end{rem}

\begin{defn} [Laplacian]
    \[
        \Delta f(x) = \lim_{e \to 0} \dfrac{1}{\epsilon^2}\sum_{y \in \bbz^d, |y| = 1} [f(x + \epsilon y) - f(x)]
    \]
\end{defn}
\begin{rem}
    Just taking in each direction, not the whole ball!
\end{rem}
\begin{prop} [Representing Laplacian in partial derivatives]
    Suppose \(f\) is \(C^2\) in a neighborhood of \(x\) in \(\bbr^d\). Then \(\Delta f(x) \)exists at \(x\) and \[
        \Delta f(x) = \sum_{j=1}^{d} \partial_{jj} f(x)
    \]
\end{prop}
\begin{proof}
    This comes naturally from the above definition of \(\Delta f(x)\), as well as the approximation one can make from the \(C^2\) smoothness of \(f(x)\).
\end{proof}

\begin{prop}
    If \(f\) is \(C^2\) in a neighborhood of \(x\), then \begin{equation}
        \dfrac{1}{2d}  \Delta f(x) = \lim_{\epsilon \to 0} \dfrac{MV(f; x, \epsilon) - f(x)}{\epsilon^2}
    \end{equation}
\end{prop}

\begin{thm}

    \textbf{BIG THEOREM! Stating the equivalence of a harmonic function with its Laplacian operator}
    
    A function in a domain \(U\) is harmonic if and only if \(f\) is \(C^2\) with \(\Delta f(x) = 0 \forall x \in U\)
\end{thm}
\newpage

\section*{Acknowledgments}  You should thank anyone who deserves thanks, and for sure you should
thank your mentor.   ``It is a pleasure to thank my mentor,
his/her name, for ....  ".   Or add anyone else, for example ``I thank [another participant] for helping
me understand [something or other]"

\section{Bibliography}
\begin{thebibliography}{9}

    \bibitem{ams} http://www.ams.org/publications/authors/tex/amslatex

    \bibitem{amsshort}
    Michael Downes.
    Short Math Guide for \LaTeX.
    http://tex.loria.fr/general/downes-short-math-guide.pdf

    \bibitem{May}
    J. P. May.
    A Concise Course in Algebraic Topology.
    University of Chicago Press. 1999.

    \bibitem{notsoshort}
    Tobias Oekiter, Hubert Partl, Irene Hyna and Elisabeth Schlegl.
    The Not So Short Introduction to \LaTeX 2e.
    https://tobi.oetiker.ch/lshort/lshort.pdf

\end{thebibliography}

\end{document}

