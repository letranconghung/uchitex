\documentclass[openany, amssymb, psamsfonts]{amsart}
\usepackage{mathrsfs,comment}
\usepackage[usenames,dvipsnames]{color}
\usepackage[normalem]{ulem}
\usepackage{url}
\usepackage[all,arc,2cell]{xy}
\UseAllTwocells
\usepackage{enumerate}
%%% hyperref stuff is taken from AGT style file
\usepackage{hyperref}  
\hypersetup{%
  bookmarksnumbered=true,%
  bookmarks=true,%
  colorlinks=true,%
  linkcolor=blue,%
  citecolor=blue,%
  filecolor=blue,%
  menucolor=blue,%
  pagecolor=blue,%
  urlcolor=blue,%
  pdfnewwindow=true,%
  pdfstartview=FitBH}   
  
\let\fullref\autoref
%
%  \autoref is very crude.  It uses counters to distinguish environments
%  so that if say {lemma} uses the {theorem} counter, then autrorefs
%  which should come out Lemma X.Y in fact come out Theorem X.Y.  To
%  correct this give each its own counter eg:
%                 \newtheorem{theorem}{Theorem}[section]
%                 \newtheorem{lemma}{Lemma}[section]
%  and then equate the counters by commands like:
%                 \makeatletter
%                   \let\c@lemma\c@theorem
%                  \makeatother
%
%  To work correctly the environment name must have a corrresponding 
%  \XXXautorefname defined.  The following command does the job:
%
\def\makeautorefname#1#2{\expandafter\def\csname#1autorefname\endcsname{#2}}
%
%  Some standard autorefnames.  If the environment name for an autoref 
%  you need is not listed below, add a similar line to your TeX file:
%  
%\makeautorefname{equation}{Equation}%
\def\equationautorefname~#1\null{(#1)\null}
\makeautorefname{footnote}{footnote}%
\makeautorefname{item}{item}%
\makeautorefname{figure}{Figure}%
\makeautorefname{table}{Table}%
\makeautorefname{part}{Part}%
\makeautorefname{appendix}{Appendix}%
\makeautorefname{chapter}{Chapter}%
\makeautorefname{section}{Section}%
\makeautorefname{subsection}{Section}%
\makeautorefname{subsubsection}{Section}%
\makeautorefname{theorem}{Theorem}%
\makeautorefname{thm}{Theorem}%
\makeautorefname{cor}{Corollary}%
\makeautorefname{lem}{Lemma}%
\makeautorefname{prop}{Proposition}%
\makeautorefname{pro}{Property}
\makeautorefname{conj}{Conjecture}%
\makeautorefname{defn}{Definition}%
\makeautorefname{notn}{Notation}
\makeautorefname{notns}{Notations}
\makeautorefname{rem}{Remark}%
\makeautorefname{quest}{Question}%
\makeautorefname{exmp}{Example}%
\makeautorefname{ax}{Axiom}%
\makeautorefname{claim}{Claim}%
\makeautorefname{ass}{Assumption}%
\makeautorefname{asss}{Assumptions}%
\makeautorefname{con}{Construction}%
\makeautorefname{prob}{Problem}%
\makeautorefname{warn}{Warning}%
\makeautorefname{obs}{Observation}%
\makeautorefname{conv}{Convention}%


%
%                  *** End of hyperref stuff ***

%theoremstyle{plain} --- default
\newtheorem{thm}{Theorem}[section]
\newtheorem{cor}{Corollary}[section]
\newtheorem{prop}{Proposition}[section]
\newtheorem{lem}{Lemma}[section]
\newtheorem{prob}{Problem}[section]
\newtheorem{conj}{Conjecture}[section]
%\newtheorem{ass}{Assumption}[section]
%\newtheorem{asses}{Assumptions}[section]

\theoremstyle{definition}
\newtheorem{defn}{Definition}[section]
\newtheorem{ass}{Assumption}[section]
\newtheorem{asss}{Assumptions}[section]
\newtheorem{ax}{Axiom}[section]
\newtheorem{con}{Construction}[section]
\newtheorem{exmp}{Example}[section]
\newtheorem{notn}{Notation}[section]
\newtheorem{notns}{Notations}[section]
\newtheorem{pro}{Property}[section]
\newtheorem{quest}{Question}[section]
\newtheorem{rem}{Remark}[section]
\newtheorem{warn}{Warning}[section]
\newtheorem{sch}{Scholium}[section]
\newtheorem{obs}{Observation}[section]
\newtheorem{conv}{Convention}[section]

%%%% hack to get fullref working correctly
\makeatletter
\let\c@obs=\c@thm
\let\c@cor=\c@thm
\let\c@prop=\c@thm
\let\c@lem=\c@thm
\let\c@prob=\c@thm
\let\c@con=\c@thm
\let\c@conj=\c@thm
\let\c@defn=\c@thm
\let\c@notn=\c@thm
\let\c@notns=\c@thm
\let\c@exmp=\c@thm
\let\c@ax=\c@thm
\let\c@pro=\c@thm
\let\c@ass=\c@thm
\let\c@warn=\c@thm
\let\c@rem=\c@thm
\let\c@sch=\c@thm
\let\c@equation\c@thm
\numberwithin{equation}{section}
\makeatother

\bibliographystyle{plain}

%--------Meta Data: Fill in your info------
\title{DGQT Technical Assessment}

\author{Hung Le}

\begin{document}

\maketitle

\section{Buffet}

\subsection*{Problem 1.1} \mbox{}

\noindent \textbf{Solution:}

Let \textbf{P1}, \textbf{P2} be 2 players of the game, with \textbf{P1} playing first.
Let $k$ be the probability that the person that plays first wins. It follows that:
\begin{enumerate}
  \item The probability that \textbf{P1} wins is $k$
  \item The probability that the second player wins is $(1-k)$ 
\end{enumerate}

On the first turn, if \textbf{P1} flips H, he then wins. If \textbf{P1} flips T, the game essentially resets with \textbf{P2} playing first and \textbf{P1} becomes the second player, at which point his probability of winning is $(1-k)$.

Therefore:
\begin{align*}  
  k &= P(\text{P1 flips H})\times P(\text{P1 wins}|\text{P1 flips H})\\
  &+ P(\text{P1 flips T})\times P(\text{P1 wins}|\text{P1 flips T})\\
  & = \frac{1}{2} \times 1 + \frac{1}{2} \times (1-k) \\
  & = 1 - \frac{k}{2}\\
  \frac{3k}{2} &= 1\\
  \Rightarrow k &= \frac{2}{3}
\end{align*}

Therefore the probability that \textbf{P1} wins is $\frac{2}{3}$, that \textbf{P2} wins is $1- \frac{2}{3} = \frac{1}{3}$ (probability of neither player winning $= \lim_{n\rightarrow\infty}{\left(\frac{1}{2}\right)^n} = 0$). Since $ \frac{2}{3} > \frac{1}{3}$, it does matter who goes first (the first player has a higher probability of winning) and I would thus prefer to go first.

\subsection*{Problem 1.4} \mbox{}

\noindent \textbf{Solution:}

Using put-call parity:
\begin{align*}
  c + Ke^{-rT} &= p + S_0 \\
  12.35 + 140e^{-4 \% \times \frac{3}{12}} &= p + 142.16 \\
  p &\approx 8.79698
\end{align*}

Thus the price of said put option is \$8.79698.
\subsection*{Problem 1.5} \mbox{}

\noindent\textbf{Solution:}

Following the Black-Scholes Differential Equation, the parameter drift rate $\mu$ is absent from the equation itself, therefore does not affect the solution. Since all other parameters are equal, current prices $f_1 = f_2$.

\section{Challenge Problems}
\subsection*{Problem 2.2} \mbox{}

\noindent\textbf{Solution:}

Given:
\[
  \left\{
    \begin{array}{ccl}
      S_0 &=&50\\
      \sigma &=& 0.25\\
    \end{array}  
  \right.
\]

(a) Annual expected return = $(1+14\%)^4-1\approx0.68896 \approx 68.90\%$

(b) Since the price follows a geometric Brownian motion:
\[
    dS = \mu Sdt + \sigma SdW
\]

Integrating both sides:
\begin{equation} \label{eq:brownian}
  \begin{split}
    S_t &= S_0e^{(\mu - \frac{\sigma^2}{2})t+\sigma W_t}\\
    &= 50e^{(0.68896-\frac{0.25^2}{2})t + 0.25W_t}\\
    &= 50e^{0.65771t + 0.25W_t}
  \end{split}  
\end{equation}

(c) From Equation (\ref{eq:brownian}) it follows that:

\[  
  \ln\left(\frac{S_t}{50}\right) = 0.65771t + 0.25W_t
\]

Since $W_t \sim \mathcal{N}(0, t)$:

\[\Rightarrow \ln\left(\frac{S_t}{50}\right) \sim \mathcal{N}(0.65771t, 0.25^2t)\]

At 9 months, $t = \frac{9}{12} = 0.75$. Therefore:

\[\ln\left(\frac{S_{0.75}}{50}\right) \sim \mathcal{N}(0.65771 \times 0.75, 0.25^2\times 0.75) = \mathcal{N}(0.4932825, 0.046875) \]

\[
  \Rightarrow \ln(S_{0.75}) \sim \mathcal{N}(\ln50 + 0.4932825, 0.046875) 
\]

Thus after 9 months the stock price will follow a lognormal distribution with the above parameters.

(d) The probability that the stock price will be greater than \$65 in 9 months:
\[
  \begin{split}
    \mathbb{P}(S_{0.75} > 65) &= \mathbb{P}(\ln S_{0.75} > \ln 65)\\
    &= 1 - normcdf(\ln 65,\ln 50 + 0.4932825,0.046875) \\
    &= 0.856916
  \end{split}
\]

(e)
At t = 0.75:
\[
  \begin{split}
    \mathbb{E}[X_t] &= X_0e^{\mu t}\\
    &= 50e^{0.68896\times 9 \div 12}\\
    &\approx 83.8260 \\
    \Rightarrow \text{Expected payoff} &\approx \$83.8260 - \$65\\
    &= \$18.8260
  \end{split}  
\]
\end{document}

