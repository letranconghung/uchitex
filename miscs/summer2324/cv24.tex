% Document class and font size
\documentclass[a4paper, 11pt]{extarticle}

% Packages
\usepackage[utf8]{inputenc} % For input encoding
\usepackage{geometry} % For page margins
\geometry{a4paper, margin=.75in} % Set paper size and margins
\usepackage{titlesec} % For section title formatting
\usepackage{enumitem} % For itemized list formatting
\usepackage{hyperref} % For hyperlinks
\usepackage{amsmath,amsfonts,mathtools,amsthm,amssymb}
\usepackage[usenames,dvipsnames,pdftex]{xcolor}

\hypersetup{
    % Enable highlighting links.
    colorlinks,
    % Change the color of links to blue.
    urlcolor=[RGB]{4, 46, 214},
    % Change the color of citations to black.
    citecolor={black},
    % Change the color of url's to blue with some black.
    linkcolor={blue!20!black}
}
% Formatting
\setlist{noitemsep} % Removes item separation
\titleformat{\section}{\large\bfseries}{\thesection}{1em}{}[\titlerule] % Section title format
\titlespacing*{\section}{0pt}{\baselineskip}{\baselineskip} % Section title spacing

% Begin document
\begin{document}

% Disable page numbers
\pagestyle{empty}

% Header
\begin{center}
\textbf{\Large Hung C. Le Tran}\\
5500 S University Ave, Chicago, IL \\
\href{mailto:conghunglt@uchicago.edu}{conghunglt@uchicago.edu} \quad  (872) 219--6917
\end{center}

% Education Section
\section*{EDUCATION}
\noindent
\textbf{University of Chicago} \hfill Chicago, IL\\
Bachelor of Science in Mathematics \hfill Expected, June 2026\\
Overall GPA: 3.9, Major GPA: 4.0

\section*{HONORS AND AWARDS}
\begin{itemize}
    \item Dean's List, 2022--2023
    \item Gold Medal in Singapore Mathematics Olympiad, 2019, 2021
    \item Silver Medal in Singapore National Olympiad in Informatics, 2021
\end{itemize}

\section*{EXPERIENCE}
\textbf{Math REU Apprentice Program} \hfill University of Chicago \\
\textit{Program Participant} \hfill June 2023--August 2023
\begin{itemize}
    \item Attended 5 weeks of the Apprentice Program on group theory, quadratic reciprocity, graph theory. Attended 8 weeks of Probability and Analysis lectures on random walks, geometric measure theory, Fourier analysis, stochastic calculus.
    \item Wrote an \href{https://math.uchicago.edu/~may/REU2023/REUPapers/LeTran.pdf}{\underline{expository paper}} on partial differential equations, specifically linear PDE techniques, the Laplacian operator, and Laplace's and heat equations with their probabilistic interpretations.
\end{itemize}

\noindent
\textbf{Directed Reading Program} \hfill University of Chicago \\
\textit{Mentee} \hfill January 2024--present
\begin{itemize}
    \item Read \href{https://arxiv.org/pdf/1511.00442.pdf}{\underline{papers}} on Kakeya sets and their relation to geometric measure theory and complexity theory with a graduate mentor.
\end{itemize}

\section*{RELEVANT COURSEWORK}
\textbf{Current}
\begin{itemize}
    \item MATH 20800: Honors Analysis in $\mathbb{R}^n$ II
    \item MATH 26200: Point-Set Topology
    \item TTIC 31020: Introduction to Machine Learning (Graduate)
\end{itemize}
\textbf{Completed}
\begin{itemize}
    \item MATH 20700: Honors Analysis in $\mathbb{R}^n$ I
    \item MATH 20250: Abstract Linear Algebra
    \item MATH 16100--16300: Honors Calculus I--III
    \item PHYS 18500: Intermediate Mechanics
    \item PHYS 14200, 14300: Honors Electricity and Magnetism, Honors Waves, Optics and Heat
    \item CMSC 25300: Mathematical Foundations of Machine Learning
\end{itemize}

\section*{REFERENCES}
Beniada Shabani, Assistant Instructional Professor, University of Chicago\\
\href{mailto:bshabani@uchicago.edu}{\underline{bshabani@uchicago.edu}}\\
\\
Mark Cerenzia, Lecturer, Princeton University\\
\href{mailto:cerenzia@princeton.edu}{\underline{cerenzia@princeton.edu}}
\end{document}