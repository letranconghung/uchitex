% Document class and font size
\documentclass[a4paper, 11pt]{extarticle}

% Packages
\usepackage[utf8]{inputenc} % For input encoding
\usepackage{geometry} % For page margins
\geometry{a4paper, margin=.75in} % Set paper size and margins
\usepackage{titlesec} % For section title formatting
\usepackage{enumitem} % For itemized list formatting
\usepackage{hyperref} % For hyperlinks
\usepackage{amsmath,amsfonts,mathtools,amsthm,amssymb}
\usepackage[usenames,dvipsnames,pdftex]{xcolor}

\hypersetup{
    % Enable highlighting links.
    colorlinks,
    % Change the color of links to blue.
    urlcolor=[RGB]{4, 46, 214},
    % Change the color of citations to black.
    citecolor={black},
    % Change the color of url's to blue with some black.
    linkcolor={blue!20!black}
}
% Formatting
\setlist{noitemsep} % Removes item separation
\titleformat{\section}{\large\bfseries}{\thesection}{1em}{}[\titlerule] % Section title format
\titlespacing*{\section}{0pt}{\baselineskip}{\baselineskip} % Section title spacing

% Begin document
\begin{document}

% Disable page numbers
\pagestyle{empty}
\begin{center}
    \textbf{\Large{Statement of Purpose}}
\end{center}

Dear Professor Flick,\\

My name is Hung Le Tran, and I am a second year at the University of Chicago majoring in mathematics. I am applying to the CMU Math REU in Geometry and Topology  in a   Discrete Setting (GTDS), and am very interested in attending. If accepted, I would gain immense mathematical maturity both in breadth and depth, acquire new experiences that would shape my academic journey in mathematics, and build invaluable relationships with like-minded peers and mentors.\\

For a long time, I have been involved in Math Olympiads, in which the challenge to flexibly use and connect ideas in algebra, geometry and number theory to solve presented problems always keeps me excited. My first taste of independent mathematical “research”, however, came in my last year of high school, when I wrote my Extended Essay on the general formula for orthogonal regression. The most fulfilling moment then was when I learnt of Marden's Theorem while working on the 3-point case, a theorem that relates roots of complex polynomials to the Steiner inellipse of a triangle, two concepts that I had initially thought to be wholly separate. This was when I started appreciating the beauty of abstract mathematics and its intricate links, for a basic example, here, between geometry and analysis.\\

This past summer, I participated in the UChicago Math REU Apprentice Program, my best mathematics experience yet. In 5 weeks, I was introduced to group theory, quadratic reciprocity, and graph theory, while also attending the 8-week Probability and Analysis lectures of the Full Program, which covered random walks, geometric measure theory, Fourier analysis, stochastic calculus, as well as the Number Theory lectures where broad ideas on analytic number theory were discussed.\\

Most importantly, throughout the REU, I really enjoyed working through Lawler's \textit{Random Walk and the Heat Equation} and Evans' \textit{Partial Differential Equations} under the mentorship of Prof. Beniada Shabani, building up to an expository work on the PDE topics I read, unifying the analytic and probabilistic viewpoints employed in my readings on Laplace's and heat equations. This, again, was fascinating to me, when one can model and view the same problem from varying perspectives. I thoroughly loved both the self-directed and mentored aspects of the reading process, and certainly look forward to the same experience again this coming summer, doing a lot of self-initiated work while receiving close mentorship, if I get the opportunity to be involved in GTDS.\\

This year, in pursuit of more mentored reading experience, I signed up for the Directed Reading Program, in which I am reading papers on Kakeya sets and their connection to complexity theory. Hence, the discrete setting of geometry and topology that GTDS intends to explore really excites me. For my coursework, I am taking the Honors Analysis sequence and Point-Set Topology, a topic I became interested in after seeing the power of viewing objects through the topological lens in real analysis.\\

I believe that I have a great desire to learn more mathematics and an immense passion to conduct independent studies, and thus sincerely hope that you can grant me the opportunity to be part of the program. Thank you for your time and consideration.\\

All my best,\\

Hung Le Tran
\end{document}